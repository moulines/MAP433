Le traitement du signal discret a pris son essor dans les ann\'ees
70 gr\^ace \`a
l'apparition des microprocesseurs et \`a l'utilisation de
la transform\'ee de Fourier rapide.
Il remplace progressivement le
traitement du signal analogique dans la majorit\'e des applications
telles que l'enregistrement num\'erique, la diffusion de la t\'el\'evision, le
traitement de la parole et de l'image.
Le calcul nuym\'erique permet la mise en
place d'algorithmes
nettement plus sophistiqu\'es  que le
calcul analogique dont la fiabilit\'e est limit\'ee par
les bruits de circuits et les erreurs
de calibration des composants \'electroniques.

\section{Filtrage discret homog\`ene}

Les op\'erateurs analogiques de filtrage lin\'eaire homog\`ene
s'\'etendent aux signaux discrets en rempla\c{c}ant les
int\'egrales par des sommes discr\`etes.
La transform\'ee de Fourier est alors remplac\'ee par les
s\'eries de Fourier. Les propri\'et\'es des filtres discrets
s'analysent souvent plus facilement
avec la transform\'ee en $z$ qui \'etend les
s\'eries de Fourier \`a tout le plan complexe.
Pour simplifier les notations, nous supposons que l'intervalle d'\'echantillonnage
est $T=1$ et les \'echantillons d'un signal discret sont not\'es
$\sequence{f}[k][\zset]$.

\subsection{Convolutions discr\`etes}
\label{sec-conv}
Dans le cas discret, l'homog\'en\'eit\'e temporelle se limite
\`a des translations sur la grille d'\'echantillonnage.
Un op\'erateur lin\'eaire discret $L$ est homog\`ene dans le
temps si et seulement si pour tout $\sequence{x}[t][\nset]$ et tout
d\'ecalage $ = f[n-p]$ avec $p \in \Z$


La lin\'earit\'e et l'invariance temporelle impliquent
\[
Lf[n] = \sum_{p=-\infty}^{+\infty} f[p] h [n-p] = f \star h [n].
\]
Un op\'erateur lin\'eaire homog\`ene est donc un produit de
convolution discret.
\\
\\
{\bf Stabilit\'e et causalit\'e}
Un filtre discret $L$
est
{\it causal} si et seulement si $Lf[p]$ ne
d\'epend que des valeurs de $f[n]$ pour $n \leq p$. Cela
implique donc que $h_n = 0$ si $n < 0$.
La r\'eponse impulsionnelle $h_n$ est causale.
On repr\'esente souvent un signal causal gr\^ace \`a la fonction
de Heavyside discr\`ete
\begin{equation}
\gamma [n] =
   \left \{ \begin{array}{ll}
            1 & \mbox{si $n \geq 0$}\\
            0 & \mbox{si $n < 0$}
            \end{array}
   \right.
\end{equation}
car $h_n = h_n \gamma [n]$.


Pour qu'un signal
d'entr\'ee born\'e produise un signal de sortie born\'e il suffit que
\begin{equation}
\sum_{n=- \infty}^{+ \infty} |h [ n  ]| < + \infty ,
\end{equation}
car
\[
|Lf[n]| \leq \sup_{n \in \Z} |f[n]|
\sum_{k=- \infty}^{+ \infty} |h [ k  ]| .
\]
On peut v\'erifier (exercice) que cette condition suffisante
est aussi n\'ecessaire.
On dit alors que le filtre et la r\'eponse impulsionnelle
sont {\it stables}.


\subsection{Fonction de transfert}
Comme dans le cas continu, les vecteurs propres de ces
op\'erateurs de convolutions sont des exponentielles complexes
$e_\om[k] = \rme^{i\om k}$,
\begin{equation}
L e_\om [n] = \sum_{k=- \infty}^{+ \infty}
\rme^{i\om (n-k)} h _{k} =
\rme^{i \om n} \sum_{k=- \infty}^{+ \infty} h_k \rme^{-i \om k} .
\end{equation}
Les valeurs propres correspondantes sont donc obtenues
par la s\'erie de Fourier
\begin{equation}
\hat h ( \rme^{\rmi \lambda} ) = \sum_{k=- \infty}^{+ \infty} h_k \rme^{-i\lambda k} ,
\end{equation}
que l'on appelle fonction de transfert du filtre.

\subsection{S\'eries de Fourier et transform\'e de Fourier \`a temps discret}
La transform\'ee de Fourier d'un signal discret $\sequence{x}[t][\nset]$
est d\'efinie par
\begin{equation}
\label{serie-Fourier}
\hat f (\rme^{\rmi \lambda} ) = \sum_{k=- \infty}^{+ \infty}  x_t \rme^{-\rmi \lambda t} \eqsp.
\end{equation}
Toutes les propri\'et\'es de la transform\'ee de Fourier
(\ref{symm}-\ref{impaire}) restent  valables

Rappelons que la famille $\sequence{e}[k][\zset]$, où $e_k(\lambda)= \rme^{\rmi k \lambda}$ est une
base orthogonale de $\ltwo(\tore)$ muni du produit scalaire
\[
\pscal{a}{b} =
\frac 1 {2 \pi} \int_{- \pi}^{\pi} a(\lambda) b^*(\lambda) d \lambda \eqsp.
\]
Si $\sequence{f}[k][\zset] \in \ltwo(\zset)$,
la s\'erie (\ref{serie-Fourier}) peut
alors s'interpr\'eter
comme la d\'ecomposition de la fonction $\lambda \mapsto \hat{f}({\rme^{\rmi \lambda}}) \in \ltwo(\tore)$
dans cette base orthoponale. Les coefficients de d\'ecomposition
sont obtenus par produit scalaire
\begin{equation}
\label{recons_discret}
f_n = \pscal{\hat{f}}{e_n} = \frac 1 {2 \pi}
\int_{- \pi}^{\pi} \hat \hat{f}(\rme^{\rmi \lambda}) \rme^{\rmi \lambda n} \rmd \lambda \eqsp.
\end{equation}
et
\begin{equation}
\sum_{n=-\infty}^{+\infty} |f_n|^2 =
\frac 1 {2 \pi} \int_{- \pi}^{\pi} |\hat f(\rme^{\rmi \lambda })|^2 \rmd \lambda \eqsp.
\end{equation}

\subsection{Filtrage discret}
Les exponentielles complexes \'etant les vecteurs propres des
op\'erateurs de convolution discr\`ete, il en r\'esulte
le th\'eor\`eme suivant.

\begin{theorem} [Convolution discr\`ete]
Soient $\sequence{f}[n][\zset]$ et $\sequence{h}[n][\zset]$ deux signaux dans $\pltwo$.
La transform\'ee de Fourier de $g_n = (f \star h)_n$ est
\begin{equation}
\hat{g}(\rme^{\rmi \lambda}) = \hat{f}(\rme^{\rmi \lambda}) \hat h(\rme^{\rmi \lambda}) .
\end{equation}
\end{theorem}
La d\'emonstration est
formellement identique \`a la d\'emonstration du th\'eor\`eme
\ref{th_convol} si on
remplace les int\'egrales par des sommes discr\`etes
et que l'on suppose que $\sequence{f}[n][\zset]$ et $\sequence{h}[n][\zset]$ sont dans $\plone$.
Le m\^eme r\'esultat dans $\pltwo$ s'obtient par un argument de densit\'e.

La formule de reconstruction (\ref{recons_discret})
montre qu'un signal filtr\'e s'\'ecrit
\begin{equation}
f \star h_n = \frac 1 {2 \pi}
\int_{- \pi}^{\pi} \hat h({\rme^{\rmi \lambda}}) \hat f(\rme^{\rmi \lambda }) \rme ^{\rmi \lambda n} \rmd \lambda \eqsp.
\end{equation}
La fonction de transfert $\hat{h}({\rme^{\rmi \lambda}})$ amplifie ou att\'enue
les composantes fr\'equentielles $\hat f({\rme^{\rmi \lambda}})$ du signal $\sequence{f}[n][\zset]$
dans l'intervalle de fr\'equence $\tore$.

On v\'erifie de m\^eme qu'une multiplication temporelle est
\'equivalente \`a une convolution dans le domaine fr\'equentiel.
Si $g_n = f_n w_n$ alors
\[
\hat{g}({\rme^{\rmi \lambda}}) = \frac 1 {2 \pi} \int_{-\pi}^{\pi}
\hat f(\rme^{\rmi \xi }) \hat{w} (\rme^{\rmi(\lambda -\xi)}) \rmd \xi \eqsp.
\]
\begin{example}
La moyenne discr\`ete uniforme d\'efinie par
\[
Lf[n] = \frac 1 {2N+1} \sum_{p=-N}^{+N} f_{n-p} ,
\]
est un filtre dont la r\'eponse impulsionnelle est
\begin{equation}
h_n =
   \left \{ \begin{array}{ll}
            \frac 1 {2N+1}& \mbox{si $-N \leq  n \leq N$}\\
            0 & \mbox{si $|n| > N$}
            \end{array}
   \right.
\end{equation}
La fonction de transfert est la s\'erie de Fourier
\[
\hat h({\rme^{\rmi \lambda}}) = \frac 1 {2N+1}
\sum_{n=-N}^{+N} \rme^{-\rmi n\lambda} = \frac 1 {2N+1}
\frac {\sin (N+\half)\lambda} {\sin\lambda/2} .
\]
Ce filtre att\'enue surtout les fr\'equences
au-del\`a de $ {2\pi}/ (2N+1)$.
\end{example}
\begin{example}[S\'election fr\'equentielle id\'eale]
La fonction de transfert d'un
filtre discret \'etant $2\pi$ p\'eriodique,
elle est sp\'ecifi\'ee sur l'intervalle $[-\pi,\pi]$.
La fonction de transfert
du filtre discret passe-bas id\'eal est d\'efinie pour
$\lambda \in [-\pi,\pi]$ par
\begin{equation}
\label{passe-bas-discret}
\hat h_0 ( \rme^{\rmi \lambda}  ) =
   \left \{ \begin{array}{ll}
            1 & \mbox{si $| \lambda | \leq \lambda_c$}\\
            0 & \mbox{si $| \lambda | > \lambda_c$}
            \end{array}
   \right.
\end{equation}
Sa r\'eponse impulsionnelle calcul\'ee gr\^ace
\`a l'int\'egrale (\ref{recons_discret}) est
\begin{equation}
h_0 [n] = \frac{\sin \lambda_c n} { \pi n} .
\end{equation}
C'est l'\'echantillonnage uniforme de la fonction de transfert
d'un filtre analogique passe-bas id\'eal.

La fonction transfert d'un filtre passe-bande discret id\'eal
est
\begin{equation}
\hat h_1 ( {\rme^{\rmi \lambda}}  ) =
   \left\{ \begin{array}{ll}
1 & \mbox{si $|\lambda| \in [\lambda_0 - \lambda_c , \lambda_0 + \lambda_c ]$}\\
0 & \mbox{ailleurs}
\end{array}
   \right.
\end{equation}
Comme $\hat h_1({\rme^{\rmi \lambda}}) = \hat h_0 (\rme^{\rmi(\lambda-\lambda_0) }) +
\hat h_0 (\rme^{\rmi(\lambda+\lambda_0) }) $, on peut en d\'eduire que
sa r\'eponse impulsionnelle est
\[
h_1 [n] =
2 \cos (\lambda_0 n)~ h_0[n].
\]

La convolution discr\`ete d'un signal $\sequence{f}[n][\zset]$ avec un filtre
passe-bas ou passe-bande id\'eal ne peut se calculer exactement
avec un nombre fini d'op\'erations. Il est donc n\'ecessaire
d'approximer ces filtres par des op\'erateurs de convolutions
qui se calculent en temps fini.
\end{example}


\section{Transform\'ee en z}

Pour \'etudier plus facilement les propri\'et\'es des fonctions
de transfert des filtres discrets, et en particulier des
filtres r\'ecursifs,
on introduit la transform\'ee en $z$
qui \'etend la s\'erie de Fourier
\begin{equation}
\hat{h} ({\rme^{\rmi \lambda}}) = \sum_{n=-\infty}^{+\infty} h_n {\rme^{-\rmi \lambda}}
\end{equation}
\`a tout le plan complexe $z \in \cset$, avec la s\'erie de Laurent
\begin{equation}
\hat h(z) = \sum_{n=-\infty}^{+\infty} h_n z^{-n} .
\end{equation}
\\
\\
\subsection{Anneau de convergence}
On dit que la s\'erie de Laurent $\hat h(z)$ est convergente si
\[
\sum_{n=-\infty}^{+\infty} |h_n|\, |z|^{-n} < + \infty .
\]
Le domaine de convergence ne d\'epend que de $|z|$ et est donc
isotrope. La proposition suivante
montre
que le domaine de convergence est un anneau dans le plan complexe.

\begin{proposition}
\label{conv-z}
Il existe $\rho_1$
et $\rho_2$ tels que $\hat h(z)$ est convergente
pour $\rho_1 < |z| < \rho_2$ et divergente
pour $|z| < \rho_1$ ou $|z| > \rho_2$.
On note $A(\hat h)$ l'intervale de $|z|$ sur lequel $\hat h(z)$
est convergente.
\end{proposition}

La d\'emonstration est laiss\'ee en exercice.
Dans le cas o\`u la transform\'ee en z est convergente pour
$|z| = 1$, la transform\'ee de Fourier est \'egale \`a
la restriction
de la transform\'ee en z au cercle unit\'e du plan complexe.
\\
\\
{\bf Stabilit\'e et causalit\'e} Le domaine de convergence (absolu)
de la transform\'ee en $z$ d\'epend des
propri\'et\'es de causalit\'e et de stabilit\'e du filtre.
Le filtre est causal si
$h_n = 0$ pour $n < 0$ d'o\`u l'on d\'eduit que si
$\hat h (z)$ converge pour $|z| = \rho$ alors il converge pour
$|z| \geq \rho$. L'anneau de convergence s'\'etend donc \`a l'infini
($\rho_2 = +\infty$).

Le filtre est stable si et seulement si
\[
\sum_{n=0}^{+\infty} |h_n| < +\infty .
\]
Cela signifie que l'anneau de convergence contient $|z| = 1$.
Si le filtre est causal et stable, on d\'eduit donc que
$\hat h(z)$ est convergente pour $|z| \geq 1$.
\begin{example}
\label{exam:exponential-1}
Consid\'erons le signal $f_n= a^n$ pour $n \geq 0$. Nous avons donc
\[
\hat{f}(z) = \sum_{n=0}^\infty a^n z^{-n} = \sum_{n=0}^{\infty} (a z^{-1})^n \eqsp,
\]
qui est convergente si 
\[
\sum_{n=0}^{\infty} |a z^{-1}|^n < \infty \eqsp.
\]
Le domaine de converge est donc la couronne $\ensemble{z \in \cset}{|z| > |a|}$. A l'int\'erieur du domaine de convergence la série est sommable:
\[
\hat{f}(z)= \sum_{n=0}^\infty (a z^{-1})^n = \frac{1}{1 - a z^{-1}} \eqsp.
\]
Remarquons que si $|a| > 1$, le signal diverge exponentiellement, et la s\'erie de Fourier n'est pas d\'efinie.
\end{example}
\begin{example}
\label{exam:exponential-2}
Consid\'erons maintenant le signal $\sequence{f}[n][\nset]$ donn\'e par $f_n = - a^n$ pour $n \leq -1$. La transform\'e en $z$ de ce signal est donn\'e par 
\begin{align*}
\hat{f}(z)
&= - \sum_{n=-\infty}^{-1} a^n z^{-n} \\ 
&= - \sum_{n=1}^\infty a^{-n} z^n = 1 - \sum_{n=0}^\infty (a^{-1} z)^n \eqsp.
\end{align*}
Le domaine de convergence est le disque $\ensemble{z \in \cset}{|z| < |a|}$ et sur ce domaine 
\[
\hat{f}(z)= 1 - \frac{1}{1-a^{-1} z}= \frac{1}{1-a z^{-1}}= \frac{z}{z-a} \eqsp.
\]
Remarquons que si $|a| < 1$, le signal diverge exponentiellement et la transform\'e de Fourier n'est pas d\'efinie.
\end{example}
Remarquons que les transformées en $z$ pour \Cref{exam:exponential-1} et \Cref{exam:exponential-2} co\"incident alors que les signaux associ\'es sont diff\'erents ! Il est tr\`es important quand on utilise la transfor\'ee en $z$ de sp\'ecifier le domaine sur lequel la transform\'ee est d\'efinie.

\subsection{Inversion de la transformée en $z$}
%La transform\'ee en z s'inverse par une int\'egrale de
%Cauchy sp\'ecifi\'ee par le th\'eor\`eme suivant.
%
%\begin{theorem} [Int\'egrale de Cauchy]
%\label{cauchy}
%Soit $C$ un contour qui entoure l'origine dans
%le plan complexe $\C$ et qui est inclus le domaine de
%convergence de
%\[
%\hat h(z) = \sum_{n=-\infty}^{+\infty} h_n z^{-n} .
%\]
%Alors
%\begin{equation}
%\label{residu}
%h_k = \frac 1 {2 \pi i} \oint_C \hat h(z) z^{k-1} dz  ,
%\end{equation}
%avec une int\'egration orient\'ee dans le sens trigonom\'etrique
%direct.
%\end{theorem}
%
%\noindent{\bf D\'emonstration}
%Ce r\'esultat est une cons\'equence de la
%propri\'et\'e suivante de l'int\'egrale
%de Cauchy \cite{bony}.
%Soit $C$ un contour qui entoure l'origine dans $\C$
%\begin{equation}
%\frac 1 {2 \pi i} \oint_C z^{-k} dz =
%\left\{
%\begin{array}{ll}
%1 & \mbox{ si $k=1$}\\
%0 & \mbox{ si $k\neq1$}
%\end{array} .
%\right.
%\end{equation}
%L'int\'egration est calcul\'ee dans le sens direct.
%
%Supposons que $C$ soit inclus dans le domaine de
%convergence de $\hat h(z)$.
%\[
%\frac 1 {2 \pi i} \oint_C \hat h(z) z^{k-1} dz =
%\frac 1 {2 \pi i} \oint_C
%\sum_{n=-\infty}^{+\infty} h_n z^{-n+k-1} dz .
%\]
%A l'int\'erieur du domaine de convergence, on peut inverser
%ces deux int\'egrales et le th\'eor\`eme de Cauchy implique
%\[
%\frac 1 {2 \pi i} \oint_C \hat h(z) z^{k-1} dz =
%\sum_{n=-\infty}^{+\infty} h_n
%\frac 1 {2 \pi i} \oint_C
%z^{-n+k-1} dz = h_k .
%\]
%$\Box$.\\
%
%L'int\'egrale de Cauchy (\ref{residu}) g\'en\'eralise
%la formule d'inversion des s\'eries de
%Fourier (\ref{recons_discret}).
%En effet, si le cercle unit\'e est inclus dans la r\'egion de
%convergence, l'int\'egration le long du cercle unit\'e
%est obtenue
%avec $z = \rme^{\rmi \lambda}$ et (\ref{residu}) devient
%\[
%h_k = \frac 1 {2 \pi} \int_{-\pi}^{\pi}
%\hat h ({\rme^{\rmi \lambda}}) e^{i k \lambda} d\lambda .
%\]
%
La transform\'ee en $z$ peut s'inverser mais le calcul de
$h_k$ \`a partir de $\hat h (z)$ d\'epend du domain de
convergence choisi. La formule g\'en\'erale d'inversion se
fait par une int\'egrale de Cauchy qui calcule $h_k$
en integrant $\hat h (z)$ le long d'un contour inclu
dans l'anneau de convergence.
Dans le cas o\`u l'anneau de convergence
inclu le cercle unit\'e, cette int\'egrale peut se faire le
long du cercle unit\'e, auquel cas on obtient la transform\'ee
de Fourier inverse
\[
h_k = \frac 1 {2 \pi} \int_{-\pi}^{\pi}
\hat h ({\rme^{\rmi \lambda}}) e^{i k \lambda} d\lambda .
\]

Pour montrer que
$h_k$ ne d\'epend pas seulement de
$\hat h(z)$ mais aussi du domaine de convergence choisi,
prenons par exemple
\[
\hat h(z) = \frac 1 {1 - a z^{-1}} .
\]
La r\'eponse impulsionnelle correspondant \`a la r\'egion de
convergence \`a l'ext\'erieur du cercle
$|z| = a$ est causale et se calcule par un d\'evelopement
en s\'erie de $\frac 1 {1 - x}$
\[
\hat h(z) = \sum_{n=0}^{+\infty} a^n z^{-n} ,
\]
d'o\`u $h_n = a^n \gamma[n]$.
Pour que la r\'egion de convergence soit $|z| < a$ on
r\'e\'ecrit
\[
\hat h(z) = \frac {-a^{-1} z} {1 - a^{-1}z} .
\]
En utilisant la d\'ecomposition en s\'erie de $\frac 1 {1-x}$
on obtient une r\'eponse impulsionnelle anticausale
\[
h_n = \left\{
\begin{array}{ll}
-a^n & \mbox{si $n \leq -1$}\\
0 & \mbox{si $n \geq 0$}\\
\end{array}
\right.
\]
\\
\\
\begin{example}
On utilise g\'en\'eralement un filtre causal, ce qui
impose que l'anneau de convergence s'\'etende \`a l'infini.\\
$\bullet$ Si $h_n = \delta _{n-k}$ alors
\begin{equation}
\label{transl-z}
\hat h(z) = z^{-k}
\end{equation}
et $A(\hat h) = ]0 , +\infty [$.\\
$\bullet$ Si $h_n = a^n \gamma [n]$ alors
\begin{equation}
\label{un-pole2}
\hat h(z) = \frac 1 {1 - a z^{-1}}
\end{equation}
$A(\hat h) = ]|a| , +\infty [$.\\
$\bullet$ Si $h_n = n a^n \gamma [n]$ alors
\[
\hat h(z) = \frac {a z^{-1}} {1 - a z^{-1}}
\]
$A(\hat h) = ]|a| , +\infty [$.
\end{example}

\subsection{Convolution}
Toutes les propri\'et\'es de la transform\'ee
de Fourier s'\'etendent directement
\`a la transform\'ee en $z$.
En particulier, si $g_n = (f \star h)_n$ alors la transform\'ee
en $z$ de $g_n$ est le produit
\[
\hat g(z) = \hat f(z) \hat h(z)
\]
et son anneau de convergence est
\[
A(\hat g) = A(\hat f ) \bigcap A(\hat h ) .
\]


\section{Synth\`ese de filtres discrets}

Lors de la synth\`ese de filtres discrets, tout comme dans le
cas analogique, on impose des conditions
d'att\'enuation sur le gain du filtre $|\hat h ({\rme^{\rmi \lambda}})|$.
Le probl\`eme est d'obtenir des filtres tels
que $|\hat h ({\rme^{\rmi \lambda}})|$ satisfasse aux conditions d'att\'enuation et
dont la structure permette de calculer les convolutions discr\`etes
avec le moins d'op\'erations possibles.

\subsection{Filtres r\'ecursifs}
\label{recursifs}
Pour effectuer des calculs num\'eriques, on utilise
une classe de filtres pour lesquels la convolution discr\`ete
se calcule avec un nombre fini d'op\'erations par
\'echantillon.
La sortie $\sequence{g}[n][\zset]$ est reliée à l'entrée $\sequence{f}[n][\nset]$
par une \'equation de diff\'erences
\begin{equation}
\label{recurr}
\sum_{k=0}^N a_k g_{n-k} = \sum_{k=0}^M b_k f_{n-k} ,
\end{equation}
o\`u $a_k$ et $b_k$ sont des r\'eels et $a_0 \neq 0$.
Donc
\[
g[n] = \frac 1 {a_0} \left(
\sum_{k=0}^M b_k f_{n-k} - \sum_{k=1}^N a_k g_{n-k} \right)
\]
se calcule \`a partir de son pass\'e et de $\sequence{f}[n][\zset]$ \`a l'aide de
$N+M$ multiplications et additions.

Etant donn\'e un signal causal $\sequence{f}[n][\zset]$ (avec $f_k=0$ pour $k \leq -1$), 
le calcul de $g_n$ pour $n \geq 0$ n\'ecessite la connaissance de ``conditions initiales'',
par exemple $N$ valeurs cons\'ecutives de $g_n$.
Si l'on impose que $g_n = 0$ pour $-N \leq n < 0$, alors
$g_n$ est enti\`erement caract\'eris\'e pour tout $n \in \zset$.
L'op\'erateur $L$ est alors un filtre lin\'eaire homog\`ene causal.

Si $N = 0$ alors le fitre a une r\'eponse impulsionnelle $\sequence{h}[n][\nset]$ finie
de longueur $M$
\[
g_n =
\sum_{k=0}^M \frac {b_k} {a_0} f_{n-k} = (h \star f)_n \eqsp.
\]
Si $M = 0$, on dit que le filtre est autor\'egressif
\[
g_n = \frac {b_0} {a_0} f_n - \sum_{k=1}^N \frac{a_k} {a_p} g_{n-k} \eqsp.
\]
\subsection{Fonction de transfert}
Pour caract\'eriser la classe des op\'erateurs de convolutions $L$
qui satisfont (\ref{recurr}),
nous \'evaluons la condition impos\'ee
sur la fonction de transfert en calculant
la transform\'ee de Fourier de chaque c\^ot\'e de l'\'egalit\'e
(\ref{recurr}).
Si $\hat f({\rme^{\rmi \lambda}})$ est la transform\'ee de Fourier de $f[n]$
alors la transform\'ee de Fourier
de $f_{n-k}$ est $e^{-i k \lambda} \hat f({\rme^{\rmi \lambda}})$. La transform\'ee
de Fourier de (\ref{recurr}) est donc
\[
\sum_{k=0}^N a_k \,\rme^{-ik \lambda} \hat g({\rme^{\rmi \lambda}})
=
\sum_{k=0}^M b_k \,\rme^{-ik\lambda} \hat f({\rme^{\rmi \lambda}}) ,
\]
d'o\`u l'on d\'eduit que
\begin{equation}
\label{RecruTrasfFunc}
\hat h({\rme^{\rmi \lambda}}) = \frac {\hat g({\rme^{\rmi \lambda}})} {\hat f({\rme^{\rmi \lambda}})} =
\frac {\sum_{k=0}^M b_k \,\rme^{-ik\lambda}} {\sum_{k=0}^N a_k \,
\rme^{-ik\lambda} }.
\end{equation}
La fonction de transfert d'un filtre r\'ecursif est donc un rapport
de polyn\^{o}mes en $\rme^{-i\lambda}$.

Les propri\'et\'es du module
et de la phase s'analysent plus facilement en calulant les
p\^oles $d_k$ et les z\'eros $c_k$ de la fonction rationnelle
(\ref{RecruTrasfFunc})
\[
\hat h({\rme^{\rmi \lambda}}) = \frac {b_0} {a_0} \frac
{\prod_{k=1}^M (1 - c_k \rme^{-\rmi \lambda})} {\prod_{k=1}^N (1 - d_k \rme^{-\rmi \lambda})} .
\]
Le module de la transform\'ee de Fourier est donc
\[
|\hat h({\rme^{\rmi \lambda}})| = \frac {|b_0|} {|a_0|} \frac
{\prod_{k=1}^M |1 - c_k {\rme^{-\rmi \lambda}}|} {\prod_{k=1}^N |1 - d_k {\rme^{-\rmi \lambda}}|} .
\]
L'amplitude de la fonction de transfert
est le plus souvent calcul\'ee en d\'ecibels (db) qui mesurent
\[
20 \log_{10} |\hat h({\rme^{\rmi \lambda}})| =
10 \log_{10} \frac {|b_0|^2} {|a_0|^2} +
\sum _{k=1}^M 10 \log_{10} |1 - c_k {\rme^{-\rmi \lambda}}|^2
- \sum_{k=1}^N 10 \log_{10} |1 - d_k {\rme^{-\rmi \lambda}}|^2 .
\]
Les p\^oles et les z\'eros ne se distinguent donc que par un changement
de signe. La phase complexe de $\hat h({\rme^{\rmi \lambda}})$ se mesure de m\^eme par
\[
\arg \hat h({\rme^{\rmi \lambda}}) = \arg \frac {b_0} {a_0} +
\sum _{k=1}^M \arg (1 - c_k {\rme^{'\rmi \lambda}})
- \sum_{k=1}^N \arg (1 - d_k {\rme^{'\rmi \lambda}}) .
\]
\begin{example}
Prenons le cas d'un p\^ole ou d'un z\'ero situ\'e en $re^{i\theta}$
et \'etudions le module et la phase de $(1-r \rme^{i \theta} {\rme^{'\rmi \lambda}})$.
\[
10 \log_{10} |1-r \rme^{i \theta} {\rme^{'\rmi \lambda}}|^2 =
10 \log_{10} [ 1 + r^2 - 2r\cos(\lambda - \theta) ] .
\]
Le module est donc minimum pour $\lambda = \theta$ o\`u il vaut
$20 \log_{10} |1-r|$ et
maximum en $\lambda = \theta + \pi$ o\`u il vaut
$20 \log_{10} |1+r|$.
Suivant que ce facteur est un p\^ole ou un z\'ero, il produit
une att\'enuation ou une
amplification au voisinage de $\lambda = \theta$.
La phase complexe est
\[
\arg\hat h({\rme^{\rmi \lambda}}) = \arctan \left[
\frac {r \sin(\lambda - \theta)} {1 -r \cos(\lambda - \theta)} \right] .
\]
\end{example}
\subsection{Filtres r\'ecursifs}
Nous avons vu en (\ref{RecruTrasfFunc}) que la fonction de
transfert d'un filtre r\'ecursif
est une fonction rationnelle. Sa tranform\'ee en $z$ peut
donc s'\'ecrire
\[
\hat h(z) =
\frac {\sum_{k=0}^M b_k z^{-k}} {\sum_{k=0}^N a_k z^{-k} }.
\]
La r\'eponse impulsionnelle causale $h_n$ se calcule
facilement en d\'ecomposant $\hat h (z)$ en \'el\'ements simples.
Si $\hat h (z)$ a des p\^oles simples situ\'es
en $d_k$, on peut montrer par identification des coefficients
(exercice) qu'il peut s'\'ecrire sous la forme
\[
\hat h (z) = \sum_{r=0}^{M-N} B_r z^{-r} +
\sum_{k=0}^{N} \frac {A_k} {1 - d_k z^{-1}} .
\]
Le filtre causal correspondant \`a une r\'eponse impulsionelle
qui se calcule avec (\ref{transl-z}) et (\ref{un-pole2})
\[
h_n =
\sum_{r=0}^{M-N} B_r \delta[n-r] +
\sum_{k=0}^{N} {A_k} (d_k)^n \gamma[n] .
\]
Dans le cas de p\^oles multiples, la d\'ecomposition fractionnelle
s'\'etend avec des puissances aux d\'enominateurs des fractions.
On distingue les filtres \`a r\'eponse impulsionnelle finie
dont la transform\'ee en $z$ est un polyn\^{o}me en $z^{-1}$ (N=0)
et les filtres \`a r\'eponse impulsionnelle infinie pour lesquels
$N > 0$.

On observe que la r\'eponse impulsionelle $h_n$ est
causale et stable si et seulement si pour tout $k$, $|d_k| < 1$.
Cela signifie donc que tous les p\^oles de $\hat h (z)$ ont
un module plus petit que 1.

\subsection{Approximation de filtres s\'electifs en fr\'equence}

Tout comme pour la synth\`ese de filtres analogiques, on
approxime un filtre passe-bas id\'eal (\ref{passe-bas-discret})
par un filtre r\'ecursif dont la fonction de transfert satisfait
les conditions impos\'es par un
gabarit qui limite les oscillations dans
la bande passante et la bande d'att\'enuation
(voir figure \ref{gabarit}).
La technique de synth\`ese la plus courante
est de transformer un filtre passe-bas analogique rationnel
\[
\hat h_a (\lambda) = \frac {N(i\lambda)} {D(i \lambda)}
\]
en un filtre discret r\'ecursif par un changement de variable
\[
i\lambda = F({\rme^{\rmi \lambda}}) ,
\]
o\`u F est une fonction rationnelle de ${\rme^{\rmi \lambda}}$ qui envoie
$]-\pi,\pi[$ sur
$]-\infty,+\infty[$.
La fonction de transfert
\[
\hat h_d ({\rme^{\rmi \lambda}}) = \frac {N(F({\rme^{\rmi \lambda}}))} {D(F({\rme^{\rmi \lambda}}))}
\]
est une fonction rationnelle de ${\rme^{\rmi \lambda}}$
et donc la fonction de transfert
d'un filtre discret r\'ecursif.
Le changement de variable $F({\rme^{\rmi \lambda}})$ qui associe
$]-\pi,\pi[$ \`a l'axe r\'eel $\R$ doit \^etre aussi ``r\'egulier''
que possible pour ne pas trop modifier les propri\'et\'es
de la fonction de transfert $\hat h(\lambda)$.
On utilise souvent l'application
\[
F({\rme^{\rmi \lambda}}) = \frac 2 T \tan (\frac {\lambda} 2 ) =
\frac {2} T \frac{1 - e^{-i\lambda}}{1 + \rme^{-i\lambda}} .
\]
Le facteur $T$ est un param\`etre de dilatation qui peut \^etre
ajust\'e arbitrairement.

Par exemple,
on peut construire un filtre passe-bas dont la fr\'equence
de coupure est en $\lambda_c$ \`a partir d'un filtre analogique de
Butterworth (\ref{butterworth}).
On obtient
\[
|\hat h({\rme^{\rmi \lambda}})|^2 = \frac 1 {1 + \left(
\frac {\tan (\lambda/2)} {\tan (\lambda_c /2)}\right)^{2N}} .
\]
L'ordre $N$ du filtre doit \^etre adapt\'e aux conditions
impos\'ees
par le gabarit du filtre passe-bas.

\subsection{Factorisation spectrale}

Lors de la synth\`ese d'un filtre r\'ecursif,
une fois que l'on a calcul\'e
$|\hat h({\rme^{\rmi \lambda}})|^2$ pour satisfaire les conditions d'amplification
ou d'att\'enuation, il reste \`a
adapter la phase pour que $\hat h ({\rme^{\rmi \lambda}})$ soit un filtre
causal et stable. Le module est donn\'e par
\[
|\hat h({\rme^{\rmi \lambda}})|^2 = \hat h({\rme^{\rmi \lambda}}) \hat h^* ({\rme^{\rmi \lambda}}) = \hat h(z)
\hat h^* (1 / z^* ) ,
\]
avec $z = {\rme^{\rmi \lambda}}$.
Pour un filtre r\'ecursif,
\[
\hat h(z) = \frac {b_0} {a_0} \frac
{\prod_{k=1}^M (1 - c_k z^{-1})} {\prod_{k=1}^N (1 - d_k z^{-1})} .
\]
et
\[
C(z) = \hat h(z) \hat h^* (1 / z^* ) =
\frac {|b_0|^2} {|a_0|^2} \frac
{\prod_{k=1}^M (1 - c_k z^{-1})(1 -  c^*_k  z)
} {\prod_{k=1}^N (1 - d_k z^{-1})(1 - d^*_k z)} .
\]
La donn\'ee de $|\hat h({\rme^{\rmi \lambda}})|^2$ impose la position des z\'eros et
des p\^oles de $C(z)$. Les z\'eros et les p\^oles de $C(z)$
vont par paires $(c_k, 1/{ c^*_k})$ et
$(d_k, 1/{d^*_k})$. Pour chaque paire, il y a un \'el\'ement
dans le cercle unit\'e et l'autre \`a l'ext\'erieur, \`a moins qu'ils
ne soient confondus sur le cercle unit\'e.
On peut construire $\hat h(z)$ en choisissant
arbitrairement un p\^ole et un z\'ero dans
chaque paire.
Pour que $\hat h(z)$ soit la transform\'ee en z d'un syst\`eme
stable
et causal, nous avons vu que tous les p\^oles doivent \^etre
strictement dans le
cercle unit\'e. Cela laisse libre le choix des z\'eros.
Un choix particulier des z\'eros est de les prendre tous
dans le cercle unit\'e. Un filtre dont les z\'eros et les p\^oles
sont dans le cercle unit\'e est appel\'e filtre \`a phase minimale.
\\
\\
{\bf Filtre inverse}
Le filtre inverse d'un filtre $h$ est
le filtre $h_i$ tel que
pour tout $f[n]$
\[
f \star h \star h_i [n] = f[n] .
\]
Cela signifie que les zones de convergence
de $\hat h (z)$ et de $\hat h_i (z)$ s'intersectent et que
sur ce domaine
\[
\hat h(z) \hat h_i (z) = 1.
\]
Le filtre inverse d'un filtre \`a phase minimale
est stable et causal.
En effet,
les p\^oles de $\hat h_i (z)$ sont les z\'eros
de $\hat h(z)$ et inversement.
Or, pour que $\hat h_i (z)$ soit stable et causal, il faut et
il suffit que ses
p\^oles soient dans le cercle unit\'e et donc que les z\'eros de
$\hat h(z)$ soient dans le cercle unit\'e.

\section{Signaux finis}
\label{finite-sig}
Nous avons suppos\'e jusqu'\`a pr\'esent que nos signaux discrets
$f[n]$ sont d\'efinis pour tout $n \in \Z$. Le plus souvent,
$f[n]$ est connu sur un domaine fini, disons $0 \leq n < N$.
Il faut donc adapter les calculs de convolutions en tenant
compte des effets de bord en $n=0$ et $n= N-1$.
Par ailleurs, pour utiliser la transform\'ee de Fourier comme outil
de calcul num\'erique, il faut pouvoir la red\'efinir sur des
signaux discrets finis.
Ces deux probl\`emes sont r\'esolus en p\'eriodisant les
signaux finis. L'algorithme de transform\'ee de Fourier
rapide est d\'ecrit avec une application au calcul rapide
des convolutions.


\subsection{Convolutions circulaires}

Soient $\tilde f[n]$ et $\tilde h_n$ des signaux de $N$
\'echantillons.
Pour calculer la convolution
\[
\tilde f \star \tilde h [n] = \sum_{p=-\infty}^{+\infty}
\tilde f[p] \tilde h[n-p]
\]
pour $0 \leq n < N$, il nous faut conna\^{\i}tre
$\tilde f[n]$ et $\tilde h_n$ au-del\`a de $0 \leq n < N$.
Une approche possible est d'\'etendre
$\tilde f[n]$ et $\tilde h_n$ avec une p\'eriodisation
sur $N$ \'echantillons
\[
f[n] =  \tilde f[n ~\mbox{modulo}~ N]
~~,~~
h_n =  \tilde h[n ~\mbox{modulo}~ N] .
\]
La convolution circulaire de ces deux signaux de p\'eriode $N$
est r\'eduite \`a une somme sur leur p\'eriode
\[
f \cistar h [n]
= \sum_{p=0}^{N-1}
f[p] h[n-p] .
\]

Les vecteurs propres d'un op\'erateur de convolution circulaire
\[
Lf[n] = f \cistar h_n
\]
sont les exponentielles discr\`etes
$e_k[n] = \rme^{ \frac {i2 \pi k} N n  }$.
En effet,
\[
L e_k [n] = \rme^{ \frac {i2 \pi k} N n  }
\sum_{p=0}^{N-1}
h[p]  \rme^{ \frac {-i2 \pi k} N p  },
\]
et les valeurs propres sont donn\'ees par la
transform\'ee de Fourier discr\`ete de $h_n$
\[
\hat h_k = \sum_{p=0}^{N-1}
h[p]  \rme^{ \frac {-i2 \pi k} N p  } .
\]

\subsection{Transform\'ee de Fourier discr\`ete}
\label{transf-four-discr}

L'espace des signaux discrets
de p\'eriode $N$ est de dimension $N$ et l'on note le produit
scalaire
\begin{equation}
\label{inn-prod}
<f,g> = \sum_{n=0} ^{N-1} f[n]  g^*[n] .
\end{equation}
Le th\'eor\`eme suivant d\'emontre que tout signal
de p\'eriode $N$ peut s'\'ecrire comme une transform\'ee de
Fourier discr\`ete.

\begin{theorem}
La famille d'exponentielles discr\`etes
$(e_k[n])_\ZkNU$
\begin{equation}
e_k[n] = \rme^{ \frac {i2 \pi k} N n  },
\end{equation}
est une base orthogonale de l'espace des signaux
de p\'eriode $N$.
\end{theorem}

Pour prouver ce th\'eor\`eme, il suffit de montrer que
cette famille de $N$ vecteurs est orthogonale (exercice).
Comme l'espace est
de dimension $N$, c'est donc une base de l'espace.
Tout signal $\sequence{f}[n][\zset]$ de p\'eriode $N$ peut se d\'ecomposer
dans cette base
\begin{equation}
f_n = \sum_{k=0} ^{N-1} \frac {\pscal{f}{e_k}} {\|e_k \|^2}
e_{k,n} .
\end{equation}
La transform\'ee de Fourier discr\`ete de $f[n]$ est
\begin{equation}
\label{fourier-discret}
\hat{f}_k  = \pscal{f}{e_k}
\sum_{n=0}^{N-1} f_n W_N^{kn}    \eqsp, W_N = \rme^{\frac {-\rmi 2 \pi }{N} } .
\end{equation}
Comme $\sum_{n=0}^{N-1} |W_N^{kn}|^2 = N$,
\begin{equation}
\label{fourier-discret-inverse}
f_n = \frac 1 N \sum_{k=0}^{N-1} \hat f_k W_N^{-kn} .
\end{equation}
L'orthogonalit\'e implique une formule de Plancherel
\begin{equation}
\label{planch-discret}
\sum_{n=0} ^{N-1} |f_n|^2 = \frac{1}{N} \sum_{k=0} ^{N-1} |\hat{f}_k|^2 .
\end{equation}
\subsection{Effets de bord}
La transform\'ee de Fourier discr\`ete
d'un signal de p\'eriode $N$ se calcule \`a partir des
valeurs de $f[n]$ pour $0 \leq n < N$. Pourquoi se soucier
du fait que ce soit un signal de p\'eriode $N$ plut\^ot qu'un signal
de $N$ \'echantillons ?
La somme de Fourier (\ref{fourier-discret})
d\'efinit un signal de p\'eriode $N$ pour lequel l'\'echantillon
$f[0]$ \'etant le m\^eme que $f[N]$ se retrouve
plac\'e \`a c\^ot\'e de $f[N-1]$.
Si $f[0]$ et $f[N-1]$ sont tr\`es diff\'erents, cela induit une
transition brutale dans le signal p\'eriodis\'e qui se traduit
par l'apparition de coefficients de Fourier de relativement
grande amplitude aux hautes fr\'equences.
Par exemple, le signal apparemment r\'egulier
$f[n] = n$ pour $0 \leq n < N$
a une transition brutale en $n=pN$ pour $p \in \Z$,
une fois p\'eriodis\'e. Cette transition appara\^{\i}t
dans sa s\'erie de Fourier.
\\
\\
{\bf Filtrage fini}
Comme $\rme^{ \frac {i2 \pi k} N n  }$ sont les
vecteurs propres des op\'erateurs de convolution
circulaire, on d\'eduit un th\'eor\`eme de convolution.

\begin{theorem} [Convolution Circulaire]
\label{circulaire}
La convolution circulaire
$g_n = f \cistar h [n]$ est un signal de p\'eriode $N$ dont la
transform\'ee de Fourier discr\`ete est
\begin{equation}
\hat g[k] = \hat f[k] \hat h_k
\end{equation}
\end{theorem}

La d\'emonstration de ce th\'eor\`eme est identique \`a la
d\'emonstration des deux th\'eor\`emes de convolution pr\'ec\'edents
et laiss\'ee en exercice.
Ce th\'eor\`eme montre que la convolution circulaire est
un filtrage fr\'equentiel. Il ouvre aussi la porte au
calcul rapide de convolutions en utilisant la
transform\'ee de Fourier rapide.

\subsection{Transform\'ee de Fourier rapide}

Pour un signal $f[n]$ de $N$ points, le calcul direct
de la transform\'ee de Fourier discr\`ete
\begin{equation}
{\hat f[k]}  = \sum_{n=0} ^{N-1} f[n] e^*{ \frac {-i2 \pi k} N n  } ,
\end{equation}
pour $0 \leq k < N$,
demande $O(N^2)$ multiplications et additions.
Il est cependant possible de
r\'eduire le nombre d'op\'erations \`a $O(N \log_2 N)$
en r\'eorganisant les calculs.
Lorsque $k$ est pair, on regroupe les termes
$n$ et $n+\frac N 2$
\begin{equation}
\label{four-paire}
{\hat f [2k]}  = \sum_{n=0} ^{{\frac N 2}-1} (f[n]+f[n+ N/ 2 ])
\rme^{ \frac {-i2 \pi k} {{\frac N 2}}n} .
\end{equation}
Lorsque $k$ est impair, le m\^eme regroupement devient
\begin{equation}
\label{four-impaire}
{\hat f [2k+1]}  =
\sum_{n=0} ^{{\frac N 2}-1} \rme^{ \frac {-i2 \pi} N n}
(f[n]-f[n+{\frac N 2}])
\rme^{ \frac {-i2 \pi k} {{\frac N 2}} n} .
\end{equation}
L'\'equation (\ref{four-paire})
montre que les fr\'equences paires sont obtenues
en calculant la transform\'ee de Fourier discr\`ete du signal
de p\'eriode ${\frac N 2}$
\[
f_p [n] = f[n] + f[n+{\frac N 2}],
\]
tandis que (\ref{four-impaire})
permet de calculer les fr\'equences impaires par la
transform\'ee de Fourier discr\`ete du signal de p\'eriode ${\frac N 2}$
\[
f_i [n] = \rme^{ \frac {-i2 \pi} N n}  (f[n] - f[n+{\frac N 2}] ).
\]
\\
\\
{\bf Complexit\'e}
Une transform\'ee de Fourier d'un signal de taille $N$ s'obtient
donc en calculant deux transform\'ees de Fourier de signaux de
taille ${\frac N 2}$.
Le signal $f[n]$ \'etant complexe, le calcul
des signaux $f_p [n]$ et $f_i[n]$
demande
$N$ additions complexes et ${\frac N 2}$ multiplications complexes,
ce qui fait $3N$ additions et $2N$ multiplications r\'eelles.
Soit $C(N)$ le nombre d'op\'erations d'une transform\'ee de Fourier
rapide d'un signal de p\'eriode $N$. On a donc
\begin{equation}
\label{recurrence}
C(N) = 2 C({\frac N 2}) + K N ,
\end{equation}
avec $K = 5$.
La transform\'ee de Fourier d'un signal d'un seul point \'etant
\'egale \`a lui-m\^eme, $C(1) = 0$.
Avec le changement de variable $l = \log_2 N$ et de fonction
$T(l) = C(N)/N$, on d\'eduit de (\ref{recurrence}) que
\[
T(l) = T(l-1) + K .
\]
Comme $T(0) = 0$, $T(l) = Kl$ et donc
\[
C(N) = K N \log_2 (N) .
\]

Il existe diff\'erents algorithmes de transform\'ee de Fourier
rapide qui d\'ecomposent une transform\'ee de Fourier
de taille $N$ en deux transform\'ees de Fourier de taille ${\frac N 2}$
plus $O(N)$ op\'erations. L'algorithme le plus performant \`a ce
jour est un peu plus compliqu\'e que celui que nous venons de
d\'ecrire, mais ne demande que $N \log_2 N$ multiplications
et $3N \log_2 N$ additions.

La transform\'ee de Fourier inverse se calcule \`a partir
de l'algorithme rapide en observant que
\begin{equation}
f^*[n] = \frac 1 N \sum_ {k=0}^ {N-1}  \hat f^*[k]
\rme^{\frac {-i 2 \pi kn} N} .
\end{equation}
La tranform\'ee de Fourier discr\`ete inverse est donc obtenue en
calculant la transform\'ee de Fourier directe du complexe conjugu\'e
du signal, et en calculant le complexe conjugu\'e du r\'esultat.


\subsection{Convolution rapide}
\label{convol-rap-sec}

L'algorithme rapide de la transform\'ee de Fourier
discr\`ete permet d'utiliser le th\'eor\`eme \ref{circulaire} pour
calculer efficacement les convolutions discr\`etes de
deux signaux \`a support fini.
Soient $f[n]$ et $h_n$ deux signaux ayant
des \'echantillons non nuls pour $0 \leq n < M$.
Le signal causal
\begin{equation}
\label{somme}
g_n = f \star h [n] = \sum_{k=- \infty}^{+ \infty} f[k] h _{n-k} ,
\end{equation}
n'a de valeurs non-nulles que pour $0 \leq n < 2M$.
Le calcul direct de ce produit
de convolution, en \'evaluant la somme (\ref{somme}), demande
$O(M^2)$ additions et multiplications.
Le th\'eor\`eme de convolution circulaire \ref{circulaire}
sugg\`ere un proc\'ed\'e plus rapide
bas\'e sur des convolutions circulaires.

Pour r\'eduire le calcul de
la convolution non-circulaire (\ref{somme})
\`a une convolution circulaire, on d\'efinit deux signaux
de p\'eriode $2M$
\begin{equation}
a [ n  ] =
   \left \{ \begin{array}{ll}
            f[n]   & \mbox{si $0 \leq n < M$}\\
            0  & \mbox{si $ M \leq n < 2M$}
            \end{array}
   \right.
\end{equation}
\begin{equation}
b [ n  ] =
   \left \{ \begin{array}{ll}
            h_n &  \mbox{si $ 0 \leq n < M$}\\
            0  & \mbox{si $M \leq n < 2M$}
            \end{array}
   \right.
\end{equation}
Il est maintenant facile de v\'erifier que la convolution circulaire
\[
c[n] = a \cistar b [n]
\]
satisfait
\begin{equation}
c [ n  ] =  g_n   \mbox{~~~pour } 0 \leq n < 2M .
\end{equation}
On peut donc calculer les valeurs non-nulles de
$g_n$ en calculant les transform\'ees de Fourier discr\`etes
de $a[n]$ et $b[n]$, en les multipliant et
en calculant la transform\'ee de Fourier inverse du r\'esultat.
En utilisant l'algorithme de transform\'ee de Fourier rapide,
ce calcul demande un total de $O(M \log_2 M)$ additions
et multiplications, au lieu de $O(M^2)$.

Cet algorithme rapide de calcul de convolutions est utilis\'e
pour la convolution de signaux par des filtres de r\'eponse
impulsionnelle finie. Si le signal $f[n]$ a $N$ points
non-nuls et le filtre $h_n$ a $L$ \'echantillons non-nuls,
l'algorithme de convolution peut \^etre
modifi\'e pour calculer la convolution $h \cistar f[n]$
en $O(N log_2 L )$ op\'erations (exercice).
%\end{document}








