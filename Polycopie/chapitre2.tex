\chapter{Traitement du signal analogique}
\label{analogique-chap}


Le traitement du signal analogique repose essentiellement sur 
l'utilisation d'op\'erateurs lin\'eaires qui modifient les 
propri\'et\'es d'un signal de fa\c{c}on homog\`ene dans le
temps. La transform\'ees de Fourier 
diagonalise ces op\'erateurs et appara\^{\i}t donc comme
le principal outil d'analyse math\'ematique.
Nous \'etudions la synth\`ese de filtres homog\`enes et une
application \`a la transmission par modulation d'amplitude.

\section{Filtrage lin\'eaire homog\`ene}

Un signal analogique mono-dimensionnel
est une fonction $f(t)$
d'une variable continue $t \in \R$, 
que nous supposerons \^etre le temps. 
De nombreux algorithmes de traitement du signal tels
que la transmission par modulation d'amplitude, le d\'ebruitage
de signaux stationnaires ou le codage par pr\'ediction,
s'impl\'ementent avec des op\'erateurs lin\'eaires homog\`enes dans le
temps. 

L'homog\'en\'eit\'e temporelle d'un op\'erateur $L$ signifie que
si l'entr\'ee $f_\tau (t) = f(t-\tau)$ est retard\'ee
par $\tau \in \R$ alors la sortie $L[f_\tau (t)]$
est aussi retard\'ee par $\tau$
\begin{equation}
\label{invariant}
L[f(t)] = g(t) 
\Rightarrow L [ f_\tau (t) ]= g(t-\tau) . 
\end{equation}
Pour garantir une stabilit\'e num\'erique, nous supposons aussi
que $L$ a une continuit\'e faible.
Pour tout $t$, la sortie
$L[f] (t)$ est peu perturb\'ee si
$f(t)$ est un signal r\'egulier qui est l\'eg\`erement modifi\'e.
Cette continuit\'e peut \^etre formalis\'ee dans le cadre de la th\'eorie
des distributions \cite{bony2}.

\subsection{Dirac}

Un Dirac est une masse ponctuelle qui est souvent utilis\'ee pour
simplifier les calculs. C'est une ``distribution'' $\delta (t)$
dont le support est r\'eduit au point $t = 0$ et d'int\'egrale unit\'e,
si bien que pour toute fonction continue $f(t)$
\[
\int_{-\infty}^{+\infty} f (u) \delta (u) du = f(0).
\]
La th\'eorie des distributions \cite{bony2} d\'efinie formellement
cette int\'egrale
comme une forme lin\'eaire qui associe a toute 
fonction sa valeur en $t = 0$.
Nous nous contentons ici
d'une d\'efinition plus intuitive d'un Dirac
comme limite de ``bosses'' qui sont contract\'ees ind\'efiniment.

Soit $\phi (t)$ une fonction continue \`a support dans $[-1,1]$ et
de masse unit\'e
\begin{equation}
\label{Dirac0}
\int_{-\infty}^{+\infty} \phi (u) du = 1 .
\end{equation}
La fonction $\phi_s (t) = \frac 1 s \phi ( \frac t s )$ a un
support dans $[-s,s]$. Avec le changement de variable
$t' = \frac t s$ on montre que
\[
\int_{-\infty}^{+\infty} \phi_s (u) du =
\int_{-\infty}^{+\infty} \frac 1 s \phi ( \frac u s ) du = 1 .
\]

Soit $f(t)$ une fonction continue, on v\'erifie facilement que
\[
\lim_{s \rightarrow 0} 
\int_{-\infty}^{+\infty} f(u) \phi_s (u) du = f(0) .
\]
Un Dirac se d\'efinit par
\[
\delta (t) = \lim_{s \rightarrow 0} \phi_s (t),
\]
o\`u la limite doit \^etre formellement prise au sens o\`u
pour toute fonction continue $f(t)$
\begin{equation}
\lim_{s \rightarrow 0} 
\int_{-\infty}^{+\infty} f(u) \phi_s (u) du = 
\int_{-\infty}^{+\infty} f (u) \delta (u) du = f(0).
\end{equation}

Un Dirac $\delta (t)$ n'est pas une fonction mais la th\'eorie
des distributions montre que cela se manipule
comme une fonction dans les calculs. On oubliera donc son 
statut de distribution par la suite.
Ainsi, on peut d\'efinir un Dirac translat\'e en $\tau$ par
\[
\delta (t-\tau) = \lim_{s \rightarrow 0} \phi_s (t-\tau),
\]
et on montre que pour toute fonction continue $f(t)$
\begin{equation}
\label{dirac-trans}
\int_{-\infty}^{+\infty} f (u) \delta (u-\tau) du = f(\tau).
\end{equation}
On peut cependant \'eviter d'utiliser une limite
et d\'eduire simplement ce resultat de (\ref{Dirac0}) par un
changement de variable $u' = u + \tau$.\\

Un Dirac est sym\'etrique $\delta (t) = \delta (-t)$ car
\[
\int_{-\infty}^{+\infty} f (u) \delta (-u) du = 
\int_{-\infty}^{+\infty} f (-u) \delta (u) du = f(0).
\]
On peut donc r\'e\'ecrire (\ref{dirac-trans}) 
comme une d\'ecomposition de $f(t)$ en une somme de Diracs
translat\'es en diff\'erents points
\[
f(t) = \int_{-\infty}^{+\infty} f(u) \delta (t-u) du .
\]

\subsection{R\'eponse impulsionnelle}

En d\'ecomposant un signal comme une somme de Diracs translat\'es, nous
montrons que tout op\'erateur homog\`ene peut s'\'ecrire comme un
produit de convolution.
On a vu que
\[
f(t) = \int_{-\infty}^{+\infty} f(u) \delta (t-u) du .
\]
La continuit\'e et la lin\'earit\'e de $L$ montrent que
\[
L[f (t)] = \int_{-\infty}^{+\infty} f(u) L[\delta(t-u)] du .
\]
Soit $h(t)$ la r\'eponse de $L$ pour une impulsion $\delta (t)$
\[
h(t) = L[\delta (t)] .
\]
L'homog\'en\'eit\'e temporelle implique 
$L[\delta (t-u)] = h(t-u)$ et donc
\[
L[f (t)] = \int_{-\infty}^{+\infty} f(u) h (t-u)  du 
= \int_{-\infty}^{+\infty} h (u) f(t-u) du .
\]
On note le produit de convolution de $h$ avec $f$
\[
L[f (t)] = 
h \star f (t) = \int_{-\infty}^{+\infty} h (u) f(t-u) du .
\]
Un op\'erateur lin\'eaire homog\`ene se calcule donc par un produit
de convolution avec la r\'eponse impulsionnelle.

On rappelle quelques propri\'et\'es importantes du produit de convolution :
\begin{itemize}
\item Commutativit\'e
\begin{equation}
f \star h (t) = h \star f (t) .
\end{equation}
\item La convolution de $f(t)$ avec un Dirac translat\'e
$\delta_\tau (t) = \delta (t-\tau)$ translate
$f(t)$ par $\tau$
\begin{equation}
f \star \delta_\tau (t) = 
\int_{-\infty}^{+\infty} f (t-u) \delta_\tau (u) du  = f(t-\tau).
\end{equation}
\end{itemize}

{\bf Stabilit\'e et causalit\'e}
Un filtre est {\it causal} si et seulement si
$L[f(t)]$ ne d\'epend que des valeurs $f(u)$ pour $u < t$.
Comme
\[
L[f(t)]  = \int_{-\infty}^{+\infty} h (u) f(t-u) du ,
\]
cela signifie que $h(u) = 0$ pour $u < 0$. On dit alors que
$h(t)$ est une fonction {\it causale}. On exprime souvent les
fonctions causale comme un produit avec une 
fonction \'echelon
\[
h(t) = h(t) \gamma (t)
\]
avec
\begin{equation}
\label{Heavyside}
\gamma (t) = \left\{
\begin{array}{ll}
0 & \mbox{si $t < 0$}\\
1 & \mbox{si $t \geq 0$}
\end{array}
\right.
\end{equation}

Lorsque $f(t)$ est born\'ee on
veut garantir que $L[f(t)]$ est aussi born\'ee.
On dit alors que le filtre $L$ et $h(t)$ sont {\it stables}.
Comme
\begin{equation}
\label{stab-cond}
|L[f(t)]| \leq \sup_{u \in \R} |f(u)| 
\int_{-\infty}^{+\infty} |h (u)| du ,
\end{equation}
il suffit que $h \in \LU$
\[
\int_{-\infty}^{+\infty} |h (u)| du < + \infty .
\]
On v\'erifie (exercice) que si $h(t)$ est une fonction definie
pour presque tout $t \in \R$
la condition $h \in \LU$ est aussi n\'ecessaire.\\
\\
\noindent{\bf Exemples}\\

$\bullet$ Un syst\`eme 
{\it d'amplification} par $\lambda$ et de {\it d\'elai}
par $\tau$ est d\'efini par
\[
L[f(t)] = \lambda f(t - \tau) .
\]
La r\'eponse impulsionnelle de ce filtre est 
\[
h(t) = \lambda \delta(t - \tau) .
\]

$\bullet$ La {\it moyenne uniforme} de $f(t)$ dans un voisinage de 
taille $T$ est 
\[
L[f(t)] = \frac 1 T \int_{t-\frac T 2}^{t+\frac T 2} f( u ) d u .
\]
Cette int\'egrale peut \^etre r\'e\'ecrite comme un produit de 
convolution avec 
\[
h(t) = \left\{ 
\begin{array}{ll}
 \frac 1 T & \mbox{si $t \in [-\frac T 2 , \frac T 2 ]$} \\
 0 & \mbox{si $|t| > \frac T 2$}
\end{array}
\right.
\]

$\bullet$ 
Une {\it moyenne pond\'er\'ee} correspond \`a une
r\'eponse impulsionnelle $h(t)$ telle que
\[
\int_{- \infty} ^{+ \infty} h(u) ~d u = 1 .
\]
L'int\'egrale
\[
L[f(t)] = \int_{-\infty}^{+\infty} h (u) f(t-u) du 
\]
peut \^etre interpr\'et\'ee comme une moyenne pond\'er\'ee par $h(u)$
de $f(u)$ au voisinage de $t$. Si
$f(t) = c$ alors on v\'erifie que $L[f(t)] = c$. On verra
comment optimiser
le choix de $h(u)$ pour enlever au mieux les fluctuations 
irr\'eguli\`eres de $f(t)$ dues \`a un bruit de mesure.

\subsection{Fonction de transfert}

Les exponentielles complexes $e^{i \om t}$
sont les vecteurs propres des op\'erateurs de convolution.
En effet
\[
L[e^{i \om t}] = 
\int_{- \infty} ^ {+ \infty}  h ( u ) e^{i (t-u) \om}~d u ,
\]
ce qui nous donne
\[
L[e^{i \om t}] = 
e ^ {i t \om} \int_{- \infty} ^ {+ \infty}  h ( u ) 
e^{-i \om u}d u = 
e ^ {i t \om} \hat h ( \om ),
\]
avec pour valeur propre
\[
\hat h ( \om ) = \int_{- \infty} ^ {+ \infty}  h ( u ) 
e^{-i \om u }~d u .
\]
La fonction $\hat h(\om)$ est la transform\'ee de Fourier
de $h(t)$ et est appel\'ee fonction de transfert
du filtre.
Les exponentielles \'etant les vecteurs propres d'un syst\`eme
lin\'eaire homog\`ene, il est tentant d'essayer d'exprimer le
signal $f(t)$ comme somme d'exponentielles complexes,
de fa\c{c}on \`a facilement calculer la r\'eponse du filtre.
L'analyse de Fourier prouve qu'une telle d\'ecomposition est
possible, en imposant des conditions tr\`es faibles sur $f(t)$.

\section{Analyse de Fourier}


\subsection{Transform\'ee de Fourier dans $\LU$}

Pour s'assurer que l'int\'egrale de Fourier
\begin{equation}
\label{fourier}
\hat f ( \om ) = \int_{- \infty}^{+ \infty} 
f(t ) e^{-i \om t} dt
\end{equation}
existe, on suppose que $f(t)$ est int\'egrable
$f(t) \in \LU$. Cela nous permet d'\'etudier ses propri\'et\'es
principales avant de l'\'etendre \`a d'autres classes de fonctions.

Lorsque $f(t) \in \LU$, 
\begin{equation}
|\hat f(\om)| \leq \int_{- \infty}^{+ \infty} |f(t )| dt .
\end{equation}
La transform\'ee de Fourier est alors born\'ee et l'on peut
v\'erifier que $\hat f (\om)$ est continue (exercice).
Si $\hat f (\om) \in \LU$, on peut
prouver \cite{bony} que l'op\'erateur de Fourier s'inverse et que
\begin{equation}
\label{inverse}
f ( t ) = \frac 1 {2 \pi} \int_{- \infty}^{+ \infty} 
\hat f(\om ) e^{i \om t} d \om .
\end{equation}
La transform\'ee de Fourier $\hat f(\om)$ peut donc \^etre 
interpr\'et\'ee
comme l'amplitude de la composante sinuso\"{\i}dale $e^{i \om t}$
de fr\'equence $\om$ dans $f(t)$.
Au lieu de d\'ecrire $f(t)$ par ses valeurs \`a chaque instant,
la transform\'ee de Fourier donne une description de $f(t)$
en somme de sinuso\"{\i}des
totalement d\'elocalis\'ees dans le temps.\\

{\bf Regularit\'e}
Les composantes irr\'eguli\`eres de $f(t)$ sont reconstruites
par les sinuso\"{\i}des $e^{i \om t}$ qui oscillent rapidement 
et donc de hautes fr\'equences
$\om$. Si la transform\'ee de Fourier $\hat f(\om)$ d\'ecro\^{\i}t
rapidement, cela signifie donc que $f(t)$ est une fonction
r\'eguli\`ere. Cette propri\'et\'e se formalise en montrant 
que si
\[
\int_{-\infty}^{+\infty}|\hat f(\om)| (|\om|^p + 1 ) < +\infty ,
\]
alors $f(t)$ est $p$ fois contin\^ument d\'erivable. Pour d\'emontrer
ce r\'esultat, on prouve d'abord que la transform\'ee de Fourier
de $\frac {d^p f(t)} {dt^p}\in \LU$ est 
$(i \om)^p \hat f(\om)$ (exercice). On utilise ensuite le fait
que si $\hat g (\om) \in \LU$ alors $g(t)$ est born\'ee et
continue, ce qui se montre en utilisant (\ref{inverse}).
L'\'equivalence entre r\'egularit\'e temporelle et d\'ecroissance 
du module de la transform\'ee de Fourier est particuli\`erement
importante pour analyser les propri\'et\'es d'un signal $f(t)$.

Pour les applications au traitement du signal, 
le r\'esultat le plus important est 
le th\'eor\`eme de convolution.

\begin{theorem} [Convolution]
\label{th_convol}
Soit $f(t) \in \LU$ et $h(t) \in \LU$. La
fonction $g(t) = h \star f (t) \in \LU$ et sa transform\'ee de
Fourier est 
\begin{equation}
\label{convolution}
\hat g (\om) = \hat h(\om) \hat f(\om) .
\end{equation}
\end{theorem}

\noindent{\bf D\'emonstration} \\
\[
\hat g (\om) = \int_{-\infty}^{+\infty} e ^{-it\om}
\left \{ \int_{-\infty}^{+\infty} f(t-u) h(u) du \right\} dt .
\]
Comme $|f(t-u)| |h(u)|$ est sommable dans $\R^2$, on peut
appliquer le th\'eor\`eme de Fubini et le changement de
variable $(t,u) \rightarrow (v=t-u,u)$ nous donne
\begin{eqnarray*}
\hat g (\om) & = & \int_{-\infty}^{+\infty} 
\int_{-\infty}^{+\infty} e ^{-i(u+v)\om} f(v) h(u) du dv \\
& = &
\left\{\int_{-\infty}^{+\infty} e ^{-iv\om} f(v) dv \right\}
\left\{\int_{-\infty}^{+\infty} e ^{-iu\om} h(u) du \right\} ,
\end{eqnarray*}
ce qui v\'erifie (\ref{convolution}). $\Box$
\\
\\
{\bf Filtrage}
Le th\'eor\`eme de convolution prouve que
la transform\'ee de Fourier d'un filtrage
$L[f] (t) = f \star h (t)$ est $\hat f(\om) \hat h(\om)$.
La formule 
de reconstruction 
\begin{equation}
f ( t ) = \frac 1 {2 \pi} \int_{- \infty}^{+ \infty} 
\hat f(\om ) e^{i \om t} d \om .
\end{equation}
appliqu\'ee \`a $L[f](t)$ implique donc
\begin{equation}
L[f] ( t ) = \frac 1 {2 \pi} \int_{- \infty}^{+ \infty} 
\hat h(\om) \hat f(\om ) e^{i \om t} d \om .
\end{equation}
Chaque composante fr\'equentielle $e^{i \om t}$ de $f(t)$ 
d'amplitude $\hat f(\om)$ est
donc amplifi\'ee ou att\'enu\'ee par $\hat h(\om)$. Ceci n'est pas
surprenant puisque nous avons d\'ej\`a prouv\'e que 
les exponentielles $e^{i \om t}$ sont des
vecteurs propres d'une convolution.
Si $\hat h(\om) = 0$ pour $\om \in [\om_1,\om_2]$, les 
composantes fr\'equentielles de $f(t)$
pour $\om \in [\om_1,\om_2]$ sont annul\'ees par l'op\'erateur
$L$, d'o\`u l'appellation ``filtre''.

\newpage

\noindent {\bf Propri\'et\'es g\'en\'erales}
Le tableau qui suit r\'esume certaines propri\'et\'es importantes
de la tranform\'ee de Fourier. Les d\'emonstrations se font
le plus souvent par un simple changement de variable dans
l'int\'egrale de Fourier.

\begin{eqnarray}
\mbox{\bf Propri\'et\'e} & \mbox{\bf Fonction} & 
\mbox{\bf Transform\'ee de Fourier} \nonumber \\ 
&  f(t) & \hat f(\om) \nonumber \\
\label{symm}
\mbox{Inverse} & \hat f(t) & 2 \pi f(-\om) \\
\label{convol}
\mbox{Convolution} & f_1 \star f_2 & \hat f_1 (\om) \hat f_2 (\om) \\
\label{mult}
\mbox{Multiplication} & f_1(t) f_2(t) & \frac 1 {2\pi}\hat f_1 \star \hat f_2 (\om) \\
\label{trans}
\mbox{Translation} & f(t-t_0) & e ^{-i t_0 \om} \hat f(\om) \\
\label{modul}
\mbox{Modulation} & e ^{i \om_0 t} f(t) & \hat f(\om - \om_0) \\
\label{dilate}
\mbox{Dilatation} & f(a t) & \frac 1 {|a|} \hat f (\frac \om a ) \\
\label{diffentiation}
\mbox{Diff\'erentiation} & \frac {d^p f(t)} {dt^p} & (i \om)^p \hat f(\om) \\
\label{polynome}
\mbox{Multiplication Polyn\^omiale } & (- i t)^p f(t)  & \frac {d^p \hat
f(\om)} {d \om^p} \\
\label{conjugue}
{\rm Complexe~ Conjugu\acute e} & f^*(t) & \hat f^* (-\om) \\
\label{symmHerm}
{\rm Sym\acute etrie~ Hermitienne} & f(t) = \Reel f(t) & 
\hat f (-\om) = \hat f ^* (\om) \\
\label{reelle}
{\rm Composante ~R \acute{e}elle} & \Reel f(t) & \frac {\hat f(\om) + 
\hat f^*(-\om)} 2 \\
\label{imaginaire}
\mbox{Composante Imaginaire} & \Ima f(t) & \frac {\hat f(\om) - \hat f^* (-\om)} {2i} \\
\label{paire}
\mbox{Composante Paire} & \frac {f(t)+f^*(-t)} 2 & \Reel \hat f(\om) \\
\label{impaire}
\mbox{Composante Impaire} & \frac {f(t)-f^*(-t)} 2 & \Ima \hat f(\om) 
\end{eqnarray}


\subsection{Transform\'ee de Fourier dans $\LD$ et Diracs}

Plut\^ot que de travailler dans $\LU$, il est souvent plus facile
de consid\'erer les signaux comme des \'el\'ements de 
l'espace de Hilbert $\LD$ car on a alors acc\`es \`a toutes les
facilit\'es donn\'ees par l'existence d'un produit scalaire.
Le produit scalaire de $f(t) \in \LD$ et $g(t) \in \LD$
est d\'efini par
\[
<f,g> = \int_{- \infty}^ {+ \infty} f(t)  g^* (t) dt  ,
\]
et la norme
\[
\|f\|^2 = <f,f> = \int_{- \infty}^ {+ \infty} |f(t)|^2 dt  .
\]
Pour facilement travailler dans cette structure Hilbertienne,
il nous faut y d\'efinir la transform\'ee de Fourier.
Cela pose un probl\`eme car l'int\'egrale
de Fourier (\ref{fourier}) d'une fonction de carr\'e int\'egrable n'est pas
toujours convergente. 
\\
\\
{\bf Conservation d'\'energie}
La transform\'ee de Fourier s'\'etend
\`a partir de $\LU$ par un argument de densit\'e qui utilise
le fait qu'\`a une constante pr\`es,
la norme et les angles dans $\LD$ ne
sont pas modifi\'es par transform\'ee de Fourier.

\begin{theorem}
Soient $f(t)$ et $h(t)$ dans $\LU \cap \LD$,
\begin{equation}
\label{parseval}
<f,h> = \int_{- \infty}^ {+ \infty} f(t) h^* (t) dt = 
\frac 1 {2 \pi} 
\int_{- \infty}^{+ \infty} \hat f( \om ) \hat h ^*(\om) d\om = 
\frac 1 {2 \pi} < \hat f , \hat h >.
\end{equation}
Pour $h=f$ on obtient la formule de Plancherel
\begin{equation}
\label{plancherel}
|| f ||^2 = 
\int_{- \infty}^ {+ \infty} |f(t)|^2 dt = 
\frac 1 {2 \pi} 
\int_{- \infty}^{+ \infty} |\hat f( \om )|^2 d\om = 
\frac 1 {2 \pi} || \hat f ||^2 .
\end{equation}
\end{theorem}

\noindent{\bf D\'emonstration de (\ref{parseval})}\\
Prenons $g(t) = f \star \tilde h (t)$ avec 
$\tilde h (t) = h^*(-t)$. 
Le th\'eor\`eme de convolution en (\ref{convolution}) et 
(\ref{conjugue}) montre que
$\hat g (\om) = \hat f(\om) \hat h^*(\om)$. 
La formule de reconstruction (\ref{inverse}) appliqu\'ee
\`a $g(0)$ nous donne
\begin{equation}
\int_{- \infty}^ {+ \infty} f(t) h^* (t) dt = 
g(0) = 
\frac 1 {2 \pi} \int_{-\infty}^{+ \infty} \hat g (\om) d \om =
\frac 1 {2 \pi} 
\int_{- \infty}^{+ \infty} \hat f( \om ) \hat h ^*(\om) d\om . 
\end{equation}
$\Box$
\\
\\
\noindent{\bf Extension dans $\LD$}
Pour d\'efinir la transform\'ee de Fourier de
$f(t) \in \LD$, on construit
une famille $\{f_n\}_\nZ$ de fonctions dans 
$\LU \cap \LD$ qui convergent vers $f$
\[
\lim_{n \rightarrow + \infty} \|f - f_n \| = 0.
\]
Ceci est possible car
$\LU \cap \LD$ est dense dans $\LD$.
La famille $\{f_n (t)\}_\nZ$ est une suite de Cauchy, 
et donc
$\| f_n - f_p \|$ est arbitrairement petit
pour $n$ et $p$ suffisamment grands.
Comme $f_n (t) \in \LD$, la formule de Plancherel montre que
$\hat f_n (\om) \in \LD$.
De plus, $\{\hat f_n (\om)\}_\nZ$ est aussi une suite de Cauchy car
\[
\| \hat f_n - \hat f_p \| = \sqrt{2 \pi}  \| f_n - f_p \| 
\]
est arbitrairement petit pour $n$ et $p$ suffisamment grands.
Comme toute suite de Cauchy converge dans
$\LD$, il existe $\hat f(\om) \in \LD$ tel que
\[
\lim_{n \rightarrow + \infty} \|\hat f - \hat f_n \| = 0.
\]
Par d\'efinition, $\hat f(\om)$ est la tranform\'ee de
Fourier de $f(t)$.
On peut alors v\'erifier que 
le th\'eor\`eme de convolution, la formule
de Parseval et les propri\'et\'es
(\ref{symm}-\ref{impaire}) restent valables dans $\LD$.
\\
\\
\noindent{\bf Dirac}
La transform\'ee de Fourier d'un Dirac peut simplement se
calculer en se souvenant que pour toute fonction continue
$f(t)$
\[
\int^{+\infty}_{-\infty} f (t) \delta (t) dt = f(0).
\]
La transformee de Fourier de $\delta (t)$ est donc
\[
\int^{+\infty}_{-\infty} e^{-i\om t} \delta (t) dt  = 1 .
\]
C'est la fonction constante \'egale \`a $1$.
La th\'eorie des distributions \cite{bony2} montre que l'on
peut definir la transform\'ee de Fourier pour toute distribution
temp\'er\'ee.

\subsection{Exemples de transform\'ee de Fourier}

$\bullet$ Une 
{\it Gaussienne} $f(t) = e^{- t^2}$ \'etant une fonction
de la classe de Schwartz, sa transform\'ee de Fourier est
aussi une fonction $\bC^\infty$ \`a d\'ecroissance rapide.
Pour calculer sa transform\'ee de Fourier,
on montre par une int\'egration par partie (exercice)
que sa transform\'ee de Fourier
\[
\hat f(\om) = \int_{-\infty}^{+\infty} e^{-t^2} e^{-i \om t} dt 
\]
satisfait l'\'equation diff\'erentielle
\[
2 \hat f '(\om) + \om \hat f (\om) = 0.
\]
La solution est une gaussienne
\[
\hat f(\om) = K e^{-\om^2/4} ,
\]
et comme
\[
\hat f(0) = \int_{-\infty}^{+\infty} e^{-t^2} dt = \sqrt \pi ,
\]
\begin{equation}
\label{gaussian-fourier}
\hat f(\om) = \sqrt \pi e^{-\om^2/4} .
\end{equation}

$\bullet$ La 
{\it fonction indicatrice}
$f(t) = 1_{[-T,T]} (t)$ est discontinue et donc a une
transform\'ee de Fourier qui n'est pas dans $\LU$ mais qui
est dans $\LD$
\begin{equation}
\label{indicator}
\hat f (\om) =  \int_{-T}^{T} e^{-i \om t} dt = \frac {2 \sin (T \om)} { \om} .
\end{equation}

$\bullet$ 
Le 
{\it sinus cardinal} $f(t) = \frac{ \sin \pi t} {\pi t}$ est dans $\LD$ mais pas
dans $\LU$. Sa transform\'ee de Fourier se 
d\'eduit de (\ref{indicator}) gr\^ace \`a la propri\'et\'e de sym\'etrie 
(\ref{symm})
\begin{equation}
\label{fourier-sinc}
\hat f(\om) = 1_{[-\pi,\pi]}(\om) .
\end{equation}

$\bullet$ Un {\it Dirac} translat\'e
$\delta_\tau (t) = \delta (t-\tau)$ a une
transform\'ee de Fourier qui se calcul directement
\begin{equation}
\label{dirac_transl}
\hat \delta_\tau (\om) = 
\int_{-\infty}^{+\infty} \delta_\tau (t) e^{-i\om t} dt = 
e^{-i\om \tau} .
\end{equation}

$\bullet$ La {\it valeur principale} $f(t) = \vp \frac 1 {\pi t}$
d\'efinit par convolution la transform\'ee de Hilbert 
\begin{equation} 
{\cal H}[ g] (t) = g \star \vp \frac 1 {\pi t } = 
\frac 1 \pi \int_{- \infty}^{+ \infty} 
{g( u )} \vp \frac 1 {t - u} d u .
\end{equation} 
On calcule la transform\'ee de 
Fourier de $\vp \frac 1 {\pi t}$
en observant que
\[
t f(t) = \frac 1 \pi ,
\]
ce qui se traduit en Fourier gr\^ace \`a
(\ref{polynome}) par
\[
\frac {d \hat f (\om)} {d \om} = -2i \delta (\om) .
\]
Donc 
\begin{equation}
\label{hilbert_fourier}
\hat f (\om) = -i ~ \sign (\om) + c
\end{equation}
avec
\[
\sign(\om) = 
\left\{
\begin{array} {ll}
1 & \mbox{ si $\om > 0$}\\
0 & \mbox{ si $\om = 0$}\\
-1 & \mbox{ si $\om < 0$}
\end{array}
\right.
\]
Comme $f(t)$ est r\'eelle antisym\'etrique, sa transform\'ee
de Fourier est imaginaire pure antisym\'etrique ce qui
prouve que $c = 0$.

$\bullet$ Le {\it peigne de Dirac}
\begin{equation}
\label{peigne}
c(t) = \sum_{n=-\infty}^{+ \infty} \delta (t -nT)
\end{equation}
est une distribution 
dont la transform\'ee de Fourier se d\'eduit de (\ref{dirac_transl})
\begin{equation}
\hat c( \om) = \sum_{n=-\infty}^{+ \infty} e^{-inT \om}.
\end{equation}
La formule de Poisson prouve que $\hat c (\om)$ est
aussi \'egal \`a un peigne de Dirac dont l'espacement est
$\frac{2 \pi} T$.

\begin{theorem}[Formule de Poisson]
\label{poisson-lemma}
\begin{equation}
\label{poisson}
\sum_{n=-\infty}^{+ \infty} e^{-inT \om} =
\frac {2 \pi} T \sum_{k=-\infty}^{+\infty}
\delta(\om - \frac {2\pi k} T) .
\end{equation}
\end{theorem}


\noindent {\bf D\'emonstration} 
Comme $C(\om) = \sum_{n=-\infty}^{+ \infty} e^{-inT \om}$ est
${2 \pi} / T$ p\'eriodique, il suffit de prouver que sa
restriction \`a $[-\pi/T,\pi/T]$ est \'egale \`a
${2 \pi}/ T \delta (\om)$.
Pour tout
$\phi (\om) \in \bC^\infty_0$ dont le support est inclus dans
$[-\pi/T,\pi/T]$, on veut montrer que
\begin{equation}
\label{converge}
<C , \phi> = 
\lim_{N \rightarrow + \infty} \int_{-\infty}^{+\infty}
\sum_{n=-N}^{+ N} e^{-inT \om}
\phi(\om) d\om = 
\frac {2\pi} T \phi(0) .
\end{equation}
La somme de cette s\'erie g\'eom\'etrique est
\begin{equation}
\label{geom-sum}
\sum_{n=-N}^{+N} e^{-inT\om} = 
\frac {\sin [(N+1/2)T\om]} {\sin[T\om/2]} .
\end{equation}
Donc
\begin{equation}
<C , \phi> = \lim_{N \rightarrow + \infty} 
\frac {2 \pi} T 
\int_{-\pi/T}^{+\pi/T} \frac {\sin [(N+1/2)T\om]} {\pi \om} 
\frac {T\om/2} {\sin[T\om/2]} \phi(\om) d\om .
\end{equation}
Pour $|\om| < \pi/T$, on d\'efinit
\[
\hat \psi (\om) = {\phi(\om)}  \frac {T\om/2} {\sin[T\om/2]} 
\]
tandis que $\hat \psi (\om) = 0$ si $|\om| > \pi /T$.
Cette fonction est la transform\'ee de Fourier de
$\psi (t) \in \LD$.
Comme ${2 \sin (a \om)}/{\om}$ est la transform\'ee de Fourier de
$1_{[-a,a]}(t)$, la formule de Parseval 
(\ref{parseval}) implique
\begin{equation}
<C , \phi> = 
\frac {2 \pi} T 
\int_{-\infty}^{+\infty} \frac {\sin [(N+1/2)T \om]}{\pi \om} 
\hat \psi (\om) d\om = \frac {2 \pi} T 
\int_{-(N+1/2)T}^{(N+1/2)T} \psi(t) dt .
\end{equation}
Lorsque $N$ tend vers $+ \infty$ l'int\'egral converge vers
$\hat \psi(0) = \phi(0)$. $\Box$\\


\section{Synth\`ese de filtres}

\subsection{Filtres passe-bandes}

La transform\'ee
de Fourier d'un signal filtr\'e $g(t) = f \star h (t)$ est
\[
\hat g (\om) = \hat f (\om) \hat h (\om) .
\]
De nombreuses applications n\'ecessitent
d'isoler les composantes du signal
dans diff\'erentes bandes de fr\'equences.

Un filtre
passe-bas id\'eal a une fonction de transfert d\'efinie par
\begin{equation} 
\label{passe-bas}
\hat h_0 ( \om  ) = 
   \left\{ \begin{array}{ll} 
1 & \mbox{si $|\om| \leq \om_c$}\\
0 & \mbox{si $|\om| > \om_c$}
\end{array}
   \right.  
\end{equation}
Il \'elimine donc toutes les fr\'equences de $\hat f (\om)$
au del\`a de $\om_c$.
On d\'eduit de (\ref{fourier-sinc}) que la r\'eponse impulsionnelle 
de ce filtre est 
\[
h_0(t) = \frac{\sin (\om_c t)} {\pi t} .
\]
Ce filtre passe-bas id\'eal est ni causal
ni stable. Le paragraphe suivant
explique comme l'approximer avec
un syst\`eme physiquement r\'ealisable.

Un filtre passe-bande r\'eel 
a une fonction de transfert qui supprime
toute composante fr\'equentielle en dehors de deux intervalles
sym\'etriques par rapport \`a $\om = 0$
\begin{equation} 
\hat h_1 ( \om  ) = 
   \left\{ \begin{array}{ll} 
1 & \mbox{si $|\om| \in [\om_0 - \om_c , \om_0 + \om_c ]$}\\
0 & \mbox{ailleurs} 
\end{array}
   \right.  
\end{equation}
Un tel filtre peut se d\'eduire d'un filtre passe-bas. Comme 
\[
\hat h_1(\om) = \hat h_0 (\om-\om_0)+ \hat h_0 (\om+\om_0) .
\]
Comme la transform\'ee de Fourier de
$h_0 (t) e^{i\om_0 t}$ est $\hat h_0 (\om-\om_0)$ on d\'eduit que
\[
h_1(t) = 2 \cos (\om_0 t)~ h_0 (t) = 2 \cos (\om_0 t)~  
\frac{\sin \om_c t} {\pi t} .
\]
Ce filtre est g\'en\'eralement approxim\'e par un filtre
causal et stable, en rempla\c{c}ant
$h_0 (t)$ par une approximation stable et causale.

\subsection{Filtrage par circuits \'electroniques}
\label{circuits}

Un filtrage lin\'eaire analogique est le plus
souvent impl\'ement\'e avec un
circuit \'electronique. Le signal $f(t)$ est
repr\'esent\'e par une diff\'erence de potentiel $u (t) = f(t)$
appliqu\'ee \`a l'entr\'ee du circuit. 
Pour certaines r\'eponses impulsionnelles $h(t)$, nous allons montrer
que l'on peut configurer le circuit de fa\c con \`a ce que
la diff\'erence de potentiel
$v (t)$ \`a la sortie soit \'egale au produit de convolution
$v (t) = u \star h (t)$ (voire figure \ref{circuit-fig}).

Les circuits VLSI analogiques sont 
essentiellement compos\'es de r\'esistances, de capaci\'t\'es, et
d'amplificateurs op\'erationnels, construits avec  
des transistors. Les inductances ne sont pas utilis\'ees 
car elles demandent trop de place sur le silicium,
mais elles sont remplac\'ees par 
des circuits \'equivalents.
Ce type de circuit relie les diff\'erences de
potentiels \`a l'entr\'ee et \`a la sortie par une \'equation 
diff\'erentielle
\`a coefficients constants
\begin{equation}
\label{equa-diff}
a_N \frac {d^N v (t)}{dt^N} + 
... + a_{1} \frac {d v (t)}{dt} + a_0 v (t) = 
b_M \frac {d^M u (t)}{dt^M} + 
... + b_{1} \frac {d u (t)}{dt} + b_0 u (t) .
\end{equation}

On suppose que $u (t)$ est un signal causal, $u (t) = 0$ pour
$t < 0$, et l'on veut calculer la solution $v (t)$ de 
cette \'equation diff\'erentielle. 
Cette solution d\'epend des conditions initiales
\`a la sortie du circuit sp\'ecifi\'ees par 
$ \{ \frac {d^k v (0)} {dt^k} \}_{0 \leq k < N}$.
Nous supposerons que le circuit est initialement au repos ce qui
signifie que toutes ces d\'eriv\'ees sont nulles. 
La sortie $v(t)$ est alors reli\'ee a $u(t)$ par un op\'erateur
lin\'eaire homog\`ene dont nous calculons la fonction de transfert.
\\
\\
{\bf Fonction de transfert}
La propri\'et\'e de diff\'erentiation 
(\ref{diffentiation}) permet de calculer
la transform\'ee de Fourier de chaque c\^ot\'e de 
l'\'egalit\'e (\ref{equa-diff})
\begin{eqnarray*}
{a_N (i\om)^N \hat v (\om) + ... + a_{1} (i \om) \hat v (\om) 
+ a_0} \hat v (\om) & = &\\
{b_M (i\om)^M \hat u (\om) + ... + b_{1} (i \om) \hat u (\om)
+ b_0} \hat u (\om). & &
\end{eqnarray*}
La fonction de transfert est donc
\begin{equation}
\label{electr-transf}
\hat h (\om) = \frac {\hat v (\om)}{\hat u (\om)} =
\frac {b_M (i\om)^M + ... + b_{1} (i \om) + b_0} 
{a_N (i\om)^N + ... + a_{1} (i \om) + a_0} .
\end{equation}
Cette fonction de transfert est aussi appel\'ee
l'imp\'edance du circuit. 

Dans le cas d'un circuit \'electronique, on a 
$N \geq M$, car $|\hat h(\om)|$ ne peut pas tendre ver $+\infty$ \`a
haute fr\'equences.
La sortie du circuit initialement au repos peut s'\'ecrire
\[
v (t) = \int_{-\infty}^{+\infty} h(\tau) u(t-\tau) d\tau =
\int_{0}^{+\infty} h(\tau) u(t-\tau) d\tau ,
\]
car la r\'eponse impulsionnelle $h(t)$ est causale.
\\
\\
{\bf Exemple}
Le circuit RC avec amplification 
de la figure \ref{circuit-fig} est un exemple particuli\`erement
simple qui relie l'entr\'ee et la sortie par l'\'equation
\[
RC \frac {d v (t)}{dt} + v (t) = (1 + \frac {R_2} {R_1})u (t).
\]
L'imp\'edance est donc
\[
\hat h(\om) = \frac {1 + \frac {R_2} {R_1}} {1 + RC i \om} .
\]
One peut v\'erifier (exercice) que la r\'eponse impulsionnelle 
du filtrage homog\`ene est causale et s'exprime \`a partir
de la fonction \'echelon (\ref{Heavyside}) par
\[
h(t) = \frac 1 {RC} \,(1 + \frac {R_2} {R_1})\, 
e^{-\frac {t}{RC}} \, \gamma (t) .
\]

\begin{figure}[bhtp]
\centerline{
	\leavevmode\epsfbox{/home/mallat/X/TREX/figures/SigFig/MALLATFIG2.1-EPS}}
%\vspace{6cm}
\caption{Circuit RC avec un amplificateur op\'erationnel}
\label{circuit-fig}
\end{figure}


\subsection{Approximations par filtres rationnels}

Nous avons vu qu'un circuit \'electronique
impl\'emente 
un filtre dont la fonction de transfert
est une fonction rationnelle de $i \om$
\begin{equation}
\label{rationnel}
\hat h(\om) = \frac {N(i\om)} {D(i\om)} ,
\end{equation}
o\`u $N(u)$ et $D(u)$ sont des polyn\^omes \`a coefficients
r\'eels. 
On peut d\'emontrer (exercice) que le filtre est causal et
stable si et seulement si $D(s)$ est un polyn\^ome dont les
racines ont des parties r\'eelles strictement n\'egatives. 
Par ailleurs on montre aussi que si 
$P(\om)$ est une fonction
rationnelle de $i \om$ avec $P(\om) \geq 0$ pour tout $\om \in \R$,
alors il existe une fonction de transfert rationelle, 
correspondant \`a un filtre causal et
stable, qui satisfait
\[
|\hat h (\om)|^2 = P(\om).
\]

Un filtre 
passe-bas id\'eal 
\[
\hat h_0 (\om) = 1_{[-\om_c,\om_c]} (\om)
\] 
n'est pas r\'ealisable
par un circuit \'electrique car sa fonction de transfert
n'est pas rationnelle. 
Le nombre d'\'el\'ements (r\'esistances, capacit\'es,
amplificateurs) n\'ecessaires pour impl\'ementer 
une fonction de transfert rationnelle $\hat h(\om)$ est
proportionnel au degr\'e du d\'enominateur.
Pour limiter la complexit\'e
du circuit, on veut donc
approximer $|\hat h_0 (\om)|^2$ par une
fonction rationnelle de faible degr\'e, tout en minimisant l'erreur
d'approximation. 
L'erreur
d'approximation est d\'efinie par un gabarit illustr\'e
par la figure \ref{gabarit}, qui sp\'ecifie 
l'amplitude maximum des oscillations de
$|\hat h(\om)|^2$ dans les bandes de passage et d'att\'enuation
ainsi que la largeur de la bande de transition.
Le probl\`eme est donc de trouver des fonctions rationnelles
de degr\'e le plus faible possible, qui satisfont les
contraintes de gabarit impos\'ees par une application.
Les polyn\^omes de Butterworth ou de Chebyshev ont des
propri\'et\'es particuli\`erement bien adapt\'ees \`a ce type 
d'approximation.

\begin{figure}[bhtp]
\centerline{
	\leavevmode\epsfbox{/home/mallat/X/TREX/figures/SigFig/MALLATFIG2.2-EPS}}
\caption{Le gabarit d'un filtre sp\'ecifie
l'amplitude maximum des oscillations $\epsilon_p $ et $\epsilon_a $
dans les bandes de passage et d'att\'enuation du filtre, ainsi que
la largeur $\Delta \omega$ de la bande de transition}
\label{gabarit}
\end{figure}

\noindent {\bf Filtres de Butterworth}
Un filtre de Butterworth d'ordre $n$ est d\'efini par
\begin{equation}
\label{butterworth}
|\hat h_n^b(\om)|^2 = \frac 1 {1 + (\om / \om_c)^{2n}} .
\end{equation}
Plus $n$ augmente plus le 
filtre est plat au voisinage de 
$\om = 0$ car
les $2n-1$ premi\`eres d\'eriv\'ees de $|\hat h(\om)|^2$ sont nulles
en $\om = 0$. A la fr\'equence de coupure $\om_c$, 
$|\hat h^b_n(\om_c)|^2 = \half$.
Ces filtres convergent vers le filtre passe-bas id\'eal
(\ref{passe-bas}), au sens \`ou
pour tout $\om \in \R -\{ \om_c\}$
\[
\lim_{n \rightarrow + \infty} |\hat h_n^b (\om)|^2 = 
|\hat h_0 (\om)|^2 .
\]
\\
\\
{\bf Filtres de Chebyshev}
Les filtres de Chebyshev ne sont pas plats au
voisinage de $\om = 0$ mais ils ont des oscillations de
tailles constantes dans la bande de passage. A
degr\'e \'egal, ils ont
une bande de transition plus faible que les filtres
de Butterworth. Ils sont d\'efinis par
\[
|\hat h_n^c (\om)|^2 = \frac 1 {1 + \eps^2 C_n^2 (\om / \om_c)} ,
\]
o\`u $C_n (\om)$ est un polyn\^ome de Chebyshev de degr\'e $n$
qui peut s'\'ecrire
\begin{equation}
C_n(\om) = 
   \left\{ \begin{array}{ll} 
\cos (n \cos^{-1} \om )  & \mbox{si $0 \leq |\om| \leq 1$}\\
\cosh (n \cosh^{-1} \om ) &\mbox{si $|\om| \geq 1$}
\end{array}
   \right.  
\end{equation}
Ces polyn\^omes peuvent aussi \^etre caract\'eris\'es par r\'ecurrence
\[
C_0 (\om) = 1 ~~~,~~~C_1 (\om) = \om ,
\]
\[
C_{n+1} (\om) = 2 \om C_n (\om) - C_{n-1} (\om) .
\]
Pour $|\om| < 1$,
$|C_n(\om)|^2$ oscille r\'eguli\`erement
entre $0$ et $1$ tandis que lorsque
$|\om| > 1$, le
cosinus hyperbolique augmente
de fa\c{c}on monotone.
En cons\'equence $|\hat h_n^c (\om)|^2$ 
oscille entre 1 et
$\frac 1 {1 + \eps^2}$ lorsque $0 \leq |\om| /\om_c \leq 1$.
Lorsque $|\om| /\om_c \geq 1$ alors 
$|\hat h_n^c (\om)|^2$ a une d\'ecroissance
monotone vers $0$.

Il existe d'autres familles de fonctions rationnelles utilis\'ees
pour l'approximation du filtre passe-bas id\'eal et le choix
d'une approximation pour une application est un art
qui d\'epend du type de distortions que l'on peut admettre.


\section{Modulation d'amplitude}
\label{modulation}

A travers un canal unique de transmission il est souvent n\'ecessaire
de transmettre plusieurs signaux simultan\'ement, comme par exemple
des \'emissions de radio ou des conversations t\'el\'ephoniques.
Lorsque ces signaux peuvent \^etre bien approxim\'es par des fonctions
dont la transform\'ee de Fourier est \`a support compact, la
modulation d'amplitude permet de multiplexer ces signaux
pour les transmettre en m\^eme temps.
L'audition n'\'etant essentiellement sensible qu'\`a des
sons entre 300Hz et 3300Hz, on peut limiter les sons 
par filtrage passe-bas \`a l'intervalle de fr\'equence 
$[-3300, 3300]$,
lors de leur transmission t\'el\'ephonique ou radio.

On suppose que 
les signaux $\{ f_n \}_{\ZnN}$ que l'on veut multiplexer ont tous
une transform\'ee de Fourier dont le support est inclu
dans $[-b, b]$. 
La modulation d'amplitude permet de multiplexer ces $N$ signaux
en un seul signal dont la bande de
fr\'equence est $N$ fois plus grande.
Pour cela on 
transforme chaque
signal $f_n(t)$ en un signal modul\'e
$f^m_n (t)$ dont la transform\'ee
de Fourier a un support \'egal \`a 
$[-\om_n-b,-\om_n] \cup [\om_n,b+\om_n]$. 
En choisissant $\om_n = n b$,
le support de $f_n^m (\om)$ n'intersecte pas
le support de $f_p^m (\om)$ si $n \neq p$. 
A partir du signal multiplex\'e
\begin{equation}
\label{sig-mult}
M(t) = \sum_{n=0}^{N-1}  f^m_n (t) 
\end{equation}
on peut r\'ecup\'erer chaque signal $f^m_n$ par filtrage
passe-bande, puis on reconstruit
$f_n (t)$ par d\'emodulation. 
Les paragraphes suivants expliquent le calcul
de ces diff\'erentes \'etapes.

\subsection{Signal analytique et transform\'ee de Hilbert}

Les signaux r\'eels ayant une transform\'ee de Fourier
\`a sym\'etrie hermitienne, leur
support est sym\'etrique par rapport \`a $\om = 0$.
On veut donc 
transposer les fr\'equences
de $f_n (t)$ de l'intervalle
$[-b,b]$ \`a un double intervalle sym\'etrique
$[-\om_n-b, -\om_n] \cup [\om_n, \om_n+b]$.
Pour cela nous consid\'erons s\'epar\'ement les fr\'equences 
positives et n\'egatives de $f_n(t)$, comme l'illustre la
figure \ref{Multi-fig}.


Comme $\hat f_n(\om) = \hat f_n^*(-\om)$, $\hat f_n(\om)$ est
enti\`erement caract\'eris\'e par sa restriction \`a $\om > 0$
donn\'ee par
\begin{equation} 
\hat f_n^z ( \om ) = 
   \left \{ \begin{array}{ll} 
            2 \hat f_n ( \om ) & \mbox{si $\om > 0$}\\
            \hat f_n ( 0 ) & \mbox{si $\om = 0$}\\
            0 & \mbox{si $ \om < 0$}
            \end{array}
   \right.  
\end{equation}
La fonction
$f_n^z (t)$ est appel\'ee partie analytique de $f_n(t)$
car on peut d\'emontrer
qu'elle admet une extension analytique dans la partie sup\'erieure
du plan complexe que l'on construit gr\^ace \`a la transform\'ee
de Laplace [Bony]. 
Les propri\'et\'es $(\ref{reelle},\ref{imaginaire})$ de
la transform\'ee de Fourier montrent que 
la transform\'ee de Fourier de la partie r\'eelle de
$f^z_n (t)$ est
\[
\frac {\hat f_n^z ( \om ) + \hat f_n^{z*} ( - \om )} 2 = \hat f_n(\om) ,
\]
et donc que $\Reel  f_n^z (t) = f_n(t)$. De m\^eme
la transform\'ee de Fourier de sa partie imaginaire est
\begin{equation} 
\frac {\hat f_n^z ( \om ) - \hat f_n^{z*} ( - \om )} {2 i} 
= -i ~\mbox{sign} ( \om ) \hat f_n ( \om ).
\end{equation} 
Nous avons vu en (\ref{hilbert_fourier}) que 
$-i ~\sign ( \om )$ est la fonction de transfert du
filtre de Hilbert.
La partie imaginaire de $f_n^z (t)$ est donc la transform\'ee
de Hilbert de $f(t)$
\begin{equation} 
\Ima f_n^z (t) = {\cal H}[ f_n] (t) = 
f_n \star \vp \frac 1 {\pi t } = 
\frac 1 \pi \int_{- \infty}^{+ \infty} 
{f_n( u )} ~\vp \frac 1 {t - u} d u .
\end{equation} 
\\
\\
{\bf Modulation d'amplitude}
Pour transposer en fr\'equence $\hat f_n (\om)$ 
et obtenir une fonction
$ \hat f_n^m ( \om)$ dont le support est 
$[-b-\om_n,-\om_n] \cup [\om_n,\om_n+b]$, on
d\'ecale de $\om_n$ les
fr\'equences positives $\hat f_n^z (\om)$ et de $-\om_n$
les fr\'equences n\'egatives $\hat f_n^z(-\om)$ (figure \ref{Multi-fig})
\begin{equation} 
\hat f^m_n (\om ) = \frac {\hat f^z_n( \om - \om_n ) + 
\hat f^{z *}_n( - \om - \om_n )} 2 .
\end{equation} 
On calcule avec (\ref{reelle}) et (\ref{modul}) la
transform\'ee de Fourier inverse du c\^ot\'e droit
de cette \'equation, ce qui nous donne
\begin{equation} 
f^m_n (t) = \Reel [ f_n^z (t) e^{i \om_n t} ] .
\end{equation} 
Comme $f_n^z (t) = f_n(t) + i {\cal H}[f_n](t)$, 
la modulation d'amplitude s'exprime \`a partir de la
transform\'ee de Hilbert par
\begin{equation} 
\label{modulation-amplitude}
f^m_n(t) = f_n(t) cos (\om_n t) - {\cal H} [f_n] (t) sin (\om_n t) .
\end{equation} 



%\begin{figure}[bhtp]
%\centerline{
%	\leavevmode\epsfbox{/home/mallat/X/TREX/figures/SigFig/FIG2.3.EPS.txt}}
%\caption{Multiplexage par modulation d'amplitude}
%\label{Multi-fig}
%\end{figure}

\begin{figure}[bhtp]
\vspace{13cm}
\caption{Multiplexage par modulation d'amplitude}
\label{Multi-fig}
\end{figure}

\subsection{D\'emodulation et d\'etection synchrone}

Le signal multiplex\'e $M(t)$ (\ref {sig-mult}) est la
somme de signaux modules $f^m_n(t)$ 
dont les supports fr\'equentiels
ne s'intersectent pas.
On d\'efinit un filtre passe-bande dont le support est
le m\^eme que celui de $\hat f_n^m (\om)$
\begin{equation} 
\hat h_n ( \om ) = 
   \left \{ \begin{array}{ll} 
            1 & \mbox{si~~ $|\om| \in  [ \om_n ,  \om_n + b ]$}\\
            0 & \mbox{ailleurs } 
            \end{array}
   \right.  
\end{equation}
On a alors 
\[
f_n^m (t) = M \star h_n (t) .
\]

Le signal $f_n(t)$ se reconstruit \`a partir de $f_n^m (t)$ 
en supprimant la modulation d'amplitude.
L'\'equation (\ref{modulation-amplitude}) montre que
\begin{equation} 
g_n (t) = 2 f_n^m (t) cos(\om_n t ) = f_n (t) + 
f_n (t) cos (2 \om_n t) - 
{\cal H} [f_n] (t) sin (2 \om_n t) ,
\end{equation} 
ce qui s'\'ecrit en Fourier
\begin{eqnarray}
\label{demodulation}
\hat g_n (\om)
= \hat f_n (\om)& + &
\frac {\hat f_n (\om-2\om_n) + \hat f_n (\om+2\om_n)} 2 \\
&+ &
\frac {\hat {{\cal H}[f_n]}(\om-2\om_n) - 
\hat {{\cal H}[f_n]}(\om+2\om_n)} {2i}. \nonumber 
\end{eqnarray}
Comme le support de $\hat f_n (\om)$ et 
de $\hat {{\cal H}[f_n]}(\om )$ est $[-b,b]$ et que
$\om_n > b$, on s\'epare $\hat f(\om)$ des autres composantes
fr\'equentielles avec la fonction de transfert
\begin{equation} 
\hat h_0 ( \om ) = 
   \left \{ \begin{array}{ll} 
            1 &\mbox{si $|\om | \leq b$}\\
            0 & \mbox{ailleurs}
            \end{array} .
   \right.  
\end{equation}
On d\'eduit de (\ref{demodulation}) que
\[
f_n (t) = g_n \star h_0  (t) = (2 f_n^m (u) cos(\om_n u ) 
\star h_0 (u))(t) .
\]
