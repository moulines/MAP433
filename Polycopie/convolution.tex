\section{D\'efinition}
\begin{definition}
La convolution de deux fonctions $f$ et $g$ de $\rset$ \`a valeurs dans $\cset$ est une fonction $f*g$ d\'efinie par
$$
f*g(x)=\int_{\rset}f(x-t)g(t)\rmd t=\int_{\rset}f(u)g(x-u) \rmd u.
$$
\end{definition}
L'existence du produit de convolution requiert bien entendu des hypoth\`eses que nous pr\'eciserons dans la suite.
\begin{example}
\label{example:convolution}
\begin{enumerate}[label=(\roman*),wide=0pt, labelindent=\parindent]
\item Let $f=g=\1_{[0,1]}$. Alors
$$
\int_{\rset}f(x-t)g(t) \rmd t=\int_{0}^{1}\1_{[0,1]}(x-t) \rmd t= \Leb(\ccint{0,1} \cap \ccint{x-1,\ x})
$$
qui est la fonction "triangle"
$$
f*g(x)=\begin{cases}
0 &  x\leq 0,\\
x &  0\leq x\leq 1,\\
2-x & 1\leq x\leq 2,\\
0 &  x\geq 2.
\end{cases}
$$
Le produit de convolution de deux fonctions discontinues est donc continue.
\item  Soit $f\in \lone(\rset)$ and $g= (2h)^{-1} \1_{\ccint{1-h,+h}}$ o\`u $h>0$.
$$
f*g(x)=\frac{1}{2h}\int_{-h}^{+h}f(x-t) \rmd t=\frac{1}{2h}\int_{x-h}^{x+h}f(u) \rmd  u,
$$
qui est la moyenne de $f$ sur l'intervalle $\ccint{x-h,\ x+h}$. Le lecteur montrera que la fonction $f*g$ est continue.
\end{enumerate}
\end{example}


\section{Convolution dans $\lone(\rset)$}

\begin{definition}[support d'une fonction  measurable]
Soit $f$ : $\rset\rightarrow \cset$ une fonction mesurable. Soit $O_{i}, i\in I$ la  famille de tous les ouverts de $\rset$ tels que pour tout $i\in I$, $f=0$ p.p. sur $O_{i}$. Soit $O=\bigcup_{i\in I}O_{i}$ et on d\'efinit le support de $f, \supp(f)$ , comme l'ensemble ferm\'e $\rset\backslash O$.
\end{definition}
On v\'erifie ais\'ement que si $f=g$ p.p. alors $\supp(f)=\supp(g)$. Dans l'example~\ref{example:convolution} $\supp(f)=\supp(g)=[0,1]$, et la  convolution $f*g$ a pour effet d'\'etendre le support $\supp(f*g)=[0,2]$. De fa\c{c}on g\'en\'erale, nous avons
\begin{lemma}
\label{lem:support-convolution}
Soient $f$ et $g$ deux fonctions telles que $f*g$ existe. Alors
$$
\supp(f*g)\subset\overline{\supp(f) + \supp(g)}.
$$
\end{lemma}
\begin{proof}
  Notons $S= \rset\setminus (\supp(f)+\supp(g))$ et $S^{o}$ l'int\'erieur de $S$. Soit $x\in S$. Pour tout $t\in \supp(f)$ nous avons $(x-t)\not\in \supp(g)$ et par cons\'equent $\displaystyle \int_{\rset}g(x-t)f(t)\rmd t=0$. Soit $O_{f*g}$ le plus grand ensemble ouvert sur lequel $f*g=0$ p.p.  Nous avons montr\'e que si $x\in S^{o}$ alors $x\in O_{f*g}$. Par cons\'equent $\rset\setminus O_{f*g}=\supp(f*g) \subset \rset\setminus S^{o}$. La preuve d\'ecoule de $\rset\setminus S^{o}=\overline{\rset\setminus S}$.
\end{proof}

\begin{proposition}
\label{prop:20-2-1}
Si $f$ et $g$ sont des \'el\'ements $\lone(\rset)$ , alors:
\begin{enumerate}[label=(\roman*)]
\item $f*g$ est d\'efini presque-partout et $f*g \in \lone(\rset)$ .
\item La convolution est op\'erateur bilin\'eaire continu de $\lone(\rset)\times \lone(\rset)$ \`a valeurs dans $\lone(\rset)$ et
\begin{equation}
\label{eq:inegalite-norm}
\norm{f*g}[1] \leq \norm{f}[1] \, \norm{ g}[1] \, .
\end{equation}
\end{enumerate}
\end{proposition}
\begin{proof}
 \begin{enumerate}[label=(\roman*), wide=0pt, labelindent=\parindent]
\item Comme $f, g \in \lone(\rset)$, le th\'eor\`eme de Fubini montre que la fonction $(y,\ z)\mapsto f(y)g(z)$ appartient \`a $L_{1}(\rset^{2})$. Le changement de variables $y=x-t$ et $z=t$ montre
$$
\iint_{\rset\times \rset}f(y)g(z)dydz=\iint_{\rset\times \rset}f(x-t)g(t)\rmd x\rmd t.
$$
La fonction $x\displaystyle \mapsto\int_{\rset}f(x-t)g(t)\rmd t$ est donc d\'efinie p.p. et appartient \`a  $\lone(\rset)$, en appliquant encore le th\'eor\`eme de Fubini.
\item Pour \'etablir l'in\'egalit\'e \eqref{eq:inegalite-norm}, nous \'ecrivons
$$
|f*g(x)|\leq\int_{\rset}|f(x-t)||g(t)|\rmd t=|f|*|g|(x)\ .
$$
Par cons\'equent, nous avons
\begin{align*}
\int_{\rset}|f*g|(x)\rmd x
&\leq\int_{\rset}|f|*|g|(x)\rmd x=\int_{\rset}\rmd x\int_{\rset}|f(x-t)||g(t)|\rmd t \\
&=\int_{\rset}|g(t)|(\int_{\rset}|f(x-t)|\rmd x)\rmd t= \norm{g}[1] \norm{f}[1].
\end{align*}
\end{enumerate}
\end{proof}

\begin{proposition}
Supposons que $f\in \L1loc(\rset)$ et que $g\in \lone(\rset)$ .
\begin{enumerate}[label=(\roman*)]
\item Si $\supp(g)$ est born\'e, alors $f*g(x)$ est d\'efini p.p .et $f * g \in \L1loc(\rset)$ .
\item Si $f$ est born\'ee alors $f*g(x)$ est d\'efini pour tout $x$ et  $f * g \in L_{\infty}(\rset)$ .
\end{enumerate}
\end{proposition}
\begin{proof}
\begin{enumerate}[label=(\roman*), wide=0pt, labelindent=\parindent]
\item $g = 0$ p.p. sur le compl\'ementaire d'un intervalle $\ccint{-a,\ a}$. Soit $x \in \ccint{\alpha,\ \beta}$. Pour tout $t\in \ccint{-a,\ a}$ et $x\in \ccint{\alpha,\ \beta}$,
$f(x-t)g(t)=\1_{[\alpha-a,\beta+a]}(x-t)f(x-t)g(t)$  et donc
$$
f*g(x)=\int_{-a}^{+a}f(x-t)g(t)\rmd t=(\1_{[\alpha-a,\beta+a]}f)*g(x)\ .
$$
$f*g$ coincide sur $[\alpha,\ \beta]$ avec la convolution de deux fonctions appartenant \`a $\lone(\rset)$. \Cref{prop:20-2-1} montre que cette fonction est d\'efinie p.p. et est int\'egrable. Donc $f*g$ est d\'efinie p.p. et est int\'egrable sur tout ensemble compact.
\item Si $f\in L_{\infty}(\rset)$ , alors
$$
|\int_{\rset}f(u)g(x-u)\rmd u|\leq\norm{f}[\infty]\int_{\rset}|g(x-u)|,\ \rmd u=\norm{f}[\infty] \,
\norm{g}[1] \, .
$$
pour tout $x$ et $\norm{f*g}[\infty]\leq \norm{f}[\infty] \, \norm{g}[1].$
\end{enumerate}
\end{proof}
\section{Convolution dans $L_{p}(\rset)$}



\begin{proposition}
Supposons que $f\in L_{p}(\rset)$ et $g\in L_{q}(\rset)$ o\`u $p^{-1}+q^{-1}=1$. Alors:
\begin{enumerate}[label=(\roman*)]
\item $f*g$ est d\'efini est continue et born\'ee sur  $\rset.$
\item $\norm{f*g}[\infty]\leq \norm{f}[p] \, \norm{g}[q]$.
\end{enumerate}
\end{proposition}
\begin{proof}
Nous allons \'etablir ce r\'esultat pour: $p=1,  q=+\infty$ and $p=2, q=2.$
Consid\'erons tout d'abord le cas $p=1, q=+\infty.$
Nous avons d\'ej\`a \'etabli  que $f*g$ est d\'efini partout et born\'e; seule la continuit\'e reste \`a \'etablir. Nous avons
\begin{align*}
|f*g(x)-f*g(y)| &\leq\int|f(x-t)-f(y-t)||g(t)|\rmd t \\
&\leq\Vert g\Vert_{\infty}\int|f(x-t)-f(y-t)|\rmd t.
\end{align*}
 Supposons tout d'abord que $f$ est continue \`a suppot compact, $\supp(f) \subset\ooint{-a,\ a}$.
 Pour $|x-y|$ suffisamment petit,
\begin{align*}
\int_{\rset}|f(x-t)-f(y-t)|\rmd t
&=\int_{\rset}|f(x-y+u)-f(u)|\rmd u \\
&=\int_{-a}^{+a}|f(x-y+u)-f(u)|\rmd u \leq 2a\sup_{|u|\leq a}|f(x-y+u)-f(u)|.
\end{align*}
Comme $f$ est uniform\'ement continue sur $[-a,\ a]$,  $f*g$ est (uniform\'ement) continue sur $\rset$.

Les fonctions continues \`a support compact sont denses dans $\lone(\rset)$.
Soit $\sequence{f}[n][\nset]$ une suite de fonctions continues \`a support compact telles que
$\lim_{n\rightarrow\infty} \norm{f_{n}-f}[1]=0$. Nous avons
\begin{align*}
|f*g(x)-f*g(y)|&\leq|f*g(x)-f_{n}*g(x)|+|f_{n}*g(y)-f*g(y)| +|f_{n}*g(x)-f_{n}*g(y)|,
\end{align*}
qui implique
$$
|f*g(x)-f*g(y)|\leq 2 \norm{g}[\infty] \norm{f-f_{n}}[1]+|f_{n}*g(x)-f_{n}*g(y)|.
$$
Le premier terme tend vers $0$ lorsque $n \to \infty$, et le second terme est uniform\'ement continu pour tout $n$;
par cons\'equent $f*g$ est uniform\'ement continue sur $\rset.$



Consid\'erons le cas $p=2$ et  $q=2$. L'in\'egalit\'e de Cauchy-Schwarz montre que
$$
|f*g(x)|\displaystyle \leq\int_{\rset}|f(x-t)||g(t)|\rmd t\leq \left(\int_{\rset}|f(x-t)|^{2}\rmd t\right)^{1/2} \left(\int_{\rset}|g(t)|^{2}\rmd t \right)^{1/2}
$$
 et donc que $\norm{f*g}[\infty] \leq \norm{f}[2] \, \norm{g}[2]$.
  La continuit\'e d\'ecoule de la densit\'e de $C_{c}^{0}(\rset)$  (fonctions continues \`a support compact) dans $L_{2}(\rset)$.

La preuve pour $p\neq 1,2$ est similaire en rempla\c{c}ant l'in\'egalit\'e de Cauchy-Schwarz par l'in\'egalit\'e de H\"older et en utilisant la densit\'e de $C_c^0(\rset)$ dans $L_p(\rset)$.
\end{proof}

Nous consid\'erons dans la proposition suivante la convolution d'une fonction de $\lone(\rset)$
 et d'une fonction de $L_{2}(\rset)$.
\begin{proposition}
Si $f\in \lone(\rset)$ et $g\in L_{2}(\rset)$ , alors
\begin{enumerate}[label=(\roman*)]
\item $f*g(x)$ est d\'efini presque-partout.
\item  $f*g  \in L_{2}(\rset)$  et $\norm{f*g}[2]\leq \norm{f}[1] \, \norm{g}[2]$.
\end{enumerate}
\end{proposition}
\begin{proof}
\begin{enumerate}[label=(\roman*), wide=0pt, labelindent=\parindent]
\item Nous avons
\begin{equation}
|f(u)g(x-u)|=(|f(u)||g(x-u)|^{2})^{1/2}(|f(u)|)^{1/2}
\end{equation}
Comme $|f|\in \lone(\rset)$ et $|g|^{2}\in \lone(\rset)$ , la fonction $u\mapsto|f(u)||g(x-u)|^{2}$ est int\'egrable pour presque-tout
$x$. The right-hand term of (20.4), being the product of two square integrable functions, is integrable. Thus $f*g(x)$ is defined for almost all $x.$

\item En utilisant l'in\'egalit\'e de Cauchy-Schwarz
\begin{align*}
|f*g(x)| &\leq\int_{\rset}|f(u)||g(x-u)|\rmd u \\
&\leq \left(\int_{\rset}|f(u)||g(x-u)|^{2}\rmd u \right)^{1/2} \left(\int_{\rset}|f(u)|\rmd u \right)^{1/2}
\end{align*}
et donc
$$
|f*g(x)|^{2}\leq(|f|*|g|^{2}(x))\norm{f}[1].
$$
En int\'egrant les deux membres de la seconde in\'egalit\'e, nous obtenons,
\begin{align*}
\int_{\rset}|f*g(x)|^{2}\rmd x &\leq \norm{f}[1] \int_{\rset}|f|*|g|^{2}(x)\rmd x \\
&\leq\norm{f}[1] \norm{f}[1] \norm{g^{2}}[1],
\end{align*}
ce qui implique $\norm{f*g}[2] \leq \norm{f}[1] \norm{g}[2]$.
\end{enumerate}
\end{proof}
La version convolu{\'e}e $f\star K$ de
$f$ adopte la r{\'e}gularit{\'e} du \emph{noyau} $K$. Ce principe est donn{\'e} par le
lemme suivant.

\begin{lemma}\label{lem:regulairsation-est-reguliere}
  Soient $f,g\in \lone(\rset)$. Alors on a les propri{\'e}t{\'e}s suivantes
  \begin{enumerate}[label=(\roman*)]
  \item\label{item:approx-cont-S-1} Si $g$ est $\mathcal{C}^k$, $f\star g$ est
  $\mathcal{C}^k$ et $(f\star g)^{(k)}=f\star g^{(k)}$.
\item\label{item:approx-cont-S-2} On a pour tout entier $p\geq0$,
  \begin{equation}
    \label{eq:decroissance-concolution}
    \|(f\star g)\times m_p\|_\infty\leq \|f\times (1+|m_p|)\|_\infty \;
    \|g\times (1+|m_p|)\|_1  \;,
  \end{equation}
o{\`u} $m_p$ d{\'e}signe le mon{\^o}me de degr{\'e} $p$, $m_p(x)=x^p$.
\item \label{item:approx-cont-S-2prim} Si $f$ et $g$ sont {\`a} d{\'e}croissance rapide,
  il en est de m{\^e}me de $f\star g$.
\item\label{item:approx-cont-S-3}  Si $f$ est {\`a} d{\'e}croissance rapide et
  $g\in\mcs$, $f\star g\in\mcs$.
  \end{enumerate}
\end{lemma}
\begin{proof}
  La propri{\'e}t{\'e}~\ref{item:approx-cont-S-1} est une simple application du lemme
  de d{\'e}rivation sous le signe somme.

Montrons la propri{\'e}t{\'e}~\ref{item:approx-cont-S-2}.
Pour tous $t,x\in\rset$, on a $|x|^p\leq(|x-t|+|t|)^p\leq|x-t|^p+|t|^p$.
D'o{\`u}
$$
|f\star g(x)||x|^p \leq \int |f(x-t)|\,|x-t|^p\,|g(t)|\rmd t+
\int |f(x-t)|\,|t|^p\,|g(t)|\rmd t \;.
$$
On obtient donc~(\ref{eq:decroissance-concolution}).

La propri{\'e}t{\'e}~\ref{item:approx-cont-S-2prim} est obtenu en appliquant
de~\ref{item:approx-cont-S-2} pour tout $p\geq1$.

La propri{\'e}t{\'e}~\ref{item:approx-cont-S-3} d{\'e}coule de \ref{item:approx-cont-S-1}
et~\ref{item:approx-cont-S-2prim}.
\end{proof}

\begin{remark}
On remarque facilement que l'on peut prendre dans les lemmes pr{\'e}c{\'e}dents
$K=g_1$, c'est-{\`a}-dire $K_\sigma=g_\sigma$, o{\`u} $g_\sigma$ est d{\'e}finie
en~(\ref{eq:gauss_func}). On peut parler dans ce cas de \emph{r{\'e}gularisation
  gaussienne}.

On peut prendre aussi
\begin{equation}
\label{eq:noyau-support-borne}
K(x)=
\begin{cases}
c^{-1} \rme^{-1/(1-x^2)} & |x| \leq 1 \\
0    & \text{sinon}
\end{cases}
\quad \text{avec} \quad c = \int_{-1}^{+1} \rme^{-(1-x^2)^{-1}} \rmd x \eqsp.
\end{equation}

\end{remark}

\begin{theorem}[densit\'e de $\mcs$ dans $L_p(\rset)$]
\label{theo:densite-mcs-Lp}
Soit $p \geq 1$ et $f \in L_p(\rset)$. Pour tout $\epsilon > 0$, il existe $g_\epsilon \in \mcs$ telle que
$\norm{f - g_\epsilon}[p] \leq \epsilon$.
\end{theorem}
\begin{proof}
Pour tout $\epsilon > 0$, il existe $f_\epsilon \in C_c^0(\rset)$ telle que $\norm{f - f_\epsilon}[p] \leq \epsilon$.
Pour $\sigma > 0$, on note $g_{\epsilon,\sigma} = f_\epsilon * K_\sigma$ o\`{u} $K$ est le noyau donn\'e par \eqref{eq:noyau-support-borne}. \Cref{lem:regularisation} montre que $g_{\epsilon,\sigma}$ converge uniform\'ement vers $f_\epsilon$, \ie\ $\lim_{\sigma \to 0} \norm{f_\epsilon - g_{\epsilon,\sigma}}[\infty]= 0$. Comme $f_\epsilon$ et $K_\sigma$ sont \`{a} supports compacts, \Cref{lem:support-convolution} montre que $\supp(g_{\epsilon,\sigma}) \subset \ccint{a,b}$ avec $-\infty < a < b < \infty$.
On a donc aussi
$$
\lim_{\sigma \to 0} \norm{f_\epsilon - g_{\epsilon,\sigma}}[p] = 0 \eqsp,
$$
ce qui implique que, pour tout $\epsilon > 0$,
$$
\lim_{\sigma \to 0} \norm{f - g_{\epsilon,\sigma}}[p] \leq \norm{f-f_\epsilon}[p] \leq \epsilon \eqsp.
$$
\Cref{lem:regulairsation-est-reguliere} montre que comme $f$ est \`{a} d\'ecroissance rapide et $K_\sigma \in \mcs$, alors $g_{\epsilon,\sigma} f * K_\sigma \in \mcs$.
\end{proof}


\section{Une premi\`ere extension de la transformation de Fourier inverse}
Ici nous l'appliquons dans le cadre $\lone(\rset)$ grace au r{\'e}sultat suivant, qui nous permettra de compl{\'e}ter la
proposition~\ref{prop:inversionFourierL1} par un th{\'e}or{\`e}me d'inversion.
\begin{proposition}\label{prop:DualiteSchwartz}
Soient deux fonctions $f$ et $g$ dans $\lone(\rset)$. Si, pour toute fonction \textit{test} $\phi$ dans $\mcs$, on a
$$
\int f(x)\,\phi(x)\,\rmd x= \int g(x)\,\phi(x)\,\rmd x,
$$
alors $f=g$ (au sens $\lone(\rset)$).
\end{proposition}
\begin{proof}
En prenant la diff{\'e}rence entre les deux membre de l'{\'e}galit{\'e} de l'hypoth{\`e}se, on voit qu'il suffit de montrer ce r{\'e}sultat pour
$g=0$. De plus, comme les fonctions continues sont denses dans l'ensemble des fonctions int\'egrables, on peut se contenter de prendre $f$ continue, le cas g{\'e}n{\'e}ral {\'e}tant
obtenu par passage {\`a} la limite. Or, pour $f$ continue et $g=0$, le r{\'e}sultat est imm{\'e}diat par application du principe de
r{\'e}gularisation en choisissant une fonction $K\in\mcs$ positive int{\'e}grant {\`a} 1 (par exemple $g_1$) puis en appliquant le
lemme~\ref{lem:regularisation} en tout point de la droite r{\'e}elle.
\end{proof}

On en d{\'e}duit le r{\'e}sultat annonc{\'e} qui compl{\`e}te la proposition~\ref{prop:inversionFourierL1}.
\begin{theorem}
\label{theo:inversion-L1}
Soit $f\in \lone(\rset)$ et supposons que $\hat{f}$ appartiennent aussi {\`a} $\lone(\rset)$. Alors
la fonction (continue) $\bar{\TF} \hat{f}$ est l'unique
repr{\'e}sentant continu de $f$.
\end{theorem}
\begin{proof}
La continuit{\'e} de $\bar{\TF} \hat{f}$ d{\'e}coule du th{\'e}or{\`e}me~\ref{thm:rieman-lebesgue}.
Pour toute fonction test $\phi$ de $\mcs$, on a, d'apr{\`e}s la proposition~\ref{prop:echangeTF} et le
th{\'e}or{\`e}me~\ref{thm:Schwarz},
$$
\int f(x) \phi(x)\,\rmd x=\int \hat{f}(\xi) \TFC(\phi)(\xi)\,d\xi.
$$
Mais comme $\hat{f}$, on peut r{\'e}appliquer l'{\'e}quivalent de la proposition~\ref{prop:echangeTF} mais pour la transform{\'e}e
inverse, ce qui donne alors directement
$$
\int \hat{f}(\xi) \TFC(\phi)(\xi)\,d\xi=\int \TFC(\hat{f})(x) \phi(x) \,\rmd x.
$$
D'o{\`u} le r{\'e}sultat en appliquant la proposition~\ref{prop:DualiteSchwartz}.
\end{proof}

\section{Convolution et transform\'ee de Fourier dans $\lone(\rset)$}
\begin{proposition}
\label{prop:convolution-L1}
Soient $f,g \in \lone(\rset)$.
\begin{enumerate}[label=(\roman*)]
\item \label{item:convolution-L1-i} $\widehat{f * g}(\xi)= \hat{f}(\xi) \, \hat{g}(\xi)$ pour tout $\xi \in \rset$,
\item Si de plus $\hat{f}, \hat{g} \in \lone(\rset)$, alors $\widehat{f \cdot g}(\xi)= \hat{f}(\xi) \, \hat{g}(\xi)$.
\end{enumerate}
\end{proposition}
\begin{proof}
\begin{enumerate}[label=(\roman*),wide=0pt, labelindent=\parindent]
\item Comme $f,g \in \lone(\rset)$, \Cref{prop:20-2-1} shows that $f * g \in \lone(\rset)$. On peut donc calculer $\widehat{f * g}(\xi)$. En utilisant le th\'eor\`eme de Fubini, nous avons
\begin{align*}
\int \rme^{-\rmi 2 \pi \xi x} f * g(x) \rmd x
&= \int \rme^{-\rmi 2 \pi \xi x} \left( \int f(x-t) g(t) \rmd t \right) \\
&= \int g(t) \left( \int \rme^{-\rmi 2 \pi \xi x} f(x-t) \rmd t \right) \rmd t \\
&= \int g(t) \rme^{- \rmi 2 \pi \xi t} \hat{f}(\xi) \rmd t = \hat{g}(\xi) \cdot \hat{f}(\xi)
\end{align*}
\item On peut appliquer \eqref{item:convolution-L1-i} avec $\TFC$ en rempla\c{c}ant $\rmi$ par $-\rmi$. Comme
$\hat{f}$ et $\hat{g} \in \lone(\rset)$, on applique \eqref{item:convolution-L1-i} \`{a} $\hat{f}$ et $\hat{g}$ ce qui montre
\[
\TFAC{\hat{f} * \hat{g}}(x)= \TFAC{\hat{f}}(x) \cdot \TFAC{\hat{g}} (x)= f(x) \cdot g(x) \eqsp, \pp
\]
\end{enumerate}
\end{proof}
\begin{corollary}
Soient $f,g \in \lone(\rset)$.
\begin{enumerate}[label=(\roman*)]
\item $\widehat{f * g}= \hat{f} \cdot \hat{g}$.
\item $\widehat{f \cdot g}= \hat{f} * \hat{g}$.
\end{enumerate}
\end{corollary}
\section{Applications aux filtres analogiques gouvern\'es par une \'equation diff\'erentielle}
La transform\'ee de Fourier permet d'\'etudier les filtres d\'efinis par une \'equation diff\'erentielle ordinaire \`{a} coefficients constants.
\begin{equation}
\label{eq:entre-sortie}
\sum_{k=0}^{q}b_{k}g^{(k)}=\sum_{j=0}^{p}a_{j}f^{(j)},\ a_{p}\cdot b_{q}\neq 0 \eqsp,
\end{equation}
o\`{u} $f$ est l'entr\'ee et $g=A(f)$ est la sortie. Nous allons tout d'abord supposer que $f \in \mcs$
C'est un cas tr\`es particulier car il n'y a pas de raisons que l'entr\'ee soit aussi r\'eguli\`ere. Nous verrons dans la suite que l'\'etude de ce cas permet de consid\'erer ensuite des situations plus g\'en\'erales.

Nous supposons que $f\in \mcs$ et nous recherchons des solutions $g \in \mcs$.
Si une telle solution $g$ existe, nus pouvons calculer la transform\'ee de Fourier des deux membres de \eqref{eq:entre-sortie}:
\begin{equation}
\label{eq:entre-sortie-Fourier}
\sum_{k=0}^{q}b_{k}(2\rmi\pi\lambda)^{k}\hat{g}(\lambda)=\sum_{j=0}^{p}a_{j}(2\rmi \pi\lambda)^{j}\hat{f}(\lambda) \eqsp.
\end{equation}
Consid\'erons les deux polyn\^omes: $P(X)=\displaystyle \sum_{j=0}^{p}a_{j}X^{j}$ and $Q(X)=\displaystyle \sum_{k=0}^{q}b_{k}X^{k}$
et supposons que la fraction rationnelle $P(z)/Q(z)$ ne poss\`{e}de  pas de p\^oles sur l'axe imaginaire.
Alors $P(2 \rmi \pi\lambda)/Q(2 \rmi \pi\lambda)$ est d\'efinie pour tout $\lambda \in \rset$,et \eqref{eq:entre-sortie-Fourier}
est \'equivalent \`{a}
\begin{equation}
\label{eq:soln-entre-sortie}
\hat{g}(\lambda)=\frac{P(2\rmi\pi\lambda)}{Q(2\rmi\pi\lambda)}\hat{f}(\lambda)= H(\lambda) \hat{f}(\lambda).
\end{equation}
Cette identit\'e d\'etermine $g \in \mcs$ compl\`etement, si cette solution existe, et prouve l'unicit\'e de la solution de \eqref{eq:entre-sortie} dans $\mcs$. L'existence d'une solution d\'ecoule aussi de \eqref{eq:soln-entre-sortie}, comme la fonction
$$
G(\lambda)=\frac{P(2 \rmi \pi\lambda)}{Q(2 \rmi \pi\lambda)}\hat{f}(\lambda)
$$
est un \'el\'ement de $\mcs$ d\`es que  $f \in \mcs$.
En appliquant \Cref{thm:Schwarz}, nous avons $g= \TFAC{G}$ est l'unique solution de \eqref{eq:entre-sortie} dans $\mcs$.

\begin{proposition}
\label{prop:unicite-mcs}
 Si $P(x)/Q(x)$ n'a pas de p\^oles sur l'axe imaginaire et si $f \in \mcs$, alors \eqref{eq:entre-sortie} poss\`ede une unique solution $g\in \mcs$.
\end{proposition}
Pour $f \in \mcs$, appelons $g= A(f)$ l'unique solution de \eqref{eq:entre-sortie} dans $\mcs$. Soient $f_1,f_2 \in \mcs$ et $\alpha_1,\alpha_2 \in \cset$. Posons $f= \alpha_1 f_1+ \alpha_2 f_2$. On a clairement $f \in \mcs$ et par lin\'{e}arit\'{e} de la transform\'{e}e de Fourier $\hat{f}= \alpha_1 \hat{f}_1 + \alpha_2 \hat{f}_2$. Comme
$$
g= A(f)= \TFAC{ H \hat{f}} = \alpha_1 \TFAC{ H \hat{f}_1} + \alpha_2 \TFAC{ H \hat{f}_2} = \alpha_1 A(f_1) + \alpha_2 A(f_2)
$$
l'application $A: \mcs \to \mcs$ est un lin\'{e}aire (c'est un endomorphisme de $\mcs$). En traitement du signal, c'est ce que l'on appelle le principe de \emph{superposition}, la r\'{e}ponse du syst\`{e}me \`{a} une combinaison lin\'{e}aire des entr\'{e}es est la combinaison lin\'{e}aire (avec les m\^{e}mes poids) des sorties.

Pour $\tau \in \rset$, appelons $L_\tau: \mcs \to \mcs$ l'op\'{e}rateur de retard: $f_\tau = L_\tau(f)$. Il est imm\'{e}diat de voir que pour tout $\tau \in \rset$, $L_\tau$ est un op\'{e}rateur lin\'{e}aire de $\mcs \to \mcs$. Pour $f \in \mcs$, notons $f_\tau = L_\tau f$. Etudions maintenant l'image par $A$ de $f_\tau$. En utilisant le formulaire \Cref{sec:formulaire}
\begin{align*}
A( f_\tau)
&= \TFAC{H \widehat{f_\tau}} = \TFAC( \xi \mapsto H(\xi) \rme^{-2 \rmi \pi \tau \xi} \hat{f}(\xi)) \\
&= L_\tau( A f) \eqsp.
\end{align*}
Par cons\'{e}quent, pour tout $f \in \mcs$, nous avons $A \circ L_\tau (f)= L_\tau \circ A (f)$.
\begin{definition}
\label{def:lineaire-invariance}
Une application $A: \mcs \to \mcs$ est lin\'{e}aire et invariante si
\begin{enumerate}[label=(\roman*)]
\item pour tout $f_1,f_2 \in \mcs$ et $\alpha_1,\alpha_2 \in \cset$, $A(\alpha_1 f_1 + \alpha_2 f_2)= \alpha_1 A(f_1) + \alpha_2 A(f_2) $.
\item pour tout $\tau \in \rset$, $L_\tau \circ A= A \circ L_\tau$.
\end{enumerate}
\end{definition} 