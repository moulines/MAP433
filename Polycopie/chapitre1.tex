\chapter{Introduction}

Initialement appliqu\'e aux t\'el\'ecommunications, le traitement
du signal se retrouve \`a pr\'esent
dans tous les domaines n\'ecessitant d'analyser et transformer
de l'information num\'erique.
La manipulation de donn\'ees obtenues par capteurs
bio-m\'edicaux, lors d'exp\'eriences physiques ou biologiques,
sont aussi des probl\`emes de traitement du signal.
Le t\'el\'ephone, la radio et la t\'el\'evision ont motiv\'e l'\'elaboration
d'algorithmes de filtrage lin\'eaires permettant de coder
des sons ou des images, de les transmettre, et de supprimer
certains bruits de transmission.


Lorsque la rapidit\'e de calcul le permet,
le filtrage par circuits d'\'electronique
analogique est remplac\'e par des calculs num\'eriques
sur des signaux digitaux. Le calcul digital est en effet plus
fiable et offre une flexibilit\'e algorithmique bien plus
importante.
La conversion analogique-digitale est \'etudi\'ee
dans le chapitre \ref{discret-chap}
ainsi que l'extension des op\'erateurs de filtrage
aux signaux discrets. L'introduction de
la transform\'ee de Fourier discr\`ete rapide
par Cooley et Tuckey en 1965
a fait de l'analyse de Fourier un outil algorithmique
puissant qui se retrouve dans la
plupart des calculs rapides de traitement du signal.

Lorsque l'on veut d\'ecrire les propri\'et\'es d'une classe de signaux,
comme un m\^eme son prononc\'e par diff\'erents locuteurs,
il est utile de se placer dans un cadre probabiliste.
La vari\'et\'e des signaux d'une telle classe peut en effet
\^etre caract\'eris\'ee par un processus al\'eatoire.
La mod\'elisation de signaux par proc\'essus stationnaires est
introduite dans le
chapitre \ref{aleatoire-chap}
ou nous \'etudions plus particuli\`erement
les mod\`eles autor\'egressifs.
L'estimation lin\'eaire, la suppression de bruit
et la pr\'ediction sont \'etudi\'ees dans le chapitre
\ref{wiener-chap} \`a travers le filtrage de Wiener.


La notion d'information contenue dans un signal peut se formaliser
par la th\'eorie de Shannon qui la relie au nombre de bits minimum
pour coder le signal. Le chapitre \ref{comp-code-chap}
en d\'eduit des algorithmes de compression
qui suppriment la redondance interne
d'un signal et le repr\'esentent avec un nombre de bits r\'eduit.
De tels algorithmes augmentent consid\'erablement les capacit\'es de
stockage, et permettent de transmettre
des signaux \`a travers des canaux \`a d\'ebits r\'eduits.
Ainsi, le nouveau
standard de compression pour la t\'el\'evision haute d\'efinition peut
diffuser une image de bien plus haute r\'esolution spatiale
et temporelle, par les m\^emes canaux de transmission
que la t\'el\'evision actuelle.

Cependant, la th\'eorie de Shannon ne permet pas d'extraire
l'information ``utile'' d'un signal.
La reconnaissance de la parole a
motiv\'e un grand nombre de travaux
sur ce sujet.
La performance des syst\`emes de reconnaissance de parole
a progress\'e beaucoup plus lentement que les
projections optimistes des ann\'ees 50.
Les algorithmes de traitement doivent
s'adapter au contenu tr\`es complexe du
signal, et sont donc beaucoup plus sophistiqu\'es que des filtrages
lin\'eaires homog\`enes.
Le chapitre \ref{parole-chap} \'etudie
l'application des mod\`eles autor\'egressifs, et
l'extraction d'information
locale en temps et en fr\'equence,
gr\^ace \`a une transform\'ee de Fourier \`a fen\^etre.


