\chapter{Traitement de la Parole}
\label{parole-chap}

Le traitement de la parole n\'ecessite une analyse
physiologique des m\'ecanismes
de production, ce qui permet de comprendre les propri\'et\'es
particuli\`eres de ce type de signal.
Cela motive notament l'utilisation de mod\`eles autor\'egressifs.
L'identification des param\`etres de ces mod\`eles se fait
par r\'egression lin\'eaire. On peut ainsi
effectuer un codage compacte des signaux de parole pour
la t\'el\'ephonie cellulaire.


\section{Mod\'elisation du signal de parole}
\label{modelisation-paro}

\subsection{Production}


La production de la parole se fait en trois \'etapes.
Les poumons compressent de l'air qui est envoy\'e
\`a travers la trach\'ee. Cet air passe par le larynx qui est
compos\'e d'un syst\`eme de cartilages et de muscles incluant les
cordes vocales. Le larynx produit alors un signal d'excitation qui
se propage \`a travers le conduit vocal. C'est la
d\'eformation du conduit vocal qui produit l'articulation de la
parole. Les \'el\'ements principaux de cette articulation sont la
langue, les l\`evres et la machoire inf\'erieure. La figure
\ref{parole-physio} illustre l'appareil phonatoire.
\begin{figure}
%(voire Genat/Karar p. 10)
\vspace{7cm}
\caption{Syst\`eme phonatoire \protect \cite{karar}.}
\label{parole-physio}
\end{figure}
\\
\\
{\bf Excitation}
Le larynx peut produire des signaux d'excitation diff\'erents.
Les sons vois\'es tels que les voyelles sont produits par
vibration des cordes vocales. L'air est forc\'e \` a travers les
cordes vocales qui vibrent comme les l\`evres d'un trompettiste. Cela
produit un train de quasi-impulsions, illustr\'e en
figure \ref{quasi-impulsions},
qui est envoy\'e dans le conduit vocal.
La fr\'equence des r\'ep\'etitions, appel\'ee
``pitch'', est essentiellement contr\^ol\'ee par la tension des cordes
vocales. Elle correspond \`a la fr\'equence fondamentale (hauteur) du son.
Dans le cas de la voix parl\'ee, le pitch est typiquement
entre 100Hz et 300Hz. Une soprano peut cependant augmenter cette
fr\'equence jusqu'\`a 3600Hz.

\begin{figure}
%(voire fig. 3-6 p. 65, Parson)
\vspace{2cm}
\caption{Train de quasi-impulsions \'emises par les cordes
vocales}.
\label{quasi-impulsions}
\end{figure}

Pour un chuchotement, les cordes vocales ne vibrent pas
mais laissent un passage
\'etroit entre les cartilage du larynx, qui envoie un
air turbulent dans le conduit vocal.
Cet air turbulent peut \^etre mod\'elis\'e par un bruit Gaussien, dont
la puissance spectrale a un large support fr\'equentiel.
\\
\\
{\bf Articulation}
Le conduit vocal donne l'articulation au son qui caract\'erise chaque
phon\`eme. Nous avons vu que
pour un son vois\'e, le larynx \'emet un train d'onde
riche en harmoniques qui est filtr\'e par le conduit
vocal. Ce conduit vocal a des r\'esonances appel\'ees ``formants''.
La d\'eformation du conduit vocal d\'eplace les fr\'equences de
r\'esonance, ce qui permet de former
toutes les voyelles et certaines consonnes.
Pour des sons non-vois\'es, le conduit vocal peut aussi effectuer
des constrictions qui produit des fricatives ou des sons
chuint\'es telles que le
[s] et le [ch].
Les sons plosifs sont produits par une fermeture du conduit
vocal ce qui cr\'ee une occlusion. Le rel\^achement brutal de
cette occlusion produit alors une plosive telle que [p] ou [t].
L'articulation du son est aussi affect\'ee par
l'ouverture de la cavit\'e nasale.
Des cat\'egorisations tr\`es d\'etaill\'ees des
diff\'erents sons de parole ont \'et\'e faites
par les phon\'eticiens.


\subsection{Conduit vocal}
\label{conduit-sec}
Le conduit vocal peut se mod\'eliser comme la juxtaposition
de plusieurs cylindres de m\^eme longueur \'egale \`a
$\Delta$ mais de diam\`etres variables,
comme l'illustre la figure \ref{conduit-voc}.
Chaque cylindre est un syst\`eme lin\'eaire avec en entr\'ee
une onde directe $f_{n-1} (t)$ mesurant le d\'ebit du flot d'air
qui passe \`a l'entr\'ee du cylindre par unit\'e de temps,
et une onde inverse
$b_{n-1} (t)$. En sortie, on a onde directe $f_n (t)$
et une onde inverse $b_n (t)$ qui est reli\'ee \`a l'entr\'ee
par une matrice $T_n$ qui d\'epend de l'imp\'edance acoustique
des cylindres $n-1$ et $n$
\[
\left(
\begin{array} {l}
f_n (t)\\
b_n (t)
\end{array}
\right) = T_n
\left(
\begin{array} {l}
f_{n-1} (t)\\
b_{n-1} (t)
\end{array}
\right)
\]
Si l'on cascade la r\'eponse de chaque cylindre, on obtient
\[
\left(
\begin{array} {l}
f_p (t)\\
b_p (t)
\end{array}
\right) = T
\left(
\begin{array} {l}
f_{0} (t)\\
b_{0} (t)
\end{array}
\right)
\]
avec
\[
T = \prod_{n=1}^{p} T_n .
\]


\begin{figure}
% (voire p. 109 Parson)
\vspace{3.5cm}
\caption{Le conduit vocal peut se mod\'eliser comme une
succession de cylindres de m\^eme longueur $\Delta$
ayant des diam\`etres
diff\'erents.}
\label{conduit-voc}
\end{figure}

On discr\'etise ce syst\`eme avec un pas d'\'echantillonnage de
$\frac \Delta c$ qui est le temps de propagation de l'onde
acoustique dans chaque
cylindre. Le signal d'entr\'ee est le signal discret
\[
f_0 [n] = f_0 (\frac {n \Delta} c ) - b_0 (\frac {n \Delta} c ) ,
\]
tandis que le signal de sortie est
\[
f_p [n] = f_p (\frac {n \Delta} c ) - b_p (\frac {n \Delta} c ) .
\]
En \'ecrivant les \'equations de propagation des ondes et
les conservations de d\'ebit et de pression au travers des
jonctions des cylindres, on peut calculer la fonction de
transfert qui relie la transform\'ee en z de $f_0 [n]$ et de
$f_p [n]$ \`a partir des diam\`etres de chacun des cylindres
\[
\frac {\hat f_p (z)} {\hat f_0 (z)} = \hat h(z) .
\]
En l'absence de perte le long du syst\`eme, on peut montrer que
$\hat h(z)$ a $p$ p\^oles mais n'a pas de z\'eros.
C'est donc un filtre autor\'egressif qui peut s'\'ecrire
\[
\hat h (z) = \frac 1 {a[0] + a[1] z^{-1}+ ... + a[p] z^{-p}} .
\]
Cette condition reste valable tant que le conduit vocal peut
\^etre repr\'esent\'e par un seul tube sans embranchement. On n\'eglige
donc l'influence introduite par le conduit nasal ainsi que
les pertes d'\'energie dues aux vibrations
des parois du conduit vocal et aux frictions.
\\
\\
{\bf Formants}
Ce mod\`ele simplifi\'e montre que le conduit vocal peut \^etre repr\'esent\'e
par un filtre autor\'egressif dont les param\`etres $a[k]$ d\'ependent
de la configuration du conduit vocal.
Ce filtre \' etant causal et stable, nous savons que les p\^oles de
$\hat h (z)$ sont tous de module plus petit que 1.
Les p\^oles de $\hat h(z)$ qui sont proches du cercle unit\'e
produisent un pic dans le module de la r\'eponse fr\'equentielle
$|\hat h (\eom)|$.
Ces pics de fr\'equence sont appel\'es formants. Ils ont une
importance particuli\`ere pour la reconnaissance des sons produits.
Plus le z\'ero est proche du cercle unit\'e, plus le formant est
prononc\'e. Les formants de plus grande amplitude
sont g\'en\'eralement les
2 premiers, apparaissant aux plus basses fr\'equences.
La figure \ref{autoreg}
donne un exemple de filtre autor\'egressif ayant 8 p\^oles dont
la position est indiqu\'ee dans
le plan complexe. La r\'eponse fr\'equentielle
$|\hat h (\eom)|_\db$ est donn\'ee \`a droite.

\begin{figure}
% voire p.23-25 genat/karar
\vspace{8cm}
\caption{La figure de gauche donne la position des 8 p\^oles
d'un filtre autor\'egressif tandis que la figure de droite
donne le module $|\hat h (\eom)|_\db$.}
\label{autoreg}
\end{figure}


\subsection {Excitation}

{\bf Sons vois\'es}
En mode vibratoire, les cordes vocales \'emettent un train
d'onde qui peut s'\'ecrire
\[
f(t) = \sum_{n=-\infty}^{+\infty} g(t - n T) =
g(t) \star \sum_{n=-\infty}^{+\infty} \delta (t - n T) .
\]
o\`u $g(t)$ est une onde \'el\'ementaire dont le support est
petit devant $T$ (voire figure \ref{quasi-impulsions}).
On peut supposer que cette
onde est ind\'ependante de la forme du conduit vocal.
En appliquant la formule de Poisson (\ref{poisson}), on calcule
la transform\'ee de Fourier de $f(t)$
\[
\hat f (\om) = \hat g(\om)
\frac {2 \pi} T \sum_{n=-\infty}^{+\infty}
\delta (\om - \frac {2 n \pi} T ) =
\frac {2 \pi} T \sum_{n=-\infty}^{+\infty}
\hat g(\frac {2 n \pi} T ) \delta (\om - \frac {2 n \pi} T ) .
\]
C'est une succession d'harmoniques dont l'enveloppe est \'egale
a $\hat g (\om)$.

On a vu que le conduit vocal est \'equivalent \`a un filtre lin\'eaire
de fonction de transfert $\hat h (\om)$. La transform\'ee
de Fourier du son \'emis est donc
\[
\hat f(\om) \hat h (\om ) =
\hat h (\om ) \hat g(\om)
\frac {2 \pi} T \sum_{n=-\infty}^{+\infty}\delta (\om - \frac {2 n \pi} T ) .
\]
Cette r\'eponse peut \^etre mod\'elis\'ee comme un train d'impulsions
de Diracs qui passe \`a travers un filtre dont la fonction
de transfert est sp\'ecifi\'ee par l'enveloppe totale
$\hat h (\om) \hat g (\om)$.

On sait que la discr\'etisation de l'onde se propageant dans le
conduit vocal peut se mod\'eliser par un filtrage autor\'egressif.
La discr\'etisation de $g(t)$
peut de m\^eme \^etre approxim\'ee par un filtre autor\'egressif.
Un signal de parole vois\'e discr\'etis\'e se mod\'elise donc par
un train d'impulsion de Diracs discrets
$\sum_{k=-\infty}^{+\infty} \delta [n-kT]$ filtr\'e par
un filtre autor\'egressif qui d\'epend \`a la fois du conduit
vocal et de la forme de l'impulsion $g(t)$ produite par
les cordes vocales.
\\
\\
{\bf Sons non vois\'es}
Les sons non vois\'es sont produits par un signal turbulent
\'emis par le larynx qui est ensuite modifi\'e par le conduit vocal.
La discr\'etisation de ce signal turbulent
peut \^etre mod\'elis\'ee par un processus Gaussien
stationnaire $Y[n]$ dont la puissance spectrale
$\hat R_{Y} (\eom)$
\`a une \'energie qui est r\'epartie sur
une large bande de fr\'equence. Un tel processus peut
aussi s'ecrire comme
un bruit blanc Gaussien $W[n]$ filtr\'e par un filtre $g[n]$
\[
Y[n] = W \star g [n].
\]
Le th\'eor\`eme \ref{cov_conv_th}  prouve que la puissance spectrale
de $Y[n]$ est reli\'ee a $\hat g (\eom)$ par
\[
\hat R_{Y} (\eom) = |\hat g (\eom)|^2 .
\]

Nous avons vu que le conduit vocal se comporte
comme un filtre AR de r\'eponse impulsionelle $h [n]$.
Le son produit est alors mod\'elis\'e par le processus
\[
Y \star h [n] = W \star g \star h [n] .
\]
Un tel signal discr\'etis\'e se mod\'elise donc par un bruit blanc discret
$W[n]$ filtr\'e par $g \star h [n]$ dont la fonction de transfert
est $\hat g (z) \hat h (z)$.
En g\'en\'eral, $\hat g (z)$
peut avoir des z\'eros. Ces z\'eros sont neglig\'es car leur importance
perceptuelle est secondaire \`a c\^ote des p\^oles. On mod\'elise donc
$\hat g (z) \hat h (z)$ par un seul filtre autor\'egressif.
\\
\\
{\bf Mod\`ele synth\'etique}
Suivant que le son est vois\'e ou non, le signal de parole peut
se mod\'eliser comme un train d'impulsion ou
comme la r\'ealisation d'un bruit
blanc filtr\'e par un filtre autor\'egressif
dont les caract\'eristiques d\'ependent
du son prononc\'e. Dans le cas d'un son vois\'e, la p\'eriode des
impulsions (pitch) est un param\`etre qui doit \^etre d\'etermin\'e.
Ce mod\`ele est illustr\'e par la
figure \ref{modele-parole}.


\begin{figure}[bhtp]
%(p.55 Genat/Kara)
\centerline{
	\leavevmode\epsfbox{/home/mallat/X/TREX/figures/TrexFig/MALLATFIG7.7-eps}}
\caption{Mod\'elisation d'un son de parole par une excitation
p\'eriodique ou al\'eatoire, filtr\'ee par un filtre autor\'egressif.}
\label{modele-parole}
\end{figure}
\\
\\
\noindent
{\bf Stationnarit\'e}
Le conduit vocal ne se comporte comme un filtre lin\'eaire
homog\`ene que sur des intervalles de temps relativement petits.
Sur une dur\'ee plus longue,
un signal de parole est non stationnaire
puisque les sons changent au cours du temps. La taille des  intervalles
sur lesquels le son peut \^etre approxim\'e par une excitation
modul\'ee par un filtre homog\`ene d\'epend de la nature du son.
Pour une voyelle, cette approximation est valable sur environ
$10^{-2}$ seconde alors que beaucoup de sons de
consonnes telles que les plosives
ne restent pas stationnaires sur cette dur\'ee.
La variation de ces intervalles de stationnarit\'e est l'une des
difficult\'es du traitement de la parole.

\section{Estimation d'un mod\`ele de parole}
\label{est-par-sec}

Nous avons expliqu\'e qu'un signal de parole peut localement
\^etre approxim\'e par une excitation $e[n]$
filtr\'ee par un filtre AR dont
les param\`etres d\'ependent du son prononc\'e. Pour des applications
de reconnaissance et de codage, on veut identifier l'excitation
ainsi que les param\`etres du filtre.
On isole une portion d'un signal de parole $f[n]$
en le multipliant avec une fen\^etre $w[n]$ de taille $P$
centr\'ee en un instant $pP$
\[
x[n] = f[n] w[n-pP] ,
\]
comme le montre la figure \ref{fenetrage}.
On prend $P$
suffisamment petit pour que le son isol\'e puisse \^etre consid\'er\'e
comme stationnaire. Par exemple, la fen\^etre de
Hamming est d\'efinie par
\[
w[n] =
\left\{
\begin{array}{ll}
0.54 + 0.46 \cos (\frac {2\pi n} P ) &
\mbox{si $|n| < \frac P 2$}\\
0 & \mbox{sinon}
\end{array}
\right.
\]
Le signal r\'esultant $x[n]$ poss\`ede
au plus $P$ coefficients non-nuls.
Dans un mod\`ele de parole,
$x[n]$ est produit par une excitation
$e[n]$ filtr\'ee par un filtre AR dont la fonction de transfert
est renormalis\'ee pour s'\'ecrire
\[
\hat h (z) = \frac 1 {1 - a[1] z^{-1} - ... - a[N] z^{-N}} .
\]

\begin{figure}
% (p. 55 Genat/Karar)
\vspace{4cm}
\caption{Des portions du signal de parole sont isol\'ees par
des fen\^etres, qui couvrent des intervalles de temps o\`u
le signal peut \^etre consid\'er\'e comme stationnaire.}
\label{fenetrage}
\end{figure}
\\
\\
\noindent
{\bf Calcul de l'excitation} De nombreuses techniques
ont \'et\'e d\'evelopp\'ees
pour d\'eterminer le voisement et le pitch d'un son.
Une approche particuli\`erement simple est bas\'ee sur
la somme des diff\'erences du signal \`a intervalles $k$ variables
\begin{equation}
\label{AMDF}
D[k] = \frac 1 P \sum_{n} |x[n] - x[n-k]| .
\end{equation}
Si le signal est vois\'e et que la fr\'equence des impulsions est
$T$ alors $D[k]$ a un minimum a $k = T$. Lorsque le minimum
de $D[k]$ n'est pas suffisamment bas,
on en d\'eduit que le son est non vois\'e.
Cet algorithme a l'avantage de ne n\'ecessiter aucune multiplication
et donc de s'impl\'ementer tr\`es rapidement.


\subsection{R\'egression lin\'eaire}

Pour identifier tous les param\`etres d'un son, il nous faut
estimer les param\`etres $a[k]$ du filtre AR qui sp\'ecifient
les propri\'et\'es du conduit vocal.
Le signal $x[n]$ satisfait l'\'equation r\'ecurrente
\[
x[n] - \sum_{k=1}^N a[k] x[n-k] = e[n] .
\]
On peut interpr\'eter
\begin{equation}
\label{estim-dex}
\tilde x [n] =  \sum_{k=1}^N a[k] x[n-k]
\end{equation}
comme une estimation de $x[n]$ \`a partir de $N$ valeurs pass\'ees,
auquel cas l'erreur d'estimation n'est autre que l'excitation
\[
x [n] - \tilde x[n] = e[n] .
\]
Si $e[n]$ est la r\'ealisation d'un bruit blanc, donc non corr\'el\'ee
d'un \'echantillon \`a l'autre, chaque $e[n]$ peut \^etre consid\'er\'e
comme l'innovation apport\'ee par l'excitation relativement \`a la
pr\'ediction de $x[n]$ par son pass\'e. Ce point de vue permet de poser
l'identification des param\`etres  $a[k]$
du filtre AR comme un probl\`eme
de pr\'ediction lin\'eaire.
Etant donn\'e un signal $x[n]$, on veut calculer les
coefficients de r\'egression tels que l'estimation (\ref{estim-dex})
g\'en\`ere une erreur
\begin{equation}
\label{err-prod}
\sum_{n=-\infty}^{+\infty} |x[n] - \tilde x[n]|^2 =
\sum_{n=-\infty}^{+\infty} |e[n]|^2
\end{equation}
qui est minimum.
Pour effectuer ce calcul on introduit
l'autocorr\'elation empirique du signal $x[n]$
\begin{equation}
\label{autocro-x}
r_x[k] = \sum_{n=-\infty}^{+\infty}  x[n-k] x[n] .
\end{equation}
Cette somme s'\'etend seulement sur $P$ valeurs car $x[n]$ a au plus
$P$ valeurs cons\'ecutives non nulles.
Le th\'eor\`eme suivant caract\'erise les coefficients de r\'egression
de $\tilde x[n]$.

\begin{theorem} [Pr\'ediction lin\'eaire]
Les coefficients de r\'egression $\{a[k]\}_\UkN$ du
signal $\tilde x [n] =  \sum_{k=1}^N a[k] x[n-k]$ qui
minimise
\[
\epsilon_N = \sum_{n=-\infty}^{+\infty} |x[n] - \tilde x[n]|^2
\]
sont solutions du syst\`eme
\begin{equation}
\label{yule-walker2}
\left( \begin{array}{cccc}
r_x[0] &r_x[1] & ...&r_x[N] \\
r_x[-1]&r_x[0]& ...&r_x[N-1]\\
. &. &...&.  \\
.&.&...&.\\
.&.&...&.\\
r_x[-N]&r_x[-N+1]&...&r_x[0]\\
\end{array}
\right)
\left( \begin{array}{c}
a[1]\\
a[2]\\
.\\
.\\
.\\
a[N]\\
\end{array}
\right)
=
\left( \begin{array}{c}
r_x[1]\\
r_x[2]\\
.\\
.\\
.\\
r_x[N]\\
\end{array}
\right) .
\end{equation}
L'erreur r\'esultante est
\begin{equation}
\label{erro-regre}
\epsilon_N = r_x [0] - \sum_{p=1}^N a[p] r_x[p] .
\end{equation}
\end{theorem}

{\bf D\'emonstration} La d\'emonstration se fait par une
interpr\'etation g\'eom\'etrique du probl\`eme de minimisation
Le signal $x[n]$ a une \'energie finie et donc appartient
\`a l'espace $\lD$ muni de la
norme
\[
\| x[n] \|^2 = \sum_{n=-\infty}^{+\infty} |x[n]|^2
\]
et du produit scalaire
\[
<x[n],y[n]> = \sum_{n=-\infty}^{+\infty} x[n] y[n] .
\]
Le vecteur $\tilde x[n]$ est une combinaison lin\'eaire des
vecteurs $\{x_k [n] = x[n-k]\}_{1 \leq k \leq N}$
et appartient donc \`a l'espace $\V$
g\'en\'er\'e par ces $N$ vecteurs.
Minimiser l'erreur de pr\'ediction (\ref{err-prod}) revient
donc \`a trouver un vecteur $\tilde x[n]$ de $\V$ qui minimise
$\|x - \tilde x \|^2$.
Le th\'eor\`eme de projection prouve que ce vecteur
est la projection orthogonale de $\tilde x$ dans $\V$.
C'est donc un vecteur tel que $x - \tilde x$ est orthogonal \`a
tous les vecteurs de $\V$ et en particulier aux vecteurs
$\{ x[n-k] \}_\UkN$
\[
<x[n] - \tilde x [n] , x[n-k]> =
\sum_{n=-\infty}^{+\infty} (x[n] - \tilde x [n])  x[n-k] =  0 .
\]
En ins\'erant (\ref{estim-dex})
ces \'equations se r\'e\'ecrivent
\begin{equation}
\label{proj-equai}
\sum_{n=-\infty}^{+\infty} x[n] x[n-k] -
\sum_{1=0}^N a[p] \sum_{n=-\infty}^{+\infty} x[n-p] x[n-k] = 0 .
\end{equation}
En ins\'erant (\ref{autocro-x}) on obtient
\[
\sum_{p=1}^N a[p] r_x[k-p] = r_x [k]~~,~~\mbox{pour $1 \leq k \leq N$} ,
\]
qui correspond au syst\`eme de $N$ \'equations
\`a $N$ inconnues (\ref{yule-walker2}).

Comme $x[n] - \tilde x[n]$ est orthogonal
\`a tout vecteur dans
$\V$ et donc \`a $\tilde x[n]$, l'\'energie de l'erreur
peut s'\'ecrire
\[
\epsilon_N =  <x[n] - \tilde x[n] , x[n] - \tilde x[n]> =
<x[n] - \tilde x[n] , x[n]>.
\]
En rempla\c{c}ant $\tilde x[n]$
par son expression (\ref{estim-dex}), on obtient
\[
\epsilon_N = \sum_{n=-\infty}^{+\infty}  x[n] x[n] -
\sum_{p=1}^N a[p] \sum_{n=-\infty}^{+\infty}  x[n-p] x[n]
\]
et donc (\ref{erro-regre})
$\Box$
\\
\\
{\bf Filtre AR pour sons non-vois\'es}
Un son non-vois\'e est modelis\'e par bruit blanc traversant
un filtre AR.
Le paragraphe \ref{proc-autoregre-se} montre qu'un
filtre AR excit\'e par un bruit blanc $W[n]$ produit
un processus autor\'egressif $X[n]$ dont l'autocorr\'elation
$R_X [k]$ satisfait les \'equations de
Yule-Walker (\ref{yule-walker}).
En comparant le syst\`eme de Yule-Walker
(\ref{yule-walker}) et le syst\`eme (\ref{yule-walker2}),
on s'aper\c{c}oit que ces \'equations sont identiques lorsque l'on
remplace l'autocorr\'elation $R_X [k]$ du processus
$X[n]$ par l'autocorr\'elation empirique
$r_x [k]$ du signal $x[n]$.
Si $e[n]$ est une r\'ealisation du bruit blanc $W[n]$ alors $x[n]$
est une r\'ealisation du processus autor\'egressif $X[n]$.
L'autocorr\'elation empirique
$r_x [k]$ peut donc s'interpr\'eter
comme une estimation de la v\'eritable
autocorr\'elation $R_X [k]$ du processus.
Les \'equations de pr\'edictions lin\'eaires
(\ref{yule-walker2}) sont donc une
approximation des \'equations de Yule-Walker (\ref{yule-walker}),
qui permettent d'estimer les coefficients $a[k]$ du filtre AR.
La r\'esolution du syst\'eme (\ref{yule-walker2}) peut
se faire en utilisant l'algorithme rapide
de Levinson-Durbin qui n\'ecessite $O(N^2)$ op\'erations.
\\
\\
{\bf Filtre AR pour son vois\'e}
Lorsque le son est vois\'e, l'algorithme de r\'egression lin\'eaire
donne aussi une bonne estimation du filtre AR. La justification
est cependant plus compliqu\'ee et nous n'en donnons qu'une explication
superficielle.
Le signal de parole $f[n]$ est construit en filtrant
un train d'impulsion
\[
e[n] = \sum_{k=-\infty}^{+\infty} \delta [n -k T]
\]
avec un filtre AR $h[n]$ et $x[n]$ est obtenu
en multipliant $f[n]$ par la fen\^etre $w[n - p P]$
\[
x[n] = w[n-pP] \times  (e \star h)[n] .
\]
On peut en d\'eduire que le spectre $\hat x (\eom)$ est la
somme de composantes harmoniques situ\'ees autour des fr\'equences
$\om = \frac {2 k \pi} T$ dont l'amplitude est proportionnelle
a $\hat h (e ^{\frac {2 k \pi} T})$.
Pour v\'erifier que le filtre obtenu par r\'egression lin\'eaire
est proche du filtre $h[n]$ on montre que l'optimisation des
coefficients $a[k]$ de la r\'egression lin\'eaire calcule un filtre
autor\'egressif qui interpole approximativement
les valeurs $\hat h (e ^{\frac {2 k \pi} T})$.
Comme le filtre $\hat h (\eom)$ est lui-m\^eme autor\'egressif,
on en d\'eduit que
la r\'egression lin\'eaire optimale calcule une approximation
de ce filtre.
%
%La transform\'ee de Fourier de $e \star h[n]$ est
%\[\
%hat e (\eom) \hat h(\eom)
%= \sum_{k=-\infty}^{+\infty} H(e^{i \frac{2 k \pi} T})
%\delta (\om - \frac{2 k \pi} T) ,
%\]
%d'o\`u
%\begin{eqnarray*}
%\hat x(\eom) &=& \hat e (\eom) \hat h(\eom) \star \hat w (\eom)
%e^{-i pT \om} \\
%&=&
%\frac 1 {2 \pi} e^{-i pT \om}
%\sum_{k=-\infty}^{+\infty} \hat h(e^{i \frac{2 k \pi} T})
%\hat w (e^I{\om - \frac{2 k \pi} T}) .
%\end{eqnarray*}
%C'est un train d'ondes \'el\'ementaires qui ressemblent \`a des
%impulsions, qui sont les transform\'ees de Fourier translat\'ees de
%la fen\^etre $w[n]$. Comme ces impulsions ont des supports qui ne
%se recouvrent pratiquement pas,
%\[
%|X(\eom)|^2 \approx
%\frac 1 {4 \pi^2}
%\sum_{k=-\infty}^{+\infty} |H(e^{i \frac{2 k \pi} T}) |^2
%|W(e^{i(\om - \frac{2 k \pi} T)})|^2 .
%\]
%La valeur de $|X(\eom)|^2$ est non n\'egligeable qu'aux voisinage
%des frequences $\frac{2 k \pi} T$ et l'on a
%\[
%|X(e^{i\frac{2 k \pi} T})|^2 \approx |W(e^0)|^2
%|H(e^{i \frac{2 k \pi} T}) |^2 .
%\]
%La fonction de transfert du filtre AR de
%coefficients $a[k]$ est
%\[
%H(\eom) = \frac 1 {\sum_{k=0}^P a[k] e^{-i k\om}} .
%\]
%Puisque $x [n] = h \star \epsilon [n]$, leur transform\'ee de Fourier
%satisfait
%\[
%E(\eom) = \frac 1 {H(\eom)} X(\eom) .
%\]
%En appliquant la formule de Parseval (?) on peut calculer
%l'\'energie de cette excitation
%\begin{eqnarray*}
%E &=& \|e[n]\|^2 = \frac 1 {2 \pi} \int_{-\pi}^{\pi}
%|E(\eom)|^2 d\om \\
%&=& \frac 1 {2 \pi} \int_{-\pi}^{\pi} \frac {|X(\eom)|^2}
%{|H(\eom)|^2} d\om .
%\end{eqnarray*}
%Le filtre $H(\eom)$ est le filtre AR qui minimise cette expression.
%Notons que l'on a la contrainte suivante
%\[
%\int_{-\pi}^{\pi} \frac 1 {H(\eom)} d\om = a[0] = 1.
%\]
%Lorsque $|X(\eom)|^2$ la valeur de $|H(\eom)|^2$ influence tr\`es
%peu l'erreur E. Cette valeur n'est importante que lorsque
%$|X(\eom)|$ est grand. Nous calculons donc \`a pr\'esent la valeur
%du spectre de $x[n]$.
%
%
%Pour minimiser l'erreur $E$ on verifie alors que le filtre fait
%une interpolation des valeurs $H(e^{i \frac{2 k \pi} T})$
%\[
%H(e^{i \frac{2 k \pi} T}) \approx \lambda
%H(e^{i \frac{2 k \pi} T}) .
%\]
%
%Si l'on excite ce filtre avec le m\^eme train d'impulsion
%$e[n]$ et que l'on multiplie le signal r\'esultant par la m\^eme
%fen\^etre $w[n]$, on obtient alors un signal $x_1 [n]$ dont
%la transform\'ee de Fourier est
%\[
%|X_1 (\eom)|^2 =
%\frac 1 {4 \pi^2}
%\sum_{k=-\infty}^{+\infty} |H(e^{i \frac{2 k \pi} T}) |^2
%|W(e^{i(\om - \frac{2 k \pi} T)})|^2 .
%\]
%On obtient donc un signal dont l'\'energie en Fourier est similaire
%\`a celle du signal original. La phase de ce signal est par contre
%modifi\'ee mais cela est peut perceptible d'un point de vue auditif.


%\subsection{Filtrage en Treillis}
%
%Le syst\'eme de pr\'ediction lin\'eaire d\'eterministe
%(\ref{yule-walker2}) est identique au syst\'eme de pr\'ediction
%lin\'eaire de Wiener-Hopf (\ref{WH_pred}) si l'on remplace
%l'autocorr\'elation $R_X [k]$ du processus $X[n]$ par
%l'autocorr\'elation empirique $r_x [k]$ de $x[n]$.
%La r\'esolution du syst\'eme (\ref{yule-walker2}) peut donc
%se faire en utilisant le m\^eme algorithme rapide
%de Levinson-Durbin qui n\'ecessite $O(N^2)$ op\'erations.
%\\
%\\
%\noindent {\bf Algorithme de Levinson-Durbin}\\
%{\it Initialisation:}  $\epsilon_{0} = r_x[0] $.\\
%{\it Boucle:} Pour $m$ allant de $1$ \`a $N$\\
%Calcul de $K_m$\\
%\indent Si $\epsilon_{m-1} > 0$\\
%\begin{equation}
%K_m = \frac {r_X [m] - \sum_{k=1}^{m-1} a_{m-1} [k] r_x[m-k]}
%{\epsilon_{m-1} }
%\end{equation}
%\indent Sinon
%\[
%K_m = 0 .
%\]
%Calcul de $a_m$\\
%\[
%\left\{ \begin{array}{l}
%a_m[k] =  a_{m-1} [k] - K_m a_{m-1} [m-k] ~~,~~
%1 \leq k \leq m-1 \\
%a_m[m] =  K_m
%\end{array}
%\right.
%\]
%Calcul de $\epsilon_m$\\
%\[
%\epsilon_m = \epsilon_{m-1} (1 - K_m^2 ) .
%\]
%\\
%\\
%{\bf Impl\'ementation en treillis}
%L'algorithme de Levinson-Durbin est bas\'e sur le calcul
%d'erreurs de pr\'ediction progressives et r\'etrogrades,
%d\'efinis par des formules identiques \`a
%(\ref{error_formula}) et (\ref{error-retro})
%\begin{equation}
%\label{error_formula-det}
%w^p_m [n+1] = x[n] - \sum_{k=1}^m a_m [k] x[n-k] ,
%\end{equation}
%\begin{equation}
%\label{error-retro-det}
%w_m^r [n] = x[n] - \sum_{k=1}^m a_m[k] x[n+k] ,
%\end{equation}
%avec
%\[
%w_N^p [n] = e[n]
%\]
%et
%\[
%a_N [k] = a[k] .
%\]
%Cet algorithme
%revient \`a effectuer une
%orthogonalization de Gram-Schmidt rapide gr\^ace aux \'equations
%r\'ecurrentes d\'emontr\'ees par le
%Th\'eore\`eme \ref{levinson}
%\begin{equation}
%\label{erreur-rec-det}
%\left\{ \begin{array}{l}
%w_m^p [n] = w_{m-1}^p [n] -
%K_m ~w^r_{m-1} [n-N] \\
%w_m^r [n-N] = w_{m-1}^r [n-N] -
%K_m ~w^p_{m-1} [n]
%\end{array}
%\right.
%\end{equation}
%Ces \'equations
%s'impl\'ementent pas le filtrage en treillis
%illustr\'e en figure \ref{treilli-directe}.
%Ce filtre en treillis a une fonction de
%transfert \'egale \`a $1 - \sum_{k=1}^N a [k] z^{-k}$.
%Il est \'equivalent au filtre en
%\'echelle illustr\'e par la figure \ref{filtre-echelle}.
%\begin{figure}[bhtp]
%%\centerline{
%%	\leavevmode\epsfbox{/home/mallat/X/TREX/figures/SigFig/FIG6.1.EPS.txt}}
%\vspace{6cm}
%\caption{Impl\'ementation en \'echelle d'un filtre dont la fonction de
%transfert est $1 - \sum_{k=1}^N a [k] z^{-k}$.}
%\label{filtre-echelle}
%\end{figure}
%
%
%\begin{figure}[bhtp]
%%\centerline{
%%
%%\leavevmode\epsfbox{/home/mallat/X/TREX/figures/SigFig/FIG6.3.EPS.txt}}
%\vspace{5cm}
%\caption{Impl\'ementation en treillis d'un filtre dont la fonction de
%transfert est $1 - \sum_{k=1}^N a [k] z^{-k}$.}
%\label{treilli-directe}
%\end{figure}
%\\
%\\
%{\bf Filtre pr\'edicteur inverse}
%Pour synth\'etiser $x[n]$ \`a partir de l'excitation $e[n]$
%on utilise le filtre inverse
%\[
%H(z) = \frac {1} {1 - \sum_{k=1}^N a [k] z^{-k}}.
%\]
%Ce filtre peut s'impl\'ementer avec la r\'ealization classique
%en \'echelle illustr\'ee par la figure \ref{AR-echele}
%ou en inversant le
%filtre en treillis comme l'illustre la figure \ref{treilli-inv}
%L'inversion de ce filtre en treillis est bas\'e sur le
%syst\`eme d'\'equations d\'eduit de (\ref{erreur-rec-det})
%\begin{equation}
%\label{erreur-rec-det-inv}
%\left\{ \begin{array}{l}
%w_{m-1}^p [n] = w_m^p [n] + K_m ~w^r_{m-1} [n-N] \\
%w_m^r [n-N] = w_{m-1}^r [n-N] -
%K_m ~w^p_{m-1} [n]
%\end{array}
%\right.
%\end{equation}
%
%Le filtrage en treillis de la figure \ref{treilli-inv}
%peut \^etre reli\'e aux \'equations
%de propagation dans une succession de cylindres
%de m\^eme taille $\Delta$, qui
%mod\'elisent le conduit vocal, comme nous l'avons vu
%dans le paragraphe \ref{conduit-sec}, et illustr\'e par la
%figure \ref{conduit-voc}.
%Les constantes $K_m$
%s'interpr\`etent comme des constantes
%de r\'eflexion pour les ondes directes et inverses,
%\`a l'interface de deux cylindres d'imp\'edence diff\'erentes.
%\begin{figure}[bhtp]
%%\centerline{
%%	\leavevmode\epsfbox{/home/mallat/X/TREX/figures/SigFig/FIG6.2.EPS.txt}}
%\vspace{5cm}
%\caption{Impl\'ementation en \'echelle d'un filtre autor\'egressif.}
%\label{AR-echele}
%\end{figure}
%
%
%\begin{figure}[bhtp]
%%\centerline{
%%	\leavevmode\epsfbox{/home/mallat/X/TREX/figures/SigFig/FIG6.4.EPS.txt}}
%\vspace{5cm}
%\caption{Impl\'ementation en treillis d'un filtre autor\'egressif.}
%\label{treilli-inv}
%\end{figure}
%\\
%\\
%{\bf Stabilit\'e}
%Lors de la synth\`ese d'un son par filtrage AR, il faut
%s'assurer que le filtre est stable, ce qui revient
%\`a montrer que les p\^oles de $H(z)$ ont tous un module
%strictement plus petit que $1$.
%Pour une application de compression, les
%coefficients du filtre $a[k]$ doivent \^etre quantifi\'es afin
%de pouvoir les stocker avec un nombre r\'eduit de bits.
%De petites modifications de ces coefficients peut bouger
%de fa\c{c}on importante les racines du polyn\^ome
%${1 - \sum_{k=1}^N a [k] z^{-k}}$ et introduire des racines
%\`a l'ext\'erieur du cercle unit\'e.
%
%La r\'ealization par filtrage en treillis permet de s'assurer
%simplement que le filtre reste stable.
%Nous avons vu lors du calcul des constantes $K_m$ par
%l'agorithme de Levinson-Durbin que n\'ecessairement
%$K_m \leq 1$. On peut d\'emontrer que
%le filtre AR impl\'ement\'e par le
%treillis inverse de la figure \ref{treilli-inv}
%est strictement stable si
%et seulement si $K_m < 1$ pour $1 \leq m \leq N$.


\subsection{Compression par pr\'ediction lin\'eaire}
\label{compre-LPC}

Pour la t\'el\'ephonie et en particulier les t\'el\'ephones cellulaires,
le d\'ebit d'information est limit\'e par la gamme de fr\'equences
utilisable pour la transmission. Au contraire, la demande augmente
constamment, ce qui n\'ecessite de transmettre toujours plus
de conversations. Une solution est de comprimer le signal de parole
pour augmenter le nombre de conversations sous contrainte
d'un d\'ebit fixe.
La qualit\'e du signal de parole peut \^etre
d\'egrad\'ee mais le codage doit maintenir une bonne intelligibilit\'e
des sons prononc\'es.
Pour des forts taux de compression,
le codage par pr\'ediction lin\'eaire est actuellement la
technique la plus efficace.

Le standard LPC-10 demande 2400bits/s pour coder un
signal de parole \'echantillonn\'e \`a 8kHz. Le signal de parole est
divis\'e sur des fen\^etres de $P = 180$ \'echantillons.
Un filtre AR d'ordre $N=10$ est calcul\'e pour chaque fen\^etre, \`a partir de l'autocorr\'elation du signal, par r\'egression
lin\'eaire. Le voisement et le
pitch sont d\'etermin\'es en
testant l'amplitude des diff\'erences (\ref{AMDF}) \`a intervalles
variables.
Pour les signaux vois\'es, on code aussi l'intervalle $T$ du
pitch, en quantifiant uniform\'ement $\log T$.
Les algorithmes de quantification sont pr\'esent\'es
dans le paragraphe \ref{scalar-quant-sec}.

La quantificaton des coefficients $a[k]$ va d\'eplacer
les p\^oles du filtre AR, qui risquent de sortir du cercle
unit\'e. Le filtre r\'esultant est alors instable.
Pour guarantir la stabilit\'e du filtre AR, on quantifie
plut\^ot les coefficients de r\'eflexion
$\{K_m\}_{1 \leq m \leq N}$,
calcul\'es par l'algorithme de Levinson-Durbin.
Ces coefficients charact\'erisent les valeurs des $a[k]$
pour $1 \leq k \leq N$ et on peut v\'erifier que le
filtre AR est stable si et seulement si $|K_m| < 1$
pour ${1 \leq m \leq N}$. Il suffit donc de s'assurer que les
valeurs quantifi\'ees des $K_m$ restent plus petites que $1$
pour obtenir un filter AR stable.

A la r\'eception, on restaure un signal
dans chaque fen\^etre de 180 \'echantillons en utilisant les param\`etres
du code. Si l'excitation est cod\'ee comme \'etant un bruit blanc,
elle est reproduite avec un
g\'en\'erateur de nombres al\'eatoires. Sinon, on g\'en\`ere
un train d'impulsions
s\'epar\'ees par un intervalle $T$ dont la valeur a \'et\'e cod\'ee.
Cette excitation est ensuite filtr\'ee par le filtre AR
dont les coefficients de r\'eflection ont \'et\'e transmis.

La qualit\'e de ce code peut \^etre am\'elior\'ee en reproduisant
plus fid\`element l'excitation $e[n]$. Au lieu de coder cette
excitation comme un bruit blanc ou un train d'impulsions,
des techniques de quantifications vectorielles
permettent de restaurer des propri\'et\'es importantes de cette
excitation de fa\c{c}on a synth\'etiser des voix de meilleure  qualit\'e.
Ces codes sont appel\'es ``Coded Excited Linear Predictive Filters''
(CELP).
