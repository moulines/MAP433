\section{Espace des fonctions de carr{\'e} int{\'e}grable}
Soit $\mathcal{L}_2(\rset)$ l'espace des fonctions d{\'e}finies sur $\rset$ et
{\`a} valeurs complexes, $f:\rset\rightarrow \cset$, de carr{\'e} sommable
c'est-{\`a}-dire telles que:
$$
\int |f(x)|^2\,\rmd x<\infty \eqsp.
$$
On note $L_2(\rset)$ l'espace des classes d'{\'e}quivalence de
$\mathcal{L}_2(\rset)$ pour la relation d'{\'e}quivalence ``$f=g$ p.p.''.
Pour $I$ un sous ensemble bor{\'e}lien de $\rset$ (et en particulier, un intervalle), on peut d{\'e}finir
de la m{\^e}me fa\c{c}on l'espace $\mathcal{L}_2(I)$ des fonctions  de carr{\'e} sommable sur $I$, $\int_I |f(x)|^2 \rmd x < \infty$
et l'espace $L_2(I)$ des classes d'{\'e}quivalence de $\mathcal{L}_2(I)$ par rapport {\`a} la relation d'{\'e}quivalence d'{\'e}galit{\'e} presque-partout.


Pour $f$ et $g \in L_2(\rset)$, d{\'e}finissons.
\begin{equation}
\label{eq:definitionpscal}
\pscal{f}{g}_I =\int_I f(x) \bar{g}(x)\,\rmd x
\end{equation}
o{\`u}, pour tout $z \in \cset$, $\bar{z}$ est le conjugu{\'e} de $z$. Lorsque $I= \rset$, nous omettons  l'indice $I$.
Cette int{\'e}grale est bien d{\'e}finie pour 2 repr{\'e}sentants de $f$ et $g$ car $|f(x) \bar{g}(x)|\leq (|f(x)|+ |\bar{g}(x)|)/2$ et
sa valeur ne d{\'e}pend {\'e}videmment pas du choix de ses repr{\'e}sentant. D'auter part, $L_2(I)$ est bien le plus ``gros'' espace
fonctionnel sur lequel ce produit scalaire est bien d{\'e}fini puisqu'il impose justement $\pscal{f}{f}_I<\infty$.
Mentionnons aussi que, de m\^eme que pour $(\lone(\rset),\|\cdot\|_1)$,  $(L_2(\rset),\|\cdot\|_2)$ est un espace de Banach
(espace vectoriel norm{\'e} complet), o{\`u} la norme $\|\cdot\|_2$ est d{\'e}finie par
$$
\|f\|_2:=\sqrt{\pscal{f}{f}}=\left(\int |f(x)|^2\,\rmd x\right)^{1/2}.
$$
Cette norme {\'e}tant un norme induite par un produit scalaire, on dit que
$(L_2(\rset),\pscal{\cdot}{\cdot})$ est un espace
\textit{de Hilbert}.

\begin{theorem}
  \label{thm:lunDense}
  L'ensemble des fonctions int{\'e}grables et de carr{\'e} int{\'e}grable, $\lone(\rset)\cap
  L_2(\rset)$ est un sous-espace vectoriel dense de $(L_2(\rset),\|\cdot\|_2)$.
\end{theorem}
\begin{proof}
Pour tout $f\in L_2(\rset)$, on note $f_n$ la fonction {\'e}gale {\`a} $f$ sur $[-n,n]$ et nulle ailleurs.
Alors $f_n\in \lone(\rset)$ pour tout $n$, et par convergence monotone, $\|f_n-f\|_2\to0$ quand $n\to\infty$ (on dit que $f_n$ tend vers $f$ au sens de $L_2(\rset)$). On en conclut que $\lone(\rset)\cap L_2(\rset)$ est dense dans
$(L_2(\rset),\|\cdot\|_2)$.
\end{proof}
On a imm{\'e}diatement que la proposition~\ref{prop:DualiteSchwartz} s'adapte {\`a} l'espace $L_2(\rset)$.
\begin{corollary}  \label{cor:DualiteCalS}
Soient deux fonctions $f$ et $g$ dans $L_2(\rset)$. Si, pour toute fonction \textit{test} $\phi$ dans $\mcs$, on a
$$
\int f(x)\,\phi(x)\,\rmd x= \int g(x)\,\phi(x)\,\rmd x,
$$
alors $f=g$ (au sens $L_2(\rset)$).
\end{corollary}

Nous verrons que $L_2(\rset)$ pose un certain nombre de probl{\`e}mes
th{\'e}oriques pour d{\'e}finir la transform{\'e}e de Fourier qui ne se pose pas
pour une fonction de $\lone(\rset)$. Or, en passant de $\lone(\rset)$ {\`a}
$L_2(\rset)$, on impose {\`a} la fonction des conditions locales plus
contraignantes (toute restriction d'une fonction de $L_2(\rset)$ {\`a} un compact
est $\lone(\rset)$ mais l'inverse n'est pas vrai) et on autorise des
comportements en $t=\pm\infty$ un peu plus g{\'e}n{\'e}raux. D{\`e}s lors, on
peut s'interroger sur l'int{\'e}r{\^e}t d'{\'e}tudier les fonctions  de $L_2(\rset)$ plut{\^o}t que de
$\lone(\rset)$, qui plus est quand, en pratique, une fonction n'est
jamais observ{\'e} sur un temps infini. La r{\'e}ponse {\`a} cette question est
la suivante. Outre que les propri{\'e}t{\'e}s d'espace de Hilbert de
$L_2(\rset)$ sont fondamentales dans la th{\'e}orie, elles ont un lien
physique {\'e}vident dans les applications puisque le carr{\'e} de la norme
d'un signal dans $L_2(\rset)$ n'est rien d'autre que son {\'e}nergie.
Le fait qu'en pratique les "signaux" observ{\'e}s soient dans
$\lone(\rset)\cap L_2(\rset)$ explique que l'on peut en g{\'e}n{\'e}ral ne
pas se pr{\'e}occuper des subtilit{\'e}s entre transform{\'e}e de Fourier dans
$\lone(\rset)$ et transform{\'e}e de Fourier dans $L_2(\rset)$, mais,
pour {\'e}tablir les r{\'e}sultats g{\'e}n{\'e}raux que l'on utilise pour {\'e}tudier
les fonctions de carr{\'e} sommable, il serait dommage de les {\'e}noncer dans
le cas particulier $\lone(\rset)\cap L_2(\rset)$ alors qu'ils sont
valables dans $L_2(\rset)$, m{\^e}me si l'on doit pour cela donner des
preuves qui peuvent appara{\^i}tre plus abstraites.

\section{Transform{\'e}e de Fourier sur $L_2(\rset)$}
L'id{\'e}e de base de la construction consiste {\`a} {\'e}tendre la transform{\'e}e de Fourier
de $\lone(\rset)$ {\`a} $L_2(\rset)$ par un argument de densit{\'e}.
\begin{proposition}
\label{prop:plancherelparseval}
Soit $f$ et $g$ dans $\mcs$. On a:
\begin{gather*}
\int \hat{f}(\xi) \bar{\hat{g}}(\xi) d \i = \int f(x) \bar{g}(x) dx \\
\int |\hat{f}(\xi)|^2 d \xi = \int |f(x)|^2 dx \eqsp.
\end{gather*}
\end{proposition}
\begin{proof}
Appliquons la formule d'{\'e}change (Proposition \ref{prop:echangeTF}). On pose $h(\xi)= \bar{\hat{g}}(\xi)$. On a:
$$
\int \hat{f}(\xi) h(\xi) d \xi = \int f(x) \hat{h}(x) dx \eqsp.
$$
Mais $\bar{\hat{g}}(\xi)= \TFC \bar{g}(\xi)$, d'o{\`u} $\hat{h}= \bar{g}$.
\end{proof}

\begin{proposition}
Soient $E$ et $F$ deux espaces vectoriels norm{\'e}s, $F$ complet, et $G$ un sous-espace vectoriel dense dans $E$.
Si $A$ est un op{\'e}rateur lin{\'e}aire continu de $G$ dans $F$, alors il existe un prolongement unique $\tilde{A}$
lin{\'e}aire continu de $E$ dans $F$ et la norme de $\tilde{A}$ est {\'e}gale {\`a} la norme de $A$.
\end{proposition}
\begin{proof}
Soit $f \in E$. Comme $G$ est dense dans $E$, il existe une suite $f_n$ dans $G$ telle que
$\lim_{n \to \infty} \| f_n - f \| = 0$. La suite $f_n$ {\'e}tant convergente, elle est de
Cauchy. $A$ {\'e}tant lin{\'e}aire continu on a
$$
\| A f_n- A f_m \| \leq \| A \| \| f_n- f_m \| \eqsp.
$$
On en d{\'e}duit que $A f_n$ est une suite de Cauchy de $F$ qui est complet.
La suite $A f_n$ est donc convergente vers un {\'e}l{\'e}ment $g$ de $F$.
On v{\'e}rifie facilement que $g$ ne d{\'e}pend pas de la suite $f_n$ et on pose donc $A f = g$.
$\tilde{A}$ est lin{\'e}aire par construction et de plus on a
$$
\|  \tilde{A} f \| =  \lim_{n \to \infty} \| A f_n \| \leq \lim_{n \to \infty} \| A \| \| fn \| = \| A \| \| f \|\eqsp,
$$
ce qui prouve que $\| \tilde{A} \| \leq  \| A \|$. Comme $\tilde{A} f = A f$ pour tout $f \in G$, on a $\| \tilde{A} \| = \| A \|$.
Enfin, $G$ {\'e}tant dense dans $E$, il est clair que $\tilde{A}$ est unique.
\end{proof}
D'apr{\`e}s la proposition~\ref{prop:plancherelparseval}, $\TF$ est une isom{\'e}trie sur $\mcs$ muni du produit scalaire
$\pscal{\cdot}{\cdot}$. On applique le r{\'e}sultat pr{\'e}c{\'e}dent avec $E = F = L_2(\rset)$, $G = \mcs$ (voir \Cref{theo:densite-mcs-Lp}). On obtient
\begin{theorem}\label{thm:prolong}
La transformation de Fourier $\TF$ (respectivement la transformation inverse $\TFC$)
se prolonge en une isom{\'e}trie de $L_2(\rset)$ sur $L_2(\rset)$.
D{\'e}signons toujours par $\TF$ (resp. $\TFC$) ce prolongement. On a en particulier
\begin{enumerate}
\item (Inversion) pour tout $f \in L_2(\rset)$, $\TF \TFC f = \TFC \TF f = f$,
\item (Plancherel) pour tout $f,g \in L_2(\rset)$, $\pscal{f}{g} = \pscal{\TF f}{\TF g}$
\item (Parseval) pour tout $f \in L_2(\rset)$, $\| f \|_2 = \| \TF f \|_2$.
\end{enumerate}
\end{theorem}

Remarquons que l'{\'e}galit{\'e} de Parseval peut se r{\'e}{\'e}crire, pour tout $f$ et $g$ dans $L_2(\rset)$,
\begin{equation}\label{eq:echangeL2}
\pscal{ \TF f}{g} = \pscal{f}{\TF g}
\end{equation}

\begin{proposition}
\label{prop:prolongementL1L2}
Le prolongement de $\TF$ sur $\mcs$ par continuit{\'e} {\`a} $(L_2(\rset),\|\cdot\|_2)$ est compatible avec la d{\'e}finition de
$\TF$ donn{\'e}e pr{\'e}c{\'e}demment dur $\lone(\rset)$. Plus pr{\'e}cis{\'e}ment
\begin{enumerate}
\item Pour tout $f\in \lone(\rset) \cap L_2(\rset)$, $\TF f$ d{\'e}fini par le th{\'e}or{\`e}me~\ref{thm:prolong} admet un repr{\'e}sentant
 $\hat{f}\in C_\infty$ v{\'e}rifiant
$$
\hat{f}(\xi) = \int_{\rset} \rme^{- \rmi 2 \pi \xi x} f(x) dx,\quad\xi\in\rset.
$$
\item Si $f \in L_2(\rset)$, $\TF f$ est la limite dans $L_2(\rset)$ de la suite $g_n$, d{\'e}finie par $g_n(\xi) =
  \int_{-n}^n \rme^{- \rmi 2 \pi \xi x} f(x) dx$.
\end{enumerate}
\end{proposition}
\begin{proof}
Notons $\hat{f}$ la transform{\'e}e de Fourier sur $\lone(\rset)$ et $\TF f$ celle sur $L_2(\rset)$.
Prenons $f \in \lone(\rset) \cap L_2(\rset)$. En appliquant la proposition \ref{prop:echangeTF} puis
Parseval (voir~(\ref{eq:echangeL2})), on a pour tout $\psi \in \mcs$,
$$
\int \psi \hat{f} = \int \hat{\psi} f = \int \TFA{\psi} f= \int \psi \TFA{f}
$$
d'o{\`u} $\int  (\hat{f} - \TFA{f}) \psi  = 0$ pour tout $\psi \in \mcs$. Le corollaire~\ref{cor:DualiteCalS} fournit alors le
premier r{\'e}sultat.

Posons $f_n = f \1_{[-n,n]}$. Par convergence domin{\'e}e, on a $\lim_n \| f_n - f \|_2^2= 0$.
Comme $f_n  \in \lone(\rset) \cap L_2(\rset)$ on {\'e}crit $g_n = \hat{f}_n = \TFA{f_n}$ et par continuit{\'e} il vient
$ \lim_{n \to \infty} \| \TF f - g_n \|_2^2= 0$.
\end{proof}

\section{Application au calcul de transform\'ees de Fourier}
\begin{proposition}
\label{prop:calcul}
\begin{enumerate}[label=(\roman*)]
\item Soit $f \in L_2(\rset)$ On a $\TF \TF f = f_\sigma$ p.p. o\`u $f_\sigma(x)= f(-x)$
\item Si $f \in \lone(\rset) \cap L^2(\rset)$, $\TFA{\hat{f}} = f$.
\end{enumerate}
\end{proposition}
\begin{proof}
\begin{enumerate}[label=(\roman*), wide=0pt, labelindent=\parindent]
\item Montrons que $\TFA f = \TFAC f_\sigma$. Soit $\sequence{f}[n][\nset]$ une suite de fonctions de $\mcs$ telle que
$\lim_{n \to \infty} \norm{f-f_n}[2]=0$ (voir \Cref{theo:densite-mcs-Lp}). On a $\TFA f_n = \TFA (f_n)_\sigma$. En passant \`{a} la limite, on a donc $\TFA{f}= \TFAC{f_\sigma}$ (remarquons que $\norm{f_\sigma - (f_n)_\sigma}[2]=0$).
\item R\'esulte imm\'ediatement du fait que $\lone(\rset) \cap L_2(\rset)$, $\TFA{f} = \hat{f}$.
\end{enumerate}
\end{proof}
\begin{example}
Consid\'erons la fonction $f_\pm(x)= \rme^{\pm x a x} u(\pm x)$ et $\Re(a) >0$. On a $\hat{f}_{\pm}(\xi)= \pm 1 / (\pm a + 2 \rmi \pi \xi)$. $\hat{f}_{\pm} \not \in \lone(\rset)$ mais \Cref{prop:calcul} montre que $\TFA{\hat{f}}= (f_{\pm})_\sigma$. On a donc
pour $a \in \cset$, $\Re(a) > 0$,
\begin{align*}
(a + 2 \rmi \pi x)^{-1} &\TFyield \rme^{a \xi} u(-\xi) \\
(a - 2 \rmi \pi x)^{-1} &\TFyield \rme^{-a \xi} u(\xi) \eqsp.
\end{align*}
En proc\'edant de la m\^{e}me fa\c{c}on on obtient
\[
\sin(x) / x \TFyield \pi \1_{\ccint{-(2\pi)^{-1},(2 \pi)^{-1}}} (\xi) \eqsp.
\]
\end{example}
\section{Principe d'incertitude : r\'esolution en temps et en fr\'equence}

\section{Convolution et transformation de Fourier dans $L_2(\rset)$}

%%% Local Variables:
%%% mode: latex
%%% ispell-local-dictionary: "francais"
%%% TeX-master: "Polycopie-Fourier-L1L2"
%%% End:
