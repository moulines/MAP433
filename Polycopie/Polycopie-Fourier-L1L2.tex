\documentclass[a4paper,10pt]{book}
\usepackage{latexsym}
\usepackage{amsmath}
\usepackage{amssymb,amsthm}
\usepackage{bm}
\usepackage{graphicx}
\usepackage{wrapfig}
\usepackage{fancybox}
\usepackage{hyperref}
\usepackage[latin1]{inputenc}
\usepackage[francais]{babel}
\usepackage{amssymb,amsmath,amsfonts,epsfig,natbib,bbm,ifthen,graphicx,twoopt,enumerate}
\usepackage[draft]{fixme}
%\FXRegisterAuthor{em}{aem}{EM}
\FXRegisterAuthor{fr}{afr}{FR}

\newtheorem{theorem}{Th�or�me}[section]
\newtheorem{definition}[theorem]{D�finition}
\newtheorem{corollary}[theorem]{Corollaire}
\newtheorem{proposition}[theorem]{Proposition}
\newtheorem{lemma}[theorem]{Lemme}
\theoremstyle{break}
\newtheorem{example}[theorem]{Exemple}
\newtheorem{remark}[theorem]{Remarque}
\def\1{\mathbbm{1}}
\def\mcb{\ensuremath{\mathcal{B}}}
\def\mcc{\ensuremath{\mathcal{C}}}
\def\mce{\ensuremath{\mathcal{E}}}
\def\mcf{\ensuremath{\mathcal{F}}}
\def\nset{\ensuremath{\mathbb{N}}}
\def\qset{\ensuremath{\mathbb{Q}}}
\def\rset{\ensuremath{\mathbb{R}}}
\def\zset{\ensuremath{\mathbb{R}}}
\def\cset{\ensuremath{\mathbb{C}}}
\def\rsetc{\ensuremath{\overline{\rset}}}
\def\Xset{\ensuremath{\mathsf{X}}}
\def\Tset{\ensuremath{\mathsf{T}}}
\def\Yset{\ensuremath{\mathsf{Y}}}
\def\pp{\ensuremath{\mathrm{p.p.}}}
\pagestyle{plain}
\def\rmd{\mathrm{d}}
\def\Qint{\ensuremath{\mathrm{QInt}}}
\def\Int{\ensuremath{\mathrm{Int}}}
\def\eqdef{\ensuremath{\stackrel{\mathrm{def}}{=}}}
\def\eqsp{\;}
\def\lleb{\lambda^{\mathrm{Leb}}}
\newcommand{\coint}[1]{\left[#1\right[}
\newcommand{\ocint}[1]{\left]#1\right]}
\newcommand{\ooint}[1]{\left]#1\right[}
\newcommand{\ccint}[1]{\left[#1\right]}



\newcommand{\rmi}{\mathrm{i}}
\newcommand{\rme}{\mathrm{e}}

\def\1{\mathbbm{1}}
\def\mcb{\ensuremath{\mathcal{B}}}
\def\mcc{\ensuremath{\mathcal{C}}}
\def\mce{\ensuremath{\mathcal{E}}}
\def\mcf{\ensuremath{\mathcal{F}}}
\def\nset{\ensuremath{\mathbb{N}}}
\def\qset{\ensuremath{\mathbb{Q}}}
\def\rset{\ensuremath{\mathbb{R}}}
\def\zset{\ensuremath{\mathbb{R}}}
\def\cset{\ensuremath{\mathbb{C}}}
\def\rsetc{\ensuremath{\overline{\rset}}}
\def\Xset{\ensuremath{\mathsf{X}}}
\def\Tset{\ensuremath{\mathsf{T}}}
\def\Yset{\ensuremath{\mathsf{Y}}}
\def\rmd{\mathrm{d}}
\def\Qint{\ensuremath{\mathrm{QInt}}}
\def\Int{\ensuremath{\mathrm{Int}}}
\def\eqdef{\ensuremath{\stackrel{\mathrm{def}}{=}}}
\def\eqsp{\;}
\def\lleb{\lambda^{\mathrm{Leb}}}
\newcommand{\rmi}{\mathrm{i}}
\newcommand{\rme}{\mathrm{e}}
\def\supp{\mathrm{supp}}



%notation fourier
\def\1{\mathbbm{1}}
\def\mcb{\ensuremath{\mathcal{B}}}
\def\mcc{\ensuremath{\mathcal{C}}}
\def\mce{\ensuremath{\mathcal{E}}}
\def\mcf{\ensuremath{\mathcal{F}}}
\def\nset{\ensuremath{\mathbb{N}}}
\def\qset{\ensuremath{\mathbb{Q}}}
\def\rset{\ensuremath{\mathbb{R}}}
\def\zset{\ensuremath{\mathbb{R}}}
\def\cset{\ensuremath{\mathbb{C}}}
\def\rsetc{\ensuremath{\overline{\rset}}}
\def\Xset{\ensuremath{\mathsf{X}}}
\def\Tset{\ensuremath{\mathsf{T}}}
\def\Yset{\ensuremath{\mathsf{Y}}}
\def\rmd{\mathrm{d}}
\def\Qint{\ensuremath{\mathrm{QInt}}}
\def\Int{\ensuremath{\mathrm{Int}}}
\def\eqdef{\ensuremath{\stackrel{\mathrm{def}}{=}}}
\def\eqsp{\;}
\def\lleb{\lambda^{\mathrm{Leb}}}
\newcommand{\coint}[1]{\left[#1\right[}
\newcommand{\ocint}[1]{\left]#1\right]}
\newcommand{\ooint}[1]{\left]#1\right[}
\newcommand{\ccint}[1]{\left[#1\right]}


\newcommand{\TF}{\mathcal{F}}
\newcommand{\TFC}{\overline{\mathcal{F}}}
\newcommand{\TFA}[1]{\mathcal{F}\left( #1 \right)}
\newcommand{\TFAC}[1]{\overline{\mathcal{F}}\left( #1 \right)}

\def\TFyield{\stackrel{\mathcal{F}}{\mapsto}}

\def\tore{\mathbb{T}}
\def\btore{\mathcal{B}(\tore)}
\def\espaceproba{(\Omega,\mathcal{A},\PP)}
\def\limn{\lim_{n \rightarrow \infty}}
\newcommand{\ps}{\ensuremath{\text{p.s.}}}
\newcommand{\pp}{\ensuremath{\text{p.p.}}}
\def\cA{\mathcal{A}}
\def\cC{\mathcal{C}}
\def\cL{\mathcal{L}}
\def\cM{\mathcal{M}}
\def\cN{\mathcal{N}}
\def\cO{\mathcal{O}}
\def\cP{\mathcal{P}}
\def\cS{\mathcal{S}}
\newcommand{\filtop}[1]{\operatorname{F}_{#1}}
\def\bfphi{{\boldsymbol{\phi}}}
\def\bfpsi{{\boldsymbol{\psi}}}
\def\bfgamma{{\boldsymbol{\gamma}}}
\def\bfpi{{\boldsymbol{\pi}}}
\def\bfsigma{{\boldsymbol{\sigma}}}
\def\bftheta{{\boldsymbol{\theta}}}
\def\bfhphi{{\hat{\boldsymbol{\phi}}}}
\def\bfhrho{{\hat{\boldsymbol{\rho}}}}
\def\bfhgamma{{\hat{\boldsymbol{\gamma}}}}

\def\ltwo{L_2}
\newcommand{\lone}{\ensuremath{\lone}}

\newcommand{\pltwo}{\ensuremath{\ell^2}}
\def\calG{\mathcal{G}}
\def\calM{\mathcal{M}}
\def\calI{\mathcal{I}}
\def\calH{\mathcal{H}}


\newcommand\BL[1]{\mathrm{BL}(#1)}%bande limit{\'e}e
%Espace de Schwarz
\def\mcs{\ensuremath{\mathcal{S}}}
%produit scalaire
\newcommand{\pscal}[2]{\left\langle #1, #2 \right\rangle}
\newcommand{\proj}[3][]{
\ifthenelse{\equal{#1}{}}{\ensuremath{\operatorname{proj}\left( \left. #2\right|#3\right)}}
{\ensuremath{\operatorname{proj}_{#1}\left( \left. #2 \right|#3\right)}}
}
%espaces engendr{\'e}s
\newcommand{\lspan}{\mathrm{Vect}}
\newcommand{\cspan}{\overline{\mathrm{Vect}}}
\def\oplusperp{\stackrel{\perp}{\oplus}}
\def\ominusperp{\ominus}%\def\ominusperp{\stackrel{\perp}{\ominus}}


%Operation sur les fonctions/distributions

\newcommand{\translation}{\mathcal{T}}
\newcommand{\multiplication}{\mathcal{M}}


%
\def\Rset{\mathbb{R}}
\def\Cset{\mathbb{C}}
\def\Zset{\mathbb{Z}}
\def\Nset{\mathbb{N}}
\def\Tset{\mathrm{T}}
% et d'autres
\newcommand{\vvec}[1]{\mathbf{#1}}
\newcommand{\signe}{\mathrm{sgn}}
\newcommand{\rect}{\mathrm{rect}}
\newcommand{\sinc}{\mathrm{sinc}}
\newcommand{\cov}{\mathrm{cov}}
\newcommand{\corr}{\mathrm{corr}}
\newcommand{\vp}{\mathrm{vp}}
\newcommand{\erf}{\mathrm{erf}}
\def\mod{{\ \rm mod\ }}
\def\cF{\mathcal{F}}
\def\cE{\mathcal{E}}
\def\cB{\mathcal{B}}
\def\cH{\mathcal{H}}
\def\cG{\mathcal{G}}
\def\cI{\mathcal{I}}
\def\PP{\mathbb{P}}
\newcommand\PE[1]{{\mathbb E}\left[ #1 \right]}
\newcommand{\Var}[1]{\mathrm{Var}\left( #1 \right)}
\def\BB{\mathrm{B.B.}}
\def\BBF{\mathrm{B.B.F.}}
\newcommandx{\norm}[2][2=]{\Vert #1 \Vert_{#2}}
\def\L1loc{L_{1,\mathrm{loc}}}
\def\Leb{\mathrm{Leb}}


\newcommandx\sequence[3][2=,3=]
{\ifthenelse{\equal{#3}{}}{\ensuremath{\{ #1_{#2}\}}}{\ensuremath{\{ #1_{#2}, \eqsp #2 \in #3 \}}}}
\newcommandx\sequencePar[3][2=,3=]
{\ifthenelse{\equal{#3}{}}{\ensuremath{\{ #1({#2})\}}}{\ensuremath{\{ #1({#2}), \eqsp #2 \in #3 \}}}}
\def\pp{\ensuremath{\mathrm{p.p.}}}
\def\ie{i.e.} 

\begin{document}
\title{Transform�e de Fourier des fonctions}
\author{E.~Moulines, F.~Roueff}


\chapter{Espaces de Hilbert}

\input{hilbert}

\chapter{Espaces $L^p$}

\input{espaceslp}

\chapter{Transform�e de Fourier dans $L^1$}

\section{Transform{\'e}e de Fourier sur $L_1(\rset)$}

Nous abordons dans cette partie la d{\'e}finition et les propri{\'e}t{\'e}s de la transform{\'e}e de Fourier des fonctions.
\begin{definition}[Transform{\'e}e de Fourier]
Soit $f \in L_1(\rset)$. On pose, pour tout $\xi\in\rset$,
\begin{align}
\label{eq:TFL1}
&[\TFA{f}](\xi)= \hat{f}(\xi) = \int_{\rset} \rme^{- \rmi 2 \pi \xi x} f(x) \rmd x \\
&[\TFAC{f}](\xi) = \int_{\rset} \rme^{ \rmi 2 \pi \xi x} f(x) \rmd x
\end{align}
On appelle $\TFA{f}$ (not{\'e} aussi $\TF f$) la \emph{Transform{\'e}e de Fourier} de $f$ et $\TFAC{f}$  (not{\'e} aussi $\TFC f$) la
transform{\'e}e de Fourier conjugu{\'e}e de $f$.
\end{definition}
Cette int{\'e}grale a un sens pour $f \in L_1(\rset)$, parce que $x\mapsto \rme^{- \rmi 2 \pi \xi x} f(x)$ est alors aussi dans
$L_1(\rset)$ pour tout $\xi\in\rset$.
\begin{example}
Soit $f = \1_{[a,b]}(x)$, la fonction indicatrice de l'intervalle $[a,b]$. Un calcul imm{\'e}diat montre que
$$
\hat{f}(\xi) =
\begin{cases}
b-a & \quad \xi = 0, \\
\frac{\sin \pi (b-a) \xi}{\pi \xi} \rme^{- \rmi \pi (a+b) \xi} & \quad \xi\neq0\eqsp.
\end{cases}
$$
On remarque que $\hat{f} \not \in L_1(\rset)$. En revanche c'est une fonction continue born{\'e}e
telle que $\lim_{|\xi|\to \infty} \hat{f}(\xi)= 0$.
\end{example}
En fait les propri{\'e}t{\'e}s d{\'e}crites pour $\hat{f}$ dans cet exemple sont des propri{\'e}t{\'e}s g{\'e}n{\'e}rales des
fonctions de $\TF(L^1(\rset))$ comme le montre le r{\'e}sultat suivant.

\begin{theorem}[Rieman-Lebesgue]
\label{thm:rieman-lebesgue}
Etant donn{\'e} $f \in L_1(\rset)$ on a
\begin{enumerate}
\item $\TF f$ est une fonction continue et born{\'e}e sur $\rset$,
\item $\TF$ est un op{\'e}rateur lin{\'e}aire et continu de $(L_1(\rset),\|\cdot\|_1)$ dans $(C_\infty,\|\cdot\|)$ (l'espace des
  fonctions continues munie de la norme sup) et $\| \hat{f} \| \leq \| f \|_1$,
\item $\lim_{|\xi| \to \pm \infty} |\hat{f}(\xi) |= 0 $.
\end{enumerate}
\end{theorem}
\begin{proof}
\begin{enumerate}
\item La fonction $\xi \mapsto \rme^{- \rmi 2 \pi \xi x} f(x)$ est continue sur $\rset$
  et major{\'e}e en module par $|f(x)|$ (qui ne d{\'e}pend pas de $\xi$), qui est dans $L_1(\rset)$. On conclut en appliquant le th\'eor\`eme de convergence domin\'ee.
\item La lin{\'e}arit{\'e} de $\TF$ d{\'e}coule directement de la lin\'earit\'e de l'int{\'e}grale. Pour tout $\xi \in \rset$, on a $|\hat{f}(\xi)| \leq
  \int |f(x)| \rmd x = \| f \|_1$. On en d{\'e}duit que $\hat{f}$ est born{\'e}e par $\| f\|_1$ et que $\TF$ est continue.
\item Supposons tout d'abord que $f$ est continue {\`a} support compact (il existe $M>0$ tel que $f(x)=0$ si $|x|>M$). Par
  changement de variable $x=t-1/(2\xi)$ dans~\eqref{eq:TFL1}, on a, pour tout $\xi\neq0$,
$$
\hat{f}(\xi) = \int_{\rset} \rme^{- \rmi 2 \pi \xi t - \rmi \pi} f(t+1/(2\xi)) \rmd t =
- \int_{\rset} \rme^{- \rmi 2 \pi \xi t} f(t+1/(2\xi)) \rmd t \eqsp.
$$
D'o{\`u} l'expression, en sommant cette {\'e}quation avec~(\ref{eq:TFL1}),  pour tout $\xi\neq0$,
$$
2 \hat{f}(\xi) = \int_{\rset} \rme^{- \rmi 2 \pi \xi x} (f(x)-f(x+1/(2\xi))) \,\rmd x
$$
Il s'en suit, par convergence domin{\'e}e (en observant que $|f(x)-f(x+1/(2\xi))|$ est major{\'e} ind{\'e}pendamment de $x$ et $\xi$ et
est nul pour $x\notin[-M-1,M+1]$ pour tout $|\xi|\geq1$),
$$
\lim_{|\xi| \to \pm \infty} |\hat{f}(\xi)| \leq \frac12 \lim_{|\xi| \to \pm \infty}
\int_{\rset} |f(x)-f(x+1/(2\xi))| \,\rmd x = 0.
$$
Soit maintenant $f \in L_1(\rset)$. Il existe une suite $\{g_n\}$ de fonctions continues dans $L_1(\rset)$ telles que $\|f - g_n \|_1 \to 0$. Comme,
$\|\hat{f} - g_n\|_\infty \leq \| f - g_n \|_1$ et $\lim_{\xi \to \pm \infty} g_n(\xi) = 0$,
on en d{\'e}duit ais{\'e}ment que $\lim_{\xi  \to \pm \infty} \hat{f}(\xi)= 0$.
\end{enumerate}
\end{proof}

La propri{\'e}t{\'e} {\`a} la fois la plus imm{\'e}diate et la plus  fondamentale de la transform{\'e}e de Fourier est son effet sur les
translations.
\begin{proposition}[Transform{\'e}e de Fourier et translation]
\label{prop:FourierTranslation}
Soit $f\in L_1(\rset)$. Alors, pour tout $t\in\rset$, les fonctions $x\mapsto f(x-x_0)$ et
$x\mapsto \rme^{\rmi 2 \pi \xi_0 x}f(x)$ sont dans $L_1(\rset)$ et v{\'e}rifient
\begin{enumerate}
\item $[\TFA{x\mapsto f(x-x_0)}](\xi)=\rme^{- \rmi 2 \pi x_0}\hat{f}(\xi)$ pour tout $\xi\in\rset$;
\item $[\TFA{x\mapsto  \rme^{\rmi 2 \pi \xi_0 x}f(x)}](\xi)=\hat{f}(\xi-\xi_0)$ pour tout $\xi\in\rset$.
\end{enumerate}
\end{proposition}
\begin{proof}
La preuve \'el\'ementaire est laiss\'ee aux lecteurs.
\end{proof}

Une des propri{\'e}t{\'e}s remarquables de la transform{\'e}e de Fourier est d'{\'e}changer la \emph{d{\'e}rivation} et la multiplication par un mon{\^o}me
\begin{proposition}[Transform{\'e}e de Fourier et D{\'e}rivation]
\label{prop:FourierDerivation}
Soit $n$ un entier naturel.
\begin{enumerate}
\item Si $x\mapsto x^k f(x)$ est dans $L_1(\rset)$  pour tout $k=0,1, \dots, n$, alors $\hat{f}$ est $n$ fois contin\^ument
  d{\'e}rivable et on a
$$
\hat{f}^{(n)} = \TFA{x\mapsto(-2 \rmi \pi x)^n f(x)}
$$
\item Si $f$ est $n$ fois contin\^ument d{\'e}rivables avec $f^{(k)} \in L_1(\rset)$ pour tout $k=0,1, \dots, n$, alors
$$
[\TF(f^{(n)})](\xi) = (2 \rmi \pi \xi)^n \hat{f}(\xi) \quad\text{pour tout $\xi\in\rset$}\eqsp.
$$
\end{enumerate}
\end{proposition}
\begin{proof}
Dans les deux cas, il suffit de d{\'e}montrer le r{\'e}sultat pour $n=1$ puis d'appliquer une r{\'e}currence {\'e}vidente.
\begin{enumerate}
\item La fonction $h:\xi \mapsto \rme^{- \rmi 2 \pi \xi x} f(x)$ est contin\^ument d{\'e}rivable
et $h'(\xi)= -2 \rmi \pi x \rme^{- \rmi 2 \pi \xi x} f(x)$. De plus $|h'(\xi)| \leq 2 \pi |xf(x)|$.
Le r{\'e}sultat d{\'e}coule du th{\'e}or{\`e}me de d{\'e}rivation sous le signe somme.
\item Comme $f' \in L_1(\rset)$, on peut calculer $\TFA{f'}$ par la formule
$$
[\TFA{f'}](\xi) = \lim_{a \to \infty} \int_{-a}^a \rme^{- \rmi 2 \pi \xi x} f'(x) \rmd x, \quad\xi\in\rset\eqsp.
$$
De plus, par int{\'e}gration par parties, pour tout $\xi\in\rset$ et tout  $a>0$,
$$
\int_{-a}^{+a} \rme^{- \rmi 2 \pi \xi x} f'(x) \rmd x = [ \rme^{- \rmi 2 \pi \xi x} f(x) ]_{-a}^a + \int_{-a}^a (2 \rmi \pi \xi) \rme^{- \rmi 2 \pi \xi x} f(x) \rmd x \eqsp.
$$
Comme $f' \in L_1(\rset)$ et $f(a) = f(0) + \int_0^a f'(t) \rmd x$, $\lim_{a \to \infty} \int_0^a f'(t) \rmd x$ existe et donc
$\lim_{a \to \infty} f(a) $ existe. Cette limite est n{\'e}cessairement nulle car $f \in L_1(\rset)$. De la m{\^e}me fa\c{c}on,
$\lim_{a \to \infty} f(-a) = 0$. D'o{\`u} le r{\'e}sultat.
\end{enumerate}
\end{proof}

La proposition suivante sera tr{\`e}s utile pour {\'e}tablir des formules d'inversion de la transform{\'e}e de Fourier.
\begin{proposition}
\label{prop:echangeTF}
Soit $f$ et $g$ deux fonctions de $L_1(\rset)$. Alors $f \hat{g}$ et $\hat{f}g$ sont dans $L_1(\rset)$ et on a
\begin{equation}
\label{eq:echange}
\int f(x) \hat{g}(x) \rmd x = \int \hat{f}(x) g(x) \rmd x \eqsp.
\end{equation}
\end{proposition}
\begin{proof}
Comme $\hat{g} \in L_\infty(\rset)$, les fonctions $f \hat{g}$ et
$\hat{f} g$ appartiennent {\`a} $L_1(\rset)$. Comme la fonction $(t,x)
\mapsto \rme^{- \rmi 2 \pi t x} f(t) g(x) \in L_1(\rset^2)$, il
r{\'e}sulte du th{\'e}or{\`e}me de Fubini que
\begin{multline*}
\int f(t) \hat{g}(t) dt = \int f(t) \left( \int \rme^{- \rmi 2 \pi t x} g(x) \rmd x \right) dt =
\\ \int g(x) \left( \int \rme^{- \rmi 2 \pi t x} f(t) dt \right) \rmd x = \int g(x) \hat{f}(x) \rmd x\eqsp.
\end{multline*}
\end{proof}


\section{D{\'e}croissance et d{\'e}rivation}
%Nous avons observ{\'e} dans la partie pr{\'e}c{\'e}dente la n{\'e}cessit{\'e} de restreindre l'espace $L_1(\rset)$
%pour pour d{\'e}finir la transform{\'e}e de Fourier inverse.
Nous avons observ{\'e} dans la partie pr{\'e}c{\'e}dente d{\`e}s le premier exemple de calcul de transform{\'e}e de Fourier
que $L_1(\rset)$  n'est pas stable sous l'effet de $\TF$.
Nous allons introduire un sous-espace de $L_1(\rset)$ stable par transformation de Fourier,
d{\'e}rivation et multiplication par un polyn{\^o}me. Cet espace introduit par Laurent Schwartz et que l'on notera $\mcs$ joue un r\^ole essentiel en analyse de Fourier.

\begin{definition}[Fonction {\`a} d{\'e}croissance rapide]\index{Fonction|{\`a}
    d{\'e}croissance rapide}
Une fonction $f : \rset \to \cset$ est dite {\`a} \emph{d{\'e}croissance rapide} si, pour tout $p \in \nset$,
on a
$$
\lim_{|x| \to \infty} |x|^p |f (x)| = 0 \eqsp.
$$
\end{definition}
C'est le cas par exemple de $f (x) = \rme^{-|x|}$.
Mais on notera que contrairement {\`a} son nom, cette d{\'e}finition n'implique aucune monotonie pour
$f$ m{\^e}me dans un voisinage de l'infini (prendre par exemple $f (x) = \rme^{-|x|} \sin x$).
Une propri{\'e}t{\'e} utile sur l'int{\'e}grabilit{\'e} des fonctions {\`a} d{\'e}croissance rapide est la suivante.
\begin{lemma}
\label{lem:decroissancerapide}
 Si $f$ est une fonction de $L_{1\mathrm{loc}}(\rset)$ {\`a} d{\'e}croissance rapide alors pour tout
 $p \in \nset$, $x \mapsto x^p f (x)$ appartient {\`a} $L_1(\rset)$.
\end{lemma}
\begin{proof}
L'indice ``loc'' signifie que la restriction de $f$ {\`a} tout compact est dans $L_1(\rset)$.
 $f$ {\'e}tant {\`a} d{\'e}croissance rapide, il existe $M > 0$ tel que pour tout $|x| \geq  M$,
on ait $|x|^{p+2} |f(x)| \leq  1$. D'o{\`u}
\begin{align*}
\int |x^p f(x)| \rmd x &\leq \int_{|x| \leq M}  |x|^p |f(x)| \rmd x+ \int_{|x| > M} |x|^{-2} |x^{p+2}  f(x)| \rmd x \\
&\leq  M^p \int_{|x| \leq M} |f(x)|\, \rmd x +  \int_{|x| > M}   x^{-2} \rmd x < \infty\eqsp.
\end{align*}
\end{proof}
On en d{\'e}duit une propri{\'e}t{\'e} remarquable de la transform{\'e}e de Fourier des fonctions {\`a} d{\'e}croissance rapide.
\begin{proposition}
\label{prop:1913}
Soit $f$ une fonction de $L_1(\rset)$ {\`a} d{\'e}croissance rapide. Alors $\hat{f}$ est ind{\'e}finiment d{\'e}rivable.
\end{proposition}
\begin{proof}
D'apr{\`e}s la proposition \ref{prop:FourierDerivation}, $\hat{f}$ est dans $C_\infty(\rset)$ d{\'e}s que, pour tout $p \in  \nset$,
$x^p f (x)$ est dans $L_1(\rset)$; ce qui est assur{\'e} par  le lemme \ref{lem:decroissancerapide}.
\end{proof}
Inversement si $f$ est dans $C_\infty(\rset)$ quelles propri{\'e}t{\'e}s poss{\`e}de $\hat{f}$ ? Le r{\'e}sultat suivant am{\`e}ne un {\'e}l{\'e}ment de r{\'e}ponse.

\begin{proposition}
\label{prop:1914}
Soit $f$ une fonction de $C_\infty(\rset)$. Si pour tout $k \in \nset$, $f^{(k)}$ est dans $L_1(\rset)$
alors $\hat{f}$ est {\`a} d{\'e}croissance rapide.
\end{proposition}
\begin{proof}
D'apr{\`e}s la proposition \ref{prop:FourierDerivation} on a, pour tout $k \in \nset$,
$\widehat{f^{(k)}}(\xi) = (2 \rmi \pi \xi)^k \hat{f}(\xi)$.
En appliquant le th{\'e}or{\`e}me de Riemann-Lebesgue,  il vient $\lim_{|\xi| \to \infty} |\xi|^k |\hat{f}(\xi)| = 0$.
\end{proof}
Autrement dit nous venons de voir que
\begin{enumerate}
\item plus $f$ d{\'e}cro{\^i}t rapidement {\`a} l'infini, plus $\hat{f}$ est r{\'e}guli{\`e}re;
\item plus $f$ est r{\'e}guli{\`e}re, plus $\hat{f}$ d{\'e}cro{\^i}t rapidement {\`a} l'infini.
En particulier si $f \in C_\infty(\rset)$ et est {\`a} d{\'e}croissance rapide, il en est de m{\^e}me pour
$\hat{f}$.
\end{enumerate}

\section{Un exemple remarquable}

Nous allons consid{\'e}rer une famille de fonctions  qui reste stable par transformation de Fourier.

On introduit pour tout $\sigma > 0$ la fonction de \emph{densit{\'e} gaussienne}
\index{Densit{\'e} gaussienne}
\begin{equation}
  \label{eq:gauss_func}
  g_\sigma(x)=\frac{1}{\sigma \sqrt{2 \pi} }\rme^{- \xi^2/2 \sigma^2}.
\end{equation}


\begin{lemma}\label{lem:gaussenneTF}
La fonction  $g_1(x)= 1 / \sqrt{2\pi} \exp( -x^2/2)$ est la densit{\'e} d'une probabilit{\'e} sur $\rset$, et sa transform{\'e}e de Fourier
est $\hat{g_1}(\xi)= \rme^{- 2 \pi^2 \xi^2}$.
\end{lemma}
\begin{proof}
La fonction  est positive et v{\'e}rifie  $\int_\rset g_1(x) \rmd x= 1$ (on peut le montrer en exercice en {\'e}crivant le carr{\'e} de
l'int{\'e}grale comme une double int{\'e}grale).

La fonction $x\mapsto xg_1(x)$ {\'e}tant aussi dans $L_1(\rset)$, on peut appliquer la
proposition~\ref{prop:FourierDerivation}(i), et on obtient, pour tout $\xi\in\rset$,
$$
\hat{g_1}'(\xi) = - 2\rmi \pi \int x \,g_1(x) \rme^{- \rmi 2 \pi \xi x} \rmd x \eqsp.
$$
Un calcul imm{\'e}diat donne $g_1'(x)= -xg_1(x)$; une int{\'e}gration par partie donne donc
$$
\hat{g_1}'(\xi) = - 2\rmi \pi  \int g_1'(x) \,(- \rmi 2 \pi \xi)\,\rme^{- \rmi 2 \pi \xi x} \rmd x \eqsp.
$$
D'o{\`u} l'on tire finalement que $\hat{g}_1'(\xi) = - 4 \pi^2 \xi \hat{g}_1(\xi)$.
La solution g{\'e}n{\'e}rale de l'{\'e}quation diff{\'e}rentielle {\`a} variables s{\'e}parables  $f'(u)= - 4 \pi^2 u f(u)$
{\'e}tant $f(u) = C \rme^{- 2 \pi^2 u^2}$, et comme on a  $\hat{g_1}(0) = \int g_1(x) \rmd x = 1$,
on voit que n{\'e}cessairement $\hat{g_1}(\xi)= \rme^{-2 \pi^2 \xi^2}$.
\end{proof}

Par un changement de variable {\'e}vident, ce r{\'e}sultat se g{\'e}n{\'e}ralise ais{\'e}ment {\`a} tout $\sigma>0$. En particulier,
\begin{equation}\label{eq:gaussienneTF}
\hat{g_\sigma}(\xi) =  \rme^{-2 \pi^2 (\xi\sigma)^2} = \frac1{\sigma\sqrt{2\pi}}\,g_{1/2\pi\sigma}(\xi),\quad\xi\in\rset.
\end{equation}

Comme annonc\'e plus haut, la famille $(g_\sigma)_{\sigma>0}$ est donc stable par
transform{\'e} de Fourier.
\section{Quelques exemples classiques}
Rappelons que  $u$ est la fonction de Heaviside, d\'efinie par $u(x)=1$ pour $x>0$ et $u(x)=0$ pour $x\leq 0$.
\begin{enumerate}[label=(\roman*)]
\item $f_{1}(x)=\rme^{-ax}u(x)$ , ${\rm Re}(a)>0.$
$$
\hat{f_{1}}(\xi)=\int_{0}^{\infty}\rme^{-2i\pi x\xi}\rme^{-ax}dx=\lim_{b\rightarrow+\infty}[\frac{-\rme^{-x(a+2i\pi\xi)}}{a+2i\pi\xi}]_{0}^{b}=\frac{1}{a+2i\pi\xi}.
$$
\item $f_{2}(x)=\rme^{ax}u(-x)$ , ${\rm Re}(a)>0.$
$$
\hat{f_{2}}(\xi)=\int_{-\infty}^{0}\rme^{-2i\pi x\xi}\rme^{ax}dx=\frac{-1}{-a+2i\pi\xi}.
$$
\item $f_{3}(x)=\displaystyle \frac{x^{k}}{k!}\rme^{-ax}u(x)$ , ${\rm Re}(a)>0.$

$f_{3}(x)=\displaystyle \frac{1}{(-2i\pi)^{k}}\frac{1}{k!}(-2i\pi x)^{k}f_{1}(x)$ , and $\displaystyle \hat{f_{3}}(\xi)=\frac{1}{k!}\frac{1}{(-2i\pi)^{k}}\hat{f_{1}}^{(k)}(\xi)$. Comme $\hat{f_{1}}^{(k)}(\xi)=k!(a+2i\pi\xi)^{-(k+1)}(-2i\pi)^{k},$
$$
\hat{f_{3}}(\xi)=\frac{1}{(a+2i\pi\xi)^{k+1}}.
$$
\item $f_{4}(x)=\displaystyle \frac{x^{k}}{k!}\rme^{ax}u(-x)$ , ${\rm Re}(a)>0.$
Nous avons
$$
\hat{f_{4}}(\xi)=\frac{-1}{(-a+2i\pi\xi)^{k+1}}.
$$
\item $f_{5}(x)=\rme^{-a|x|}, {\rm Re}(a)>0$. Nous d\'eduisons des calculs pr\'ec\'edents
$$
\hat{f_{5}}(\xi)=\frac{2a}{a^{2}+4\pi^{2}\xi^{2}}.
$$
\item $f_{6}(x)=\mathrm{sign}(x)\rme^{-a|x|}, {\rm Re}(a)>0$. Nous avons
$$
\hat{f_{6}}(\xi)=\frac{-4i\pi\xi}{a^{2}+4\pi^{2}\xi^{2}}.
$$
\end{enumerate}
\section{Formulaire}
\label{sec:formulaire}
\begin{enumerate}[label=(\roman*)]
\item
\begin{align*}
\hat{f}^{(k)}(\xi)&=[(-2\rmi\pi x)^{k}f(x)]^{\sim}(\xi)\\
\widehat{f^{(k)}}(\xi)&=(2\rmi \pi\xi)^{k}\hat{f}(\xi)
\end{align*}
\item
\begin{align*}
f(x-a) &\TFyield \rme^{-2\rmi \pi a\xi}\hat{f}(\xi) \\
\rme^{2 \rmi \pi ax}f(x)&\TFyield \hat{f}(\xi-a) \\
\end{align*}
\item $a\neq 0$.
$$ f(ax)\TFyield \frac{1}{|\xi|}\hat{f}(\frac{\xi}{a})$$
\item  $a\in \cset$, ${\rm Re}(a)>0, k=0,1,2\ldots.$
\begin{align*}
\frac{x^{k}}{k!}\rme^{-ax}u(x) &\TFyield \frac{1}{(a+2i\pi\xi)^{k+1}} \\
\frac{x^{k}}{k!}\rme^{ax}u(-x) &\TFyield \frac{-1}{(-a+2i\pi\xi)^{k+1}} \\
\rme^{-a|x|} &\TFyield \frac{2a}{a^{2}+4\pi^{2}\xi^{2}} \\
\mathrm{sign}(x)\rme^{-a|x|} &\TFyield \frac{-4i\pi\xi}{a^{2}+4\pi^{2}\xi^{2}} \\
\end{align*}
\item  $a\in \rset$, $a>0$.
\begin{align*}
\rme^{-ax^{2}}  &\TFyield \sqrt{\frac{\pi}{a}}\rme^{-\frac{\pi^{2}}{a}\xi^{2}} \\
\1_{\ccint{-a,+a}}(x) &\TFyield \frac{\sin 2a\pi\xi}{\pi\xi}
\end{align*}
\end{enumerate}
%%% Local Variables:
%%% mode: latex
%%% ispell-local-dictionary: "francais"
%%% TeX-master: "Polycopie-Fourier-L1L2"
%%% End:



\chapter{Transform�e de Fourier dans $\mcs$}
\section{Convolution}

\section{L'espace de Schwarz}

Nous avons vu comment les propri{\'e}t{\'e}s de d{\'e}croissances se traduisent par des
propri{\'e}t{\'e}s de d{\'e}rivabilit{\'e} sur la transform{\'e}e de Fourier, et \textit{vice
  versa}.
Afin de construire un espace fonctionnel stable par transformation de Fourier,
il est donc naturel d'introduire un espace contenant des fonctions comportant
{\`a} la fois ces deux propri{\'e}t{\'e}s.

\index{Espace de Schwarz}
\begin{definition} On d{\'e}signe par $\mcs(\rset)$ ou tout simplement $\mcs$
l'espace vectoriel des fonctions de $\rset$ dans $\cset$ qui v{\'e}rifient les deux propri{\'e}t{\'e}s suivantes:
\begin{enumerate}
\item $f$ est ind{\'e}finiment d{\'e}rivable sur $\rset$;
\item $f$ est {\`a} d{\'e}croissance rapide, ainsi que toutes ses d{\'e}riv{\'e}es.
\end{enumerate}
On appelle $\mcs(\rset)$ l'\emph{espace de Schwarz}.
\end{definition}

Donnons quelques exemples importants de fonctions appartenant {\`a} $\mcs(\rset)$.
\begin{example}
  Il est facile de v{\'e}rifier que la fonction $f$ d{\'e}finie par $f(x)=\rme^{-x^2}$
  appartient {\`a} $\mcs$. Par suite, il en est de m{\^e}me de la fonction $g_\sigma$
  d{\'e}finie par~(\ref{eq:gauss_func}) quelque soit $\sigma>0$.
\end{example}
Il faut travailler un peu plus pour trouver un exemple de fonction dans $\mcs$
{\`a} \emph{support compact} c'est-{\`a}-dire nulle en dehors d'un ensemble
born{\'e}.
\begin{example}\label{exple:cinfty-supp-compact}
  On consid{\`e}re la fonction $g$ d{\'e}finie par
$$
g(x)=
\begin{cases}
\rme^{-1/x} &\text{ si $x>0$}\\
0 &\text{ sinon.}
\end{cases}
$$
On montre facilement que $f$ est
$\mathcal{C}^\infty$. En revanche elle n'est pas {\`a} support compact puisque
$g(x)\to1$ quand $x\to\infty$. Cependant la fonction $f$ d{\'e}finie par
$$
f(x)=g(x) \times g(1-x) ,\quad x\in\rset\;,
$$
est nulle en dehors de $[0,1]$ et $\mathcal{C}^\infty$. C'est donc une fonction
de $\mcs$ {\`a} support compact.
\end{example}

\begin{proposition}
L'espace $\mcs$ a les propri{\'e}t{\'e}s suivantes :
\begin{enumerate}
\item $\mcs$ est stable pour la multiplication par un polyn{\^o}me.
\item $\mcs$ est stable par d{\'e}rivation ($f \in \mcs \Rightarrow f' \in  \mcs$).
\item $\mcs \subset  L_1(\rset)$.
\end{enumerate}
\end{proposition}
La d{\'e}monstration de cette proposition est laiss{\'e}e en exercice.  Le r{\'e}sultat
essentiel, qui d{\'e}coule essentiellement de propri{\'e}t{\'e}s d{\'e}j{\`a} d{\'e}montr{\'e}es, est
contenu dans le th{\'e}or{\`e}me suivant.
\begin{theorem}
\label{theo:stabiliteS}
L'espace $\mcs$ est stable par transformation de Fourier: si $f \in \mcs$ alors $\hat{f} \in \mcs$.
\end{theorem}
\begin{proof}
Soit $f \in \mcs$. Comme
$f$ est dans $L_1(\rset)$ et {\`a} d{\'e}croissance rapide, $\hat{f} \in C_\infty (\rset)$ d'apr{\`e}s la Proposition \ref{prop:1913}.
Pour tout $k \in \nset$, $f^{(k)}$ {\'e}tant {\`a} d{\'e}croissance rapide est int{\'e}grable d'apr{\`e}s le lemme \ref{lem:decroissancerapide}.
On en d{\'e}duit de la proposition \ref{prop:1914} que $\hat{f}$ est {\`a} d{\'e}croissance rapide.
Il reste {\`a} examiner la d{\'e}rivabilit{\'e} de $f$.
Comme toutes les d{\'e}riv{\'e}es de $f$ sont {\`a} d{\'e}croissance rapide, les fonctions $x\mapsto(x^q f (x))^{(p)}$ sont dans
$L_1(\rset)$ pour tout entiers $p$ et $q$; d'o{\`u}, d'apr{\`e}s la Proposition \ref{prop:FourierDerivation}, pour tout
$\xi\in\rset$,
\begin{equation}
  \label{eq:stabilite-S-en-action}
\xi^p \hat{f}^{(q)}(\xi) = \xi^p  \TFA{(-2 \rmi \pi)^q m_q\times f}(\xi) =
\frac{1}{(\rmi 2 \pi)^p} \TFA{[(- \rmi 2 \pi)^q m_q \times f]^{(p)}}(\xi) \;,
\end{equation}
o{\`u} $m_q$ est le mon{\^o}me de degr{\'e} $q$, $m_q(x)=x^q$.  Le th{\'e}or{\`e}me
de Rieman-Lebesgue montre que $\lim_{|\xi| \to \infty} |\xi^q
\hat{f}^{(q)}(\xi)|= 0$.
\end{proof}

Nous allons voir cette stabilit{\'e} se traduit en fait par une propri{\'e}t{\'e} encore
plus remarquable: $\TF$ est une bijection de $\mcs$ dans $\mcs$. Pour cela, il
faut montrer que $\TF$ admet une r{\'e}ciproque sur $\mcs$, ce sera le r{\^o}le de sa
soeur jumelle $\TFC$. Le proc{\'e}d{\'e} de r{\'e}gularisation sera fondamental pour
arriver {\`a} cette fin.


\section{R{\'e}gularisation par convolution}

On rappelle qu'{\'e}tant donn{\'e}es $f$ et $g$ deux fonctions de $\rset$ dans $\cset$,
la convolution de $f$ et $g$ est la fonction $f \star g$
d{\'e}finie par
$$
f \star g(x) = \int f(x - t)g(t) \rmd t = \int f(u)g(x - u) \rmd u\;,
$$
en tout point $x$ o{\`u} cette int{\'e}grale est correctement d{\'e}finie.\


\begin{example}
Prenons $f = g = \1_{[0,1]}$. On obtient
$$
\int f (x - t)g(t) \rmd t = \int_0^1 \1_{[0,1]}(x - t) \rmd t = \mathrm{Leb} ([0,1] \cap [x - 1,x]).
$$
La convolution de ces deux fonctions discontinues est donc continue.

Prenons maintenant $f \in L_1(\rset)$ et $g = \frac{1}{2h} \1_{[-h,h]}$ o{\`u} $h > 0$.
Calculons $f \star g$.
\begin{equation}
f * g(x) = \frac{1}{2h} \int_{x-h}^{x+h}  f (u) \rmd u
\end{equation}
ce qui repr{\'e}sente la moyenne de $f$ sur $[x - h, x + h]$.
On voit directement sur l'int{\'e}grale que $f \star g$ est une fonction continue.
\end{example}
Ces deux exemples illustrent la propri{\'e}t{\'e} essentielle de la convolution qui est de
\textit{r{\'e}gulariser} (qui est li{\'e} au fait de moyenner).


\section{R\'egularisation par convolution}
Le lemme suivant montre que la convolution peut {\^e}tre utilis{\'e}e dans le but de
fournir une approximation, ce qui constitue le premier principe de
r{\'e}gularisation.

\begin{lemma}\label{lem:regularisation}
  Soit $f\in\mathcal L^1$.
  Soit $K$ une fonction positive telle que $\int K(x)\,\rmd x=1$ et, pour une
  constante $C>0$ et un exposant $\alpha>2$, $K(x)\leq C (1+|x|)^{-\alpha}$
  pour tout $x\in\rset$.  D{\'e}finissons, pour tout $\sigma>0$, la fonction
  $K_\sigma$ par
\begin{equation}
  \label{eq:Kband}
  K_\sigma(x)=\frac1\sigma K(x/\sigma)\;.
\end{equation}
Alors, on a les propri{\'e}t{\'e}s suivantes.
\begin{enumerate}
\item\label{item:regularisation1} En tout point $t$ o{\`u} $f$ est continue, on a
$$
\lim_{\sigma\downarrow0} (f \star K_\sigma)(t) = f(t)\;.
$$
\item\label{item:regularisation2} Si $f$ est continue {\`a} support compact, alors
  $f \star K_\sigma$ converge uniform{\'e}ment vers $f$ quand $\sigma\downarrow0$.
\end{enumerate}
\end{lemma}
\begin{proof}
Comme $\int K_\sigma(x)\, \rmd x= 1$, par un changement de variable {\'e}l{\'e}mentaire,
\begin{equation}
\label{eq:inversionL1-2}
\int f(t-u) K_\sigma(u) \rmd u - f(t) = \int \left\{ f(t - u) - f(t) \right \} K_\sigma(u) \rmd u \eqsp.
\end{equation}
Montrons le point~\ref{item:regularisation1}.
Comme $f$ est continue au point $t$, pour tout $\epsilon > 0$, il existe $\eta$ tel que $|u-t| \leq \eta$
implique que $|f(u) - f(t) | \leq \epsilon$. On a pour tout $\sigma > 0$,
\begin{equation}
\label{eq:conv-local}
\int_{|u| \leq \eta} \left|  f(t-u) - f(t) \right | K_\sigma(u) \rmd u \leq \epsilon \int K_\sigma(u) \rmd u = \epsilon \eqsp.
\end{equation}
De plus,
$$
\int_{|u| \geq \eta} \left| f(t-u) - f(t) \right | K_\sigma(u) \rmd u  \leq \|f\|_1\,\sup_{|u|\geq\eta} {K}_\sigma(u)
+|f(t)|\,\int_{|u|\geq \eta} {K}_\sigma(u) \rmd u  \eqsp.
$$
D'autre part, lorsque $\sigma \downarrow 0$,
$$
\sup_{|u|\geq\eta} {K}_\sigma(u) = \frac1\sigma \sup_{|v|\geq\eta/\sigma} {K}_\sigma(v)\leq  \frac{C}\sigma (1+\eta/\sigma)^{-\alpha}
$$
et,
$$
\int_{u\geq \eta} K_\sigma(u) \rmd u \leq \frac{C}\sigma\int_{|v|\geq\eta/\sigma} (1+|v|)^{-\alpha}\,dv =O(\sigma^{\alpha-2})\eqsp.
$$
Par cons{\'e}quent, en combinant ces majorations avec $\alpha>2$,
\begin{equation}
\label{eq:conv-hors-local}
\lim_{\sigma \downarrow 0} \int_{|u| \geq \eta} \left|  f(t-u) - f(t) \right | K_\sigma(u) \rmd u =0 \eqsp.
\end{equation}
Le r{\'e}sultat suit donc avec~(\ref{eq:inversionL1-2}).

Le point~\ref{item:regularisation2} se montre de fa�on semblable.
Comme $f$ est continue {\`a} support compact, elle est uniform{\'e}ment continue.
On peut donc choisir $\eta>0$ tel que~(\ref{eq:conv-local}) soit valide pour
tout $t$. De m{\^e}me la convergence~(\ref{eq:conv-hors-local}) est uniforme en $t$
en utilisant que $f$ est born{\'e}. D'o{\`u} le r{\'e}sultat.
\end{proof}

Le second principe de r{\'e}gularisation est que la version convolu{\'e}e $f\star K$ de
$f$ adopte la r{\'e}gularit{\'e} du \emph{noyau} $K$. Ce principe est donn{\'e} par le
lemme suivant.

\begin{lemma}\label{lem:regulairsation-est-reguliere}
  Soient $f,g\in L^1$. Alors on a les propri{\'e}t{\'e}s suivantes
  \begin{enumerate}
  \item\label{item:approx-cont-S-1} Si $g$ est $\mathcal{C}^k$, $f\star g$ est
  $\mathcal{C}^k$ et $(f\star g)^{(k)}=f\star g^{(k)}$.
\item\label{item:approx-cont-S-2} On a pour tout entier $p\geq0$,
  \begin{equation}
    \label{eq:decroissance-concolution}
    \|(f\star g)\times m_p\|_\infty\leq \|f\times (1+|m_p|)\|_\infty \;
    \|g\times (1+|m_p|)\|_1  \;,
  \end{equation}
o{\`u} $m_p$ d{\'e}signe le mon{\^o}me de degr{\'e} $p$, $m_p(x)=x^p$.
\item \label{item:approx-cont-S-2prim} Si $f$ et $g$ sont {\`a} d{\'e}croissance rapide,
  il en est de m{\^e}me de $f\star g$.
\item\label{item:approx-cont-S-3}  Si $f$ est {\`a} d{\'e}croissance rapide et
  $g\in\mcs$, $f\star g\in\mcs$.
  \end{enumerate}
\end{lemma}
\begin{proof}
  La propri{\'e}t{\'e}~\ref{item:approx-cont-S-1} est une simple application du lemme
  de d{\'e}rivation sous le signe somme.

Montrons la propri{\'e}t{\'e}~\ref{item:approx-cont-S-2}.
Pour tous $t,x\in\rset$, on a $|x|^p\leq(|x-t|+|t|)^p\leq|x-t|^p+|t|^p$.
D'o{\`u}
$$
|f\star g(x)||x|^p \leq \int |f(x-t)|\,|x-t|^p\,|g(t)|\rmd t+
\int |f(x-t)|\,|t|^p\,|g(t)|\rmd t \;.
$$
On obtient donc~(\ref{eq:decroissance-concolution}).

La propri{\'e}t{\'e}~\ref{item:approx-cont-S-2prim} est obtenu en appliquant
de~\ref{item:approx-cont-S-2} pour tout $p\geq1$.

La propri{\'e}t{\'e}~\ref{item:approx-cont-S-3} d{\'e}coule de \ref{item:approx-cont-S-1}
et~\ref{item:approx-cont-S-2prim}.
\end{proof}


On remarque facilement que l'on peut prendre dans les lemmes pr{\'e}c{\'e}dents
$K=g_1$, c'est-{\`a}-dire $K_\sigma=g_\sigma$, o{\`u} $g_\sigma$ est d{\'e}finie
en~(\ref{eq:gauss_func}). On peut parler dans ce cas de \emph{r{\'e}gularisation
  gaussienne}.

\section{Formules d'inversion}

On est maintenant en mesure de montrer des r{\'e}sultats d'inversion de la
transform{\'e}e de Fourier dans $L_1(\rset)$. La preuve se base sur l'observation
suivante. D'apr{\`e}s~(\ref{eq:gaussienneTF}), $\TF(g_\sigma)$ est une fonction
r{\'e}elle paire de $\mathcal{L}_1(\rset)$ et
$\TF\TF(g_\sigma)=\TFC\TF(g_\sigma)=g_\sigma$. Le r{\'e}sultat suivant est une
premi{\`e}re g{\'e}n{\'e}ralisation de la formule d'inversion ``$\TFC\TF(f)=f$'' qui sera
par la suite {\'e}tendue {\`a} des cadres bien plus g{\'e}n{\'e}raux.

\begin{proposition}\label{prop:inversionFourierL1}
Soit $f\in\mathcal{L}_1(\rset)$ et supposons que $\hat{f}$ appartiennent aussi
{\`a} $\mathcal{L}_1(\rset)$. Alors, en tout point $x$ o{\`u} $f$ est continue, on a
\begin{equation}
\label{eq:inversionFT}
[\bar{\TF} \hat{f}](x) = f(x).
\end{equation}
\end{proposition}
\begin{proof}
On a vu ci-dessus que la fonction $\hat{g}_\sigma(x) = \rme^{- 2 \pi^2 \sigma^2 x^2}$
a ${g}_\sigma$ pour transform{\'e}e de Fourier. Donc
$[\TF(x\mapsto \hat{g}_\sigma(x)\rme^{\rme 2 \pi t x})](\xi)={g}_\sigma(\xi-t)={g}_\sigma(t-\xi)$ par la
proposition~\ref{prop:FourierTranslation} puis par parit{\'e} de $g_\sigma$. La proposition \ref{prop:echangeTF} appliqu{\'e}e avec
$x \mapsto f(x)$ et $x \mapsto \rme^{\rmi 2 \pi t x} \hat{g}_\sigma(x)$ donne donc, pour tout $t\in \rset$,
\begin{equation}
\label{eq:inversionL1-1}
\int \hat{f}(x) \hat{g}_\sigma(x) \rme^{\rmi 2 \pi t x} \rmd x = \int f(u) {g}_\sigma(t-u) \rmd u = f\star g_\sigma(t) \eqsp.
\end{equation}
Lorsque $\sigma \to 0$, on peut passer {\`a} la limite dans l'int{\'e}grale de gauche, puisque l'on a, pour tout
$x$, $\lim_{\sigma \to 0} \hat{g}_\sigma(x)= 1$ pour tout $x$ et $|\hat{f}(x) \hat{g}_\sigma(x) \rme^{\rmi 2 \pi t x}| \leq |\hat{f}(x)|$.
Comme $\hat{f} \in L_1(\rset)$, on applique le th{\'e}or{\`e}me de convergence domin{\'e}e, qui montre
$$
\lim_{\sigma \to 0} \int \hat{f}(x) \hat{g}_\sigma(x) \rme^{ \rmi 2 \pi t x} \rmd x = \int \hat{f}(x) \rme^{\rmi 2 \pi t x} \rmd x \eqsp.
$$
Le passage {\`a} la limite dans le membre de droite de~(\ref{eq:inversionL1-1}) est lui une cons{\'e}quence du
lemme~\ref{lem:regularisation}. On obtient bien le r{\'e}sultat annonc{\'e} si $f$ est continue en $t$.
\end{proof}


Le r{\'e}sultat pr{\'e}c{\'e}dent conjugu{\'e} avec le th{\'e}or{\`e}me~\ref{theo:stabiliteS} permet de
d{\'e}finir une transform{\'e}e inverse comme application r{\'e}ciproque de $\TF$ d{\'e}finie
comme application de $\mcs$ dans $\mcs$.

En effet, si $f$ est un {\'e}l{\'e}ment de $\mcs$, $\hat{f}$ est dans $\mcs$ et donc
int{\'e}grable. $f$ {\'e}tant partout continue, la formule
d'inversion~(\ref{eq:inversionFT}) est valable pour tout $x \in \rset$. Donc,
pour tout $f \in \mcs$, $f = \TFAC{\TF f}$. De la m{\^e}me fa�on, on a $f =
\TFA{\TFC f}$. $\TF$ est donc une bijection sur $\mcs$ et son inverse est
$\TFC$.

\begin{theorem}\label{thm:Schwarz}
La transformation de Fourier $\TF$ est une application lin{\'e}aire bijective de $\mcs$ sur $\mcs$.
L'application inverse est $\TF^{-1} = \TFC$.
\end{theorem}

Ayant montr{\'e} le th{\'e}or{\`e}me~\ref{thm:Schwarz}, de nombreuses formules d'inversion peuvent \^etre d{\'e}duite par dualit{\'e}.
Nous verrons dans la section suivante comment appliquer ce principe dans un cadre hilbertien.
Ici nous l'appliquons dans le cadre $L_1(\rset)$ grace au r{\'e}sultat suivant, qui nous permettra de compl{\'e}ter la
proposition~\ref{prop:inversionFourierL1} par un th{\'e}or{\`e}me d'inversion.
\begin{proposition}\label{prop:DualiteSchwartz}
Soient deux fonctions $f$ et $g$ dans $L_1(\rset)$. Si, pour toute fonction \textit{test} $\phi$ dans $\mcs$, on a
$$
\int f(x)\,\phi(x)\,\rmd x= \int g(x)\,\phi(x)\,\rmd x,
$$
alors $f=g$ (au sens $L_1(\rset)$).
\end{proposition}
\begin{proof}
En prenant la diff{\'e}rence entre les deux membre de l'{\'e}galit{\'e} de l'hypoth{\`e}se, on voit qu'il suffit de montrer ce r{\'e}sultat pour
$g=0$. De plus, comme les fonctions continues sont denses dans l'ensemble des fonctions int�grables, on peut se contenter de prendre $f$ continue, le cas g{\'e}n{\'e}ral {\'e}tant
obtenu par passage {\`a} la limite. Or, pour $f$ continue et $g=0$, le r{\'e}sultat est imm{\'e}diat par application du principe de
r{\'e}gularisation en choisissant une fonction $K\in\mcs$ positive int{\'e}grant {\`a} 1 (par exemple $g_1$) puis en appliquant le
lemme~\ref{lem:regularisation} en tout point de la droite r{\'e}elle.
\end{proof}

On en d{\'e}duit le r{\'e}sultat annonc{\'e} qui compl{\`e}te la proposition~\ref{prop:inversionFourierL1}.
\begin{theorem}
Soit $f\in L_1(\rset)$ et supposons que $\hat{f}$ appartiennent aussi {\`a} $L_1(\rset)$. Alors
la fonction (continue) $\bar{\TF} \hat{f}$ est l'unique
repr{\'e}sentant continu de $f$.
\end{theorem}
\begin{proof}
La conitnuit{\'e} de $\bar{\TF} \hat{f}$ d{\'e}coule du th{\'e}or{\`e}me~\ref{thm:rieman-lebesgue}.
Pour toute fonction test $\phi$ de $\mcs$, on a, d'apr{\`e}s la proposition~\ref{prop:echangeTF} et le
th{\'e}or{\`e}me~\ref{thm:Schwarz},
$$
\int f(x) \phi(x)\,\rmd x=\int \hat{f}(\xi) \TFC(\phi)(\xi)\,d\xi.
$$
Mais comme $\hat{f}$, on peut r{\'e}appliquer l'{\'e}quivalent de la proposition~\ref{prop:echangeTF} mais pour la transform{\'e}e
inverse, ce qui donne alors directement
$$
\int \hat{f}(\xi) \TFC(\phi)(\xi)\,d\xi=\int \TFC(\hat{f})(x) \phi(x) \,\rmd x.
$$
D'o{\`u} le r{\'e}sultat en appliquant la proposition~\ref{prop:DualiteSchwartz}.
\end{proof}

\section{Convolution et transform�e de Fourier}

On conclut se chapitre en explorant l'effet de le transform{\'e}e de Fourier sur la
convolution. Nous explorons ici uniquement le cas de la convolution d{\'e}finie sur
$L^1\times L^1$.

\begin{theorem}%\label{thm:inversionFourierL1}
Etant donn{\'e}es deux fonctions $f$ et $g$ de $L_1(\rset)$ on a:
\begin{enumerate}
\item $f \star g$ est d{\'e}finie presque partout et $f \star g$ appartient {\`a} $L_1(\rset)$.
\item La convolution est un op{\'e}rateur bilin{\'e}aire continu de $L_1(\rset) \times L_1(\rset) \to L_1(\rset)$
tel que
$$
\| f \star g \|_1 \leq \| f \|_1 \| g \|_1
$$
\item la transform{\'e}e de Fourier du produit de convolution $\widehat{f \star g}$ est {\'e}gal au produit des transform{\'e}es
de Fourier des fonctions $\hat{f}$ et $\hat{g}$: $\widehat{f \star g} = \hat{f} \hat{g}$.
\end{enumerate}
\end{theorem}
\begin{proof}
Comme $f,g \in L_1(\rset)$, la fonction $(y,z) \mapsto f(y) g(z) \in L_1(\rset^2)$ d'apr{\`e}s le th{\'e}or{\`e}me de Fubini.
En faisant le changement de variable $y = x-t$ et $z=t$, on obtient:
$$
\iint f(y) g(z) d y d z = \iint f(x-t) g(t) \rmd x \rmd t \eqsp
$$
et les int{\'e}grales des modules sont finies.
La fonction $x \mapsto \int f(x-t) g(t) \rmd t$ est donc d{\'e}finie presque partout et appartient {\`a} $L_1(\rset)$, toujours d'apr{\`e}s le th{\'e}or{\`e}me de Fubini.
La seconde in{\'e}galit{\'e} d{\'e}coule de:
$$
\int | f \star g(x)| \rmd x = \int |g(t)| \left( \int  |f(x-t) \rmd x \right) \rmd t = \|f\|_1 \| g \|_1 \eqsp.
$$
La troisi{\`e}me assertion s'obtient de fa�on similaire par une application du th{\'e}or{\`e}me de Fubini.
\end{proof}

%%% Local Variables:
%%% mode: latex
%%% ispell-local-dictionary: "francais"
%%% TeX-master: "Polycopie-Fourier-L1L2"
%%% End:


\chapter{Transform�e de Fourier dans $\mathcal{S}'$}
\input{fourierSprim}

\chapter{Transform�e de Fourier dans $L^2$}
\section{Espace des fonctions de carr{\'e} int{\'e}grable}
Soit $\mathcal{L}_2(\rset)$ l'espace des fonctions d{\'e}finies sur $\rset$ et
{\`a} valeurs complexes, $f:\rset\rightarrow \cset$, de carr{\'e} sommable
c'est-{\`a}-dire telles que:
$$
\int |f(x)|^2\,\rmd x<\infty \eqsp.
$$
On note $L_2(\rset)$ l'espace des classes d'{\'e}quivalence de
$\mathcal{L}_2(\rset)$ pour la relation d'{\'e}quivalence ``$f=g$ p.p.''.
Pour $I$ un sous ensemble bor{\'e}lien de $\rset$ (et en particulier, un intervalle), on peut d{\'e}finir
de la m{\^e}me fa�on l'espace $\mathcal{L}_2(I)$ des fonctions  de carr{\'e} sommable sur $I$, $\int_I |f(x)|^2 \rmd x < \infty$
et l'espace $L_2(I)$ des classes d'{\'e}quivalence de $\mathcal{L}_2(I)$ par rapport {\`a} la relation d'{\'e}quivalence d'{\'e}galit{\'e} presque-partout.


Pour $f$ et $g \in L_2(\rset)$, d{\'e}finissons.
\begin{equation}
\label{eq:definitionpscal}
\pscal{f}{g}_I =\int_I f(x) \bar{g}(x)\,\rmd x
\end{equation}
o{\`u}, pour tout $z \in \cset$, $\bar{z}$ est le conjugu{\'e} de $z$. Lorsque $I= \rset$, nous omettons  l'indice $I$.
Cette int{\'e}grale est bien d{\'e}finie pour 2 repr{\'e}sentants de $f$ et $g$ car $|f(x) \bar{g}(x)|\leq (|f(x)|+ |\bar{g}(x)|)/2$ et
sa valeur ne d{\'e}pend {\'e}videmment pas du choix de ses repr{\'e}sentant. D'auter part, $L_2(I)$ est bien le plus ``gros'' espace
fonctionnel sur lequel ce produit scalaire est bien d{\'e}fini puisqu'il impose justement $\pscal{f}{f}_I<\infty$.
Mentionnons aussi que, de m\^eme que pour $(L_1(\rset),\|\cdot\|_1)$,  $(L_2(\rset),\|\cdot\|_2)$ est un espace de Banach
(espace vectoriel norm{\'e} complet), o{\`u} la norme $\|\cdot\|_2$ est d{\'e}finie par
$$
\|f\|_2:=\sqrt{\pscal{f}{f}}=\left(\int |f(x)|^2\,\rmd x\right)^{1/2}.
$$
Cette norme {\'e}tant un norme induite par un produit scalaire, on dit que
$(L_2(\rset),\pscal{\cdot}{\cdot})$ est un espace
\textit{de Hilbert}.

\begin{theorem}
  \label{thm:lunDense}
  L'ensemble des fonctions int{\'e}grables et de carr{\'e} int{\'e}grable, $L_1(\rset)\cap
  L_2(\rset)$ est un sous-espace vectoriel dense de $(L_2(\rset),\|\cdot\|_2)$.
\end{theorem}
\begin{proof}
Pour tout $f\in L_2(\rset)$, on note $f_n$ la fonction {\'e}gale {\`a} $f$ sur $[-n,n]$ et nulle ailleurs.
Alors $f_n\in L_1(\rset)$ pour tout $n$, et par convergence monotone, $\|f_n-f\|_2\to0$ quand $n\to\infty$ (on dit que
$f_n$ tend vers $f$ au sens de $L_2(\rset)$. On en conclut que $L_1(\rset)\cap L_2(\rset)$ est dense dans
$(L_2(\rset),\|\cdot\|_2)$.
\end{proof}
On a imm{\'e}diatement que la proposition~\ref{prop:DualiteSchwartz} s'adapte {\`a} l'espace $L_2(\rset)$.
\begin{corollary}  \label{cor:DualiteCalS}
Soient deux fonctions $f$ et $g$ dans $L_2(\rset)$. Si, pour toute fonction \textit{test} $\phi$ dans $\mcs$, on a
$$
\int f(x)\,\phi(x)\,\rmd x= \int g(x)\,\phi(x)\,\rmd x,
$$
alors $f=g$ (au sens $L_2(\rset)$).
\end{corollary}
On  utilisera par ailleurs le r{\'e}sultat suivant qui peut se montrer directement {\`a} partir du th{\'e}or{\`e}me de densit� des
fonctions continues dans $L_1(\rset)$. 

\begin{theorem}
  \label{thm:SDenseltwo}
  L'espace $\mcs$ est un sous-espace vectoriel dense de $(L_2(\rset),\|\cdot\|_2)$.
\end{theorem}

Nous verrons que $L_2(\rset)$ pose un certain nombre de probl{\`e}mes
th{\'e}oriques pour d{\'e}finir la transform{\'e}e de Fourier qui ne se pose pas
pour une fonction de $L_1(\rset)$. Or, en passant de $L_1(\rset)$ {\`a}
$L_2(\rset)$, on impose {\`a} la fonction des conditions locales plus
contraignantes (toute restriction d'une fonction de $L_2(\rset)$ {\`a} un compact
est $L_1(\rset)$ mais l'inverse n'est pas vrai) et on autorise des
comportements en $t=\pm\infty$ un peu plus g{\'e}n{\'e}raux. D{\`e}s lors, on
peut s'interroger sur l'int{\'e}r{\^e}t d'{\'e}tudier les fonctions  de $L_2(\rset)$ plut{\^o}t que de
$L_1(\rset)$, qui plus est quand, en pratique, une fonction n'est
jamais observ{\'e} sur un temps infini. La r{\'e}ponse {\`a} cette question est
la suivante. Outre que les propri{\'e}t{\'e}s d'espace de Hilbert de
$L_2(\rset)$ sont fondamentales dans la th{\'e}orie, elles ont un lien
physique {\'e}vident dans les applications puisque le carr{\'e} de la norme
d'un signal dans $L_2(\rset)$ n'est rien d'autre que son {\'e}nergie.
Le fait qu'en pratique les "signaux" observ{\'e}s soient dans
$L_1(\rset)\cap L_2(\rset)$ explique que l'on peut en g{\'e}n{\'e}ral ne
pas se pr{\'e}occuper des subtilit{\'e}s entre transform{\'e}e de Fourier dans
$L_1(\rset)$ et transform{\'e}e de Fourier dans $L_2(\rset)$, mais,
pour {\'e}tablir les r{\'e}sultats g{\'e}n{\'e}raux que l'on utilise pour {\'e}tudier
les fonctions de carr{\'e} sommable, il serait dommage de les {\'e}noncer dans
le cas particulier $L_1(\rset)\cap L_2(\rset)$ alors qu'ils sont
valables dans $L_2(\rset)$, m{\^e}me si l'on doit pour cela donner des
preuves qui peuvent appara{\^i}tre plus abstraites.

\section{Transform{\'e}e de Fourier sur $L_2(\rset)$}
L'id{\'e}e de base de la construction consiste {\`a} {\'e}tendre la transform{\'e}e de Fourier
de $L_1(\rset)$ {\`a} $L_2(\rset)$ par un argument de densit{\'e}.
\begin{proposition}
\label{prop:plancherelparseval}
Soit $f$ et $g$ dans $\mcs$. On a:
\begin{gather*}
\int \hat{f}(\xi) \bar{\hat{g}}(\xi) d \i = \int f(x) \bar{g}(x) dx \\
\int |\hat{f}(\xi)|^2 d \xi = \int |f(x)|^2 dx \eqsp.
\end{gather*}
\end{proposition}
\begin{proof}
Appliquons la formule d'{\'e}change (Proposition \ref{prop:echangeTF}). On pose $h(\xi)= \bar{\hat{g}}(\xi)$. On a:
$$
\int \hat{f}(\xi) h(\xi) d \xi = \int f(x) \hat{h}(x) dx \eqsp.
$$
Mais $\bar{\hat{g}}(\xi)= \TFC \bar{g}(\xi)$, d'o{\`u} $\hat{h}= \bar{g}$.
\end{proof}

\begin{proposition}
Soient $E$ et $F$ deux espaces vectoriels norm{\'e}s, $F$ complet, et $G$ un sous-espace vectoriel dense dans $E$.
Si $A$ est un op{\'e}rateur lin{\'e}aire continu de $G$ dans $F$, alors il existe un prolongement unique $\tilde{A}$
lin{\'e}aire continu de $E$ dans $F$ et la norme de $\tilde{A}$ est {\'e}gale {\`a} la norme de $A$.
\end{proposition}
\begin{proof}
Soit $f \in E$. Comme $G$ est dense dans $E$, il existe une suite $f_n$ dans $G$ telle que
$\lim_{n \to \infty} \| f_n - f \| = 0$. La suite $f_n$ {\'e}tant convergente, elle est de
Cauchy. $A$ {\'e}tant lin{\'e}aire continu on a
$$
\| A f_n- A f_m \| \leq \| A \| \| f_n- f_m \| \eqsp.
$$
On en d{\'e}duit que $A f_n$ est une suite de Cauchy de $F$ qui est complet.
La suite $A f_n$ est donc convergente vers un {\'e}l{\'e}ment $g$ de $F$.
On v{\'e}rifie facilement que $g$ ne d{\'e}pend pas de la suite $f_n$ et on pose donc $A f = g$.
$\tilde{A}$ est lin{\'e}aire par construction et de plus on a
$$
\|  \tilde{A} f \| =  \lim_{n \to \infty} \| A f_n \| \leq \lim_{n \to \infty} \| A \| \| fn \| = \| A \| \| f \|\eqsp,
$$
ce qui prouve que $\| \tilde{A} \| \leq  \| A \|$. Comme $\tilde{A} f = A f$ pour tout $f \in G$, on a $\| \tilde{A} \| = \| A \|$.
Enfin, $G$ {\'e}tant dense dans $E$, il est clair que $\tilde{A}$ est unique.
\end{proof}
D'apr{\`e}s la proposition~\ref{prop:plancherelparseval}, $\TF$ est une isom{\'e}trie sur $\mcs$ muni du produit scalaire
$\pscal{\cdot}{\cdot}$. On applique le r{\'e}sultat pr{\'e}c{\'e}dent avec $E = F = L_2(\rset)$, $G = \mcs$ (voir le
th{\'e}or{\`e}me~\ref{thm:SDenseltwo}. On obtient
\begin{theorem}\label{thm:prolong}
La transformation de Fourier $\TF$ (respectivement la transformation inverse $\TFC$)
se prolonge en une isom{\'e}trie de $L_2(\rset)$ sur $L_2(\rset)$.
D{\'e}signons toujours par $\TF$ (resp. $\TFC$) ce prolongement. On a en particulier
\begin{enumerate}
\item (Inversion) pour tout $f \in L_2(\rset)$, $\TF \TFC f = \TFC \TF f = f$,
\item (Plancherel) pour tout $f,g \in L_2(\rset)$, $\pscal{f}{g} = \pscal{\TF f}{\TF g}$
\item (Parseval) pour tout $f \in L_2(\rset)$, $\| f \|_2 = \| \TF f \|_2$.
\end{enumerate}
\end{theorem}

Remarquons que l'{\'e}galit{\'e} de Parseval peut se r{\'e}{\'e}crire, pour tout $f$ et $g$ dans $L_2(\rset)$,
\begin{equation}\label{eq:echangeL2}
\pscal{ \TF f}{g} = \pscal{f}{\TF g}
\end{equation}

\begin{proposition}
\label{prop:prolongementL1L2}
Le prolongement de $\TF$ sur $\mcs$ par continuit{\'e} {\`a} $(L_2(\rset),\|\cdot\|_2)$ est compatible avec la d{\'e}finition de
$\TF$ donn{\'e}e pr{\'e}c{\'e}demment dur $L_1(\rset)$. Plus pr{\'e}cis{\'e}ment
\begin{enumerate}
\item Pour tout $f\in L_1(\rset) \cap L_2(\rset)$, $\TF f$ d{\'e}fini par le th{\'e}or{\`e}me~\ref{thm:prolong} admet un repr{\'e}sentant
 $\hat{f}\in C_\infty$ v{\'e}rifiant
$$
\hat{f}(\xi) = \int_{\rset} \rme^{- \rmi 2 \pi \xi x} f(x) dx,\quad\xi\in\rset.
$$
\item Si $f \in L_2(\rset)$, $\TF f$ est la limite dans $L_2(\rset)$ de la suite $g_n$, d{\'e}finie par $g_n(\xi) =
  \int_{-n}^n \rme^{- \rmi 2 \pi \xi x} f(x) dx$.
\end{enumerate}
\end{proposition}
\begin{proof}
Notons $\hat{f}$ la transform{\'e}e de Fourier sur $L_1(\rset)$ et $\TF f$ celle sur $L_2(\rset)$.
Prenons $f \in L_1(\rset) \cap L_2(\rset)$. En appliquant la proposition \ref{prop:echangeTF} puis
Parseval (voir~(\ref{eq:echangeL2})), on a pour tout $\psi \in \mcs$,
$$
\int \psi \hat{f} = \int \hat{\psi} f = \int \TFA{\psi} f= \int \psi \TFA{f}
$$
d'o{\`u} $\int  (\hat{f} - \TFA{f}) \psi  = 0$ pour tout $\psi \in \mcs$. Le corollaire~\ref{cor:DualiteCalS} fournit alors le
premier r{\'e}sultat.

Posons $f_n = f \1_{[-n,n]}$. Par convergence domin{\'e}e, on a $\lim_n \| f_n - f \|_2^2= 0$.
Comme $f_n  \in L_1(\rset) \cap L_2(\rset)$ on {\'e}crit $g_n = \hat{f}_n = \TFA{f_n}$ et par continuit{\'e} il vient
$ \lim_{n \to \infty} \| \TF f - g_n \|_2^2= 0$.
\end{proof}


%%% Local Variables:
%%% mode: latex
%%% ispell-local-dictionary: "francais"
%%% TeX-master: "Polycopie-Fourier-L1L2"
%%% End:




\end{document}

%%% Local Variables:
%%% mode: latex
%%% ispell-local-dictionary: "francais"
%%% TeX-master: "Polycopie-Fourier-L1L2"
%%% End:
