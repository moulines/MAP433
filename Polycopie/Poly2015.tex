\documentclass[a4paper,11pt]{book}

\usepackage[latin1]{inputenc}
\usepackage[french]{babel}
\usepackage{amsfonts,a4wide,amsmath,amssymb,bbm,fancyhdr,makeidx,latexsym,amsthm,amsbsy,amscd,amstext,enumitem}

\usepackage{graphics}
\usepackage{graphicx}
\usepackage{picins}
%\usepackage{showkeys}



\setlength{\parskip}{0.2cm}
%\setlength{\textwidth}{12.4cm}
\setlength{\textwidth}{360pt}
%\setlength{\textwidth}{14.3cm}
\setlength{\textwidth}{14.8cm}
%\setlength{\textheight}{19.3cm}
\setlength{\textheight}{20.4cm}
%\setlength{\textheight}{21.4cm}
\linespread{1}

\voffset = 1cm

\pagestyle{fancy}
% Ceci permet dÕavoir les noms de chapitre et de section
% en minuscules
\renewcommand{\chaptermark}[1]{\markboth{#1}{}}
\renewcommand{\sectionmark}[1]{\markright{\thesection\ #1}}
\fancyhf{} % supprime les en-ttes et pieds prŽdŽfinis
\fancyhead[LE,RO]{\bfseries\thepage}% Left Even, Right Odd
\fancyhead[LO]{\bfseries\rightmark} % Left Odd
\fancyhead[RE]{\bfseries\leftmark} % Right Even
\renewcommand{\headrulewidth}{0.5pt}% filet en haut de page
\addtolength{\headheight}{0.5pt} % espace pour le filet
\renewcommand{\footrulewidth}{0pt} % pas de filet en bas
\fancypagestyle{plain}{ % pages de tetes de chapitre
\fancyhead{} % supprime lÕentete
\renewcommand{\headrulewidth}{0pt} % et le filet
}

\newtheorem{theorem}{Th{\'e}or{\`e}me}[section]
\newtheorem{definition}[theorem]{D{\'e}finition}
\newtheorem{corollary}[theorem]{Corollaire}
\newtheorem{proposition}[theorem]{Proposition}
\newtheorem{lemma}[theorem]{Lemme}
\theoremstyle{break}
\newtheorem{example}[theorem]{Exemple}
\newtheorem{remark}[theorem]{Remarque}


\newtheorem {propriete} {Propri\'et\'e} [chapter]
%\newtheorem {proposition} {Proposition} [chapter]
\renewcommand{\chaptername} {Chapitre}
\renewcommand{\bibname} {Bibliographie}
\renewcommand{\contentsname} {Table des Mati\`eres}

\def \BS {{/}}
\def \om {{\omega}}
\def \omx {{\omega_x}}
\def \omy {{\omega_y}}
\def \omxy {{(\omega_x , \omega_y)}}
\def \Om {{\Omega}}
\def \Omx {{\Omega_x}}
\def \Omy {{\Omega_y}}
\def \emOm {{e ^{-i \Omega}}}
\def \ekOm {{e ^{i k\Omega}}}
\def \ekom {{\rmee ^{\rmi k\lambda}}}
\def \emkOm {{e ^{-i k\Omega}}}
\def \emkom {{e ^{-i k\omega}}}
\def \enOm {{e ^{i n\Omega}}}
\def \enom {{e ^{i n\omega}}}
\def \emnOm {{e ^{-i n\Omega}}}
\def \emnom {{e ^{-i n\omega}}}
\def \eps {{\epsilon}}
\def \R {{\bf{R}}}
\def \N {{\bf{N}}}
\def \Z {{\bf{Z}}}
\def \C {{\bf{C}}}
%\def \C {{\bf  C}}
%\def \N {{\bf  N}}
%\def \Z {{\bf  Z}}
%\def \R {{\bf  R}}
\def \bZ {{\bf  Z}}
\def \bU {{\bf  U}}
\def \bV {{\bf  V}}
\def \bT {{\bf  T}}
\def \bA {{\bf  A}}
\def \ba {{\bf  a}}
\def \bB {{\bf  B}}
\def \bC {{\bf  C}}
\def \E {{\cal  E}}
\def \cX {{\cal  X}}
\def \bX {{\bf  X}}
\def \bh {{\bf  h}}
\def \bR {{\bf  R}}
\def \bG {{\bf  G}}
\def \bY {{\bf  Y}}
\def \bW {{\bf  W}}
\def \X {{\bf  X}}
\def \S {{\cal  S}}
\def \Y {{\bf  Y}}
\def \V {{\bf  V}}
\def \D {{\bf  D}}
\def \bI {{\bf  I}}
\def \bQ {{\bf  Q}}
\def \beps {{\bf  W}}
\def \bSigma {{\bf  R}}
\def \bS {{\bf  S}}
\def \K {{\bf  K}}
\def \I {{\bf  I}}
\def \H {{\bf  H}}
\def \V {{\bf  V}}
\def \Vj {{{\bf  V}_j}}
\def \P {{\bf  P}}
\def \U {{\bf  U}}
\def \L {{\bf  L}}
\def \HU {{\bf  H^1}}
\def \Hn {{\bf  H^n}}
\def \UklN {{ 1 \leq k,l \leq N}}
\def \UkN {{ 1 \leq k \leq N}}
\def \UkP {{ 1 \leq k \leq P}}
\def \ZkN {{ 0 \leq k \leq N}}
\def \ZmM {{ 0 \leq m < M}}
\def \ZmN {{ 0 \leq m < N}}
\def \ZkNU {{ 0 \leq k \leq N-1}}
\def \ZkN {{ 0 \leq k < N}}
\def \ZkM {{ 0 \leq k < M}}
\def \UlN {{ 1 \leq l \leq N}}
\def \UkM {{ 1 \leq k \leq M}}
\def \UkK {{ 1 \leq k \leq K}}
\def \UnN {{ 1 \leq n \leq N}}
\def \UmN {{ 1 \leq m \leq N}}
\def \ZjJ {{ 0 \leq j < J}}
\def \ZnN {{ 0 \leq n < N}}
\def \ZnNU {{ 0 \leq n < N-1}}
\def \jZ {{ j \in {\Z}}}
\def \nZ {{ n \in {\Z}}}
\def \pZ {{ p \in {\Z}}}
\def \kZ {{ k \in {\Z}}}
\def \kN {{ k \in {\N}}}
\def \nN {{ n \in {\N}}}
\def \mZ {{ m \in {\Z}}}
\def \nmZ {{ (n,m) \in {\Z^2}}}
\def \jnZ {{ (j,n) \in {\Z^2}}}
\def \nkZ {{ (n,k) \in {\Z^2}}}
\def \LDD {{\bf L ^ 2 (\R^2)}}
\def \LDP {{\bf L ^ 2 (\rm P)}}
\def \LD {{\bf L ^ 2 (\R)}}
\def \lU {{\bf l ^ 1 (\Z)}}
\def \LU {{\bf L ^ 1 (\R)}}
\def \Lpi {{\bf L^2 {\rm ([-\pi,\pi])}}}
\def \LpiD {{\bf L^2 {\rm ([-\pi,\pi]^2)}}}
\def \LT {{\bf L ^2 [0,T]}}
\def \Ld {{\bf L ^2 }}
\def \ld {{\bf l ^2 (\Z)}}
\def \RDp {{\bf C ^+}}
\def \Del {{\Delta}}
\def \Delx {{\Delta_x}}
\def \Dely {{\Delta_y}}
\def \ga {{\gamma}}
\def \Ga {{\Gamma}}
\def \gaGa {{\gamma \in \Gamma}}

\def \uxi {{u,\xi}}
\def \nuxi {{u_n,\xi_n}}
\def \us {{u,s}}
\def \un {{u_n}}
\def \xin {{\xi_n}}
\def \ml {{m,l}}
\def \nk {{n,k}}
\def \nm {{n,m}}
\def \ij {{i,j}}
\def \ran {{\rm ran}}
\def \sign {{\rm sign}}
\def \arg {{\rm arg}}
\def \Cov {{\rm Cov}}
\def \Reel {{\rm R\acute{e}el}}
\def \Ima {{\rm Ima}}
\def \deg {{\rm deg}}
\def \cistar {{\mbox{$\odot \!\!\!\!\star~$}}}
\def \grad {{\vec {\bigtriangledown}}}
\def \half {{\frac 1 2}}
\def \db {{\rm db}}
\def \nin {{\in\!\!\!/}}



\def\1{\mathbbm{1}}
\def\mcb{\ensuremath{\mathcal{B}}}
\def\mcc{\ensuremath{\mathcal{C}}}
\def\mce{\ensuremath{\mathcal{E}}}
\def\mcf{\ensuremath{\mathcal{F}}}
\def\nset{\ensuremath{\mathbb{N}}}
\def\qset{\ensuremath{\mathbb{Q}}}
\def\rset{\ensuremath{\mathbb{R}}}
\def\zset{\ensuremath{\mathbb{R}}}
\def\cset{\ensuremath{\mathbb{C}}}
\def\rsetc{\ensuremath{\overline{\rset}}}
\def\Xset{\ensuremath{\mathsf{X}}}
\def\Tset{\ensuremath{\mathsf{T}}}
\def\Yset{\ensuremath{\mathsf{Y}}}
\def\rmd{\mathrm{d}}
\def\Qint{\ensuremath{\mathrm{QInt}}}
\def\Int{\ensuremath{\mathrm{Int}}}
\def\eqdef{\ensuremath{\stackrel{\mathrm{def}}{=}}}
\def\eqsp{\;}
\def\lleb{\lambda^{\mathrm{Leb}}}
\newcommand{\rmi}{\mathrm{i}}
\newcommand{\rme}{\mathrm{e}}
\def\supp{\mathrm{supp}}



%notation fourier
\def\1{\mathbbm{1}}
\def\mcb{\ensuremath{\mathcal{B}}}
\def\mcc{\ensuremath{\mathcal{C}}}
\def\mce{\ensuremath{\mathcal{E}}}
\def\mcf{\ensuremath{\mathcal{F}}}
\def\nset{\ensuremath{\mathbb{N}}}
\def\qset{\ensuremath{\mathbb{Q}}}
\def\rset{\ensuremath{\mathbb{R}}}
\def\zset{\ensuremath{\mathbb{R}}}
\def\cset{\ensuremath{\mathbb{C}}}
\def\rsetc{\ensuremath{\overline{\rset}}}
\def\Xset{\ensuremath{\mathsf{X}}}
\def\Tset{\ensuremath{\mathsf{T}}}
\def\Yset{\ensuremath{\mathsf{Y}}}
\def\rmd{\mathrm{d}}
\def\Qint{\ensuremath{\mathrm{QInt}}}
\def\Int{\ensuremath{\mathrm{Int}}}
\def\eqdef{\ensuremath{\stackrel{\mathrm{def}}{=}}}
\def\eqsp{\;}
\def\lleb{\lambda^{\mathrm{Leb}}}
\newcommand{\coint}[1]{\left[#1\right[}
\newcommand{\ocint}[1]{\left]#1\right]}
\newcommand{\ooint}[1]{\left]#1\right[}
\newcommand{\ccint}[1]{\left[#1\right]}


\newcommand{\TF}{\mathcal{F}}
\newcommand{\TFC}{\overline{\mathcal{F}}}
\newcommand{\TFA}[1]{\mathcal{F}\left( #1 \right)}
\newcommand{\TFAC}[1]{\overline{\mathcal{F}}\left( #1 \right)}

\def\TFyield{\stackrel{\mathcal{F}}{\mapsto}}

\def\tore{\mathbb{T}}
\def\btore{\mathcal{B}(\tore)}
\def\espaceproba{(\Omega,\mathcal{A},\PP)}
\def\limn{\lim_{n \rightarrow \infty}}
\newcommand{\ps}{\ensuremath{\text{p.s.}}}
\newcommand{\pp}{\ensuremath{\text{p.p.}}}
\def\cA{\mathcal{A}}
\def\cC{\mathcal{C}}
\def\cL{\mathcal{L}}
\def\cM{\mathcal{M}}
\def\cN{\mathcal{N}}
\def\cO{\mathcal{O}}
\def\cP{\mathcal{P}}
\def\cS{\mathcal{S}}
\newcommand{\filtop}[1]{\operatorname{F}_{#1}}
\def\bfphi{{\boldsymbol{\phi}}}
\def\bfpsi{{\boldsymbol{\psi}}}
\def\bfgamma{{\boldsymbol{\gamma}}}
\def\bfpi{{\boldsymbol{\pi}}}
\def\bfsigma{{\boldsymbol{\sigma}}}
\def\bftheta{{\boldsymbol{\theta}}}
\def\bfhphi{{\hat{\boldsymbol{\phi}}}}
\def\bfhrho{{\hat{\boldsymbol{\rho}}}}
\def\bfhgamma{{\hat{\boldsymbol{\gamma}}}}

\def\ltwo{L_2}
\newcommand{\lone}{\ensuremath{\lone}}

\newcommand{\pltwo}{\ensuremath{\ell^2}}
\def\calG{\mathcal{G}}
\def\calM{\mathcal{M}}
\def\calI{\mathcal{I}}
\def\calH{\mathcal{H}}


\newcommand\BL[1]{\mathrm{BL}(#1)}%bande limit{\'e}e
%Espace de Schwarz
\def\mcs{\ensuremath{\mathcal{S}}}
%produit scalaire
\newcommand{\pscal}[2]{\left\langle #1, #2 \right\rangle}
\newcommand{\proj}[3][]{
\ifthenelse{\equal{#1}{}}{\ensuremath{\operatorname{proj}\left( \left. #2\right|#3\right)}}
{\ensuremath{\operatorname{proj}_{#1}\left( \left. #2 \right|#3\right)}}
}
%espaces engendr{\'e}s
\newcommand{\lspan}{\mathrm{Vect}}
\newcommand{\cspan}{\overline{\mathrm{Vect}}}
\def\oplusperp{\stackrel{\perp}{\oplus}}
\def\ominusperp{\ominus}%\def\ominusperp{\stackrel{\perp}{\ominus}}


%Operation sur les fonctions/distributions

\newcommand{\translation}{\mathcal{T}}
\newcommand{\multiplication}{\mathcal{M}}


%
\def\Rset{\mathbb{R}}
\def\Cset{\mathbb{C}}
\def\Zset{\mathbb{Z}}
\def\Nset{\mathbb{N}}
\def\Tset{\mathrm{T}}
% et d'autres
\newcommand{\vvec}[1]{\mathbf{#1}}
\newcommand{\signe}{\mathrm{sgn}}
\newcommand{\rect}{\mathrm{rect}}
\newcommand{\sinc}{\mathrm{sinc}}
\newcommand{\cov}{\mathrm{cov}}
\newcommand{\corr}{\mathrm{corr}}
\newcommand{\vp}{\mathrm{vp}}
\newcommand{\erf}{\mathrm{erf}}
\def\mod{{\ \rm mod\ }}
\def\cF{\mathcal{F}}
\def\cE{\mathcal{E}}
\def\cB{\mathcal{B}}
\def\cH{\mathcal{H}}
\def\cG{\mathcal{G}}
\def\cI{\mathcal{I}}
\def\PP{\mathbb{P}}
\newcommand\PE[1]{{\mathbb E}\left[ #1 \right]}
\newcommand{\Var}[1]{\mathrm{Var}\left( #1 \right)}
\def\BB{\mathrm{B.B.}}
\def\BBF{\mathrm{B.B.F.}}
\newcommandx{\norm}[2][2=]{\Vert #1 \Vert_{#2}}
\def\L1loc{L_{1,\mathrm{loc}}}
\def\Leb{\mathrm{Leb}}


\newcommandx\sequence[3][2=,3=]
{\ifthenelse{\equal{#3}{}}{\ensuremath{\{ #1_{#2}\}}}{\ensuremath{\{ #1_{#2}, \eqsp #2 \in #3 \}}}}
\newcommandx\sequencePar[3][2=,3=]
{\ifthenelse{\equal{#3}{}}{\ensuremath{\{ #1({#2})\}}}{\ensuremath{\{ #1({#2}), \eqsp #2 \in #3 \}}}}
\def\pp{\ensuremath{\mathrm{p.p.}}}
\def\ie{i.e.} 

\title{Traitement du signal\\
Partie II}
\author{St\'ephane Mallat, Eric Moulines}
\date{\ }
\begin{document}
\maketitle
\tableofcontents
\chapter{Introduction}

Initialement appliqu\'e aux t\'el\'ecommunications, le traitement
du signal se retrouve \`a pr\'esent 
dans tous les domaines n\'ecessitant d'analyser et transformer
de l'information num\'erique.
La manipulation de donn\'ees obtenues par capteurs 
bio-m\'edicaux, lors d'exp\'eriences physiques ou biologiques, 
sont aussi des probl\`emes de traitement du signal.
Le t\'el\'ephone, la radio et la t\'el\'evision ont motiv\'e l'\'elaboration
d'algorithmes de filtrage lin\'eaires permettant de coder
des sons ou des images, de les transmettre, et de supprimer 
certains bruits de transmission. 
Le chapitre \ref{analogique-chap}
introduit le filtrage analogique avec une
application \`a la transmission par
modulation d'amplitude.

Lorsque la rapidit\'e de calcul le permet,
le filtrage par circuits d'\'electronique
analogique est remplac\'e par des calculs num\'eriques
sur des signaux digitaux. Le calcul digital est en effet plus
fiable et offre une flexibilit\'e algorithmique bien plus 
importante.
C'est pourquoi les disques et 
cassettes analogiques ont r\'ecemment \'et\'e remplac\'es
par les disques compacts num\'eriques et les cassettes
digitales. La conversion analogique-digitale est \'etudi\'ee
dans le chapitre \ref{discret-chap} 
ainsi que l'extension des op\'erateurs de filtrage
aux signaux discrets. L'introduction de
la transform\'ee de Fourier discr\`ete rapide 
par Cooley et Tuckey en 1965
a fait de l'analyse de Fourier un outil algorithmique
puissant qui se retrouve dans la
plupart des calculs rapides de traitement du signal.

Lorsque l'on veut d\'ecrire les propri\'et\'es d'une classe de signaux,
comme un m\^eme son prononc\'e par diff\'erents locuteurs,
il est utile de se placer dans un cadre probabiliste.
La vari\'et\'e des signaux d'une telle classe peut en effet
\^etre caract\'eris\'ee par un processus al\'eatoire.
La mod\'elisation de signaux par proc\'essus stationnaires est
introduite dans le
chapitre \ref{aleatoire-chap} 
ou nous \'etudions plus particuli\`erement
les mod\`eles autor\'egressifs.
L'estimation lin\'eaire, la suppression de bruit
et la pr\'ediction sont \'etudi\'ees dans le chapitre 
\ref{wiener-chap} \`a travers le filtrage de Wiener. 


La notion d'information contenue dans un signal peut se formaliser
par la th\'eorie de Shannon qui la relie au nombre de bits minimum
pour coder le signal. Le chapitre \ref{comp-code-chap}
en d\'eduit des algorithmes de compression 
qui suppriment la redondance interne
d'un signal et le repr\'esentent avec un nombre de bits r\'eduit.
De tels algorithmes augmentent consid\'erablement les capacit\'es de
stockage, et permettent de transmettre
des signaux \`a travers des canaux \`a d\'ebits r\'eduits. 
Ainsi, le nouveau 
standard de compression pour la t\'el\'evision haute d\'efinition peut
diffuser une image de bien plus haute r\'esolution spatiale
et temporelle, par les m\^emes canaux de transmission 
que la t\'el\'evision actuelle.

Cependant, la th\'eorie de Shannon ne permet pas d'extraire
l'information ``utile'' d'un signal.
La reconnaissance de la parole a 
motiv\'e un grand nombre de travaux
sur ce sujet.
La performance des syst\`emes de reconnaissance de parole
a progress\'e beaucoup plus lentement que les 
projections optimistes des ann\'ees 50.
Les algorithmes de traitement doivent
s'adapter au contenu tr\`es complexe du
signal, et sont donc beaucoup plus sophistiqu\'es que des filtrages
lin\'eaires homog\`enes. 
Le chapitre \ref{parole-chap} \'etudie
l'application des mod\`eles autor\'egressifs, et 
l'extraction d'information
locale en temps et en fr\'equence,
gr\^ace \`a une transform\'ee de Fourier \`a fen\^etre.



\chapter{Transform{\'e}e de Fourier: propri{\'e}t{\'e}s de base}

\section{Transform{\'e}e de Fourier sur $L_1(\rset)$}

Nous abordons dans cette partie la d{\'e}finition et les propri{\'e}t{\'e}s de la transform{\'e}e de Fourier des fonctions.
\begin{definition}[Transform{\'e}e de Fourier]
Soit $f \in L_1(\rset)$. On pose, pour tout $\xi\in\rset$,
\begin{align}
\label{eq:TFL1}
&[\TFA{f}](\xi)= \hat{f}(\xi) = \int_{\rset} \rme^{- \rmi 2 \pi \xi x} f(x) \rmd x \\
&[\TFAC{f}](\xi) = \int_{\rset} \rme^{ \rmi 2 \pi \xi x} f(x) \rmd x
\end{align}
On appelle $\TFA{f}$ (not{\'e} aussi $\TF f$) la \emph{Transform{\'e}e de Fourier} de $f$ et $\TFAC{f}$  (not{\'e} aussi $\TFC f$) la
transform{\'e}e de Fourier conjugu{\'e}e de $f$.
\end{definition}
Cette int{\'e}grale a un sens pour $f \in L_1(\rset)$, parce que $x\mapsto \rme^{- \rmi 2 \pi \xi x} f(x)$ est alors aussi dans
$L_1(\rset)$ pour tout $\xi\in\rset$.
\begin{example}
Soit $f = \1_{[a,b]}(x)$, la fonction indicatrice de l'intervalle $[a,b]$. Un calcul imm{\'e}diat montre que
$$
\hat{f}(\xi) =
\begin{cases}
b-a & \quad \xi = 0, \\
\frac{\sin \pi (b-a) \xi}{\pi \xi} \rme^{- \rmi \pi (a+b) \xi} & \quad \xi\neq0\eqsp.
\end{cases}
$$
On remarque que $\hat{f} \not \in L_1(\rset)$. En revanche c'est une fonction continue born{\'e}e
telle que $\lim_{|\xi|\to \infty} \hat{f}(\xi)= 0$.
\end{example}
En fait les propri{\'e}t{\'e}s d{\'e}crites pour $\hat{f}$ dans cet exemple sont des propri{\'e}t{\'e}s g{\'e}n{\'e}rales des
fonctions de $\TF(L^1(\rset))$ comme le montre le r{\'e}sultat suivant.

\begin{theorem}[Rieman-Lebesgue]
\label{thm:rieman-lebesgue}
Etant donn{\'e} $f \in L_1(\rset)$ on a
\begin{enumerate}
\item $\TF f$ est une fonction continue et born{\'e}e sur $\rset$,
\item $\TF$ est un op{\'e}rateur lin{\'e}aire et continu de $(L_1(\rset),\|\cdot\|_1)$ dans $(C_\infty,\|\cdot\|)$ (l'espace des
  fonctions continues munie de la norme sup) et $\| \hat{f} \| \leq \| f \|_1$,
\item $\lim_{|\xi| \to \pm \infty} |\hat{f}(\xi) |= 0 $.
\end{enumerate}
\end{theorem}
\begin{proof}
\begin{enumerate}
\item La fonction $\xi \mapsto \rme^{- \rmi 2 \pi \xi x} f(x)$ est continue sur $\rset$
  et major{\'e}e en module par $|f(x)|$ (qui ne d{\'e}pend pas de $\xi$), qui est dans $L_1(\rset)$. On conclut en appliquant le th\'eor\`eme de convergence domin\'ee.
\item La lin{\'e}arit{\'e} de $\TF$ d{\'e}coule directement de la lin\'earit\'e de l'int{\'e}grale. Pour tout $\xi \in \rset$, on a $|\hat{f}(\xi)| \leq
  \int |f(x)| \rmd x = \| f \|_1$. On en d{\'e}duit que $\hat{f}$ est born{\'e}e par $\| f\|_1$ et que $\TF$ est continue.
\item Supposons tout d'abord que $f$ est continue {\`a} support compact (il existe $M>0$ tel que $f(x)=0$ si $|x|>M$). Par
  changement de variable $x=t-1/(2\xi)$ dans~\eqref{eq:TFL1}, on a, pour tout $\xi\neq0$,
$$
\hat{f}(\xi) = \int_{\rset} \rme^{- \rmi 2 \pi \xi t - \rmi \pi} f(t+1/(2\xi)) \rmd t =
- \int_{\rset} \rme^{- \rmi 2 \pi \xi t} f(t+1/(2\xi)) \rmd t \eqsp.
$$
D'o{\`u} l'expression, en sommant cette {\'e}quation avec~(\ref{eq:TFL1}),  pour tout $\xi\neq0$,
$$
2 \hat{f}(\xi) = \int_{\rset} \rme^{- \rmi 2 \pi \xi x} (f(x)-f(x+1/(2\xi))) \,\rmd x
$$
Il s'en suit, par convergence domin{\'e}e (en observant que $|f(x)-f(x+1/(2\xi))|$ est major{\'e} ind{\'e}pendamment de $x$ et $\xi$ et
est nul pour $x\notin[-M-1,M+1]$ pour tout $|\xi|\geq1$),
$$
\lim_{|\xi| \to \pm \infty} |\hat{f}(\xi)| \leq \frac12 \lim_{|\xi| \to \pm \infty}
\int_{\rset} |f(x)-f(x+1/(2\xi))| \,\rmd x = 0.
$$
Soit maintenant $f \in L_1(\rset)$. Il existe une suite $\{g_n\}$ de fonctions continues dans $L_1(\rset)$ telles que $\|f - g_n \|_1 \to 0$. Comme,
$\|\hat{f} - g_n\|_\infty \leq \| f - g_n \|_1$ et $\lim_{\xi \to \pm \infty} g_n(\xi) = 0$,
on en d{\'e}duit ais{\'e}ment que $\lim_{\xi  \to \pm \infty} \hat{f}(\xi)= 0$.
\end{enumerate}
\end{proof}

La propri{\'e}t{\'e} {\`a} la fois la plus imm{\'e}diate et la plus  fondamentale de la transform{\'e}e de Fourier est son effet sur les
translations.
\begin{proposition}[Transform{\'e}e de Fourier et translation]
\label{prop:FourierTranslation}
Soit $f\in L_1(\rset)$. Alors, pour tout $t\in\rset$, les fonctions $x\mapsto f(x-x_0)$ et
$x\mapsto \rme^{\rmi 2 \pi \xi_0 x}f(x)$ sont dans $L_1(\rset)$ et v{\'e}rifient
\begin{enumerate}
\item $[\TFA{x\mapsto f(x-x_0)}](\xi)=\rme^{- \rmi 2 \pi x_0}\hat{f}(\xi)$ pour tout $\xi\in\rset$;
\item $[\TFA{x\mapsto  \rme^{\rmi 2 \pi \xi_0 x}f(x)}](\xi)=\hat{f}(\xi-\xi_0)$ pour tout $\xi\in\rset$.
\end{enumerate}
\end{proposition}
\begin{proof}
La preuve \'el\'ementaire est laiss\'ee aux lecteurs.
\end{proof}

Une des propri{\'e}t{\'e}s remarquables de la transform{\'e}e de Fourier est d'{\'e}changer la \emph{d{\'e}rivation} et la multiplication par un mon{\^o}me
\begin{proposition}[Transform{\'e}e de Fourier et D{\'e}rivation]
\label{prop:FourierDerivation}
Soit $n$ un entier naturel.
\begin{enumerate}
\item Si $x\mapsto x^k f(x)$ est dans $L_1(\rset)$  pour tout $k=0,1, \dots, n$, alors $\hat{f}$ est $n$ fois contin\^ument
  d{\'e}rivable et on a
$$
\hat{f}^{(n)} = \TFA{x\mapsto(-2 \rmi \pi x)^n f(x)}
$$
\item Si $f$ est $n$ fois contin\^ument d{\'e}rivables avec $f^{(k)} \in L_1(\rset)$ pour tout $k=0,1, \dots, n$, alors
$$
[\TF(f^{(n)})](\xi) = (2 \rmi \pi \xi)^n \hat{f}(\xi) \quad\text{pour tout $\xi\in\rset$}\eqsp.
$$
\end{enumerate}
\end{proposition}
\begin{proof}
Dans les deux cas, il suffit de d{\'e}montrer le r{\'e}sultat pour $n=1$ puis d'appliquer une r{\'e}currence {\'e}vidente.
\begin{enumerate}
\item La fonction $h:\xi \mapsto \rme^{- \rmi 2 \pi \xi x} f(x)$ est contin\^ument d{\'e}rivable
et $h'(\xi)= -2 \rmi \pi x \rme^{- \rmi 2 \pi \xi x} f(x)$. De plus $|h'(\xi)| \leq 2 \pi |xf(x)|$.
Le r{\'e}sultat d{\'e}coule du th{\'e}or{\`e}me de d{\'e}rivation sous le signe somme.
\item Comme $f' \in L_1(\rset)$, on peut calculer $\TFA{f'}$ par la formule
$$
[\TFA{f'}](\xi) = \lim_{a \to \infty} \int_{-a}^a \rme^{- \rmi 2 \pi \xi x} f'(x) \rmd x, \quad\xi\in\rset\eqsp.
$$
De plus, par int{\'e}gration par parties, pour tout $\xi\in\rset$ et tout  $a>0$,
$$
\int_{-a}^{+a} \rme^{- \rmi 2 \pi \xi x} f'(x) \rmd x = [ \rme^{- \rmi 2 \pi \xi x} f(x) ]_{-a}^a + \int_{-a}^a (2 \rmi \pi \xi) \rme^{- \rmi 2 \pi \xi x} f(x) \rmd x \eqsp.
$$
Comme $f' \in L_1(\rset)$ et $f(a) = f(0) + \int_0^a f'(t) \rmd x$, $\lim_{a \to \infty} \int_0^a f'(t) \rmd x$ existe et donc
$\lim_{a \to \infty} f(a) $ existe. Cette limite est n{\'e}cessairement nulle car $f \in L_1(\rset)$. De la m{\^e}me fa\c{c}on,
$\lim_{a \to \infty} f(-a) = 0$. D'o{\`u} le r{\'e}sultat.
\end{enumerate}
\end{proof}

La proposition suivante sera tr{\`e}s utile pour {\'e}tablir des formules d'inversion de la transform{\'e}e de Fourier.
\begin{proposition}
\label{prop:echangeTF}
Soit $f$ et $g$ deux fonctions de $L_1(\rset)$. Alors $f \hat{g}$ et $\hat{f}g$ sont dans $L_1(\rset)$ et on a
\begin{equation}
\label{eq:echange}
\int f(x) \hat{g}(x) \rmd x = \int \hat{f}(x) g(x) \rmd x \eqsp.
\end{equation}
\end{proposition}
\begin{proof}
Comme $\hat{g} \in L_\infty(\rset)$, les fonctions $f \hat{g}$ et
$\hat{f} g$ appartiennent {\`a} $L_1(\rset)$. Comme la fonction $(t,x)
\mapsto \rme^{- \rmi 2 \pi t x} f(t) g(x) \in L_1(\rset^2)$, il
r{\'e}sulte du th{\'e}or{\`e}me de Fubini que
\begin{multline*}
\int f(t) \hat{g}(t) dt = \int f(t) \left( \int \rme^{- \rmi 2 \pi t x} g(x) \rmd x \right) dt =
\\ \int g(x) \left( \int \rme^{- \rmi 2 \pi t x} f(t) dt \right) \rmd x = \int g(x) \hat{f}(x) \rmd x\eqsp.
\end{multline*}
\end{proof}


\section{D{\'e}croissance et d{\'e}rivation}
%Nous avons observ{\'e} dans la partie pr{\'e}c{\'e}dente la n{\'e}cessit{\'e} de restreindre l'espace $L_1(\rset)$
%pour pour d{\'e}finir la transform{\'e}e de Fourier inverse.
Nous avons observ{\'e} dans la partie pr{\'e}c{\'e}dente d{\`e}s le premier exemple de calcul de transform{\'e}e de Fourier
que $L_1(\rset)$  n'est pas stable sous l'effet de $\TF$.
Nous allons introduire un sous-espace de $L_1(\rset)$ stable par transformation de Fourier,
d{\'e}rivation et multiplication par un polyn{\^o}me. Cet espace introduit par Laurent Schwartz et que l'on notera $\mcs$ joue un r\^ole essentiel en analyse de Fourier.

\begin{definition}[Fonction {\`a} d{\'e}croissance rapide]\index{Fonction|{\`a}
    d{\'e}croissance rapide}
Une fonction $f : \rset \to \cset$ est dite {\`a} \emph{d{\'e}croissance rapide} si, pour tout $p \in \nset$,
on a
$$
\lim_{|x| \to \infty} |x|^p |f (x)| = 0 \eqsp.
$$
\end{definition}
C'est le cas par exemple de $f (x) = \rme^{-|x|}$.
Mais on notera que contrairement {\`a} son nom, cette d{\'e}finition n'implique aucune monotonie pour
$f$ m{\^e}me dans un voisinage de l'infini (prendre par exemple $f (x) = \rme^{-|x|} \sin x$).
Une propri{\'e}t{\'e} utile sur l'int{\'e}grabilit{\'e} des fonctions {\`a} d{\'e}croissance rapide est la suivante.
\begin{lemma}
\label{lem:decroissancerapide}
 Si $f$ est une fonction de $L_{1\mathrm{loc}}(\rset)$ {\`a} d{\'e}croissance rapide alors pour tout
 $p \in \nset$, $x \mapsto x^p f (x)$ appartient {\`a} $L_1(\rset)$.
\end{lemma}
\begin{proof}
L'indice ``loc'' signifie que la restriction de $f$ {\`a} tout compact est dans $L_1(\rset)$.
 $f$ {\'e}tant {\`a} d{\'e}croissance rapide, il existe $M > 0$ tel que pour tout $|x| \geq  M$,
on ait $|x|^{p+2} |f(x)| \leq  1$. D'o{\`u}
\begin{align*}
\int |x^p f(x)| \rmd x &\leq \int_{|x| \leq M}  |x|^p |f(x)| \rmd x+ \int_{|x| > M} |x|^{-2} |x^{p+2}  f(x)| \rmd x \\
&\leq  M^p \int_{|x| \leq M} |f(x)|\, \rmd x +  \int_{|x| > M}   x^{-2} \rmd x < \infty\eqsp.
\end{align*}
\end{proof}
On en d{\'e}duit une propri{\'e}t{\'e} remarquable de la transform{\'e}e de Fourier des fonctions {\`a} d{\'e}croissance rapide.
\begin{proposition}
\label{prop:1913}
Soit $f$ une fonction de $L_1(\rset)$ {\`a} d{\'e}croissance rapide. Alors $\hat{f}$ est ind{\'e}finiment d{\'e}rivable.
\end{proposition}
\begin{proof}
D'apr{\`e}s la proposition \ref{prop:FourierDerivation}, $\hat{f}$ est dans $C_\infty(\rset)$ d{\'e}s que, pour tout $p \in  \nset$,
$x^p f (x)$ est dans $L_1(\rset)$; ce qui est assur{\'e} par  le lemme \ref{lem:decroissancerapide}.
\end{proof}
Inversement si $f$ est dans $C_\infty(\rset)$ quelles propri{\'e}t{\'e}s poss{\`e}de $\hat{f}$ ? Le r{\'e}sultat suivant am{\`e}ne un {\'e}l{\'e}ment de r{\'e}ponse.

\begin{proposition}
\label{prop:1914}
Soit $f$ une fonction de $C_\infty(\rset)$. Si pour tout $k \in \nset$, $f^{(k)}$ est dans $L_1(\rset)$
alors $\hat{f}$ est {\`a} d{\'e}croissance rapide.
\end{proposition}
\begin{proof}
D'apr{\`e}s la proposition \ref{prop:FourierDerivation} on a, pour tout $k \in \nset$,
$\widehat{f^{(k)}}(\xi) = (2 \rmi \pi \xi)^k \hat{f}(\xi)$.
En appliquant le th{\'e}or{\`e}me de Riemann-Lebesgue,  il vient $\lim_{|\xi| \to \infty} |\xi|^k |\hat{f}(\xi)| = 0$.
\end{proof}
Autrement dit nous venons de voir que
\begin{enumerate}
\item plus $f$ d{\'e}cro{\^i}t rapidement {\`a} l'infini, plus $\hat{f}$ est r{\'e}guli{\`e}re;
\item plus $f$ est r{\'e}guli{\`e}re, plus $\hat{f}$ d{\'e}cro{\^i}t rapidement {\`a} l'infini.
En particulier si $f \in C_\infty(\rset)$ et est {\`a} d{\'e}croissance rapide, il en est de m{\^e}me pour
$\hat{f}$.
\end{enumerate}

\section{Un exemple remarquable}

Nous allons consid{\'e}rer une famille de fonctions  qui reste stable par transformation de Fourier.

On introduit pour tout $\sigma > 0$ la fonction de \emph{densit{\'e} gaussienne}
\index{Densit{\'e} gaussienne}
\begin{equation}
  \label{eq:gauss_func}
  g_\sigma(x)=\frac{1}{\sigma \sqrt{2 \pi} }\rme^{- \xi^2/2 \sigma^2}.
\end{equation}


\begin{lemma}\label{lem:gaussenneTF}
La fonction  $g_1(x)= 1 / \sqrt{2\pi} \exp( -x^2/2)$ est la densit{\'e} d'une probabilit{\'e} sur $\rset$, et sa transform{\'e}e de Fourier
est $\hat{g_1}(\xi)= \rme^{- 2 \pi^2 \xi^2}$.
\end{lemma}
\begin{proof}
La fonction  est positive et v{\'e}rifie  $\int_\rset g_1(x) \rmd x= 1$ (on peut le montrer en exercice en {\'e}crivant le carr{\'e} de
l'int{\'e}grale comme une double int{\'e}grale).

La fonction $x\mapsto xg_1(x)$ {\'e}tant aussi dans $L_1(\rset)$, on peut appliquer la
proposition~\ref{prop:FourierDerivation}(i), et on obtient, pour tout $\xi\in\rset$,
$$
\hat{g_1}'(\xi) = - 2\rmi \pi \int x \,g_1(x) \rme^{- \rmi 2 \pi \xi x} \rmd x \eqsp.
$$
Un calcul imm{\'e}diat donne $g_1'(x)= -xg_1(x)$; une int{\'e}gration par partie donne donc
$$
\hat{g_1}'(\xi) = - 2\rmi \pi  \int g_1'(x) \,(- \rmi 2 \pi \xi)\,\rme^{- \rmi 2 \pi \xi x} \rmd x \eqsp.
$$
D'o{\`u} l'on tire finalement que $\hat{g}_1'(\xi) = - 4 \pi^2 \xi \hat{g}_1(\xi)$.
La solution g{\'e}n{\'e}rale de l'{\'e}quation diff{\'e}rentielle {\`a} variables s{\'e}parables  $f'(u)= - 4 \pi^2 u f(u)$
{\'e}tant $f(u) = C \rme^{- 2 \pi^2 u^2}$, et comme on a  $\hat{g_1}(0) = \int g_1(x) \rmd x = 1$,
on voit que n{\'e}cessairement $\hat{g_1}(\xi)= \rme^{-2 \pi^2 \xi^2}$.
\end{proof}

Par un changement de variable {\'e}vident, ce r{\'e}sultat se g{\'e}n{\'e}ralise ais{\'e}ment {\`a} tout $\sigma>0$. En particulier,
\begin{equation}\label{eq:gaussienneTF}
\hat{g_\sigma}(\xi) =  \rme^{-2 \pi^2 (\xi\sigma)^2} = \frac1{\sigma\sqrt{2\pi}}\,g_{1/2\pi\sigma}(\xi),\quad\xi\in\rset.
\end{equation}

Comme annonc\'e plus haut, la famille $(g_\sigma)_{\sigma>0}$ est donc stable par
transform{\'e} de Fourier.
\section{Quelques exemples classiques}
Rappelons que  $u$ est la fonction de Heaviside, d\'efinie par $u(x)=1$ pour $x>0$ et $u(x)=0$ pour $x\leq 0$.
\begin{enumerate}[label=(\roman*)]
\item $f_{1}(x)=\rme^{-ax}u(x)$ , ${\rm Re}(a)>0.$
$$
\hat{f_{1}}(\xi)=\int_{0}^{\infty}\rme^{-2i\pi x\xi}\rme^{-ax}dx=\lim_{b\rightarrow+\infty}[\frac{-\rme^{-x(a+2i\pi\xi)}}{a+2i\pi\xi}]_{0}^{b}=\frac{1}{a+2i\pi\xi}.
$$
\item $f_{2}(x)=\rme^{ax}u(-x)$ , ${\rm Re}(a)>0.$
$$
\hat{f_{2}}(\xi)=\int_{-\infty}^{0}\rme^{-2i\pi x\xi}\rme^{ax}dx=\frac{-1}{-a+2i\pi\xi}.
$$
\item $f_{3}(x)=\displaystyle \frac{x^{k}}{k!}\rme^{-ax}u(x)$ , ${\rm Re}(a)>0.$

$f_{3}(x)=\displaystyle \frac{1}{(-2i\pi)^{k}}\frac{1}{k!}(-2i\pi x)^{k}f_{1}(x)$ , and $\displaystyle \hat{f_{3}}(\xi)=\frac{1}{k!}\frac{1}{(-2i\pi)^{k}}\hat{f_{1}}^{(k)}(\xi)$. Comme $\hat{f_{1}}^{(k)}(\xi)=k!(a+2i\pi\xi)^{-(k+1)}(-2i\pi)^{k},$
$$
\hat{f_{3}}(\xi)=\frac{1}{(a+2i\pi\xi)^{k+1}}.
$$
\item $f_{4}(x)=\displaystyle \frac{x^{k}}{k!}\rme^{ax}u(-x)$ , ${\rm Re}(a)>0.$
Nous avons
$$
\hat{f_{4}}(\xi)=\frac{-1}{(-a+2i\pi\xi)^{k+1}}.
$$
\item $f_{5}(x)=\rme^{-a|x|}, {\rm Re}(a)>0$. Nous d\'eduisons des calculs pr\'ec\'edents
$$
\hat{f_{5}}(\xi)=\frac{2a}{a^{2}+4\pi^{2}\xi^{2}}.
$$
\item $f_{6}(x)=\mathrm{sign}(x)\rme^{-a|x|}, {\rm Re}(a)>0$. Nous avons
$$
\hat{f_{6}}(\xi)=\frac{-4i\pi\xi}{a^{2}+4\pi^{2}\xi^{2}}.
$$
\end{enumerate}
\section{Formulaire}
\label{sec:formulaire}
\begin{enumerate}[label=(\roman*)]
\item
\begin{align*}
\hat{f}^{(k)}(\xi)&=[(-2\rmi\pi x)^{k}f(x)]^{\sim}(\xi)\\
\widehat{f^{(k)}}(\xi)&=(2\rmi \pi\xi)^{k}\hat{f}(\xi)
\end{align*}
\item
\begin{align*}
f(x-a) &\TFyield \rme^{-2\rmi \pi a\xi}\hat{f}(\xi) \\
\rme^{2 \rmi \pi ax}f(x)&\TFyield \hat{f}(\xi-a) \\
\end{align*}
\item $a\neq 0$.
$$ f(ax)\TFyield \frac{1}{|\xi|}\hat{f}(\frac{\xi}{a})$$
\item  $a\in \cset$, ${\rm Re}(a)>0, k=0,1,2\ldots.$
\begin{align*}
\frac{x^{k}}{k!}\rme^{-ax}u(x) &\TFyield \frac{1}{(a+2i\pi\xi)^{k+1}} \\
\frac{x^{k}}{k!}\rme^{ax}u(-x) &\TFyield \frac{-1}{(-a+2i\pi\xi)^{k+1}} \\
\rme^{-a|x|} &\TFyield \frac{2a}{a^{2}+4\pi^{2}\xi^{2}} \\
\mathrm{sign}(x)\rme^{-a|x|} &\TFyield \frac{-4i\pi\xi}{a^{2}+4\pi^{2}\xi^{2}} \\
\end{align*}
\item  $a\in \rset$, $a>0$.
\begin{align*}
\rme^{-ax^{2}}  &\TFyield \sqrt{\frac{\pi}{a}}\rme^{-\frac{\pi^{2}}{a}\xi^{2}} \\
\1_{\ccint{-a,+a}}(x) &\TFyield \frac{\sin 2a\pi\xi}{\pi\xi}
\end{align*}
\end{enumerate}
%%% Local Variables:
%%% mode: latex
%%% ispell-local-dictionary: "francais"
%%% TeX-master: "Polycopie-Fourier-L1L2"
%%% End:

\chapter{Convolution et régularisation}
\section{Convolution}

\section{L'espace de Schwarz}

Nous avons vu comment les propri{\'e}t{\'e}s de d{\'e}croissances se traduisent par des
propri{\'e}t{\'e}s de d{\'e}rivabilit{\'e} sur la transform{\'e}e de Fourier, et \textit{vice
  versa}.
Afin de construire un espace fonctionnel stable par transformation de Fourier,
il est donc naturel d'introduire un espace contenant des fonctions comportant
{\`a} la fois ces deux propri{\'e}t{\'e}s.

\index{Espace de Schwarz}
\begin{definition} On d{\'e}signe par $\mcs(\rset)$ ou tout simplement $\mcs$
l'espace vectoriel des fonctions de $\rset$ dans $\cset$ qui v{\'e}rifient les deux propri{\'e}t{\'e}s suivantes:
\begin{enumerate}
\item $f$ est ind{\'e}finiment d{\'e}rivable sur $\rset$;
\item $f$ est {\`a} d{\'e}croissance rapide, ainsi que toutes ses d{\'e}riv{\'e}es.
\end{enumerate}
On appelle $\mcs(\rset)$ l'\emph{espace de Schwarz}.
\end{definition}

Donnons quelques exemples importants de fonctions appartenant {\`a} $\mcs(\rset)$.
\begin{example}
  Il est facile de v{\'e}rifier que la fonction $f$ d{\'e}finie par $f(x)=\rme^{-x^2}$
  appartient {\`a} $\mcs$. Par suite, il en est de m{\^e}me de la fonction $g_\sigma$
  d{\'e}finie par~(\ref{eq:gauss_func}) quelque soit $\sigma>0$.
\end{example}
Il faut travailler un peu plus pour trouver un exemple de fonction dans $\mcs$
{\`a} \emph{support compact} c'est-{\`a}-dire nulle en dehors d'un ensemble
born{\'e}.
\begin{example}\label{exple:cinfty-supp-compact}
  On consid{\`e}re la fonction $g$ d{\'e}finie par
$$
g(x)=
\begin{cases}
\rme^{-1/x} &\text{ si $x>0$}\\
0 &\text{ sinon.}
\end{cases}
$$
On montre facilement que $f$ est
$\mathcal{C}^\infty$. En revanche elle n'est pas {\`a} support compact puisque
$g(x)\to1$ quand $x\to\infty$. Cependant la fonction $f$ d{\'e}finie par
$$
f(x)=g(x) \times g(1-x) ,\quad x\in\rset\;,
$$
est nulle en dehors de $[0,1]$ et $\mathcal{C}^\infty$. C'est donc une fonction
de $\mcs$ {\`a} support compact.
\end{example}

\begin{proposition}
L'espace $\mcs$ a les propri{\'e}t{\'e}s suivantes :
\begin{enumerate}
\item $\mcs$ est stable pour la multiplication par un polyn{\^o}me.
\item $\mcs$ est stable par d{\'e}rivation ($f \in \mcs \Rightarrow f' \in  \mcs$).
\item $\mcs \subset  L_1(\rset)$.
\end{enumerate}
\end{proposition}
La d{\'e}monstration de cette proposition est laiss{\'e}e en exercice.  Le r{\'e}sultat
essentiel, qui d{\'e}coule essentiellement de propri{\'e}t{\'e}s d{\'e}j{\`a} d{\'e}montr{\'e}es, est
contenu dans le th{\'e}or{\`e}me suivant.
\begin{theorem}
\label{theo:stabiliteS}
L'espace $\mcs$ est stable par transformation de Fourier: si $f \in \mcs$ alors $\hat{f} \in \mcs$.
\end{theorem}
\begin{proof}
Soit $f \in \mcs$. Comme
$f$ est dans $L_1(\rset)$ et {\`a} d{\'e}croissance rapide, $\hat{f} \in C_\infty (\rset)$ d'apr{\`e}s la Proposition \ref{prop:1913}.
Pour tout $k \in \nset$, $f^{(k)}$ {\'e}tant {\`a} d{\'e}croissance rapide est int{\'e}grable d'apr{\`e}s le lemme \ref{lem:decroissancerapide}.
On en d{\'e}duit de la proposition \ref{prop:1914} que $\hat{f}$ est {\`a} d{\'e}croissance rapide.
Il reste {\`a} examiner la d{\'e}rivabilit{\'e} de $f$.
Comme toutes les d{\'e}riv{\'e}es de $f$ sont {\`a} d{\'e}croissance rapide, les fonctions $x\mapsto(x^q f (x))^{(p)}$ sont dans
$L_1(\rset)$ pour tout entiers $p$ et $q$; d'o{\`u}, d'apr{\`e}s la Proposition \ref{prop:FourierDerivation}, pour tout
$\xi\in\rset$,
\begin{equation}
  \label{eq:stabilite-S-en-action}
\xi^p \hat{f}^{(q)}(\xi) = \xi^p  \TFA{(-2 \rmi \pi)^q m_q\times f}(\xi) =
\frac{1}{(\rmi 2 \pi)^p} \TFA{[(- \rmi 2 \pi)^q m_q \times f]^{(p)}}(\xi) \;,
\end{equation}
o{\`u} $m_q$ est le mon{\^o}me de degr{\'e} $q$, $m_q(x)=x^q$.  Le th{\'e}or{\`e}me
de Rieman-Lebesgue montre que $\lim_{|\xi| \to \infty} |\xi^q
\hat{f}^{(q)}(\xi)|= 0$.
\end{proof}

Nous allons voir cette stabilit{\'e} se traduit en fait par une propri{\'e}t{\'e} encore
plus remarquable: $\TF$ est une bijection de $\mcs$ dans $\mcs$. Pour cela, il
faut montrer que $\TF$ admet une r{\'e}ciproque sur $\mcs$, ce sera le r{\^o}le de sa
soeur jumelle $\TFC$. Le proc{\'e}d{\'e} de r{\'e}gularisation sera fondamental pour
arriver {\`a} cette fin.


\section{R{\'e}gularisation par convolution}

On rappelle qu'{\'e}tant donn{\'e}es $f$ et $g$ deux fonctions de $\rset$ dans $\cset$,
la convolution de $f$ et $g$ est la fonction $f \star g$
d{\'e}finie par
$$
f \star g(x) = \int f(x - t)g(t) \rmd t = \int f(u)g(x - u) \rmd u\;,
$$
en tout point $x$ o{\`u} cette int{\'e}grale est correctement d{\'e}finie.\


\begin{example}
Prenons $f = g = \1_{[0,1]}$. On obtient
$$
\int f (x - t)g(t) \rmd t = \int_0^1 \1_{[0,1]}(x - t) \rmd t = \mathrm{Leb} ([0,1] \cap [x - 1,x]).
$$
La convolution de ces deux fonctions discontinues est donc continue.

Prenons maintenant $f \in L_1(\rset)$ et $g = \frac{1}{2h} \1_{[-h,h]}$ o{\`u} $h > 0$.
Calculons $f \star g$.
\begin{equation}
f * g(x) = \frac{1}{2h} \int_{x-h}^{x+h}  f (u) \rmd u
\end{equation}
ce qui repr{\'e}sente la moyenne de $f$ sur $[x - h, x + h]$.
On voit directement sur l'int{\'e}grale que $f \star g$ est une fonction continue.
\end{example}
Ces deux exemples illustrent la propri{\'e}t{\'e} essentielle de la convolution qui est de
\textit{r{\'e}gulariser} (qui est li{\'e} au fait de moyenner).


\section{R\'egularisation par convolution}
Le lemme suivant montre que la convolution peut {\^e}tre utilis{\'e}e dans le but de
fournir une approximation, ce qui constitue le premier principe de
r{\'e}gularisation.

\begin{lemma}\label{lem:regularisation}
  Soit $f\in\mathcal L^1$.
  Soit $K$ une fonction positive telle que $\int K(x)\,\rmd x=1$ et, pour une
  constante $C>0$ et un exposant $\alpha>2$, $K(x)\leq C (1+|x|)^{-\alpha}$
  pour tout $x\in\rset$.  D{\'e}finissons, pour tout $\sigma>0$, la fonction
  $K_\sigma$ par
\begin{equation}
  \label{eq:Kband}
  K_\sigma(x)=\frac1\sigma K(x/\sigma)\;.
\end{equation}
Alors, on a les propri{\'e}t{\'e}s suivantes.
\begin{enumerate}
\item\label{item:regularisation1} En tout point $t$ o{\`u} $f$ est continue, on a
$$
\lim_{\sigma\downarrow0} (f \star K_\sigma)(t) = f(t)\;.
$$
\item\label{item:regularisation2} Si $f$ est continue {\`a} support compact, alors
  $f \star K_\sigma$ converge uniform{\'e}ment vers $f$ quand $\sigma\downarrow0$.
\end{enumerate}
\end{lemma}
\begin{proof}
Comme $\int K_\sigma(x)\, \rmd x= 1$, par un changement de variable {\'e}l{\'e}mentaire,
\begin{equation}
\label{eq:inversionL1-2}
\int f(t-u) K_\sigma(u) \rmd u - f(t) = \int \left\{ f(t - u) - f(t) \right \} K_\sigma(u) \rmd u \eqsp.
\end{equation}
Montrons le point~\ref{item:regularisation1}.
Comme $f$ est continue au point $t$, pour tout $\epsilon > 0$, il existe $\eta$ tel que $|u-t| \leq \eta$
implique que $|f(u) - f(t) | \leq \epsilon$. On a pour tout $\sigma > 0$,
\begin{equation}
\label{eq:conv-local}
\int_{|u| \leq \eta} \left|  f(t-u) - f(t) \right | K_\sigma(u) \rmd u \leq \epsilon \int K_\sigma(u) \rmd u = \epsilon \eqsp.
\end{equation}
De plus,
$$
\int_{|u| \geq \eta} \left| f(t-u) - f(t) \right | K_\sigma(u) \rmd u  \leq \|f\|_1\,\sup_{|u|\geq\eta} {K}_\sigma(u)
+|f(t)|\,\int_{|u|\geq \eta} {K}_\sigma(u) \rmd u  \eqsp.
$$
D'autre part, lorsque $\sigma \downarrow 0$,
$$
\sup_{|u|\geq\eta} {K}_\sigma(u) = \frac1\sigma \sup_{|v|\geq\eta/\sigma} {K}_\sigma(v)\leq  \frac{C}\sigma (1+\eta/\sigma)^{-\alpha}
$$
et,
$$
\int_{u\geq \eta} K_\sigma(u) \rmd u \leq \frac{C}\sigma\int_{|v|\geq\eta/\sigma} (1+|v|)^{-\alpha}\,dv =O(\sigma^{\alpha-2})\eqsp.
$$
Par cons{\'e}quent, en combinant ces majorations avec $\alpha>2$,
\begin{equation}
\label{eq:conv-hors-local}
\lim_{\sigma \downarrow 0} \int_{|u| \geq \eta} \left|  f(t-u) - f(t) \right | K_\sigma(u) \rmd u =0 \eqsp.
\end{equation}
Le r{\'e}sultat suit donc avec~(\ref{eq:inversionL1-2}).

Le point~\ref{item:regularisation2} se montre de fa�on semblable.
Comme $f$ est continue {\`a} support compact, elle est uniform{\'e}ment continue.
On peut donc choisir $\eta>0$ tel que~(\ref{eq:conv-local}) soit valide pour
tout $t$. De m{\^e}me la convergence~(\ref{eq:conv-hors-local}) est uniforme en $t$
en utilisant que $f$ est born{\'e}. D'o{\`u} le r{\'e}sultat.
\end{proof}

Le second principe de r{\'e}gularisation est que la version convolu{\'e}e $f\star K$ de
$f$ adopte la r{\'e}gularit{\'e} du \emph{noyau} $K$. Ce principe est donn{\'e} par le
lemme suivant.

\begin{lemma}\label{lem:regulairsation-est-reguliere}
  Soient $f,g\in L^1$. Alors on a les propri{\'e}t{\'e}s suivantes
  \begin{enumerate}
  \item\label{item:approx-cont-S-1} Si $g$ est $\mathcal{C}^k$, $f\star g$ est
  $\mathcal{C}^k$ et $(f\star g)^{(k)}=f\star g^{(k)}$.
\item\label{item:approx-cont-S-2} On a pour tout entier $p\geq0$,
  \begin{equation}
    \label{eq:decroissance-concolution}
    \|(f\star g)\times m_p\|_\infty\leq \|f\times (1+|m_p|)\|_\infty \;
    \|g\times (1+|m_p|)\|_1  \;,
  \end{equation}
o{\`u} $m_p$ d{\'e}signe le mon{\^o}me de degr{\'e} $p$, $m_p(x)=x^p$.
\item \label{item:approx-cont-S-2prim} Si $f$ et $g$ sont {\`a} d{\'e}croissance rapide,
  il en est de m{\^e}me de $f\star g$.
\item\label{item:approx-cont-S-3}  Si $f$ est {\`a} d{\'e}croissance rapide et
  $g\in\mcs$, $f\star g\in\mcs$.
  \end{enumerate}
\end{lemma}
\begin{proof}
  La propri{\'e}t{\'e}~\ref{item:approx-cont-S-1} est une simple application du lemme
  de d{\'e}rivation sous le signe somme.

Montrons la propri{\'e}t{\'e}~\ref{item:approx-cont-S-2}.
Pour tous $t,x\in\rset$, on a $|x|^p\leq(|x-t|+|t|)^p\leq|x-t|^p+|t|^p$.
D'o{\`u}
$$
|f\star g(x)||x|^p \leq \int |f(x-t)|\,|x-t|^p\,|g(t)|\rmd t+
\int |f(x-t)|\,|t|^p\,|g(t)|\rmd t \;.
$$
On obtient donc~(\ref{eq:decroissance-concolution}).

La propri{\'e}t{\'e}~\ref{item:approx-cont-S-2prim} est obtenu en appliquant
de~\ref{item:approx-cont-S-2} pour tout $p\geq1$.

La propri{\'e}t{\'e}~\ref{item:approx-cont-S-3} d{\'e}coule de \ref{item:approx-cont-S-1}
et~\ref{item:approx-cont-S-2prim}.
\end{proof}


On remarque facilement que l'on peut prendre dans les lemmes pr{\'e}c{\'e}dents
$K=g_1$, c'est-{\`a}-dire $K_\sigma=g_\sigma$, o{\`u} $g_\sigma$ est d{\'e}finie
en~(\ref{eq:gauss_func}). On peut parler dans ce cas de \emph{r{\'e}gularisation
  gaussienne}.

\section{Formules d'inversion}

On est maintenant en mesure de montrer des r{\'e}sultats d'inversion de la
transform{\'e}e de Fourier dans $L_1(\rset)$. La preuve se base sur l'observation
suivante. D'apr{\`e}s~(\ref{eq:gaussienneTF}), $\TF(g_\sigma)$ est une fonction
r{\'e}elle paire de $\mathcal{L}_1(\rset)$ et
$\TF\TF(g_\sigma)=\TFC\TF(g_\sigma)=g_\sigma$. Le r{\'e}sultat suivant est une
premi{\`e}re g{\'e}n{\'e}ralisation de la formule d'inversion ``$\TFC\TF(f)=f$'' qui sera
par la suite {\'e}tendue {\`a} des cadres bien plus g{\'e}n{\'e}raux.

\begin{proposition}\label{prop:inversionFourierL1}
Soit $f\in\mathcal{L}_1(\rset)$ et supposons que $\hat{f}$ appartiennent aussi
{\`a} $\mathcal{L}_1(\rset)$. Alors, en tout point $x$ o{\`u} $f$ est continue, on a
\begin{equation}
\label{eq:inversionFT}
[\bar{\TF} \hat{f}](x) = f(x).
\end{equation}
\end{proposition}
\begin{proof}
On a vu ci-dessus que la fonction $\hat{g}_\sigma(x) = \rme^{- 2 \pi^2 \sigma^2 x^2}$
a ${g}_\sigma$ pour transform{\'e}e de Fourier. Donc
$[\TF(x\mapsto \hat{g}_\sigma(x)\rme^{\rme 2 \pi t x})](\xi)={g}_\sigma(\xi-t)={g}_\sigma(t-\xi)$ par la
proposition~\ref{prop:FourierTranslation} puis par parit{\'e} de $g_\sigma$. La proposition \ref{prop:echangeTF} appliqu{\'e}e avec
$x \mapsto f(x)$ et $x \mapsto \rme^{\rmi 2 \pi t x} \hat{g}_\sigma(x)$ donne donc, pour tout $t\in \rset$,
\begin{equation}
\label{eq:inversionL1-1}
\int \hat{f}(x) \hat{g}_\sigma(x) \rme^{\rmi 2 \pi t x} \rmd x = \int f(u) {g}_\sigma(t-u) \rmd u = f\star g_\sigma(t) \eqsp.
\end{equation}
Lorsque $\sigma \to 0$, on peut passer {\`a} la limite dans l'int{\'e}grale de gauche, puisque l'on a, pour tout
$x$, $\lim_{\sigma \to 0} \hat{g}_\sigma(x)= 1$ pour tout $x$ et $|\hat{f}(x) \hat{g}_\sigma(x) \rme^{\rmi 2 \pi t x}| \leq |\hat{f}(x)|$.
Comme $\hat{f} \in L_1(\rset)$, on applique le th{\'e}or{\`e}me de convergence domin{\'e}e, qui montre
$$
\lim_{\sigma \to 0} \int \hat{f}(x) \hat{g}_\sigma(x) \rme^{ \rmi 2 \pi t x} \rmd x = \int \hat{f}(x) \rme^{\rmi 2 \pi t x} \rmd x \eqsp.
$$
Le passage {\`a} la limite dans le membre de droite de~(\ref{eq:inversionL1-1}) est lui une cons{\'e}quence du
lemme~\ref{lem:regularisation}. On obtient bien le r{\'e}sultat annonc{\'e} si $f$ est continue en $t$.
\end{proof}


Le r{\'e}sultat pr{\'e}c{\'e}dent conjugu{\'e} avec le th{\'e}or{\`e}me~\ref{theo:stabiliteS} permet de
d{\'e}finir une transform{\'e}e inverse comme application r{\'e}ciproque de $\TF$ d{\'e}finie
comme application de $\mcs$ dans $\mcs$.

En effet, si $f$ est un {\'e}l{\'e}ment de $\mcs$, $\hat{f}$ est dans $\mcs$ et donc
int{\'e}grable. $f$ {\'e}tant partout continue, la formule
d'inversion~(\ref{eq:inversionFT}) est valable pour tout $x \in \rset$. Donc,
pour tout $f \in \mcs$, $f = \TFAC{\TF f}$. De la m{\^e}me fa�on, on a $f =
\TFA{\TFC f}$. $\TF$ est donc une bijection sur $\mcs$ et son inverse est
$\TFC$.

\begin{theorem}\label{thm:Schwarz}
La transformation de Fourier $\TF$ est une application lin{\'e}aire bijective de $\mcs$ sur $\mcs$.
L'application inverse est $\TF^{-1} = \TFC$.
\end{theorem}

Ayant montr{\'e} le th{\'e}or{\`e}me~\ref{thm:Schwarz}, de nombreuses formules d'inversion peuvent \^etre d{\'e}duite par dualit{\'e}.
Nous verrons dans la section suivante comment appliquer ce principe dans un cadre hilbertien.
Ici nous l'appliquons dans le cadre $L_1(\rset)$ grace au r{\'e}sultat suivant, qui nous permettra de compl{\'e}ter la
proposition~\ref{prop:inversionFourierL1} par un th{\'e}or{\`e}me d'inversion.
\begin{proposition}\label{prop:DualiteSchwartz}
Soient deux fonctions $f$ et $g$ dans $L_1(\rset)$. Si, pour toute fonction \textit{test} $\phi$ dans $\mcs$, on a
$$
\int f(x)\,\phi(x)\,\rmd x= \int g(x)\,\phi(x)\,\rmd x,
$$
alors $f=g$ (au sens $L_1(\rset)$).
\end{proposition}
\begin{proof}
En prenant la diff{\'e}rence entre les deux membre de l'{\'e}galit{\'e} de l'hypoth{\`e}se, on voit qu'il suffit de montrer ce r{\'e}sultat pour
$g=0$. De plus, comme les fonctions continues sont denses dans l'ensemble des fonctions int�grables, on peut se contenter de prendre $f$ continue, le cas g{\'e}n{\'e}ral {\'e}tant
obtenu par passage {\`a} la limite. Or, pour $f$ continue et $g=0$, le r{\'e}sultat est imm{\'e}diat par application du principe de
r{\'e}gularisation en choisissant une fonction $K\in\mcs$ positive int{\'e}grant {\`a} 1 (par exemple $g_1$) puis en appliquant le
lemme~\ref{lem:regularisation} en tout point de la droite r{\'e}elle.
\end{proof}

On en d{\'e}duit le r{\'e}sultat annonc{\'e} qui compl{\`e}te la proposition~\ref{prop:inversionFourierL1}.
\begin{theorem}
Soit $f\in L_1(\rset)$ et supposons que $\hat{f}$ appartiennent aussi {\`a} $L_1(\rset)$. Alors
la fonction (continue) $\bar{\TF} \hat{f}$ est l'unique
repr{\'e}sentant continu de $f$.
\end{theorem}
\begin{proof}
La conitnuit{\'e} de $\bar{\TF} \hat{f}$ d{\'e}coule du th{\'e}or{\`e}me~\ref{thm:rieman-lebesgue}.
Pour toute fonction test $\phi$ de $\mcs$, on a, d'apr{\`e}s la proposition~\ref{prop:echangeTF} et le
th{\'e}or{\`e}me~\ref{thm:Schwarz},
$$
\int f(x) \phi(x)\,\rmd x=\int \hat{f}(\xi) \TFC(\phi)(\xi)\,d\xi.
$$
Mais comme $\hat{f}$, on peut r{\'e}appliquer l'{\'e}quivalent de la proposition~\ref{prop:echangeTF} mais pour la transform{\'e}e
inverse, ce qui donne alors directement
$$
\int \hat{f}(\xi) \TFC(\phi)(\xi)\,d\xi=\int \TFC(\hat{f})(x) \phi(x) \,\rmd x.
$$
D'o{\`u} le r{\'e}sultat en appliquant la proposition~\ref{prop:DualiteSchwartz}.
\end{proof}

\section{Convolution et transform�e de Fourier}

On conclut se chapitre en explorant l'effet de le transform{\'e}e de Fourier sur la
convolution. Nous explorons ici uniquement le cas de la convolution d{\'e}finie sur
$L^1\times L^1$.

\begin{theorem}%\label{thm:inversionFourierL1}
Etant donn{\'e}es deux fonctions $f$ et $g$ de $L_1(\rset)$ on a:
\begin{enumerate}
\item $f \star g$ est d{\'e}finie presque partout et $f \star g$ appartient {\`a} $L_1(\rset)$.
\item La convolution est un op{\'e}rateur bilin{\'e}aire continu de $L_1(\rset) \times L_1(\rset) \to L_1(\rset)$
tel que
$$
\| f \star g \|_1 \leq \| f \|_1 \| g \|_1
$$
\item la transform{\'e}e de Fourier du produit de convolution $\widehat{f \star g}$ est {\'e}gal au produit des transform{\'e}es
de Fourier des fonctions $\hat{f}$ et $\hat{g}$: $\widehat{f \star g} = \hat{f} \hat{g}$.
\end{enumerate}
\end{theorem}
\begin{proof}
Comme $f,g \in L_1(\rset)$, la fonction $(y,z) \mapsto f(y) g(z) \in L_1(\rset^2)$ d'apr{\`e}s le th{\'e}or{\`e}me de Fubini.
En faisant le changement de variable $y = x-t$ et $z=t$, on obtient:
$$
\iint f(y) g(z) d y d z = \iint f(x-t) g(t) \rmd x \rmd t \eqsp
$$
et les int{\'e}grales des modules sont finies.
La fonction $x \mapsto \int f(x-t) g(t) \rmd t$ est donc d{\'e}finie presque partout et appartient {\`a} $L_1(\rset)$, toujours d'apr{\`e}s le th{\'e}or{\`e}me de Fubini.
La seconde in{\'e}galit{\'e} d{\'e}coule de:
$$
\int | f \star g(x)| \rmd x = \int |g(t)| \left( \int  |f(x-t) \rmd x \right) \rmd t = \|f\|_1 \| g \|_1 \eqsp.
$$
La troisi{\`e}me assertion s'obtient de fa�on similaire par une application du th{\'e}or{\`e}me de Fubini.
\end{proof}

%%% Local Variables:
%%% mode: latex
%%% ispell-local-dictionary: "francais"
%%% TeX-master: "Polycopie-Fourier-L1L2"
%%% End:

\chapter{Transform{\'e}e de Fourier-Plancherel}
\section{Espace des fonctions de carr{\'e} int{\'e}grable}
Soit $\mathcal{L}_2(\rset)$ l'espace des fonctions d{\'e}finies sur $\rset$ et
{\`a} valeurs complexes, $f:\rset\rightarrow \cset$, de carr{\'e} sommable
c'est-{\`a}-dire telles que:
$$
\int |f(x)|^2\,\rmd x<\infty \eqsp.
$$
On note $L_2(\rset)$ l'espace des classes d'{\'e}quivalence de
$\mathcal{L}_2(\rset)$ pour la relation d'{\'e}quivalence ``$f=g$ p.p.''.
Pour $I$ un sous ensemble bor{\'e}lien de $\rset$ (et en particulier, un intervalle), on peut d{\'e}finir
de la m{\^e}me fa�on l'espace $\mathcal{L}_2(I)$ des fonctions  de carr{\'e} sommable sur $I$, $\int_I |f(x)|^2 \rmd x < \infty$
et l'espace $L_2(I)$ des classes d'{\'e}quivalence de $\mathcal{L}_2(I)$ par rapport {\`a} la relation d'{\'e}quivalence d'{\'e}galit{\'e} presque-partout.


Pour $f$ et $g \in L_2(\rset)$, d{\'e}finissons.
\begin{equation}
\label{eq:definitionpscal}
\pscal{f}{g}_I =\int_I f(x) \bar{g}(x)\,\rmd x
\end{equation}
o{\`u}, pour tout $z \in \cset$, $\bar{z}$ est le conjugu{\'e} de $z$. Lorsque $I= \rset$, nous omettons  l'indice $I$.
Cette int{\'e}grale est bien d{\'e}finie pour 2 repr{\'e}sentants de $f$ et $g$ car $|f(x) \bar{g}(x)|\leq (|f(x)|+ |\bar{g}(x)|)/2$ et
sa valeur ne d{\'e}pend {\'e}videmment pas du choix de ses repr{\'e}sentant. D'auter part, $L_2(I)$ est bien le plus ``gros'' espace
fonctionnel sur lequel ce produit scalaire est bien d{\'e}fini puisqu'il impose justement $\pscal{f}{f}_I<\infty$.
Mentionnons aussi que, de m\^eme que pour $(L_1(\rset),\|\cdot\|_1)$,  $(L_2(\rset),\|\cdot\|_2)$ est un espace de Banach
(espace vectoriel norm{\'e} complet), o{\`u} la norme $\|\cdot\|_2$ est d{\'e}finie par
$$
\|f\|_2:=\sqrt{\pscal{f}{f}}=\left(\int |f(x)|^2\,\rmd x\right)^{1/2}.
$$
Cette norme {\'e}tant un norme induite par un produit scalaire, on dit que
$(L_2(\rset),\pscal{\cdot}{\cdot})$ est un espace
\textit{de Hilbert}.

\begin{theorem}
  \label{thm:lunDense}
  L'ensemble des fonctions int{\'e}grables et de carr{\'e} int{\'e}grable, $L_1(\rset)\cap
  L_2(\rset)$ est un sous-espace vectoriel dense de $(L_2(\rset),\|\cdot\|_2)$.
\end{theorem}
\begin{proof}
Pour tout $f\in L_2(\rset)$, on note $f_n$ la fonction {\'e}gale {\`a} $f$ sur $[-n,n]$ et nulle ailleurs.
Alors $f_n\in L_1(\rset)$ pour tout $n$, et par convergence monotone, $\|f_n-f\|_2\to0$ quand $n\to\infty$ (on dit que
$f_n$ tend vers $f$ au sens de $L_2(\rset)$. On en conclut que $L_1(\rset)\cap L_2(\rset)$ est dense dans
$(L_2(\rset),\|\cdot\|_2)$.
\end{proof}
On a imm{\'e}diatement que la proposition~\ref{prop:DualiteSchwartz} s'adapte {\`a} l'espace $L_2(\rset)$.
\begin{corollary}  \label{cor:DualiteCalS}
Soient deux fonctions $f$ et $g$ dans $L_2(\rset)$. Si, pour toute fonction \textit{test} $\phi$ dans $\mcs$, on a
$$
\int f(x)\,\phi(x)\,\rmd x= \int g(x)\,\phi(x)\,\rmd x,
$$
alors $f=g$ (au sens $L_2(\rset)$).
\end{corollary}
On  utilisera par ailleurs le r{\'e}sultat suivant qui peut se montrer directement {\`a} partir du th{\'e}or{\`e}me de densit� des
fonctions continues dans $L_1(\rset)$. 

\begin{theorem}
  \label{thm:SDenseltwo}
  L'espace $\mcs$ est un sous-espace vectoriel dense de $(L_2(\rset),\|\cdot\|_2)$.
\end{theorem}

Nous verrons que $L_2(\rset)$ pose un certain nombre de probl{\`e}mes
th{\'e}oriques pour d{\'e}finir la transform{\'e}e de Fourier qui ne se pose pas
pour une fonction de $L_1(\rset)$. Or, en passant de $L_1(\rset)$ {\`a}
$L_2(\rset)$, on impose {\`a} la fonction des conditions locales plus
contraignantes (toute restriction d'une fonction de $L_2(\rset)$ {\`a} un compact
est $L_1(\rset)$ mais l'inverse n'est pas vrai) et on autorise des
comportements en $t=\pm\infty$ un peu plus g{\'e}n{\'e}raux. D{\`e}s lors, on
peut s'interroger sur l'int{\'e}r{\^e}t d'{\'e}tudier les fonctions  de $L_2(\rset)$ plut{\^o}t que de
$L_1(\rset)$, qui plus est quand, en pratique, une fonction n'est
jamais observ{\'e} sur un temps infini. La r{\'e}ponse {\`a} cette question est
la suivante. Outre que les propri{\'e}t{\'e}s d'espace de Hilbert de
$L_2(\rset)$ sont fondamentales dans la th{\'e}orie, elles ont un lien
physique {\'e}vident dans les applications puisque le carr{\'e} de la norme
d'un signal dans $L_2(\rset)$ n'est rien d'autre que son {\'e}nergie.
Le fait qu'en pratique les "signaux" observ{\'e}s soient dans
$L_1(\rset)\cap L_2(\rset)$ explique que l'on peut en g{\'e}n{\'e}ral ne
pas se pr{\'e}occuper des subtilit{\'e}s entre transform{\'e}e de Fourier dans
$L_1(\rset)$ et transform{\'e}e de Fourier dans $L_2(\rset)$, mais,
pour {\'e}tablir les r{\'e}sultats g{\'e}n{\'e}raux que l'on utilise pour {\'e}tudier
les fonctions de carr{\'e} sommable, il serait dommage de les {\'e}noncer dans
le cas particulier $L_1(\rset)\cap L_2(\rset)$ alors qu'ils sont
valables dans $L_2(\rset)$, m{\^e}me si l'on doit pour cela donner des
preuves qui peuvent appara{\^i}tre plus abstraites.

\section{Transform{\'e}e de Fourier sur $L_2(\rset)$}
L'id{\'e}e de base de la construction consiste {\`a} {\'e}tendre la transform{\'e}e de Fourier
de $L_1(\rset)$ {\`a} $L_2(\rset)$ par un argument de densit{\'e}.
\begin{proposition}
\label{prop:plancherelparseval}
Soit $f$ et $g$ dans $\mcs$. On a:
\begin{gather*}
\int \hat{f}(\xi) \bar{\hat{g}}(\xi) d \i = \int f(x) \bar{g}(x) dx \\
\int |\hat{f}(\xi)|^2 d \xi = \int |f(x)|^2 dx \eqsp.
\end{gather*}
\end{proposition}
\begin{proof}
Appliquons la formule d'{\'e}change (Proposition \ref{prop:echangeTF}). On pose $h(\xi)= \bar{\hat{g}}(\xi)$. On a:
$$
\int \hat{f}(\xi) h(\xi) d \xi = \int f(x) \hat{h}(x) dx \eqsp.
$$
Mais $\bar{\hat{g}}(\xi)= \TFC \bar{g}(\xi)$, d'o{\`u} $\hat{h}= \bar{g}$.
\end{proof}

\begin{proposition}
Soient $E$ et $F$ deux espaces vectoriels norm{\'e}s, $F$ complet, et $G$ un sous-espace vectoriel dense dans $E$.
Si $A$ est un op{\'e}rateur lin{\'e}aire continu de $G$ dans $F$, alors il existe un prolongement unique $\tilde{A}$
lin{\'e}aire continu de $E$ dans $F$ et la norme de $\tilde{A}$ est {\'e}gale {\`a} la norme de $A$.
\end{proposition}
\begin{proof}
Soit $f \in E$. Comme $G$ est dense dans $E$, il existe une suite $f_n$ dans $G$ telle que
$\lim_{n \to \infty} \| f_n - f \| = 0$. La suite $f_n$ {\'e}tant convergente, elle est de
Cauchy. $A$ {\'e}tant lin{\'e}aire continu on a
$$
\| A f_n- A f_m \| \leq \| A \| \| f_n- f_m \| \eqsp.
$$
On en d{\'e}duit que $A f_n$ est une suite de Cauchy de $F$ qui est complet.
La suite $A f_n$ est donc convergente vers un {\'e}l{\'e}ment $g$ de $F$.
On v{\'e}rifie facilement que $g$ ne d{\'e}pend pas de la suite $f_n$ et on pose donc $A f = g$.
$\tilde{A}$ est lin{\'e}aire par construction et de plus on a
$$
\|  \tilde{A} f \| =  \lim_{n \to \infty} \| A f_n \| \leq \lim_{n \to \infty} \| A \| \| fn \| = \| A \| \| f \|\eqsp,
$$
ce qui prouve que $\| \tilde{A} \| \leq  \| A \|$. Comme $\tilde{A} f = A f$ pour tout $f \in G$, on a $\| \tilde{A} \| = \| A \|$.
Enfin, $G$ {\'e}tant dense dans $E$, il est clair que $\tilde{A}$ est unique.
\end{proof}
D'apr{\`e}s la proposition~\ref{prop:plancherelparseval}, $\TF$ est une isom{\'e}trie sur $\mcs$ muni du produit scalaire
$\pscal{\cdot}{\cdot}$. On applique le r{\'e}sultat pr{\'e}c{\'e}dent avec $E = F = L_2(\rset)$, $G = \mcs$ (voir le
th{\'e}or{\`e}me~\ref{thm:SDenseltwo}. On obtient
\begin{theorem}\label{thm:prolong}
La transformation de Fourier $\TF$ (respectivement la transformation inverse $\TFC$)
se prolonge en une isom{\'e}trie de $L_2(\rset)$ sur $L_2(\rset)$.
D{\'e}signons toujours par $\TF$ (resp. $\TFC$) ce prolongement. On a en particulier
\begin{enumerate}
\item (Inversion) pour tout $f \in L_2(\rset)$, $\TF \TFC f = \TFC \TF f = f$,
\item (Plancherel) pour tout $f,g \in L_2(\rset)$, $\pscal{f}{g} = \pscal{\TF f}{\TF g}$
\item (Parseval) pour tout $f \in L_2(\rset)$, $\| f \|_2 = \| \TF f \|_2$.
\end{enumerate}
\end{theorem}

Remarquons que l'{\'e}galit{\'e} de Parseval peut se r{\'e}{\'e}crire, pour tout $f$ et $g$ dans $L_2(\rset)$,
\begin{equation}\label{eq:echangeL2}
\pscal{ \TF f}{g} = \pscal{f}{\TF g}
\end{equation}

\begin{proposition}
\label{prop:prolongementL1L2}
Le prolongement de $\TF$ sur $\mcs$ par continuit{\'e} {\`a} $(L_2(\rset),\|\cdot\|_2)$ est compatible avec la d{\'e}finition de
$\TF$ donn{\'e}e pr{\'e}c{\'e}demment dur $L_1(\rset)$. Plus pr{\'e}cis{\'e}ment
\begin{enumerate}
\item Pour tout $f\in L_1(\rset) \cap L_2(\rset)$, $\TF f$ d{\'e}fini par le th{\'e}or{\`e}me~\ref{thm:prolong} admet un repr{\'e}sentant
 $\hat{f}\in C_\infty$ v{\'e}rifiant
$$
\hat{f}(\xi) = \int_{\rset} \rme^{- \rmi 2 \pi \xi x} f(x) dx,\quad\xi\in\rset.
$$
\item Si $f \in L_2(\rset)$, $\TF f$ est la limite dans $L_2(\rset)$ de la suite $g_n$, d{\'e}finie par $g_n(\xi) =
  \int_{-n}^n \rme^{- \rmi 2 \pi \xi x} f(x) dx$.
\end{enumerate}
\end{proposition}
\begin{proof}
Notons $\hat{f}$ la transform{\'e}e de Fourier sur $L_1(\rset)$ et $\TF f$ celle sur $L_2(\rset)$.
Prenons $f \in L_1(\rset) \cap L_2(\rset)$. En appliquant la proposition \ref{prop:echangeTF} puis
Parseval (voir~(\ref{eq:echangeL2})), on a pour tout $\psi \in \mcs$,
$$
\int \psi \hat{f} = \int \hat{\psi} f = \int \TFA{\psi} f= \int \psi \TFA{f}
$$
d'o{\`u} $\int  (\hat{f} - \TFA{f}) \psi  = 0$ pour tout $\psi \in \mcs$. Le corollaire~\ref{cor:DualiteCalS} fournit alors le
premier r{\'e}sultat.

Posons $f_n = f \1_{[-n,n]}$. Par convergence domin{\'e}e, on a $\lim_n \| f_n - f \|_2^2= 0$.
Comme $f_n  \in L_1(\rset) \cap L_2(\rset)$ on {\'e}crit $g_n = \hat{f}_n = \TFA{f_n}$ et par continuit{\'e} il vient
$ \lim_{n \to \infty} \| \TF f - g_n \|_2^2= 0$.
\end{proof}


%%% Local Variables:
%%% mode: latex
%%% ispell-local-dictionary: "francais"
%%% TeX-master: "Polycopie-Fourier-L1L2"
%%% End:

\chapter{Echantillonnage}
Nous nous int{\'e}ressons dans cette partie au sous espace de $L_2(\rset)$ des fonctions {\`a} \emph{bande limit{\'e}e}
\begin{definition}[Bande Limit{\'e}e]
Une fonction $f \in L_2(\rset)$ ({\`a} valeurs r{\'e}elles) est dite {\`a} \emph{bande limit{\'e}e} s'il existe $B < \infty$ tel que: $\TF f(\xi) = 0$ pour $\xi \not \in [-B,+B]$.
On note $\BL{B}$ le sous espace vectoriel des fonctions $f \in L_2(\rset)$ telles que $\TF f(\xi) = 0$ pour (presque tout) $\xi \not \in [-B,+B]$.
\end{definition}
Pour $f \in L_2(\rset)$, la Transform{\'e}e de Fourier d{\'e}finit un isomorphisme de $L_2(\rset)$ dans $L_2(\rset)$
d'inverse $\TFC$. Soit maintenant  $f \in \BL{B}$.
Comme $\TF f$ est {\`a} support compact et dans $L_2(\rset)$ par isom{\'e}trie de $\TF$, il est aussi dans $\lone(\rset)$.
D'apr{\`e}s la proposition \ref{prop:prolongementL1L2}, appliqu{\'e}e {\`a} la transform{\'e}e de Fourier conjugu{\'e}e de $\TF f$ qui n'est
autre que $f$, $f$ admets donc un repr{\'e}sentant continue. Par la suite on identifiera tout {\'e}l{\'e}ment de $\BL{B}$
{\`a} son repr{\'e}sentant continu.



Soit  $T \leq 1/(2B)$. Nous identifierons dans la suite $T$ avec la \emph{p{\'e}riode d'{\'e}chantillonnage} et $1/T$ avec la \emph{fr{\'e}quence d'{\'e}chantillonnage}.
Consid{\'e}rons la fonction $\lambda \to F_T(\lambda)$ obtenu en p{\'e}riodisant la fonction $\lambda \to \TF f(\lambda)$ {\`a} la p{\'e}riode $1/T$:
$$
F_T (\lambda) = \sum_{n \in \zset} [\TF f] \left( \lambda - \frac{n}{T} \right) \eqsp.
$$
Par construction, la fonction $\lambda \mapsto F_T(\lambda)$ est une fonction p{\'e}riodique de p{\'e}riode $1/T$, et sur chaque
p{\'e}riode $\coint{(k-1/2)/T,(k+1/2)/T}$, la fonction $F_T$ est {\'e}gale {\`a}  $\TF f$ la translat{\'e}e de $k/T$.  Il est donc clair que
$F_T \in L_2{[-1/2T,1/2T]}$ et  admet donc un d{\'e}veloppement en s{\'e}rie de
Fourier:
\begin{equation}
\label{eq:SerieFourier}
F_T (\lambda) = \sum_{n \in \zset} c_n(F_T) \rme^{+ \rmi 2 \pi \lambda n T},
\end{equation}
o{\`u} $\{c_k(F_T)\}$ est la suite de des coefficients de Fourier de la fonction $F_T$,
d{\'e}finis pour tout $k\in\zset$ par,
\begin{equation}
\label{eq:CoefficientFourier}
c_k(F_T) = T \int_{-1/(2T)}^{1/(2T)} F_T(\lambda) \rme^{+ \rmi 2 \pi \lambda n t} d \lambda\eqsp.
\end{equation}
L'{\'e}galit{\'e} dans \eqref{eq:SerieFourier} doit {\^e}tre comprise au sens de la convergence dans l'espace de Hilbert
$L_2([-1/(2T),1/(2T)])$: la s{\'e}rie trigonom{\'e}trique $F_{N,T}(t)(\lambda)$, d{\'e}finie par
\begin{equation}
F_{N,T}(\lambda)= \sum_{k=-N}^N c_N(F_T) \rme^{+ \rmi 2 \pi \lambda n T} \eqsp,
\end{equation}
converge vers la fonction $F_T$ au sens de la topologie induite par la norme $\|\cdot\|_2$, c'est-{\`a}-dire,
$$
\lim_{N \to \infty} \int_{-1/(2T)}^{1/(2T)} |F_T(\lambda) - F_{N,T}(\lambda)|^2 d \lambda =0 \eqsp.
$$
L'{\'e}galit{\'e} de Parseval implique aussi que $\sum_{n \in \zset} |c_n(F_T)|^2 < \infty$. Comme nous avons suppos{\'e} que $T \leq
1/(2B)$, nous avons donc $[-B,+B] \subset [-1/(2T), 1/(2T)]$,
ce qui implique que, pour tout $\lambda \in [-1/(2T),1/(2T)]$, $F_T(\lambda) = \TF f(\lambda)$, ce qui implique que les
coefficients de Fourier $c_k(F_T)$ s'{\'e}crivent:
\begin{equation}
\label{eq:CoefficientFourier}
c_k(F_T) = T \int_{-B}^{B} \TF f(\lambda) \rme^{+ \rmi 2 \pi \lambda k T} d \lambda= T f(kT)\eqsp.
\end{equation}
Ils correspondent donc aux {\'e}chantillons de la fonction $f$ pr{\'e}lev{\'e}s aux instants r{\'e}guli{\`e}rement espac{\'e}s $kT$ (les instants d'{\'e}chantillonnage). La formule de Parseval pour les
coefficients de Fourier implique en particulier que $\sum_{k \in \zset} |f(kT)|^2 < \infty$. \eqref{eq:SerieFourier} se r{\'e}{\'e}crit donc
\begin{equation}
\label{eq:FormuleSommatoirePoisson}
\sum_{n \in \zset} \TF f \left( \lambda - \frac{n}{T} \right)  = T \sum_{n \in \zset} f(nT) \rme^{+ \rmi 2 \pi \lambda n T}\eqsp,
\end{equation}
qui est appel{\'e}e la \emph{formule sommatoire de Poisson}. Cette formule, que nous avons d{\'e}montr{\'e} ici pour des fonctions {\`a} bande limit{\'e}e, s'av{\`e}rent v{\'e}rifi{\'e}es
sous des hypoth{\`e}ses beaucoup plus g{\'e}n{\'e}rales. En multipliant les deux membres de l'identit{\'e} pr{\'e}c{\'e}dente par la fonction indicatrice de l'intervalle
$[-1/(2T),+1/(2T)]$ et en utilisant $\1_{[-1/(2T),1/(2T)]}(\lambda) \TF f (\lambda)= \1_{[-1/(2T),1/(2T)]}(\lambda) F_T(\lambda)$,
nous obtenons donc l'identit{\'e},
$$
\TF f (\lambda) = T \sum_{n \in \zset} f(nT) \1_{[-1/(2T),1/(2T)]}(\lambda) \rme^{+ \rmi 2 \pi \lambda n T} \eqsp.
$$
qui doit {\^e}tre comprise au sens $L_2(\rset)$,
$$
\lim_{N \to \infty} \int \left| \TF f(\lambda)  - T \sum_{n =-N}^N f(nT) \1_{[-1/(2T),1/(2T)]}(\lambda) \rme^{+ \rmi 2 \pi
    \lambda n T}  \right|^2 \rmd \lambda = 0 \eqsp.
$$
Comme l'application $\TFC$ est continue de $L_2(\rset) \to L_2(\rset)$ et que
$$
\TFAC{\1_{[-1/(2T),1/(2T)]}(\lambda) \rme^{- \rmi 2 \pi \lambda n a}}= \frac{\sin\left( \frac{\pi}{T}(t - nT)\right)}{\pi(t-nT)}
$$
(avec la convention $0/0=0$), on obtient la formule d'interpolation
\begin{equation}
\label{eq:FormuleInterpolation}
f(t) = \sum_{n \in \zset} f(nT) s_T(t-nT)  \eqsp,
\end{equation}
o{\`u} la fonction $s_T$, appel{\'e}e \emph{sinus-cardinal} est d{\'e}finie par
\begin{equation}
\label{eq:SinusCardinal}
s_T(0)=0\quad\text{et}\quad s_T(t) = \frac{\sin \left( \frac{\pi}{T} t\right) }{ \frac{\pi}{T} t}\quad\text{pour tout $t\neq0$} \eqsp.
\end{equation}
La convergence de la s{\'e}rie \eqref{eq:FormuleInterpolation} a lieu dans $L_2(\rset)$. Si de plus on a
$$
\sum_{k \in \zset} |f(kT)| < \infty,
$$
la s{\'e}rie \eqref{eq:FormuleInterpolation} est uniform{\'e}ment (car normalement au sens de la norme sup) convergente vers une
fonction $g$ continue sur $\rset$. Donc la
s{\'e}rie converge aussi dans $L_2(J)$, pour tout intervalle born{\'e} $J \subset \rset$. On en d{\'e}duit que $f(t)= g(t)$ presque-partout, et donc que $f(t)= g(t)$
pour tout $t$ r{\'e}el, puisque les fonctions $f$ et $g$ sont continues. Nous pouvons formuler le r{\'e}sultat important
\begin{theorem}
Soit $f \in \BL{B}$. Alors on pour tout $T \leq 1/(2B)$, on a
$$
\sum_{k= - \infty}^\infty |f(kT)|^2 < \infty \eqsp,
$$
et
$$
f(t) = \sum_{n \in \zset} f(nT) \frac{\sin\left( \frac{\pi}{T}(t - nT)\right)}{\frac{\pi}{T}(t-nT)} \eqsp.
$$
La convergence de la s{\'e}rie et l'{\'e}galit{\'e} ont lieu au sens de la norme de $L_2(\rset)$. Elles ont lieu au sens de la convergence
uniforme, et donc pout tout $t$ r{\'e}el si
$$
\sum_{n \in \zset} |f(nT)| < \infty \eqsp.
$$
\end{theorem}
Il est int{\'e}ressant de se poser la question de savoir ce qu'il advient du r{\'e}sultat pr{\'e}c{\'e}dent
lorsque la condition sur la fr{\'e}quence d'{\'e}chantillonnage est viol{\'e}e.
Nous supposons toujours que la fonction $f$ est {\`a} bande limit{\'e}e, $f \in \BL{B}$, mais  que la fr{\'e}quence d'{\'e}chantillonnage $1/T $ est inf{\'e}rieure
{\`a} la bande $ 2 B$ de la fonction. Comme $F_T$ est p{\'e}riodique de p{\'e}riode $1/T$ et
$F_T \in L_2([-1/(2T),1/(2T)])$, cette fonction est d{\'e}veloppable en s{\'e}rie de Fourier,
$$
F_T(\lambda)= T \sum_{n=-\infty}^{\infty} \int_{-1/(2T)}^{1/(2T)} \sum_{n \in \zset} \TF f(\lambda - n/T) \rme^{\rmi 2 \pi \lambda n T} d \lambda \rme^{\rmi 2 \pi \lambda n T}.
$$
Un calcul {\'e}l{\'e}mentaire montre que
\begin{multline*}
\int_{-1/(2T)}^{1/(2T)} \sum_{n \in \zset} \TF f(\lambda - n/T) \rme^{\rmi 2 \pi \lambda n T} d \lambda = \sum_{n \in \zset} \int_{-1/(2T)}^{1/(2T)}  \TF f(\lambda - n/T) \rme^{\rmi 2 \pi \lambda n T} d \lambda = \\
\sum_{n \in \zset} \int_{-1/(2T)-n/T}^{1/(2T)-n/T}  \TF f(\lambda) \rme^{\rmi 2 \pi \lambda n T} d \lambda = \int \TF f(\lambda) \rme^{\rmi 2 \pi \lambda n T} d \lambda = f(kT).
\end{multline*}
Par cons{\'e}quent, nous avons encore $\sum_{k \in \zset} |f(kT)|^2 < \infty$ et la formule sommatoire de Poisson \eqref{eq:FormuleSommatoirePoisson} reste valide.
En appliquant $\TF$ aux deux membres de \eqref{eq:FormuleSommatoirePoisson}, nous obtenons donc
$$
\left[\TFAC{\sum_{n \in \zset} \TF f(\lambda - n /T) \1_{[-1/(2T),1/(2T)]}(\lambda)}\right](t)
= \sum_{n \in \zset} f(nT) \mathrm{sinc} \left( \frac{\pi}{T}(t-nT) \right).
$$
Par cons{\'e}quent, si la condition sur la fr{\'e}quence d'{\'e}chantillonnage n'est pas respect{\'e}e, la transform{\'e}e de Fourier de la fonction interpol{\'e}e sera {\'e}gale
{\`a} la transform{\'e}e du signal "p{\'e}riodis{\'e}e". On parle, pour qualifier ce ph{\'e}nom{\`e}ne de  \emph{repliement spectral}, ou d'\emph{aliasing}.


Posons, pour tout entier $k$, $\phi_{T,k}(t) = s_T(t- k T)$. La fonction $\phi_{T,k}$ appartient {\`a} $L_2(\rset)$.
\begin{proposition}
\label{prop:SincBH}
La famille $\{ \phi_{T,k} \}_{k \in \zset}$ est une base Hilbertienne de l'espace $\BL{1/(2T)}$.
\end{proposition}
\begin{proof}
Montrons tout d'abord que les fonctions $\{ \phi_{k,T} \}_{k \in \zset}$ forment une famille orthogonale. Nous avons, par application de la formule
de Parseval,
$$
\int \phi_{T,k} \phi_{T,l} = \int \TFA{\phi_{T,k}} \overline{\TFA{\phi_{T,l}}} \eqsp.
$$
On a, par d{\'e}finition, $\TFA{\phi_{T,k}}= T \1_{[-1/(2T),1/(2T)]}(\lambda) \rme^{- \rmi 2 \pi \lambda k T}$. Par cons{\'e}quent,
$$
\int \phi_{T,k} \phi_{T,l} = T^2 \int_{-1/(2T)}^{1/(2T)} \rme^{- \rmi 2 \pi \lambda (k-l) T} d \lambda = \begin{cases} T & \quad k = l \\ 0 & \quad k \ne l \end{cases} \eqsp.
$$
Montrons maintenant que la famille $\{ \phi_{T,k} \}$ forment une
base totale de l'ensemble $\BL{[-1/(2T),1/(2T)]}$. La formule de reconstruction \eqref{eq:FormuleInterpolation} montre que pour tout $N$,
et tout $f \in \BL{[-1/(2T),1/(2T)]}$, nous avons
$$
\left \| f - \sum_{k=-N}^N f(kT) \phi_{T,k}  \right \|_2^2 = T \sum_{N \leq |k| < \infty} |f(kT)|^2 \eqsp,
$$
ce qui montre que l'ensemble de fonctions $\{ \phi_{T,k} \}$ est dense dans l'ensemble $\BL{1/(2T)}$.
\end{proof}
Soit $f \in L_2(\rset)$. Cette fonction n'est pas a priori {\`a} bande limit{\'e}e, o{\`u}, si elle est {\`a} bande limit{\'e}e, cette bande
n'est peut {\^e}tre pas compatible avec la fr{\'e}quence d'{\'e}chantillonnage de la fonction, $T \geq 1/(2B)$. On sait que si l'on applique
la proc{\'e}dure d'{\'e}chantillonnage d{\'e}crite ci-dessus sans pr{\'e}caution particuli{\`e}re, le signal discr{\'e}tis{\'e} et interpol{\'e} sera une version alt{\'e}r{\'e}e
du signal original (repliement de spectre). Une approche consiste, avant d'{\'e}chantillonner la fonction, de la projeter sur l'espace
$\BL{1/(2T)}$. Pour $f \in L_2(\rset)$, le calcul de cette projection est {\'e}l{\'e}mentaire. Consid{\'e}rons en effet la fonction $\tilde{f}$ d{\'e}finie
par:
$$
\tilde{f}(t) = \int \TF f (\lambda)  \1_{[-1/(2T),1/(2T)]}(\lambda) \rme^{+ \rmi 2 \pi \lambda t} d \lambda \eqsp.
$$
Par construction, nous avons $\TF \tilde{f} (\lambda) = \TF f (\lambda)  \1_{[-1/(2T),1/(2T)]}(\lambda)$, et donc $\tilde{f} \in \BL{1/(2T)}$.
Nous avons d'autre part, pour toute fonction $g \in \BL{1/(2T)}$, par l'identit{\'e} de Plancherel
$$
\| f - g \|_2^2 = \| \TF f - \TF g \|_2^2 \geq \int_{|\lambda| \geq 1/(2T)} | \TF f (\lambda)|^2 d \lambda = \| f - \tilde{f} \|_2^2 \eqsp.
$$
Par cons{\'e}quence $\tilde{f}$ est la projection de $f$ sur $\BL{1/(2T)}$.



\chapter{Traitement du signal discret}
\label{discret-chap}
Le traitement du signal discret a pris son essor dans les ann\'ees
70 gr\^ace \`a
l'apparition des microprocesseurs et \`a l'utilisation de
la transform\'ee de Fourier rapide.
Il remplace progressivement le
traitement du signal analogique dans la majorit\'e des applications
telles que l'enregistrement digital, la t\'el\'evision, le
traitement de la parole et de l'image.
Le calcul informatique permet la mise en
place d'algorithmes
nettement plus sophistiqu\'es et plus pr\'ecis que le
calcul analogique
dont la fiabilit\'e est limit\'ee par
les bruits de circuits et les erreurs
de calibrage des composants \'electroniques.
%Le traitement du
%signal analogique reste cependant beaucoup plus rapide ce qui
%est fondamental pour certaines applications en temps r\'eel.
%
%Les signaux \'etant le plus souvent d'origine analogique, nous
%\'etudions la conversion analogique-digitale et les
%conditions permettant d'effectuer la transformation
%inverse. Le filtrage homog\`ene est \'etendu au calcul discret
%et nous introduisons le calcul rapide par transform\'ee de Fourier
%discr\`ete.
%
%\section{Conversion analogique-digitale}
%
%L'approche la plus
%simple pour discr\'etiser une fonction $f(t)$
%est d'effectuer un \'echantillonnage avec un intervalle
%$T$ uniforme,
%en enregistrant les valeurs $\{f(n T )\}_\nZ$.
%Pour effectuer la transformation inverse, nous
%\'etudions l'existence d'algorithmes d'interpolation
%permettant de reconstruire $f(t)$ \`a partir de ses \'echantillons.
%
%\subsection{Echantillonnage}
%
%Pour traiter les signaux discrets dans le m\^eme cadre que les
%signaux analogiques, nous les repr\'esentons par des distributions
%de Dirac.
%Un \'echantillon $f(nT)$ est repr\'esent\'e par
%un Dirac d'amplitude $f(nT)$ centr\'e en $n T$.
%L'\'echantillonnage uniforme de $f(t)$ correspond \`a la distribution
%\begin{equation}
%\label{fdanisi}
%f_d (t) = \sum_{n=- \infty} ^{+ \infty} f(nT) \delta (t -nT) .
%\end{equation}
%Puisque $f(nT) \delta (t -nT) = f (t) \delta (t-nT)$,
%\[
%f_d (t) = f (t)  \sum_{n=- \infty} ^{+ \infty}  \delta (t -nT) .
%\]
%Un \'echantillonnage uniforme est donc obtenu par multiplication
%avec le peigne de Dirac
%\begin{equation}
%c(t) = \sum_{n=-\infty}^{+ \infty} \delta (t -nT).
%\end{equation}
%
%Les propri\'et\'es de cet \'echantillonnage s'\'etudient
%plus facilement
%dans le domaine de Fourier.
%Si $\{ f(n T)\}_\nZ$ est born\'e, $f_d (t)$ est une
%distribution temp\'er\'ee \cite{bony}
%dont la transform\'ee de Fourier $\hat f_d (\om)$
%est bien d\'efinie.
%La transform\'ee de Fourier de
%$ \delta (t -nT)$ \'etant $e^{-i nT \om}$, on d\'eduit de
%(\ref{fdanisi}) que
%$\hat f_d (\om)$ est une s\'erie de Fourier
%$\frac {2 \pi} T$ p\'eriodique
%\begin{equation}
%\hat f_d ( \om ) = \sum_{n=0}^{+ \infty} f (nT) e ^{-inT \om } .
%\end{equation}
%
%Pour comprendre comment reconstruire $f(t)$ \`a partir de ses
%\'echantillons, nous exprimons $\hat f_d (\om)$ en fonction de
%$\hat f (\om)$. Comme $f_d (t) = f (t) c(t)$, sa transform\'ee de
%Fourier peut aussi s'\'ecrire
%\[
%\hat f_d (\om) = \frac 1 {2 \pi} \hat f \star \hat c(\om).
%\]
%La formule de Poisson (\ref{poisson}) prouve que la transform\'ee de
%Fourier du peigne de Dirac $c(t)$ est
%\begin{equation}
%\hat c (\om) = \frac {2 \pi} T \sum_{k=-\infty}^{+\infty}
%\delta(\om - \frac {2\pi k} T) .
%\end{equation}
%Comme
%$\hat f \star \delta(\om - \frac {2\pi k} T) = \hat f (\om -\frac {2\pi k} T)$,
%\begin{equation}
%\label{periodize}
%\hat f_d (\om) =
%\frac 1 T \sum_{k=- \infty} ^{+ \infty}
%\hat f ( \om - \frac {2 k \pi} T ) .
%\end{equation}
%Echantillonner un signal est donc \'equivalent \`a une p\'eriodisation
%de sa transform\'ee de Fourier, obtenue en additionnant les
%translat\'ees $\hat f (\om - \frac{2k\pi}T)$.
%Le th\'eor\`eme de Nyquist donne une condition suffisante sur
%le support de
%$\hat f (\om)$ pour reconstruire
%$f(t)$ \`a partir des \'echantillons $f(nT)$.
%Cette condition garantit que $f(t)$ n'a pas d'oscillations
%violentes entre chaque paire d'\'echantillons.
%
%\begin{theorem} [Nyquist]
%\label{nyquist-th}
%Soit $f (t)$ un signal dont la transform\'ee de Fourier
%$\hat f (\om)$ a un support inclus dans $[-\frac \pi T,\frac \pi T]$.
%Alors $f(t)$ peut \^etre reconstruite en interpolant
%ses \'echantillons
%\begin{equation}
%\label{sinc-int}
%f (t) = \sum_{n=- \infty} ^{+ \infty}  f(nT) h_T(t - nT) ,
%\end{equation}
%avec
%\begin{equation}
%\label{sinc}
%h_T  (t) = \sinc (\frac {\pi t} T) =
%\frac{\sin \frac {\pi t} T} {\frac { \pi t}T } .
%\end{equation}
%\end{theorem}
%
%\noindent{\bf D\'emonstration}\\
%Comme le support de
%$\hat f ( \om )$ est inclus dans $[-\frac \pi T,\frac \pi T]$,
%si $n\neq 0$ le
%support de $\hat f ( \om -\frac{2n \pi}T)$ n'intersecte pas
%le support de $\hat f( \om )$.
%En cons\'equence  (\ref{periodize})
%prouve que
%\begin{equation}
%\label{localise}
%\hat f_d ( \om  ) = \frac {\hat f (\om )} T ~~ \mbox{si}~~
%| \om | \leq \frac {\pi } T .
%\end{equation}
%Soit $\hat h_T (\om)$ la fonction de transfert d'un
%filtre passe-bas id\'eal
%\begin{equation}
%\hat h_T ( \om  ) =
%   \left \{ \begin{array}{ll}
%            T & \mbox{si $| \om | \leq \frac \pi T$}\\
%            0 & \mbox{si $| \om | > 0$}
%            \end{array}
%   \right.
%\end{equation}
%et
%dont la r\'eponse impulsionnelle $h_T (t)$ est donn\'ee par
%(\ref{sinc}).
%On d\'eduit de (\ref{localise})
%que
%\[
%\hat f (\om ) = \hat h_T( \om ) \hat f_d (\om )
%\]
%ce qui se traduit en variable de temps par
%\[
%f (t) = h_T  \star f_d (t) = h_T \star
%\sum_{n=- \infty} ^{+ \infty} f(nT) \delta (t -nT)  =
%\sum_{n=- \infty} ^{+ \infty} f(nT) h_T(t -nT) .
%\]
%$\Box$
%
%Le th\'eor\`eme d'\'echantillonnage de Nyquist
%donne une condition n\'ecessaire pour reconstruire un
%signal \`a partir de ses \'echantillons \'etant donn\'ee
%une information
%\`a priori sur son support fr\'equentiel.
%L'\'echantillonnage et l'interpolation sont illustr\'es par
%la figure \ref{echant-interp},
%dans les domaines temporels et fr\'equentiels.
%D'autres
%caract\'erisations
%peuvent \^etre obtenues en imposant des contraintes
%diff\'erentes sur $f(t)$.
%
%\begin{figure}[bhtp]
%
%\centerline{
%	\epsfxsize=14cm
%	\leavevmode\epsfbox{/home/mallat/X/TREX/figures/SigFig/MALLATFIG3.1-EPS}}
%\caption{Echantillonnage et interpolation dans
%les domaines temporels et fr\'equentiels}
%\label{echant-interp}
%\end{figure}
%
%\subsection{Repliement spectral}
%\label{repliement-sp}
%Si le support de $\hat f(\om)$ n'est pas inclus
%dans $[-\frac \pi T, \frac \pi T]$, la formule d'interpolation
%(\ref{sinc-int}) ne reconstruit pas $f(t)$.
%Nous \'etudions les propri\'et\'es de l'erreur de reconstruction
%ainsi qu'une proc\'edure de filtrage pour la r\'eduire.
%\\
%\\
%{\bf Recouvrement fr\'equentiel}
%La transform\'ee de Fourier de
%$h_T  \star f_d (t)$ a un support inclus dans $[-\frac \pi T,\frac \pi T]$
%et donc ne peut \^etre \'egale \`a
%$f(t)$ dont la transform\'ee de Fourier a un support qui s'\'etend
%au-del\`a de $[-\frac \pi T,\frac \pi T]$.
%Nous avons vu que
%\begin{equation}
%\hat f_d (\om) =
%\frac 1 T \sum_{k=- \infty} ^{+ \infty}
%\hat f ( \om - \frac {2k\pi} T ) .
%\end{equation}
%Lorsque le support de $\hat f(\om)$ n'est pas inclus dans
%$[-\frac \pi T, \frac \pi T]$, pour certaines
%fr\'equences $\om \in [-\frac \pi T, \frac \pi T]$ il existe des entiers
%$k \neq 0$ pour lesquels
%$\hat f(\om - \frac{2k \pi}T) \neq 0$ (voire figure \ref{recouvre}).
%Dans ce cas, $\hat f_d (\om)$ est la somme de
%$\hat f(\om)$ plus certaines
%composantes de hautes fr\'equences $\hat f(\om - \frac{2k\pi}T)$.
%La valeur de
%$\hat f_d (\om) \hat h_T (\om)$ peut donc \^etre tr\`es diff\'erente
%de $\hat f(\om)$ m\^eme lorsque $\om \in [-\frac \pi T, \frac \pi T]$.
%
%Consid\'erons par exemple le signal
%\[
%f(t) = \cos (\om_0 t) = \frac {e^{i \om_0 t} + e^{-i \om_0 t}} 2
%\]
%avec $ \frac{2 \pi }T > \om_0 > \frac \pi T$.
%Sa transform\'ee de Fourier \'etant
%\[
%\hat f(\om) = \pi \Bigl(\delta(\om-\om_0) + \delta(\om+\om_0)\Bigr),
%\]
%la p\'eriodisation (\ref{periodize}) nous donne
%\[
%\hat f_d (\om) = \frac \pi T \sum_{k=-\infty}^{+\infty}
%\Bigl( {\delta(\om-\om_0 - \frac{2k\pi }T ) +
%\delta(\om+\om_0-\frac{2k\pi}T)}\Bigr) .
%\]
%Les seules composantes dans $[-\frac \pi T,\frac \pi T]$ sont
%${\delta(\om-\om_0 + \frac{2\pi}T ) +
%\delta(\om+\om_0-\frac{2\pi}T)} $ donc
%apr\`es filtrage par le filtre passe-bas $h_T (\om)$, on obtient
%\[
%f_d \star h_T (t)  =  \cos  \Bigl((\frac{2\pi}T-\om_0) t\Bigr).
%\]
%Le repliement spectral r\'eduit la
%fr\'equence du cosinus de $\om_0$ \`a
%$\frac {2\pi} T-\om_0 \in [-\frac \pi T, \frac \pi T]$.
%Ce repli fr\'equentiel s'observe lorsque l'on filme un mouvement
%trop rapide avec un nombre insuffisant d'images par seconde.
%Une roue de voiture tournant \`a grande vitesse appara\^{\i}t
%comme tournant beaucoup plus lentement dans le film.
%\\
%\\
%{\bf Pr\'efiltrage} Supposons que
%le pas d'\'echantillonnage est limit\'e \`a une valeur $T$
%par des contraintes de temps calcul ou de m\'emoire et que
%$\om_0 >  \frac{\pi} T$. A d\'efaut de reconstruire exactement
%$f(t)$, on veut r\'ecup\'erer la meilleure approximation
%de $f(t)$ par interpolation d'un \'echantillonnage avec
%$h_T (t)$. Une telle interpolation est une
%convolution avec
%$h_T (t)$ et a donc
%une transform\'ee de Fourier dont le support est
%inclus dans $[- \frac {\pi} T , \frac {\pi} T]$.
%Soit $\V$ l'espace des fonctions dont les transform\'ees de
%Fourier ont un support
%inclus dans $[- \frac {\pi} T , \frac {\pi} T]$.
%La fonction de $\V$ qui est la plus proche de $f(t)$ est la
%projection orthogonale $\P_\V f (t)$ de $f(t)$ dans $\V$
%qui minimise
%\begin{equation}
%\label{dist}
%\| f - \P_\V f \|^2 =
%\frac 1 {2 \pi} \int_{-\infty}^ {+ \infty}
%|\hat f (\om) - \hat {\P}_\V f (\om)|^2 d\om .
%\end{equation}
%Comme $\P_\V f(t) \in \V$, le support de sa transform\'ee
%de Fourier $\hat {\P}_\V f (\om)$
%est incluse dans
%$[- \frac {\pi} T , \frac {\pi} T]$.
%La distance (\ref{dist})
%est minimis\'ee si
%\[
%\hat {\P_\V f} (\om) = \hat f (\om)~~
%\mbox{pour $|\om| \leq \frac \pi T$} .
%\]
%La projection orthogonale
%est donc obtenue par le filtrage lin\'eaire
%\begin{equation}
%\label{prefiltre}
%\P_\V f (t) = \frac 1 T f \star h_T (t)
%\end{equation}
%qui enl\`eve toute composante fr\'equentielle au del\`a de
%la fr\'equence d'\'echantillonnage $\frac \pi T$.
%Puisque $\P_\V f \in \V$, le th\'eor\`eme de Nyquist prouve que
%\[
%\P_\V f (t)  = \sum_{n=- \infty} ^{+ \infty}
%\P_\V f(nT) h_T(t - nT) .
%\]
%On calcule la projection orthogonale de $f(t)$ sur $\V$
%en pr\'efiltrant $f(t)$ avec (\ref{prefiltre}) et cette
%projection orthogonale est reconstruite \`a partir de son
%\'echantillonnage uniforme.
%Un convertisseur analogique digital est donc compos\'e d'un filtre
%qui limite la bande de fr\'equence du signal \`a
%$[-\frac \pi T , \frac \pi T ]$ suivi d'un \'echantillonnage
%uniforme
%avec intervalles $T$.
%En pratique, l'impl\'ementation par circuit \'electronique
%n\'ecessite d'approximer le filtre passe-bas id\'eal $h_T (t)$
%par un filtre r\'ealisable (par exemple Butterworth ou Chebyshev).
%
%\begin{figure}[bhtp]
%\centerline{
%	\epsfxsize=14cm
%	\leavevmode\epsfbox{/home/mallat/X/TREX/figures/SigFig/MALLATFIG3.2-EPS}}
%\caption{Cette figure illustre le recouvrement
%spectral cr\'e\'e par un pas d'\'echantillonnage trop grand.
%Le signal
%reconstruit $f_d \, * \, h_T(t)$ peut \^etre tr\`es diff\'erent de
%$f(t)$}
%\label{recouvre}
%\end{figure}

\section{Filtrage discret homog\`ene}

Les op\'erateurs analogiques de filtrage lin\'eaire homog\`ene
s'\'etendent aux signaux discrets en rempla\c{c}ant les
int\'egrales par des sommes discr\`etes.
La transform\'ee de Fourier est alors remplac\'ee par les
s\'eries de Fourier. Les propri\'et\'es des filtres discrets
s'analysent souvent plus facilement
avec la transform\'ee en z qui \'etend les
s\'eries de Fourier \`a tout le plan complexe.
Pour simplifier
les notations, nous supposons que l'intervalle d'\'echantillonnage
est $T=1$ et les \'echantillons d'un signal discret sont not\'es
$f[n]$.

\subsection{Convolutions discr\`etes}
\label{sec-conv}
Dans le cas discret, l'homog\'en\'eit\'e temporelle se limite
\`a des translations sur la grille d'\'echantillonnage.
Un op\'erateur lin\'eaire discret $L$ est homog\`ene dans le
temps si et seulement si pour tout $f[n]$ et tout
d\'ecalage $f_p [n] = f[n-p]$ avec $p \in \Z$
\[
L f_p [n] = Lf[n-p] .
\]
\\
\\
{\bf R\'eponse impulsionnelle}
On note $\delta [n]$ le Dirac discret
\begin{equation}
\delta [n] =
   \left \{ \begin{array}{ll}
            1 & \mbox{si $n = 0$}\\
            0 & \mbox{si $n \neq 0$}
            \end{array} .
   \right.
\end{equation}
Tout signal $f[n]$ peut \^etre d\'ecompos\'e
comme somme de Diracs translat\'es
\[
f[n] = \sum_{p=-\infty}^{+\infty} f[p] \delta [n-p] .
\]
Soit $L\delta[n] = h[n]$ la r\'eponse impulsionnelle
de cet op\'erateur.
La lin\'earit\'e et l'invariance temporelle impliquent
\[
Lf[n] = \sum_{p=-\infty}^{+\infty} f[p] h [n-p] = f \star h [n].
\]
Un op\'erateur lin\'eaire homog\`ene est donc un produit de
convolution discret.
\\
\\
{\bf Stabilit\'e et causalit\'e}
Un filtre discret $L$
est
{\it causal} si et seulement si $Lf[p]$ ne
d\'epend que des valeurs de $f[n]$ pour $n \leq p$. Cela
implique donc que $h[n] = 0$ si $n < 0$.
La r\'eponse impulsionnelle $h[n]$ est causale.
On repr\'esente souvent un signal causal gr\^ace \`a la fonction
de Heavyside discr\`ete
\begin{equation}
\gamma [n] =
   \left \{ \begin{array}{ll}
            1 & \mbox{si $n \geq 0$}\\
            0 & \mbox{si $n < 0$}
            \end{array}
   \right.
\end{equation}
car $h[n] = h[n] \gamma [n]$.


Pour qu'un signal
d'entr\'ee born\'e produise un signal de sortie born\'e il suffit que
\begin{equation}
\sum_{n=- \infty}^{+ \infty} |h [ n  ]| < + \infty ,
\end{equation}
car
\[
|Lf[n]| \leq \sup_{n \in \Z} |f[n]|
\sum_{k=- \infty}^{+ \infty} |h [ k  ]| .
\]
On peut v\'erifier (exercice) que cette condition suffisante
est aussi n\'ecessaire.
On dit alors que le filtre et la r\'eponse impulsionnelle
sont {\it stables}.
\\
\\
{\bf Fonction de transfert}
Comme dans le cas continu, les vecteurs propres de ces
op\'erateurs de convolutions sont des exponentielles complexes
$e_\om[k] = e^{i\om k}$,
\begin{equation}
L e_\om [n] = \sum_{k=- \infty}^{+ \infty}
e^{i\om (n-k)} h [k] =
e^{i \om n} \sum_{k=- \infty}^{+ \infty} h[k] e^{-i \om k} .
\end{equation}
Les valeurs propres correspondantes sont donc obtenues
par la s\'erie de Fourier
\begin{equation}
\hat h ( e^{i \om} ) = \sum_{k=- \infty}^{+ \infty} h[k] e^{-i\om k} ,
\end{equation}
que l'on appelle fonction de transfert du filtre.

\subsection{S\'eries de Fourier}

La transform\'ee de Fourier d'un signal discret $f[n]$
est d\'efinie par
\begin{equation}
\label{serie-Fourier}
\hat f ( e^{i \om} ) = \sum_{k=- \infty}^{+ \infty} f[k] e^{-i\om k} .
\end{equation}
C'est la transform\'ee de Fourier de sa repr\'esentation
par somme de Dirac
\[
f_d (t) = \sum_{n=- \infty} ^{+ \infty} f[n] \delta (t -n) .
\]
Toutes les propri\'et\'es de la transform\'ee de Fourier
(\ref{symm}-\ref{impaire})
restent donc valables
si $f_d (t)$ est une distribution temp\'er\'ee,
ce qui est le cas si $|f[n]|$ est born\'e.

On peut aussi d\'emontrer \cite{bony} que
la famille $\{\ekom\}_\kZ$ est une
base orthonormale de $\Ld[-\pi, \pi]$ muni du produit scalaire
\[
<a(\om),b(\om)> =
\frac 1 {2 \pi}
\int_{- \pi}^{\pi} a( \om ) b^* (\om) d \om .
\]
Si $f[n] \in \lD$,
la s\'erie (\ref{serie-Fourier}) peut
alors s'interpr\'eter
comme la d\'ecomposition de $\hat f (\eom) \in \Ld[0,2\pi]$
dans cette base orthonormale.
Les coefficients de d\'ecomposition
sont obtenus par produit scalaire
\begin{equation}
\label{recons_discret}
f[n] = <\hat f(\eom),e ^{-i \om n}  > = \frac 1 {2 \pi}
\int_{- \pi}^{\pi} \hat f(e^{i \om }) e ^{i \om n} d \om .
\end{equation}
et
\begin{equation}
\sum_{n=-\infty}^{+\infty} |f[n]|^2 =
\frac 1 {2 \pi}
\int_{- \pi}^{\pi} |\hat f(e^{i \om })|^2 d \om .
\end{equation}
\\
\\
{\bf Filtrage discret}
Les exponentielles complexes \'etant les vecteurs propres des
op\'erateurs de convolution discr\`ete, il en r\'esulte
le th\'eor\`eme suivant.

\begin{theorem} [Convolution discr\`ete]
Soient $f[n]$ et $h[n]$ deux signaux dans $\lD$.
La transform\'ee de Fourier de $g[n] = f \star h [n]$ est
\begin{equation}
\hat g(e^{i \om}) = \hat f(e^{i \om}) \hat h(e^{i \om}) .
\end{equation}
\end{theorem}

La d\'emonstration est
formellement identique \`a la d\'emonstration du th\'eor\`eme
\ref{th_convol} si on
remplace les int\'egrales par des sommes discr\`etes
et que l'on suppose que $f[n]$ et $h[n]$ sont dans $\lU$.
Le m\^eme r\'esultat dans $\lD$ s'obtient par un argument de
densit\'e.

La formule de reconstruction (\ref{recons_discret})
montre qu'un signal filtr\'e s'\'ecrit
\begin{equation}
f \star h[n] = \frac 1 {2 \pi}
\int_{- \pi}^{\pi} \hat h(\eom) \hat f(e^{i \om }) e ^{i \om n} d \om .
\end{equation}
La fonction de transfert $\hat h(\eom)$ amplifie ou att\'enue
les composantes fr\'equentielles $\hat f(\eom)$ de $f[n]$
dans l'intervalle de fr\'equence $[-\pi,\pi]$.

On v\'erifie de m\^eme qu'une multiplication temporelle est
\'equivalente \`a une convolution dans le domaine fr\'equentiel.
Si $g[n] = f[n] w[n]$ alors
\[
\hat g(\eom) = \frac 1 {2 \pi} \int_{-\pi}^{\pi}
\hat f(e^{i u }) \hat w (e^{i(\om -u)}) du .
\]
\\
\noindent{\bf Exemple}\\
La moyenne discr\`ete uniforme d\'efinie par
\[
Lf[n] = \frac 1 {2N+1} \sum_{p=-N}^{+N} f[n-p] ,
\]
est un filtre dont la r\'eponse impulsionnelle est
\begin{equation}
h[n] =
   \left \{ \begin{array}{ll}
            \frac 1 {2N+1}& \mbox{si $-N \leq  n \leq N$}\\
            0 & \mbox{si $|n| > N$}
            \end{array}
   \right.
\end{equation}
La fonction de transfert est la s\'erie de Fourier
\[
\hat h(\eom) = \frac 1 {2N+1}
\sum_{n=-N}^{+N} e^{-in\om} = \frac 1 {2N+1}
\frac {\sin (N+\half)\om} {\sin\om/2} .
\]
Ce filtre att\'enue surtout les fr\'equences
au-del\`a de $ {2\pi}/ (2N+1)$.

\subsection{S\'election fr\'equentielle id\'eale}

La fonction de transfert d'un
filtre discret \'etant $2\pi$ p\'eriodique,
elle est sp\'ecifi\'ee sur l'intervalle $[-\pi,\pi]$.
La fonction de transfert
du filtre discret passe-bas id\'eal est d\'efinie pour
$\om \in [-\pi,\pi]$ par
\begin{equation}
\label{passe-bas-discret}
\hat h_0 ( e^{i \om}  ) =
   \left \{ \begin{array}{ll}
            1 & \mbox{si $| \om | \leq \om_c$}\\
            0 & \mbox{si $| \om | > \om_c$}
            \end{array}
   \right.
\end{equation}
Sa r\'eponse impulsionnelle calcul\'ee gr\^ace
\`a l'int\'egrale (\ref{recons_discret}) est
\begin{equation}
h_0 [n] = \frac{\sin \om_c n} { \pi n} .
\end{equation}
C'est l'\'echantillonnage uniforme de la fonction de transfert
d'un filtre analogique passe-bas id\'eal.

La fonction transfert d'un filtre passe-bande discret id\'eal
est
\begin{equation}
\hat h_1 ( \eom  ) =
   \left\{ \begin{array}{ll}
1 & \mbox{si $|\om| \in [\om_0 - \om_c , \om_0 + \om_c ]$}\\
0 & \mbox{ailleurs}
\end{array}
   \right.
\end{equation}
Comme $\hat h_1(\eom) = \hat h_0 (e^{i(\om-\om_0) }) +
\hat h_0 (e^{i(\om+\om_0) }) $, on peut en d\'eduire que
sa r\'eponse impulsionnelle est
\[
h_1 [n] =
2 \cos (\om_0 n)~ h_0[n].
\]

La convolution discr\`ete d'un signal $f[n]$ avec un filtre
passe-bas ou passe-bande id\'eal ne peut se calculer exactement
avec un nombre fini d'op\'erations. Il est donc n\'ecessaire
d'approximer ces filtres par des op\'erateurs de convolutions
qui se calculent en temps fini.


\section{Synth\`ese de filtres discrets}

Lors de la synth\`ese de filtres discrets, tout comme dans le
cas analogique, on impose des conditions
d'att\'enuation sur le gain du filtre $|\hat h (\eom)|$.
Le probl\`eme est d'obtenir des filtres tels
que $|\hat h (\eom)|$ satisfasse aux conditions d'att\'enuation et
dont la structure permette de calculer les convolutions discr\`etes
avec le moins d'op\'erations possibles.

\subsection{Filtres r\'ecursifs}
\label{recursifs}
Pour effectuer des calculs num\'eriques, on utilise
une classe de filtres pour lesquels la convolution discr\`ete
se calcule avec un nombre fini d'op\'erations par
\'echantillon.
La sortie $g[n] = Lf[n]$ est reli\'ee \`a $f[n]$
par une \'equation de diff\'erences
\begin{equation}
\label{recurr}
\sum_{k=0}^N a_k g[n-k] = \sum_{k=0}^M b_k f[n-k] ,
\end{equation}
o\`u $a_k$ et $b_k$ sont des r\'eels et $a_0 \neq 0$.
Donc
\[
g[n] = \frac 1 {a_0} \left(
\sum_{k=0}^M b_k f[n-k] - \sum_{k=1}^N a_k g[n-k] \right)
\]
se calcule \`a partir de son pass\'e et de $f[n]$ avec
$N+M$ multiplications et additions.

Etant donn\'e un signal causal $f[n]$, le calcul de $g[n]$
n\'ecessite la connaissance de ``conditions initiales'',
par exemple $N$ valeurs cons\'ecutives de $g[n]$.
Si l'on impose que $g[n] = 0$ pour $-N \leq n < 0$, alors
$g[n]$ est enti\`erement caract\'eris\'e pour tout $n \in \Z$.
L'op\'erateur $L$ est alors un filtre lin\'eaire homog\`ene causal.

Si $N = 0$ alors le fitre a une r\'eponse impulsionnelle $h[n]$ finie
de taille $M$
\[
g[n] =
\sum_{k=0}^M \frac {b_k} {a_0} f[n-k] = h \star f [n] .
\]
Si $M = 0$, on dit que le filtre est autor\'egressif
\[
g[n] = \frac {b_0} {a_0} f[n] - \sum_{k=1}^N \frac{a_k} {a_p}
g[n-k] .
\]
\\
\\
{\bf Fonction de transfert}
Pour caract\'eriser la classe des op\'erateurs de convolutions $L$
qui satisfont (\ref{recurr}),
nous \'evaluons la condition impos\'ee
sur la fonction de transfert en calculant
la transform\'ee de Fourier de chaque c\^ot\'e de l'\'egalit\'e
(\ref{recurr}).
Si $\hat f(\eom)$ est la transform\'ee de Fourier de $f[n]$
alors la transform\'ee de Fourier
de $f[n-k]$ est $e^{-i k \om} \hat f(\eom)$. La transform\'ee
de Fourier de (\ref{recurr}) est donc
\[
\sum_{k=0}^N a_k \,e^{-ik \om} \hat g(\eom)
=
\sum_{k=0}^M b_k \,e^{-ik\om} \hat f(\eom) ,
\]
d'o\`u l'on d\'eduit que
\begin{equation}
\label{RecruTrasfFunc}
\hat h(\eom) = \frac {\hat g(\eom)} {\hat f(\eom)} =
\frac {\sum_{k=0}^M b_k \,e^{-ik\om}} {\sum_{k=0}^N a_k \,
e^{-ik\om} }.
\end{equation}
La fonction de transfert d'un filtre r\'ecursif est donc un rapport
de polyn\^{o}mes en $e^{-i\om}$.

Les propri\'et\'es du module
et de la phase s'analysent plus facilement en calulant les
p\^oles $d_k$ et les z\'eros $c_k$ de la fonction rationnelle
(\ref{RecruTrasfFunc})
\[
\hat h(\eom) = \frac {b_0} {a_0} \frac
{\prod_{k=1}^M (1 - c_k e^{-i\om})} {\prod_{k=1}^N (1 - d_k e^{-i\om})} .
\]
Le module de la transform\'ee de Fourier est donc
\[
|\hat h(\eom)| = \frac {|b_0|} {|a_0|} \frac
{\prod_{k=1}^M |1 - c_k \emom|} {\prod_{k=1}^N |1 - d_k \emom|} .
\]
L'amplitude de la fonction de transfert
est le plus souvent calcul\'ee en d\'ecibels (db) qui mesurent
\[
20 \log_{10} |\hat h(\eom)| =
10 \log_{10} \frac {|b_0|^2} {|a_0|^2} +
\sum _{k=1}^M 10 \log_{10} |1 - c_k \emom|^2
- \sum_{k=1}^N 10 \log_{10} |1 - d_k \emom|^2 .
\]
Les p\^oles et les z\'eros ne
se distinguent donc que par un changement
de signe.
La phase complexe de $\hat h(\eom)$ se mesure de m\^eme par
\[
\arg \hat h(\eom) = \arg \frac {b_0} {a_0} +
\sum _{k=1}^M \arg (1 - c_k \emom)
- \sum_{k=1}^N \arg (1 - d_k \emom) .
\]
\\
\\
{\bf Exemple}
Prenons le cas d'un p\^ole ou d'un z\'ero situ\'e en $re^{i\theta}$
et \'etudions le module et la phase de $(1-r e^{i \theta} \emom)$.
\[
10 \log_{10} |1-r e^{i \theta} \emom|^2 =
10 \log_{10} [ 1 + r^2 - 2r\cos(\om - \theta) ] .
\]
Le module est donc minimum pour $\om = \theta$ o\`u il vaut
$20 \log_{10} |1-r|$ et
maximum en $\om = \theta + \pi$ o\`u il vaut
$20 \log_{10} |1+r|$.
Suivant que ce facteur est un p\^ole ou un z\'ero, il produit
une att\'enuation ou une
amplification au voisinage de $\om = \theta$.
La phase complexe est
\[
\arg\hat h(\eom) = \arctan \left[
\frac {r \sin(\om - \theta)} {1 -r \cos(\om - \theta)} \right] .
\]

\subsection{Transform\'ee en z}

Pour \'etudier plus facilement les propri\'et\'es des fonctions
de transfert des filtres discrets, et en particulier des
filtres r\'ecursifs,
on introduit la transform\'ee en z
qui \'etend la s\'erie de Fourier
\begin{equation}
\hat h(\eom) = \sum_{n=-\infty}^{+\infty} h[n] \emnom
\end{equation}
\`a tout le plan complexe $z \in \C$, avec la s\'erie de Laurent
\begin{equation}
\hat h(z) = \sum_{n=-\infty}^{+\infty} h[n] z^{-n} .
\end{equation}
\\
\\
{\bf Anneau de convergence}
On dit que la s\'erie de Laurent $\hat h(z)$ est convergente si
\[
\sum_{n=-\infty}^{+\infty} |h[n]|\, |z|^{-n} < + \infty .
\]
Le domaine de convergence ne d\'epend que de $|z|$ et est donc
isotrope. La proposition suivante
montre
que le domaine de convergence est un anneau dans le plan complexe.

\begin{proposition}
\label{conv-z}
Il existe $\rho_1$
et $\rho_2$ tels que $\hat h(z)$ est convergente
pour $\rho_1 < |z| < \rho_2$ et divergente
pour $|z| < \rho_1$ ou $|z| > \rho_2$.
On note $A(\hat h)$ l'intervale de $|z|$ sur lequel $\hat h(z)$
est convergente.
\end{proposition}

La d\'emonstration est laiss\'ee en exercice.
Dans le cas o\`u la transform\'ee en z est convergente pour
$|z| = 1$, la transform\'ee de Fourier est \'egale \`a
la restriction
de la transform\'ee en z au cercle unit\'e du plan complexe.
\\
\\
{\bf Stabilit\'e et causalit\'e} Le domaine de convergence (absolu)
de la transform\'ee en $z$ d\'epend des
propri\'et\'es de causalit\'e et de stabilit\'e du filtre.
Le filtre est causal si
$h[n] = 0$ pour $n < 0$ d'o\`u l'on d\'eduit que si
$\hat h (z)$ converge pour $|z| = \rho$ alors il converge pour
$|z| \geq \rho$. L'anneau de convergence s'\'etend donc \`a l'infini
($\rho_2 = +\infty$).

Le filtre est stable si et seulement si
\[
\sum_{n=0}^{+\infty} |h[n]| < +\infty .
\]
Cela signifie que l'anneau de convergence contient $|z| = 1$.
Si le filtre est causal et stable, on d\'eduit donc que
$\hat h(z)$ est convergente pour $|z| \geq 1$.
\\
\\
{\bf Inverse}
%La transform\'ee en z s'inverse par une int\'egrale de
%Cauchy sp\'ecifi\'ee par le th\'eor\`eme suivant.
%
%\begin{theorem} [Int\'egrale de Cauchy]
%\label{cauchy}
%Soit $C$ un contour qui entoure l'origine dans
%le plan complexe $\C$ et qui est inclus le domaine de
%convergence de
%\[
%\hat h(z) = \sum_{n=-\infty}^{+\infty} h[n] z^{-n} .
%\]
%Alors
%\begin{equation}
%\label{residu}
%h[k] = \frac 1 {2 \pi i} \oint_C \hat h(z) z^{k-1} dz  ,
%\end{equation}
%avec une int\'egration orient\'ee dans le sens trigonom\'etrique
%direct.
%\end{theorem}
%
%\noindent{\bf D\'emonstration}
%Ce r\'esultat est une cons\'equence de la
%propri\'et\'e suivante de l'int\'egrale
%de Cauchy \cite{bony}.
%Soit $C$ un contour qui entoure l'origine dans $\C$
%\begin{equation}
%\frac 1 {2 \pi i} \oint_C z^{-k} dz =
%\left\{
%\begin{array}{ll}
%1 & \mbox{ si $k=1$}\\
%0 & \mbox{ si $k\neq1$}
%\end{array} .
%\right.
%\end{equation}
%L'int\'egration est calcul\'ee dans le sens direct.
%
%Supposons que $C$ soit inclus dans le domaine de
%convergence de $\hat h(z)$.
%\[
%\frac 1 {2 \pi i} \oint_C \hat h(z) z^{k-1} dz =
%\frac 1 {2 \pi i} \oint_C
%\sum_{n=-\infty}^{+\infty} h[n] z^{-n+k-1} dz .
%\]
%A l'int\'erieur du domaine de convergence, on peut inverser
%ces deux int\'egrales et le th\'eor\`eme de Cauchy implique
%\[
%\frac 1 {2 \pi i} \oint_C \hat h(z) z^{k-1} dz =
%\sum_{n=-\infty}^{+\infty} h[n]
%\frac 1 {2 \pi i} \oint_C
%z^{-n+k-1} dz = h[k] .
%\]
%$\Box$.\\
%
%L'int\'egrale de Cauchy (\ref{residu}) g\'en\'eralise
%la formule d'inversion des s\'eries de
%Fourier (\ref{recons_discret}).
%En effet, si le cercle unit\'e est inclus dans la r\'egion de
%convergence, l'int\'egration le long du cercle unit\'e
%est obtenue
%avec $z = e^{i \om}$ et (\ref{residu}) devient
%\[
%h[k] = \frac 1 {2 \pi} \int_{-\pi}^{\pi}
%\hat h (\eom) e^{i k \om} d\om .
%\]
%
La transform\'ee en $z$ peut s'inverser mais le calcul de
$h[k]$ \`a partir de $\hat h (z)$ d\'epend du domain de
convergence choisi. La formule g\'en\'erale d'inversion se
fait par une int\'egrale de Cauchy qui calcule $h[k]$
en integrant $\hat h (z)$ le long d'un contour inclu
dans l'anneau de convergence.
Dans le cas o\`u l'anneau de convergence
inclu le cercle unit\'e, cette int\'egrale peut se faire le
long du cercle unit\'e, auquel cas on obtient la transform\'ee
de Fourier inverse
\[
h[k] = \frac 1 {2 \pi} \int_{-\pi}^{\pi}
\hat h (\eom) e^{i k \om} d\om .
\]

Pour montrer que
$h[k]$ ne d\'epend pas seulement de
$\hat h(z)$ mais aussi du domaine de convergence choisi,
prenons par exemple
\[
\hat h(z) = \frac 1 {1 - a z^{-1}} .
\]
La r\'eponse impulsionnelle correspondant \`a la r\'egion de
convergence \`a l'ext\'erieur du cercle
$|z| = a$ est causale et se calcule par un d\'evelopement
en s\'erie de $\frac 1 {1 - x}$
\[
\hat h(z) = \sum_{n=0}^{+\infty} a^n z^{-n} ,
\]
d'o\`u $h[n] = a^n \gamma[n]$.
Pour que la r\'egion de convergence soit $|z| < a$ on
r\'e\'ecrit
\[
\hat h(z) = \frac {-a^{-1} z} {1 - a^{-1}z} .
\]
En utilisant la d\'ecomposition en s\'erie de $\frac 1 {1-x}$
on obtient une r\'eponse impulsionnelle anticausale
\[
h[n] = \left\{
\begin{array}{ll}
-a^n & \mbox{si $n \leq -1$}\\
0 & \mbox{si $n \geq 0$}\\
\end{array}
\right.
\]
\\
\\
{\bf Exemples}
On utilise g\'en\'eralement un filtre causal, ce qui
impose que l'anneau de convergence s'\'etende \`a l'infini.\\
$\bullet$ Si $h[n] = \delta [n-k]$ alors
\begin{equation}
\label{transl-z}
\hat h(z) = z^{-k}
\end{equation}
et $A(\hat h) = ]0 , +\infty [$.\\
$\bullet$ Si $h[n] = a^n \gamma [n]$ alors
\begin{equation}
\label{un-pole2}
\hat h(z) = \frac 1 {1 - a z^{-1}}
\end{equation}
$A(\hat h) = ]|a| , +\infty [$.\\
$\bullet$ Si $h[n] = n a^n \gamma [n]$ alors
\[
\hat h(z) = \frac {a z^{-1}} {1 - a z^{-1}}
\]
$A(\hat h) = ]|a| , +\infty [$.
\\
\\
{\bf Convolution} Toutes
les propri\'et\'es de la transform\'ee
de Fourier s'\'etendent directement
\`a la transform\'ee en z.
En particulier, si $g[n] = f \star h [n]$ alors la transform\'ee
en $z$ de $g[n]$ est le produit
\[
\hat g(z) = \hat f(z) \hat h(z)
\]
et son anneau de convergence est
\[
A(\hat g) = A(\hat f ) \bigcap A(\hat h ) .
\]
\\
\\
{\bf Filtres r\'ecursifs}
Nous avons vu en (\ref{RecruTrasfFunc}) que la fonction de
transfert d'un filtre r\'ecursif
est une fonction rationnelle. Sa tranform\'ee en $z$ peut
donc s'\'ecrire
\[
\hat h(z) =
\frac {\sum_{k=0}^M b_k z^{-k}} {\sum_{k=0}^N a_k z^{-k} }.
\]
La r\'eponse impulsionnelle causale $h[n]$ se calcule
facilement en d\'ecomposant $\hat h (z)$ en \'el\'ements simples.
Si $\hat h (z)$ a des p\^oles simples situ\'es
en $d_k$, on peut montrer par identification des coefficients
(exercice) qu'il peut s'\'ecrire sous la forme
\[
\hat h (z) = \sum_{r=0}^{M-N} B_r z^{-r} +
\sum_{k=0}^{N} \frac {A_k} {1 - d_k z^{-1}} .
\]
Le filtre causal correspondant \`a une r\'eponse impulsionelle
qui se calcule avec (\ref{transl-z}) et (\ref{un-pole2})
\[
h[n] =
\sum_{r=0}^{M-N} B_r \delta[n-r] +
\sum_{k=0}^{N} {A_k} (d_k)^n \gamma[n] .
\]
Dans le cas de p\^oles multiples, la d\'ecomposition fractionnelle
s'\'etend avec des puissances aux d\'enominateurs des fractions.
On distingue les filtres \`a r\'eponse impulsionnelle finie
dont la transform\'ee en $z$ est un polyn\^{o}me en $z^{-1}$ (N=0)
et les filtres \`a r\'eponse impulsionnelle infinie pour lesquels
$N > 0$.

On observe que la r\'eponse impulsionelle $h[n]$ est
causale et stable si et seulement si pour tout $k$, $|d_k| < 1$.
Cela signifie donc que tous les p\^oles de $\hat h (z)$ ont
un module plus petit que 1.

\subsection{Approximation de filtres s\'electifs en fr\'equence}

Tout comme pour la synth\`ese de filtres analogiques, on
approxime un filtre passe-bas id\'eal (\ref{passe-bas-discret})
par un filtre r\'ecursif dont la fonction de transfert satisfait
les conditions impos\'es par un
gabarit qui limite les oscillations dans
la bande passante et la bande d'att\'enuation
(voir figure \ref{gabarit}).
La technique de synth\`ese la plus courante
est de transformer un filtre passe-bas analogique rationnel
\[
\hat h_a (\om) = \frac {N(i\om)} {D(i \om)}
\]
en un filtre discret r\'ecursif par un changement de variable
\[
i\om = F(\eom) ,
\]
o\`u F est une fonction rationnelle de $\eom$ qui envoie
$]-\pi,\pi[$ sur
$]-\infty,+\infty[$.
La fonction de transfert
\[
\hat h_d (\eom) = \frac {N(F(\eom))} {D(F(\eom))}
\]
est une fonction rationnelle de $\eom$
et donc la fonction de transfert
d'un filtre discret r\'ecursif.
Le changement de variable $F(\eom)$ qui associe
$]-\pi,\pi[$ \`a l'axe r\'eel $\R$ doit \^etre aussi ``r\'egulier''
que possible pour ne pas trop modifier les propri\'et\'es
de la fonction de transfert $\hat h(\om)$.
On utilise souvent l'application
\[
F(\eom) = \frac 2 T \tan (\frac {\om} 2 ) =
\frac {2} T \frac{1 - e^{-i\om}}{1 + e^{-i\om}} .
\]
Le facteur $T$ est un param\`etre de dilatation qui peut \^etre
ajust\'e arbitrairement.

Par exemple,
on peut construire un filtre passe-bas dont la fr\'equence
de coupure est en $\om_c$ \`a partir d'un filtre analogique de
Butterworth (\ref{butterworth}).
On obtient
\[
|\hat h(\eom)|^2 = \frac 1 {1 + \left(
\frac {\tan (\om/2)} {\tan (\om_c /2)}\right)^{2N}} .
\]
L'ordre $N$ du filtre doit \^etre adapt\'e aux conditions
impos\'ees
par le gabarit du filtre passe-bas.

\subsection{Factorisation spectrale}

Lors de la synth\`ese d'un filtre r\'ecursif,
une fois que l'on a calcul\'e
$|\hat h(\eom)|^2$ pour satisfaire les conditions d'amplification
ou d'att\'enuation, il reste \`a
adapter la phase pour que $\hat h (\eom)$ soit un filtre
causal et stable. Le module est donn\'e par
\[
|\hat h(\eom)|^2 = \hat h(\eom) \hat h^* (\eom) = \hat h(z)
\hat h^* (1 / z^* ) ,
\]
avec $z = \eom$.
Pour un filtre r\'ecursif,
\[
\hat h(z) = \frac {b_0} {a_0} \frac
{\prod_{k=1}^M (1 - c_k z^{-1})} {\prod_{k=1}^N (1 - d_k z^{-1})} .
\]
et
\[
C(z) = \hat h(z) \hat h^* (1 / z^* ) =
\frac {|b_0|^2} {|a_0|^2} \frac
{\prod_{k=1}^M (1 - c_k z^{-1})(1 -  c^*_k  z)
} {\prod_{k=1}^N (1 - d_k z^{-1})(1 - d^*_k z)} .
\]
La donn\'ee de $|\hat h(\eom)|^2$ impose la position des z\'eros et
des p\^oles de $C(z)$. Les z\'eros et les p\^oles de $C(z)$
vont par paires $(c_k, 1/{ c^*_k})$ et
$(d_k, 1/{d^*_k})$. Pour chaque paire, il y a un \'el\'ement
dans le cercle unit\'e et l'autre \`a l'ext\'erieur, \`a moins qu'ils
ne soient confondus sur le cercle unit\'e.
On peut construire $\hat h(z)$ en choisissant
arbitrairement un p\^ole et un z\'ero dans
chaque paire.
Pour que $\hat h(z)$ soit la transform\'ee en z d'un syst\`eme
stable
et causal, nous avons vu que tous les p\^oles doivent \^etre
strictement dans le
cercle unit\'e. Cela laisse libre le choix des z\'eros.
Un choix particulier des z\'eros est de les prendre tous
dans le cercle unit\'e. Un filtre dont les z\'eros et les p\^oles
sont dans le cercle unit\'e est appel\'e filtre \`a phase minimale.
\\
\\
{\bf Filtre inverse}
Le filtre inverse d'un filtre $h$ est
le filtre $h_i$ tel que
pour tout $f[n]$
\[
f \star h \star h_i [n] = f[n] .
\]
Cela signifie que les zones de convergence
de $\hat h (z)$ et de $\hat h_i (z)$ s'intersectent et que
sur ce domaine
\[
\hat h(z) \hat h_i (z) = 1.
\]
Le filtre inverse d'un filtre \`a phase minimale
est stable et causal.
En effet,
les p\^oles de $\hat h_i (z)$ sont les z\'eros
de $\hat h(z)$ et inversement.
Or, pour que $\hat h_i (z)$ soit stable et causal, il faut et
il suffit que ses
p\^oles soient dans le cercle unit\'e et donc que les z\'eros de
$\hat h(z)$ soient dans le cercle unit\'e.

\section{Signaux finis}
\label{finite-sig}
Nous avons suppos\'e jusqu'\`a pr\'esent que nos signaux discrets
$f[n]$ sont d\'efinis pour tout $n \in \Z$. Le plus souvent,
$f[n]$ est connu sur un domaine fini, disons $0 \leq n < N$.
Il faut donc adapter les calculs de convolutions en tenant
compte des effets de bord en $n=0$ et $n= N-1$.
Par ailleurs, pour utiliser la transform\'ee de Fourier comme outil
de calcul num\'erique, il faut pouvoir la red\'efinir sur des
signaux discrets finis.
Ces deux probl\`emes sont r\'esolus en p\'eriodisant les
signaux finis. L'algorithme de transform\'ee de Fourier
rapide est d\'ecrit avec une application au calcul rapide
des convolutions.


\subsection{Convolutions circulaires}

Soient $\tilde f[n]$ et $\tilde h[n]$ des signaux de $N$
\'echantillons.
Pour calculer la convolution
\[
\tilde f \star \tilde h [n] = \sum_{p=-\infty}^{+\infty}
\tilde f[p] \tilde h[n-p]
\]
pour $0 \leq n < N$, il nous faut conna\^{\i}tre
$\tilde f[n]$ et $\tilde h[n]$ au-del\`a de $0 \leq n < N$.
Une approche possible est d'\'etendre
$\tilde f[n]$ et $\tilde h[n]$ avec une p\'eriodisation
sur $N$ \'echantillons
\[
f[n] =  \tilde f[n ~\mbox{modulo}~ N]
~~,~~
h[n] =  \tilde h[n ~\mbox{modulo}~ N] .
\]
La convolution circulaire de ces deux signaux de p\'eriode $N$
est r\'eduite \`a une somme sur leur p\'eriode
\[
f \cistar h [n]
= \sum_{p=0}^{N-1}
f[p] h[n-p] .
\]

Les vecteurs propres d'un op\'erateur de convolution circulaire
\[
Lf[n] = f \cistar h[n]
\]
sont les exponentielles discr\`etes
$e_k[n] = e^{ \frac {i2 \pi k} N n  }$.
En effet,
\[
L e_k [n] = e^{ \frac {i2 \pi k} N n  }
\sum_{p=0}^{N-1}
h[p]  e^{ \frac {-i2 \pi k} N p  },
\]
et les valeurs propres sont donn\'ees par la
transform\'ee de Fourier discr\`ete de $h[n]$
\[
\hat h[k] = \sum_{p=0}^{N-1}
h[p]  e^{ \frac {-i2 \pi k} N p  } .
\]

\subsection{Transform\'ee de Fourier discr\`ete}
\label{transf-four-discr}

L'espace des signaux discrets
de p\'eriode $N$ est de dimension $N$ et l'on note le produit
scalaire
\begin{equation}
\label{inn-prod}
<f,g> = \sum_{n=0} ^{N-1} f[n]  g^*[n] .
\end{equation}
Le th\'eor\`eme suivant d\'emontre que tout signal
de p\'eriode $N$ peut s'\'ecrire comme une transform\'ee de
Fourier discr\`ete.

\begin{theorem}
La famille d'exponentielles discr\`etes
$(e_k[n])_\ZkNU$
\begin{equation}
e_k[n] = e^{ \frac {i2 \pi k} N n  },
\end{equation}
est une base orthogonale de l'espace des signaux
de p\'eriode $N$.
\end{theorem}

Pour prouver ce th\'eor\`eme, il suffit de montrer que
cette famille de $N$ vecteurs est orthogonale (exercice).
Comme l'espace est
de dimension $N$, c'est donc une base de l'espace.
Tout signal $f[n]$ de p\'eriode $N$ peut se d\'ecomposer
dans cette base
\begin{equation}
f[n] =
\sum_{k=0} ^{N-1} \frac {<f,e_k>} {\|e_k \|^2}
e_k [n] .
\end{equation}
La transform\'ee de Fourier discr\`ete de $f[n]$ est
\begin{equation}
\label{fourier-discret}
{\hat f [k]}  = <f,e_k> =
\sum_{n=0} ^{N-1} f[n] e^{\frac {-i2 \pi n} N k  } .
\end{equation}
Comme $\| e_k [n] \|^2 = N$,
\begin{equation}
\label{fourier-discret-inverse}
f[n] =
\frac 1 N \sum_{k=0} ^{N-1} \hat f [k] e^{ \frac {i2 \pi k} N n  } .
\end{equation}
L'orthogonalit\'e implique une formule de Plancherel
\begin{equation}
\label{planch-discret}
\sum_{n=0} ^{N-1} |f[n]|^2 = \frac 1 N
\sum_{k=0} ^{N-1} |\hat f[k]|^2 .
\end{equation}
\\
\\
{\bf Effets de bord}
La transform\'ee de Fourier discr\`ete
d'un signal de p\'eriode $N$ se calcule \`a partir des
valeurs de $f[n]$ pour $0 \leq n < N$. Pourquoi se soucier
du fait que ce soit un signal de p\'eriode $N$ plut\^ot qu'un signal
de $N$ \'echantillons ?
La somme de Fourier (\ref{fourier-discret})
d\'efinit un signal de p\'eriode $N$ pour lequel l'\'echantillon
$f[0]$ \'etant le m\^eme que $f[N]$ se retrouve
plac\'e \`a c\^ot\'e de $f[N-1]$.
Si $f[0]$ et $f[N-1]$ sont tr\`es diff\'erents, cela induit une
transition brutale dans le signal p\'eriodis\'e qui se traduit
par l'apparition de coefficients de Fourier de relativement
grande amplitude aux hautes fr\'equences.
Par exemple, le signal apparemment r\'egulier
$f[n] = n$ pour $0 \leq n < N$
a une transition brutale en $n=pN$ pour $p \in \Z$,
une fois p\'eriodis\'e. Cette transition appara\^{\i}t
dans sa s\'erie de Fourier.
\\
\\
{\bf Filtrage fini}
Comme $e^{ \frac {i2 \pi k} N n  }$ sont les
vecteurs propres des op\'erateurs de convolution
circulaire, on d\'eduit un th\'eor\`eme de convolution.

\begin{theorem} [Convolution Circulaire]
\label{circulaire}
La convolution circulaire
$g[n] = f \cistar h [n]$ est un signal de p\'eriode $N$ dont la
transform\'ee de Fourier discr\`ete est
\begin{equation}
\hat g[k] = \hat f[k] \hat h[k]
\end{equation}
\end{theorem}

La d\'emonstration de ce th\'eor\`eme est identique \`a la
d\'emonstration des deux th\'eor\`emes de convolution pr\'ec\'edents
et laiss\'ee en exercice.
Ce th\'eor\`eme montre que la convolution circulaire est
un filtrage fr\'equentiel. Il ouvre aussi la porte au
calcul rapide de convolutions en utilisant la
transform\'ee de Fourier rapide.

\subsection{Transform\'ee de Fourier rapide}

Pour un signal $f[n]$ de $N$ points, le calcul direct
de la transform\'ee de Fourier discr\`ete
\begin{equation}
{\hat f[k]}  = \sum_{n=0} ^{N-1} f[n] e^{ \frac {-i2 \pi k} N n  } ,
\end{equation}
pour $0 \leq k < N$,
demande $O(N^2)$ multiplications et additions.
Il est cependant possible de
r\'eduire le nombre d'op\'erations \`a $O(N \log_2 N)$
en r\'eorganisant les calculs.
Lorsque $k$ est pair, on regroupe les termes
$n$ et $n+\frac N 2$
\begin{equation}
\label{four-paire}
{\hat f [2k]}  = \sum_{n=0} ^{{\frac N 2}-1} (f[n]+f[n+ N/ 2 ])
e^{ \frac {-i2 \pi k} {{\frac N 2}}n} .
\end{equation}
Lorsque $k$ est impair, le m\^eme regroupement devient
\begin{equation}
\label{four-impaire}
{\hat f [2k+1]}  =
\sum_{n=0} ^{{\frac N 2}-1} e^{ \frac {-i2 \pi} N n}
(f[n]-f[n+{\frac N 2}])
e^{ \frac {-i2 \pi k} {{\frac N 2}} n} .
\end{equation}
L'\'equation (\ref{four-paire})
montre que les fr\'equences paires sont obtenues
en calculant la transform\'ee de Fourier discr\`ete du signal
de p\'eriode ${\frac N 2}$
\[
f_p [n] = f[n] + f[n+{\frac N 2}],
\]
tandis que (\ref{four-impaire})
permet de calculer les fr\'equences impaires par la
transform\'ee de Fourier discr\`ete du signal de p\'eriode ${\frac N 2}$
\[
f_i [n] = e^{ \frac {-i2 \pi} N n}  (f[n] - f[n+{\frac N 2}] ).
\]
\\
\\
{\bf Complexit\'e}
Une transform\'ee de Fourier d'un signal de taille $N$ s'obtient
donc en calculant deux transform\'ees de Fourier de signaux de
taille ${\frac N 2}$.
Le signal $f[n]$ \'etant complexe, le calcul
des signaux $f_p [n]$ et $f_i[n]$
demande
$N$ additions complexes et ${\frac N 2}$ multiplications complexes,
ce qui fait $3N$ additions et $2N$ multiplications r\'eelles.
Soit $C(N)$ le nombre d'op\'erations d'une transform\'ee de Fourier
rapide d'un signal de p\'eriode $N$. On a donc
\begin{equation}
\label{recurrence}
C(N) = 2 C({\frac N 2}) + K N ,
\end{equation}
avec $K = 5$.
La transform\'ee de Fourier d'un signal d'un seul point \'etant
\'egale \`a lui-m\^eme, $C(1) = 0$.
Avec le changement de variable $l = \log_2 N$ et de fonction
$T(l) = C(N)/N$, on d\'eduit de (\ref{recurrence}) que
\[
T(l) = T(l-1) + K .
\]
Comme $T(0) = 0$, $T(l) = Kl$ et donc
\[
C(N) = K N \log_2 (N) .
\]

Il existe diff\'erents algorithmes de transform\'ee de Fourier
rapide qui d\'ecomposent une transform\'ee de Fourier
de taille $N$ en deux transform\'ees de Fourier de taille ${\frac N 2}$
plus $O(N)$ op\'erations. L'algorithme le plus performant \`a ce
jour est un peu plus compliqu\'e que celui que nous venons de
d\'ecrire, mais ne demande que $N \log_2 N$ multiplications
et $3N \log_2 N$ additions.

La transform\'ee de Fourier inverse se calcule \`a partir
de l'algorithme rapide en observant que
\begin{equation}
f^*[n] = \frac 1 N \sum_ {k=0}^ {N-1}  \hat f^*[k]
e^{\frac {-i 2 \pi kn} N} .
\end{equation}
La tranform\'ee de Fourier discr\`ete inverse est donc obtenue en
calculant la transform\'ee de Fourier directe du complexe conjugu\'e
du signal, et en calculant le complexe conjugu\'e du r\'esultat.


\subsection{Convolution rapide}
\label{convol-rap-sec}

L'algorithme rapide de la transform\'ee de Fourier
discr\`ete permet d'utiliser le th\'eor\`eme \ref{circulaire} pour
calculer efficacement les convolutions discr\`etes de
deux signaux \`a support fini.
Soient $f[n]$ et $h[n]$ deux signaux ayant
des \'echantillons non nuls pour $0 \leq n < M$.
Le signal causal
\begin{equation}
\label{somme}
g[n] = f \star h [n] = \sum_{k=- \infty}^{+ \infty} f[k] h [n-k] ,
\end{equation}
n'a de valeurs non-nulles que pour $0 \leq n < 2M$.
Le calcul direct de ce produit
de convolution, en \'evaluant la somme (\ref{somme}), demande
$O(M^2)$ additions et multiplications.
Le th\'eor\`eme de convolution circulaire \ref{circulaire}
sugg\`ere un proc\'ed\'e plus rapide
bas\'e sur des convolutions circulaires.

Pour r\'eduire le calcul de
la convolution non-circulaire (\ref{somme})
\`a une convolution circulaire, on d\'efinit deux signaux
de p\'eriode $2M$
\begin{equation}
a [ n  ] =
   \left \{ \begin{array}{ll}
            f[n]   & \mbox{si $0 \leq n < M$}\\
            0  & \mbox{si $ M \leq n < 2M$}
            \end{array}
   \right.
\end{equation}
\begin{equation}
b [ n  ] =
   \left \{ \begin{array}{ll}
            h[n] &  \mbox{si $ 0 \leq n < M$}\\
            0  & \mbox{si $M \leq n < 2M$}
            \end{array}
   \right.
\end{equation}
Il est maintenant facile de v\'erifier que la convolution circulaire
\[
c[n] = a \cistar b [n]
\]
satisfait
\begin{equation}
c [ n  ] =  g[n]   \mbox{~~~pour } 0 \leq n < 2M .
\end{equation}
On peut donc calculer les valeurs non-nulles de
$g[n]$ en calculant les transform\'ees de Fourier discr\`etes
de $a[n]$ et $b[n]$, en les multipliant et
en calculant la transform\'ee de Fourier inverse du r\'esultat.
En utilisant l'algorithme de transform\'ee de Fourier rapide,
ce calcul demande un total de $O(M \log_2 M)$ additions
et multiplications, au lieu de $O(M^2)$.

Cet algorithme rapide de calcul de convolutions est utilis\'e
pour la convolution de signaux par des filtres de r\'eponse
impulsionnelle finie. Si le signal $f[n]$ a $N$ points
non-nuls et le filtre $h[n]$ a $L$ \'echantillons non-nuls,
l'algorithme de convolution peut \^etre
modifi\'e pour calculer la convolution $h \cistar f[n]$
en $O(N log_2 L )$ op\'erations (exercice).
%\end{document}









\chapter{Introduction au signal al\'eatoire \`a temps-discret}

Dans ce chapitre, nous introduisons des concepts de base concernant
l'analyse des s�ries temporelles. En particulier, nous d�finissons
les notions de stationnarit� et de fonction d'autocovariance.

\section{Introduction}

% Nous commen\c{c}ons par d�finir ce qu'est une s�rie temporelle.

\index{S�rie temporelle}\index{S�rie chronologique|see{S�rie temporelle}}
Une s�rie temporelle (ou s�rie chronologique) est un ensemble
d'observations $x_t$, chacune �tant enregistr�e � un instant $t$.
On rencontre des s�ries temporelles dans des domaines tr�s
vari�s tels que la m�decine, les t�l�communications ou l'�conom�trie.

% Notre but, dans cet ouvrage, est de proposer des m�thodes permettant
% de mod�liser de telles donn�es en utilisant un mod�le
% math�matique appropri� afin de pouvoir, par la suite, faire de la
% pr�diction. Pour cela, nous proposerons des m�thodes permettant
% d'estimer les param�tres du mod�le choisi ainsi que des m�thodes
% permettant de tester la validit� du mod�le propos�.


%\paragraph{Exemples de s�ries temporelles}
% Le paragraphe~\ref{sec:gene} d�finit le formalisme probabiliste
% permettant de d�crire les {\em processus al�atoires}. Les quelques
% exemples qui suivent illustrent la diversit� des situations dans
% lesquelles la mod�lisation stochastique (ou al�atoire) des s�ries
% temporelles joue un r�le important.

Dans la suite, nous proposons de consid�rer les observations comme
des r�alisations d'un processus al�atoire $(X_t)_{t\in T}$ dont nous
donnons la d�finition dans le paragraphe \ref{sec:gene}. Les quelques
exemples qui suivent illustrent la diversit� des situations dans
lesquelles la mod�lisation stochastique (ou al�atoire) des s�ries
temporelles joue un r�le important.
%\begin{example}[Trafic internet]
%La figure \ref{fig:figtraf} repr�sente les temps d'inter-arriv�es
%de paquets TCP, mesur�s en secondes, sur la passerelle du
%laboratoire Lawrence Livermore. La trace repr�sent�e a �t� obtenue
%en enregistrant 2 heures de trafic. Pendant cette dur�e, environ
%1.3 millions de paquets TCP, UDP, etc. ont �t� enregistr�s, en
%utilisant la proc�dure \emph{tcpdump} sur une station Sun.
%D'autres s�ries de ce type peuvent �tre obtenues sur
%\emph{The Internet Traffic Archive},
%\emph{http://ita.ee.lbl.gov/}.
%%======== FIGURE
% \figscale{\FIGPASSL lbl_tcp_3}
% {Trace de trafic Internet~: temps d'inter-arriv�es de paquets TCP.}
% {fig:figtraf}{0.5}
%\end{example}
\begin{example}[Parole]
La figure \ref{fig:figspeech} repr�sente un segment de signal
vocal �chantillonn� (la fr�quence d'�chantillonnage est de 8000
Hz). Ce segment de signal correspond � la r�alisation du
phon�me \emph{ch} (comme dans \emph{ch}at) qui est un son
dit \emph{fricatif}, c'est-�-dire produit par les turbulences du
flot d'air au voisinage d'une constriction (ou resserrement) du
conduit vocal.
%======== FIGURE
 \figscale{\FIGPASSL phrase}
 {Signal de parole �chantillonn� � $8000$ Hz~:
 son non vois� \emph{ch}.}{fig:figspeech}{1}
\end{example}
\begin{example}[Indice financier]
La figure~\ref{fig:SP} repr�sente les cours d'ouverture
journaliers de l'indice Standard and Poor 500, du $2$ Janvier
$1990$ au 25 Ao�t 2000. l'indice S\&P$500$ est calcul� �
partir de $500$ actions choisies parmi les valeurs cot�es au New
York Stock Exchange (NYSE) et au NASDAQ en fonction de leur
capitalisation, leur liquidit�, leur repr�sentativit� dans
diff�rents secteurs d'activit�. Cet indice est obtenu en pond�rant
le prix des actions par le nombre total d'actions, le poids de
chaque valeur dans l'indice composite �tant proportionnel � la
capitalisation.
%======== FIGURE
 \figscale{\FIGPASSL SP}
 {Cours quotidien d'ouverture de l'indice S\&P$500$~:
 entre Janvier 1990 et Ao�t 2000.}{fig:SP}{1}
\end{example}


%=================================================================
%=================================================================
%=================================================================
\section{D�finition et construction de la loi d'un processus al�atoire}
\label{sec:gene}
%=================================================================
\subsection{Processus al�atoire}

\begin{definition}[Processus al�atoire]
\index{Processus!Al�atoire}
Soient $(\Omega,\cF,\PP)$ un espace de probabilit�, $T$ un
ensemble d'indices et $(E,\cE)$ un espace mesurable. On appelle
processus al�atoire une famille $(X_t)_{t \in T}$ de v.a. �
valeurs dans $(E,\cE)$ index�es par $t \in T$.
\end{definition}
Le param�tre $t$ repr�sente par exemple le temps. Lorsque $T=
\Zset$ ou $\Nset$, nous dirons que le processus est � \emph{temps discret}
et, lorsque $T=\Rset$ ou $\Rset_+$, que le processus est �
\emph{temps continu}. Dans la suite, nous nous
int�resserons sauf exception aux processus � temps
discret avec $T= \Zset$. Quant � $(E,\cE)$, nous consid�rerons
le plus souvent $(\Rset, \cB(\Rset))$ (o� $\cB(\Rset)$ est la
tribu bor�lienne de $\Rset$) ou $(\Rset^d, \cB(\Rset^d))$.
Dans le premier cas, on dira que le processus al�atoire est
\emph{scalaire}. Dans le second, nous dirons que le processus est
\emph{vectoriel}.

Notons qu'un processus peut �tre vu comme une application $X: \Omega
\times T \rightarrow E$, $(\omega,t)\mapsto X_t(\omega)$ telle que, �
chaque instant $t \in T$, l'application $\omega \mapsto X_t(\omega)$
est une variable al�atoire de $(E,\cE)$.
\begin{definition}[Trajectoire\index{Trajectoire}]
  Pour chaque $\omega\in \Omega$, l'application $t \mapsto
  X_t(\omega)$ est une fonction de $T \rightarrow E$ qui s'appelle la
  \emph{trajectoire} associ�e � l'�preuve $\omega$.
\end{definition}

%==================================================
%==================================================
\subsection{R�partitions finies}
%==================================================

Etant donn�s 2 espaces mesurables $(E_1,\cE_1)$ et $(E_2,\cE_2)$, on d�finit l'espace mesurable produit
$(E_1\times E_2,\cE_1\otimes\cE_2)$ o� $\times$ d�signe le produit cart�sien usuel des ensembles
et $\otimes$ l'op�ration correspondante sur les tribus: $\cE_1\otimes\cE_2$ d�signe la tribu engendr�e par $\{A_1\times A_2,
A_1\in\cE_1: A_2\in\cE_2\}$, ce que l'on �crira
$$
\cE_1\otimes\cE_2=\sigma\{A_1\times A_2: A_1\in\cE_1, A_2\in\cE_2\} \; .
$$
Comme la classe d'ensembles $\{A_1\times A_2: A_1\in\cE_1,
A_2\in\cE_2\}$ est stable par intersection finie, une probabilit� sur
$\cE_1\otimes\cE_2$  est \emph{caract�ris�e} par sa restriction �
cette classe (voir \cite[Corollaire 6.1]{jacod:protter:2003}).

On d�finit de m�me un espace mesurable produit $(E_1\times\dots\times E_n,\cE_1\otimes\dots\otimes\cE_n)$
� partir d'un nombre fini $n$ d'espaces mesurables $(E_t,\cE_t)$,
$t\in T$. Si $T$ n'est pas de cardinal fini, cette d�finition se g�n�ralise en consid�rant
 la tribu engendr�e par les \emph{cylindres} sur le produit cart�sien $\prod_{t\in T} E_t$ qui contient l'ensemble des
 familles $(x_t)_{t\in T}$ telles que $x_t\in E_t$ pour tout $t\in T$. Examinons le cas qui nous servira par la suite o�
$(E_t,\cE_t)=(E,\cE)$ pour tout $t\in T$. On note alors $E^T=\prod_{t\in T} E$ l'ensemble des trajectoires $(x_t)_{t\in T}$
telles que $x_t\in E$ pour tout $t$, que l'on munit de la tribu engendr�e par les cylindres
$$
\cE^{\otimes T}=\sigma\left\{\prod_{t\in I}A_{t}\times E^{T\setminus I} : I\in\mathcal{I},\forall t\in I,\,A_t\in\cE\right\}\;,
$$
o� l'on note $\mathcal{I}$ l'ensemble des parties finies de $T$.



%Un �l�ment $I$ de $\mathcal{I}$ s'�crit $I = \{ t_1 < t_2 <\cdots < t_n \}$.
Soit $X=(X_t)_{t \in T}$ un processus d�fini sur  $(\Omega,\cF,\PP)$ �
valeurs dans $(E,\cE)$  et $I\in\mathcal{I}$.
On note $\PP_I$ la loi du vecteur al�atoire $\{ X_{t}, {t\in I} \}$, c'est-�-dire la mesure image  de $\PP$ par ce vecteur~: $\PP_I$ est
la probabilit� sur $(E^{I},\cE^{\otimes I})$ d�finie par
\begin{equation}
\label{eq:rel1}
\PP_I\left(\prod_{t\in I}A_t\right)
 = \PP\left(X_{t} \in A_t,\,t\in I\right) \eqsp,
\end{equation}
o� $A_t$, $t\in T$ sont des �l�ments quelconques de la tribu $\cE$. La probabilit� $\PP_I$ est
une \emph{probabilit� fini-dimensionnelle} ou \emph{r�partition finie} du processus $X$.
\begin{definition}
\index{Famille des r�partitions finies}
On appelle \emph{famille des r�partitions finies} l'ensemble des
r�partitions finies $(\PP_I, I \in \mathcal{I})$.
\end{definition}
La sp�cification de la mesure $\PP_I$ permet de calculer la
probabilit� d'�v�nements de la forme $\PP( \cap_{t \in I} \{
X_t \in A_t \})$ o� $\{A_t, t \in I\}$ est une famille d'�l�ments de la
tribu $\cE$, ou de mani�re �quivalente, de calculer l'esp�rance
$\PE {\prod_{t \in I} f_t(X_t) }$ o� pour tout $t\in I$, $f_t$ est une
fonction bor�lienne positive. Soit
$J \subset I$ deux parties finies ordonn�es. Soit $\Pi_{I,J}$ la
projection canonique de $E^{I}$ sur $E^{J}$ d�finie par
\begin{equation}
\label{eq:rel2} \Pi_{I,J}[ x ] = (x_t)_{t \in J}\quad \text{pour tout}\quad x=(x_t)_{t \in I} \in E^I\;.
\end{equation}
La projection canonique pr�serve uniquement les coordonn�es du
vecteur appartenant au sous ensemble d'indices $J$.
Par la d�finition~(\ref{eq:rel1}), on observe que $\PP_J$ est la mesure image de $\Pi_{I,J}$ d�finie sur
l'espace de probabilit� $(E^{I},\cE^{\otimes I},\PP_I)$:
\begin{equation}
\label{eq:rel3} \PP_I \circ \Pi_{I,J}^{-1} = \PP_J \;.
\end{equation}
Cette relation
formalise le r�sultat intuitif que la distribution
fini-dimensionnelle d'un sous-ensemble $J \subset I$ se d�duit de
la distribution fini-dimensionnelle $P_I$ en ``int�grant'' par rapport aux
variables $X_{t}$ sur l'ensemble des $t$ appartenant au
compl�mentaire de $J$ dans $I$. Cette propri�t� montre que la
famille des r�partitions finies d'un processus est fortement
structur�e. En particulier, les r�partitions finies doivent, au
moins, v�rifier les conditions de compatibilit�~(\ref{eq:rel3}).
Nous allons voir dans la suite que cette condition est en fait
aussi {\em suffisante}.


Soit $\Pi_I$ la projection canonique de $E^T$ sur $E^I$,
\begin{equation}
\label{eq:projectioncanonique}
\Pi_I( x ) = (x_t)_{t \in I}\quad \text{pour tout}\quad x=(x_t)_{t \in T} \in E^T\;.
\end{equation}
Si $I=\{s\}$ avec $s\in T$, on notera simplement
\begin{equation}
\label{eq:projectioncanoniquesingle}
\Pi_s( x ) =\Pi_{\{s\}}( x )= x_s\quad \text{pour tout}\quad x=(x_t)_{t \in T} \in E^T\;.
\end{equation}

Le th�or�me suivant montre comment on peut passer d'une famille de
r�partitions finies � une unique mesure de probabilit� sur $(E^T,\cE^{\otimes
  T})$, pourvu que la condition de compatibilit�~(\ref{eq:rel3}) soit
satisfaite.

\begin{theorem}[th�or�me de Kolmogorov]
\label{th:kolmogorov}
%On pose $(E,\cE)=(\Rset^d,\cB(\Rset^d))$ pour $d\geq1$.
Soit $(\nu_I)_{I \in \mathcal{I}}$ une famille de probabilit�s
index�es par l'ensemble des parties finies ordonn�es de $T$ telle, que pour tout $I \in \mathcal{I}$,
$\nu_I$ est une probabilit� sur $(E^I, \cE^{\otimes I})$. Supposons de plus
que la famille $\{ \nu_I, I \in \mathcal{I} \}$ v�rifie les conditions
de compatibilit� \eqref{eq:rel3}: pour tout $I,J \in \mathcal{I}$, tel que
$I \subset J$, $\nu_I \circ \Pi_{I,J}^{-1} = \nu_J$. Alors, il existe
une unique probabilit� $\PP$ sur l'espace mesurable $(E^T,\cE^{\otimes T})$
telle que, pour tout $I \in \mathcal{I}$, $\nu_I = \PP \circ \Pi_I^{-1}$.
\end{theorem}
\begin{proof}\smartqed
Remarquons que la classe des cylindres est une semi-alg�bre au sens de
\cite[p.~297]{royden:1988}. On d�finit $\PP$ sur cette classe par
$$
\PP\left(\prod_{t\in I}A_{t}\times E^{T\setminus I}\right)=\nu_I \left(\prod_{t\in I}A_{t}\right)\;,
$$
o� $I$ d�crit $\cI$ et $A_t\in\cE$ pour tout $t \in I$. La condition
de compatibilit� implique que $\PP$ v�rifie les hypoth�ses de
\cite[Proposition~9]{royden:1988}. Il s'en suit une extension unique �
l'alg�bre engendr�e par les cylindres, c'est-�-dire � la plus petite
classe d'ensembles de $E^T$ stable par intersection finie et par
passage au compl�mentaire contenant les cylindres de $E^T$. Par le
th�or�me de Carath�odory, voir~\cite[Th�or�me~8]{royden:1988}, on
obtient une unique extension de $\PP$ � la tribu $\cE^{\otimes T})$.
\qed
\end{proof}

Ceci nous permet de d�crire les r�partitions finies d'un processus
donn� � partir d'une seule probabilit� sur $(E^T,\cE^{\otimes T})$, la
\emph{loi} (ou \emph{mesure image}) du processus, d�finie comme suit.

\index{Loi}\index{Mesure image|see{Loi}}
\begin{definition}[Loi d'un processus]
\label{def:loi_proc}
Soit $X=(X_t)_{t \in T}$ un processus d�fini sur  $(\Omega,\cF,\PP)$ � valeurs dans $(E,\cE)$. La \emph{mesure image}
$\PP_X$ est l'unique probabilit� d�finie sur $(E^T,\cE^{\otimes T})$ par $\PP_X\circ\Pi_I^{-1}=\PP_I$ pour tout $I\in\mathcal{I}$, \ie
$$
\PP_X\left(\prod_{t\in I}A_{t} \times E^{T\setminus I}\right) = \PP\left(X_{t}\in A_t,\,t\in I\right)\,
$$
pour tout $(A_t)_{t\in I}\in\cE^I$.
\end{definition}
L'existence et l'unicit� de  $\PP_X$ est une cons�quence du  th�or�me~\ref{th:kolmogorov}.
Cette loi est donc {\em enti�rement} d�termin�e par la donn�e des
r�partitions finies.

La d�finition suivante permet de voir $\PP_X$ comme la probabilit�
d'une variable al�atoire � valeurs dans
$(E^T,\cE^{\otimes T})$. Cette variable al�atoire est obtenue comme la
trajectoire du \emph{processus canonique} d�fini comme suit.

\index{Processus!canonique}
\begin{definition}[Processus canonique]
\label{def:proc_canon}
Soit  $(E,\cE)$ un espace mesurable et $(E^T,\cE^T)$ l'espace mesurable des trajectoires correspondants.
La famille canonique sur $(E^T,\cE^T)$ est la famille des fonctions
mesurables  $(\xi_t)_{t\in T}$ d�finies sur
$(E^T,\cE^T)$  � valeurs dans  $(E,\cE)$ par $\xi_t(\omega)=\omega_t$ pour tout $\omega=(\omega_t)_{t\in t}\in E^T$.

Quand on munit $(E^T,\cE^T)$ de la \emph{mesure image} $\PP_X$,
on appelle la famille canonique  $(\xi_t)_{t\in T}$ d�finies sur $(E^T,\cE^T,\PP_X)$ le \emph{processus canonique} associ�
� $X$.
\end{definition}

On a suppos� jusqu'� pr�sent le processus $X=(X_t)_{t\in T}$ donn�.
Le th�or�me~\ref{th:kolmogorov} peut aussi �tre utilis� pour le
construire, sous la forme d'un processus canonique, comme le montre
l'exemple suivant, puis le paragraphe~\ref{sec:proc-gauss-reels} qui
introduit une classe particuli�re de processus~: la classe des
processus gaussiens.

\begin{example}[Suite de v.a. ind�pendantes]
\label{exple:vaindep}
Soit $(\nu_t)_{t \in T}$ une suite
de probabilit�s sur $(E,\cE)$. Pour $I\in\mathcal {I}$, on pose
\begin{equation}
\nu_I = \bigotimes_{t\in I} \nu_{t} \;,
\end{equation}
o� $\otimes$ d�signe le produit tensoriel sur les probabilit�s (loi du
vecteur � composantes ind�pendantes et de lois
marginales donn�es par les $\nu_t$, $t\in I$).
Il est clair que l'on d�finit ainsi une famille $(\nu_I)_{I \in
\cI}$ compatible, c'est-�-dire, v�rifiant la condition donn�e par
l'�quation~(\ref{eq:rel3}). Donc, si $\Omega = E^{T}$,
$X_t(\omega)= \omega_t$ et $\cF = \sigma(X_t, t \in T)$, il
existe une unique probabilit� $\PP$ sur $(\Omega,\cF)$
telle que $(X_t)_{t \in T}$ soit une suite de v.a.
ind�pendantes telles que $X_t\sim\nu_t$ pour tout $t\in T$.
\end{example}

\subsection{Processus gaussiens r�els}\label{sec:proc-gauss-reels}
%======================================================
Nous introduisons � pr�sent une classe importante de processus al�atoires en
mod�lisation stochastique~: la classe des processus gaussiens.
Rappelons tout d'abord la d�finition des variables al�atoires
gaussiennes, univari�es puis multivari�es. Une decription plus
d�taill�e peut �tre trouv�e dans~\cite[Chapter~16]{jacod:protter:2003}.


\begin{definition}[Variable al�atoire gaussienne r�elle]
\index{Variables al�atoires!gaussiennes}
 On dit que $X$ est une variable al�atoire r�elle gaussienne si
 sa loi de probabilit� a pour fonction caract�ristique~:
 $$
   \phi_X(u)=\PE{\rme^{\rmi uX}}=\exp(\rmi \mu u -\sigma^2 u^2/2)
 $$
 o� $\mu\in \Rset$ et $\sigma\in\Rset^+$.
\end{definition}
 On en d�duit que $\PE{X}=\mu$ et que $\Var{X}=\sigma^2$. Si
$\sigma\neq 0$, la loi poss�de une densit� de probabilit� qui
a pour expression~:
\begin{equation}
  \label{eq:densite-gaussienne-unidim}
 p_X(x)=\frac{1}{\sigma\sqrt{2\pi}}
  \exp\left (-\frac{(x-\mu)^2}{2\sigma^2} \right)\;.
\end{equation}

Si $\sigma=0$, on a alors $X=\mu$ p.s.
La d�finition suivante �tend cette
d�finition aux vecteurs al�atoires de dimension $n$.

\begin{definition}[Vecteur gaussien r�el]
  Un vecteur al�atoire r�el de dimension $n$
  $[X_1,\dots,X_n]^T$
% \footnote{Dans cet ouvrage, les vecteurs sont par
%     convention identifi�s sous forme matricielle � des vecteurs
%     colonnes et l'exposant $^T$ indique l'op�rateur de transposition
%     des matrices.}
  est un vecteur gaussien si toute combinaison
  lin�aire de $X_1,\dots,X_n$ est une variable al�atoire gaussienne
  r�elle.
\end{definition}
Notons $\mu$ le vecteur moyenne de $[X_1,\dots,X_n]^T$ et $\Gamma$ sa
matrice de covariance. Par d�finition d'un vecteur al�atoire gaussien,
pour tout $u\in\Rset^n$, la variable al�atoire $Y = \sum_{k=1}^n u_k
X_k=u^TX$ est une variable al�atoire r�elle gaussienne. Par
cons�quent, sa loi est compl�tement d�termin�e par sa moyenne et sa
variance qui ont pour expressions respectives~:
\[
 \PE { Y }= \sum_{k=1}^n u_k \PE {X_k}=u^T\mu
 \quad \mbox {et} \quad
 \Var{Y}= \sum_{j,k=1}^n u_j u_k \cov(X_j,X_k)=u^T \Gamma u
\]
On en d�duit l'expression, en fonction de $\mu$ et de $\Gamma$, de
la fonction caract�ristique de la loi de probabilit� d'un vecteur
gaussien $[X(1),\dots,X(n)]^T$~:
 \begin{equation}
 \label{eq:fcarac_vectgaussien}
 \phi_X(u)=\PE{ \exp( \rmi u^T X) }=\PE{ \exp( \rmi Y) }
 =
 \exp \left ( \rmi u^T \mu - \frac{1}{2}u^T \Gamma u
 \right)
\end{equation}
R�ciproquement, si un vecteur al�atoire $X$ de taille $n$ a une fonction caract�ristique de
cette forme, on obtient imm�diatement que $X$ est un vecteur gaussien en
calculant la fonction caract�ristique de ses produits scalaires.
Cette propri�t� permet d'obtenir la proposition suivante.
 \begin{proposition}\label{prop:vect_gaussiens}
   La loi d'un vecteur gaussien $X$ de taille $n$ est enti�rement caract�ris� par son vecteur
   moyenne $\mu$ et sa matrice d'autocovariance $\Gamma$. On notera
$$
X\sim\mathcal{N}_n( \mu, \Gamma) \;.
$$
R�ciproquement pour tout vecteur
   $\mu\in\Rset^n$ et toute matrice sym�trique positive $\Gamma$, il existe un
   vecteur al�atoire $X$ tel que $X\sim\mathcal{N}_n( \mu, \Gamma)$.
 \end{proposition}
 \begin{proof}\smartqed
   La premi�re partie de l'�nonc� d�coule directement de
   (\ref{eq:fcarac_vectgaussien}).  D�montrons maintenant la r�ciproque. Tout
   d'abord le r�sultat est vrai pour $n=1$ comme nous l'avons rappel� plus haut. On
   passe ais�ment au cas o� $\Gamma$ est diagonale. En effet, notons
   $\sigma_i^2$, $i=1,\dots,n$ ses �l�ments diagonaux et
   $\mu=[\mu_1,\dots,\mu_n]^T$. Alors il suffit de prendre $X_1$, \dots ,$X_n$
   ind�pendants tels que $X_i\sim\mathcal{N}_n( \mu_i, \sigma_i^2)$ pour
   $i=1,\dots,n$. On v�rifie ais�ment que $X\sim\mathcal{N}_n( \mu, \Gamma)$ en
   calculant sa fonction caract�ristique. Pour passer du cas des matrices
   digaonales � une matrice $\Gamma$ sym�trique
   positive quelconque, on utilise le lemme suivant dont la preuve est laiss�e
   � titre d'exercice.
   \begin{lemma}
     Soit $X\sim\mathcal{N}_n( \mu, \Gamma)$ avec $\mu\in\Rset^n$ et
      $\Gamma$ matrice sym�trique positive $n\times n$. Alors pour toute
      matrice $A$ de taille $p\times n$, on a  $AX\sim\mathcal{N}_n( A\mu,
      A\Gamma A^T)$.
   \end{lemma}
   Pour conclure la preuve de la proposition~\ref{prop:vect_gaussiens}, il
   suffit de remarque que toute matrice sym�trique
   positive $\Gamma$ est diagonalisable en base orthonorm�e et s'�crit donc
   $\Gamma=U\Sigma U^T$ avec $\Sigma$ matrice diagonale positive et $U$ matrice
   orthogonale. Il suffit alors de prendre $Y\sim\mathcal{N}_n( U^T\mu, \Sigma)$
   et de poser $X=UY$ et le lemme donne $X\sim\mathcal{N}_n( \mu, \Gamma)$
   comme recherch�.
 \qed
\end{proof}

On montre facilement la proposition suivante
(voir~\cite[Corollaire~16.1]{jacod:protter:2003}).

\begin{proposition}\label{prop:vect_gaussiens_indep}
   Soit $X\sim\mathcal{N}_n( \mu, \Gamma)$ avec $\mu\in\Rset^n$ et $\Gamma$
   matrice sym�trique positive $n\times n$. Alors $X$ a des composantes
   ind�pendantes si et seulement si $\Gamma$ est une matrice diagonnale.
 \end{proposition}


En utilisant le m�me proc�d� de preuve que pour la
proposition~\ref{prop:vect_gaussiens}, i.e. en consid�rant le cas $\Gamma$
diagonale puis la diagonalisation de $\Gamma$ pour passer au cas g�n�ral, on
obtient aussi le r�sultat suivant (voir~\cite[Corollaire~16.2]{jacod:protter:2003}).

 \begin{proposition}\label{prop:vect_gaussiens_densite}
   Soit $X\sim\mathcal{N}_n( \mu, \Gamma)$ avec $\mu\in\Rset^n$ et $\Gamma$
   matrice sym�trique positive $n\times n$.  Si $\Gamma$ est de rang plein,
   alors la loi de probabilit� de $X$ poss�de une densit� dans $\Rset^n$ dont
   l'expression est~:
$$
 p_X(x)=\frac{1}{(2\pi)^{n/2}\sqrt{\det(\Gamma)}}
 \exp\left ( -\frac{1}{2}(x-\mu)^T \Gamma^{-1}(x-\mu) \right ),\quad x\in\Rset^n\;.
 $$
\end{proposition}
Dans le cas o� $\Gamma$ est de rang $r<n$, c'est � dire o� $\Gamma$ poss�de
 $n-r$ valeurs propres nulles, $X$ se trouve, avec probabilit� $1$, dans un
 sous espace affine de dimension $r$ de $\Rset^n$. En effet, il existe alors
 $r-n$ vecteurs $a_i$ formant une famille libre tels que $\cov(a_i^T X) =
 0$ et donc $a_i^T X=a_i^T \mu$ p.s. $X$ n'admet donc �videmment pas de
 densit� dans ce cas.

Nous �tendons maintenant la notion de vecteur gaussien � celle de
\emph{processus gaussien}.
\begin{definition}[Processus gaussien r�el]
\index{Processus!gaussien}
 On dit qu'un processus r�el $X= (X_t)_{t \in T}$ est gaussien si,
pour tout ensemble fini d'indices $I=\{t_1, t_2, \cdots,t_n\}$,
$[X_{t_1}, X_{t_2}, \cdots, X_{t_n}]^T$ est un vecteur gaussien.
\end{definition}
Ainsi un vecteur gaussien $[X_1,\dots,X_n]^T$ peut �tre lui-m�me vu comme un
processus gaussien $\{X_t, \,t\in \{1,\dots,n\}\}$. Cette d�finition
n'a donc un int�r�t que dans le cas o� $T$ est de cardinal infini.
D'apr�s~(\ref{eq:fcarac_vectgaussien}), la famille des r�partitions finies est
 caract�ris�e par la donn�e de la fonction moyenne $\mu:t\in T \mapsto
\mu(t)\in \Rset$ et de la fonction de covariance $\gamma:(t,s)\in(T\times T)
\mapsto \gamma(t,s)\in \Rset$. De plus, pour tout ensemble fini
d'indices $I=\{t_1, t_2, \cdots,t_n\}$, la matrice $\Gamma_I$ d'�l�ments
$\Gamma_I(m,k) = \gamma(t_m, t_k)$, o� $1 \leq m,k \leq n$, est une matrice
de covariance d'un vecteur al�atoire de dimension $n$. Elle est donc
sym�trique positive.
R�ciproquement, donnons nous une fonction $\mu:t\in T \mapsto m(t)\in \Rset$ et
une fonction $\gamma:(t,s)\in(T\times T) \mapsto \gamma(t,s)\in \Rset$ telle
que, pour tout ensemble fini d'indices $I$, la matrice $\Gamma_I$ est
sym�trique positive.  On peut alors d�finir, pour tout ensemble fini d'indices
$I=\{t_1, t_2, \cdots,t_n\}$, une probabilit� gaussienne $\nu_I$ sur $\Rset^n$
par~:
\begin{equation}
\label{eq:rel11} \nu_I \eqdef \mathcal{N}_n( \mu_I, \Gamma_I)
\end{equation}
o� $\mu_I= [\mu(t_1), \dots, \mu(t_n)]^T$. La famille $(\nu_I,I \in \cI)$,
ainsi d�finie, v�rifie les conditions de compatibilit� et l'on a ainsi �tabli,
d'apr�s le th�or�me \ref{th:kolmogorov}, le r�sultat suivant~:
\begin{theorem}
  Soit $T$ un ensemble d'indices quelconque, $\mu$ une fonction r�elle d�finie
  sur $T$ et $\gamma$ une fonction r�elle d�finie sur $T\times T$ dont toutes
  les restrictions $\Gamma_I$ aux ensembles $I\times I$ avec $I\subseteq T$
  fini forment des matrices sym�triques positives. Il existe un espace de
  probabilit� $(\Omega,\cF,\PP)$ et un processus al�atoire $\{X_t, t
  \in T\}$ gaussien d�fini sur cet espace v�rifiant
  \[
  \mu(t)= \PE{ X_t } \quad \mbox{et}\quad \gamma(s,t)= \PE{ (X_s - \mu(s)) (X_t - \mu(t))}\;.
  \]
\end{theorem}
%Attention le r�sultat ci-dessus est plus subtil qu'il n'y
%para�t~: si {\em la loi} du processus est bien d�finie de mani�re unique,
%il existe n�anmoins plusieurs mani�res de construire des processus ayant cette
%loi. Pour un ensemble $T$ d'indices temporels discret, toutes les
%constructions sont �quivalentes � la construction canonique du
%th�or�me~\ref{th:kolmogorov}. Pour un ensemble $T$
%non-d�nombrable (cas des processus � temps continu), l'exemple
%ci-dessous montre que l'on cherchera � privil�gier les
%constructions qui garantissent des propri�t�s trajectorielles
%suppl�mentaires comme la continuit� des trajectoires.
%\begin{example}[Mouvement brownien]
% Pour mod�liser le mouvement d'un grain de pollen dans un liquide, le
% botaniste �cossais Brown (circa 1820) a un introduit un processus al�atoire
% $X(t)$ � valeurs dans $\Rset^2$ ayant des trajectoires ``irr�guli�res''
% caract�ris�es de la fa�on suivante~:
%\begin{enumerate}
% \renewcommand\theenumi{(\roman{enumi})}
% \item les accroissements $X(t_2)- X(t_1)$, $\cdots$, $X(t_n)
%-X(t_{n-1})$ sont ind�pendants (le processus n'a pas de
%``m�moire''),
% \item pour tout $h \in \Rset$, et tout $0 \leq
%s < t$, les v.a. $X(t+h) - X(s+h)$ et $X(t) - X(s)$ ont les
%m�mes lois, et la loi de l'incr�ment $X(t) - X(s)$ est de
%variance finie $\PE{(X(t)-X(s))^2 } < \infty$,
% \item les trajectoires sont continues.
%\end{enumerate}
%Un tel processus est appel� un mouvement brownien. Au d�but du
%XXi�me si�cle, Louis Bachelier (1900) a observ� qu'un tel
%processus � valeurs dans $\Rset$ permettait de mod�liser le cours
%d'actifs financiers, apr�s une transformation �l�mentaire.
%Albert Einstein (1905), Norbert Wiener (1923) et Paul Levy (1925)
%ont �t� les premiers � d�velopper une th�orie math�matique du
%mouvement brownien. Les utilisations d'un tel processus sont
%multiples, et touchent aujourd'hui l'ensemble des domaines des
%sciences de l'ing�nieur, de l'�conom�trie et de la finance. Nous
%nous int�resserons au mouvement Brownien sur $T= \Rset$. Observons
%tout d'abord que, pour $0 \leq t < s$, un tel processus v�rifie~:
%\begin{equation}
% \label{eq:rel12}
% X(t) - X(s)
% = \sum_{k=1}^{2^n} \left \{
% X(s + 2^{-n}k (t-s)) - X(s + 2^{-n} (k-1) (t-s))
% \right \}
%\end{equation}
%et donc l'accroissement $X(t) - X(s)$ est la somme d'un grand
%nombre de variables al�atoires ind�pendantes de m�me loi
%(stationnarit� des incr�ments) et de variance tendant vers $0$
%lorsque $n \rightarrow \infty$. Une application directe du
%th�or�me de la limite centrale montre que $X(t) - X(s)$ suit
%une loi gaussienne.
%\end{example}
%\begin{definition}[Mouvement Brownien]
%\label{def:brownien} Un processus r�el $(X(t), t \in \Rset^+)$ est
%un mouvement Brownien issu de $0$ si~:
%\begin{enumerate}
%\renewcommand\theenumi{(\roman{enumi})}
% \item $X(0) = 0$,
% \item Pour tout $t_1 < \cdots < t_n$, les variables al�atoires $X(t_2)- X(t_1)$,
%$\cdots$, $X(t_n) - X(t_{n-1})$ sont ind�pendantes,
% \item pour $t \geq s \geq 0$, l'incr�ment $(X(t)-X(s))$ est distribu�
%suivant une loi gaussienne de moyenne nulle et de variance
%$(t-s)$,
% \item les trajectoires $t \mapsto X(t,\omega)$ sont presque
%s�rement continues.
%\end{enumerate}
%\end{definition}
%Supposons qu'un tel objet existe. Alors pour tout $ t_1 < \cdots <
%t_n$, on a~:
%\begin{align*}
% X(t_1) &= X(t_1)
% \\
% X(t_2) &= X(t_1) + (X(t_2) - X(t_1)), \cdots
% \\
% X(t_n) &= X(t_1) + (X(t_2) - X(t_1)) + \cdots + (X(t_n) - X(t_{n-1}))
%\end{align*}
%et donc le vecteur $(X(t_1), \cdots, X(t_n))$ est un vecteur
%gaussien, ce qui montre que $X$ est un processus gaussien. Il s'en
%suit que $m(t)=\PE{X(t)} = 0$ et que, pour $0 \leq s < t$, la
%fonction de covariance a pour expression~:
%\[
% \gamma(s,t)
% = \PE{ X(s) X(t)} = \PE{ X(s) (X(s) + (X(t)- X(s))}
% = \PE {X(s)^2} = s
%\]
%et donc $\gamma(s,t)= s \wedge t = \min(t,s)$. Notons que la
%fonction $\gamma(s,t)$ v�rifie l'�quation (\ref{eq:rel10}). En
%effet, posant $t_0 = 0$, nous pouvons �crire~:
%\begin{equation}
% \sum_{j,k=1}^n u_j u_k t_j \wedge t_k
% = \sum_{j,k=1}^n
% \left\{ u_j u_k
% \sum_{\ell=1}^{j \wedge k} (t_{\ell} -t_{\ell-1})
% \right \}
% = \sum_{\ell=1}^n
% \left\{ (t_{\ell} - t_{\ell-1})
% \sum_{j=\ell}^n u_{\ell}^2
% \right\} \geq 0
%\end{equation}
%Ceci nous assure l'existence d'un processus gaussien r�el $(X(t),
%t \geq 0)$ tel que $\PE{ X(t) } = 0$ et $\PE{X(s) X(t)} = s
%\wedge t$. Pour $0 < s < t$, la variable al�atoire $X(t)-X(s)$ est
%distribu�e suivant une loi gaussienne de moyenne nulle et de
%variance $t-s$ et, pour $t_1 < t_2 < t_3 < t_4$, $\PE{(X(t_2) -
%X(t_1)) (X(t_4)- X(t_3))}=0$. Et donc, d'apr�s les propri�t�s
%des vecteurs gaussiens, un tel processus v�rifie les conditions
%(ii) et (iii) de la d�finition \ref{def:brownien}. On ne d�taille
%pas ici la construction qui permet de montrer qu'il est �galement
%possible de v�rifier la condition (iv).
%=================================================================
%=================================================================
%=================================================================

%\newpage
%======================================================================
%======================================================================
%======================================================================



%==================================================
%==================================================
\section{Stationnarit� stricte d'un processus � temps discret}
%==================================================
\subsection{D�finition}
La notion de stationnarit� joue un r�le central dans la th�orie
des processus al�atoires. On distingue ci-dessous deux versions de cette
propri�t�, la \emph{stationnarit� stricte} qui fait r�f�rence � l'invariance des r�partitions finies par translation de l'origine des temps,
et une notion plus faible, la \emph{stationnarit�
au second ordre}, qui impose l'invariance par translation des moments
d'ordre un et deux uniquement, lorsque ceux-ci existent.

% \clnote{pourquoi n'�crit-on pas cette d�finition pour la stationnarit�
%   stricte ?}
% \begin{definition}[Stationnarit� stricte]
% Un processus al�atoire $(X_t)_{t\in\Zset}$ est stationnaire au sens
% strict si $(X_{t_1},\dots,X_{t_k})$ et
% $(X_{t_1+h},\dots,X_{t_k+h})$ ont des lois identiques pour tout entier $k\geq
% 1$ et pour tous $t_1,\dots,t_k,h\in\Zset$.
% \end{definition}



\begin{definition}[Op�rateurs de d�calage et de retard]
\label{def:retard}
On suppose $T=\Zset$ ou $T=\Nset$.
On note $S$ et l'on appelle \emph{op�rateur de d�calage} (\emph{Shift}) l'application $E^T\to E^T$ d�finie par
$$
S(x)= (x_{t+1})_{t\in T}\quad \text{pour tout}\quad x=(x_t)_{t \in T} \in E^T\;.
$$
Pour tout $\tau\in T$, on d�finit $S^\tau$ par
$$
S^\tau(x)= (x_{t+\tau})_{t\in T}\quad \text{pour tout}\quad x=(x_t)_{t \in T} \in E^T\;.
$$
\end{definition}

\begin{definition}[Stationnarit� stricte]
\index{Stationnarit�!stricte}
On pose $T=\Zset$ ou $T=\Nset$.
Un processus al�atoire $\{X_t, t\in T \}$ est stationnaire au sens strict si $X$ et $S\circ X$ ont m�me loi, \ie\
 $\PP_{S\circ X}=\PP_X$.
\end{definition}

Par caract�risation de la loi image par les r�partitions finies, on a  $\PP_{S\circ X}=\PP_X$ si et seulement si
$$
\PP_{S\circ X}\circ\Pi_I^{-1}=\PP_X\circ\Pi_I^{-1}
$$
pour toute partie finie $I \in \mathcal{I}$.  Or $\PP_{S\circ X}\circ\Pi_I^{-1}=\PP_X\circ(\Pi_I\circ S)^{-1}$
et $\Pi_I\circ S=\Pi_{I+1}$, o� $I + 1 = \{ t+1, t \in I \}$.
On en conclut que  $\{X_t, t\in T\}$ est \emph{stationnaire au sens strict} si et seulement si, pour toute partie finie $I \in
\mathcal{I}$,
$$
\PP_I=\PP_{I+1} \; .
$$
On remarque aussi que la stationnarit� au sens strict implique que  $X$ et $S^\tau\circ X$ ont m�me loi pour tout $\tau\in T$
et donc aussi $\PP_I=\PP_{I+\tau}$, o� $I + \tau = \{ t+\tau, t \in I \}$.

% Killed by oKp on 01/10/01
% \begin{example}[Processus al�atoire binaire]
% On consid�re le processus al�atoire $X= \{ X_n, n \in \Zset \}$ �
% valeurs dans l'ensemble $\{0,1\}$. On suppose que, pour tout $n$
% et toute partie finie ordonn�e $I = \{ n_1 < \cdots < n_k \}$, les
% variables al�atoires $(X_{n_1},X_{n_2},\cdots,X_{n_k})$ sont
% ind�pendantes, de loi de Bernoulli de param�tre $\alpha_n$
% \begin{align*}
% & \PP{ X_n = x} = \alpha_n^x (1-\alpha_n)^{1-x}, \ \alpha_n \in [0,1],
% \\
% & \PP{ X_{n_1} = x_1, \cdots, X_{n_k}= x_k}
% = \prod_{i=1}^k \alpha_{n_i}^{x_i} (1-\alpha_{n_i})^{1-x_i}.
% \end{align*}
% Les r�partitions finies du processus sont caract�ris�es par la
% donn�e de la famille des param�tres des lois de Bernouilli
% $(\alpha_n)_{n \in \Zset}$. Si le param�tre $\alpha_n$ des lois de
% Bernouilli est ind�pendant de $n$, \ie $\alpha_n=\alpha$, nous
% avons, pour tout $n$, tout $I = \{ n_1 < \cdots < n_k \}$
% \[
% \PP { X_{n_1} = x_1, \cdots, X_{n_k}= x_k} = \PP{X_{n_1+n}= x_1,
% \cdots, X_{n_k+n}= x_k} = \prod_{i=1}^k \alpha_{n_i}^{x_i}
% (1-\alpha_{n_i})^{1-x_i}.
% \]
% et donc le processus est stationnaire au sens strict.
% \end{example}
\begin{example}[Processus i.i.d]
\label{exple:iid}
Soit $(Z_t)_{t\in T}$ une suite de variables al�atoires \emph{ind�pendantes et
  identiquement distribu�es} (i.i.d) � valeurs dans
$\Rset^d$. Alors $(Z_t)_{t\in T}$ est
un processus stationnaire au sens strict, car, pour toute partie finie ordonn�e
$I = \{ t_1, < t_2 < \cdots < t_n \}$ et tous bor�liens $A_1,\dots,A_n$ de
$\Rset^d$, nous
avons~:
\[
 \PP ( Z_{t_1} \in A_1, \cdots, Z_{t_n} \in A_n)
 =
 \prod_{j=1}^n \PP(Z_0 \in A_j)\;,
\]
qui ne d�pend pas de $t_1,\dots,t_n$. Notons que d'apr�s l'exemple
\ref{exple:vaindep}, pour toute probabilit� $\nu$ sur $\Rset^d$, on
sait construire un processus $(Z_t)$ i.i.d. de \emph{loi marginale} \index{Loi marginale}
$\nu$, c'est-�-dire tel que $Z_t\sim \nu$ pour tout $t\in T$.
\end{example}


\subsection{Transformations pr�servant la stationnarit�}

On pose $T=\Zset$, $E=\Cset^d$ et $\cE=\cB(\Cset^d)$ pour un entier
$d\geq1$. Commen\c{c}ons par un exemple simple.


\begin{example}[Transformation d'un processus i.i.d.]
\label{exple:trans_iid}
Soit $Z$ un processus i.i.d. (voir exemple~\ref{exple:iid}).
Soient $k$ un entier et $g$ une fonction bor�lienne de $\Rset^k$
dans $\Rset$. On peut v�rifier que le processus al�atoire
$(X_t)_{t\in\Zset}$ d�fini par
\[
 X_t= g(Z(t), Z(t-1), \cdots, Z(t-k+1))
\]
est encore un processus al�atoire stationnaire au sens strict.  Par contre, ce
processus obtenu par transformation n'est plus i.i.d dans la mesure o�, d�s que
$k \geq 1$, $X_t, X_{t+1}, \dots, X_{t+k-1}$ ont bien la m�me distribution
marginale mais sont, en g�n�ral, d�pendants car fonctions de variables
al�atoires communes. Un tel processus est dit $k$-d�pendant dans la mesure o�,
par contre, $\tau \geq k$ implique que $(X_s)_{s\leq t}$ et $(X_s)_{s\geq
  t+\tau}$ sont ind�pendants pour tout $t$. Les processus $m$-d�pendants
peuvent �tre utilis�s pour approcher une grande classe de processus d�pendants
afin d'�tudier le comportement asymptotique de statistiques usuelles telles que
la moyenne empirique.  \index{Processus!$m$-d�pendant}
Nous y reviendrons au chapitre~\ref{chap:estim_moyenne}.
\end{example}

On remarque que dans cet exemple, pour d�duire la stationnarit� de $X$, il
n'est pas n�cessaire d'utiliser que $Z$ est i.i.d. mais seulement qu'il est
stationnaire. En fait, pour v�rifier la stationnarit�, il est souvent pratique
de raisonner directement sur les lois des trajectoires en utilisant la notion
de filtrage.

\begin{definition}
  Soit $\phi$ une application mesurable de $(E^T,\cE^{\otimes T})$
  dans $(F^T,\cF^{\otimes T})$ et $X=(X_t)_{t\in T}$ un processus � valeurs dans $(E,\cE)$.
  On appelle \emph{filtr�} du processus $X$ par la transformation $\phi$ le
  processus $Y=(Y_t)_{t\in T}$ � valeurs dans $(F,\cF)$ d�fini par
  $Y=\phi\circ X$, c'est-�-dire $Y_t=\Pi_t(\phi( X))$ pour tout $t\in
  T$, o� $\Pi_t$ est d�fini par~(\ref{eq:projectioncanoniquesingle}). Si
  $\phi$ est une application lin�aire, on parlera de \emph{filtrage lin�aire}.
\end{definition}

L'exemple~\ref{exple:trans_iid} est un exemple de filtrage (en g�n�ral
non--lin�aire, � moins que $g$ soit une forme lin�aire).  La transformation
associ�e � cet exemple est l'application $\phi:\Rset^\Zset\to\Rset^\Zset$
d�finie par
$$
\phi\big((x_t)_{t\in\Zset}\big)=
\big(g(x_t,x_{t-1},\dots,x_{t-k+1})\big)_{t\in\Zset}\;.
$$


\begin{example}[D�calage]
\label{exple:decalage}
  Un exemple fondamental de filtrage lin�aire de processus est obtenu en
  prenant $\phi=S$ o� $S$ est l'op�rateur de d�calage de la
  d�finition~\ref{def:retard}. Dans ce cas $Y_t=X_{t+1}$ pour tout $t\in \Zset$.
\end{example}


\begin{example}[Filtre � r�ponse impulsionnelle finie (RIF)]
\label{exple:rif}
  Soient $n\geq1$ et $t_1<\dots < t_n$ des �l�ments de $\Zset$ et
  $\alpha_1,\dots,\alpha_n\in E$. Alors $\sum_i\alpha_i S^{-t_i}$ d�finit un
  filtrage lin�aire pour n'importe quel processus $X=(X_t)_{t\in \Zset}$ pour
  lequel la sortie est donn�e par
$$
Y_t=\sum_{i=1}^n\alpha_i X_{t-t_i},\quad t\in \Zset \; .
$$
\end{example}
\begin{example}[Diff�rentiation]
\label{exple:diff}
Un cas particulier de l'exemple pr�c�dent est donn� par l'\emph{op�rateur de diff�rentiation} $I-S^{-1}$ o� $I$ d�note l'op�rateur
identit�. Le processus obtenu en sortie s'�crit
$$
Y_t=X_t-X_{t-1},\quad t\in \Zset \; .
$$
On pourra it�rer l'op�rateur de diff�rentiation, ainsi $Y=(I-S^{-1})^kX$ est
donn�e par
$$
Y_t=\sum_{j=0}^k {{k}\choose{j}} (-1)^j X_{t-j} ,\quad t\in \Zset \; .
$$
\end{example}
\begin{example}[Retournement du temps]
\label{exple:time_reversion}
Etant donn� un processus  $X=\{X_t, t\in \Zset\}$, on appellera \emph{processus retourn�} le processus obtenu par
\emph{retournement du temps} d�fini par
$$
Y_t=X_{-t},\quad t\in \Zset \; .
$$
\end{example}
\begin{example}[Int�gration]
\label{exple:time_integration}
  Etant donn� un processus $X=(X_t)_{t\in \Zset}$ qui v�rifie
  $\sum_{t=-\infty}^0|X_t|<\infty$ p.s., on appellera \emph{processus int�gr�}
  le processus d�fini par
$$
Y_t=\sum_{s=0}^\infty X_{t-s},\quad t\in \Zset \; .
$$
Contrairement aux exemples pr�c�dents, l'application $\phi$ qui d�finit ce
filtrage doit �tre d�finie avec quelques pr�cautions. Il faut en effet tout
d'abord d�finir $\phi$ sur
$$
A=\left\{x=(x_t)_{t\in \Zset}\in E^\Zset~:~\sum_{t=-\infty}^0|x_t|<\infty\right\}\;,
$$
par $\phi(x)=\sum_{s=0}^\infty x_{t-s}$. Comme $A$ est un espace vectoriel, on
peut prolonger $\phi$ lin�airement sur $(E^\Zset,\cE^{\otimes \Zset})$. Le
point important est que ce filtrage ne sera appliqu� � $X$ que sous l'hypoth�se
$\sum_{t=-\infty}^0|X_t|<\infty$ p.s. et que ce prolongement est donc d�fini de
fa�on \emph{quelconque}.
\end{example}

On remarque que dans tous les exemples pr�c�dents les op�rateurs introduits
pr�servent la stationnarit� stricte,
c'est-�-dire, si $X$ est strictement stationnaire alors $Y$ l'est
aussi. Il est facile de construire des
filtrages lin�aires qui ne pr�serve pas la stationnarit� stricte, par exemple,
$y=\phi(x)$ avec $y_t=x_t$ pour $t$ pair et $y_t=x_t+1$ pour $t$ impaire. Une propri�t� plus
forte que la conservation de la stationnarit� est donn�e par la d�finition
suivante.

\begin{definition}
  Un filtrage lin�aire est \emph{invariant par translation} s'il commute avec
  $S$: $\phi\circ S=S\circ \phi$.
\end{definition}

Cette propri�t� implique la pr�servation de la stationnarit� mais ne lui est
pas �quivalente. Le retournement du temps est en effet un exemple de filtrage
qui ne commute pas avec $S$ puisque dans ce cas on a $\phi\circ S=S^{-1}\circ
\phi$.  En revanche tous les autres exemples ci-dessus satisfont la propri�t�
d'invariance par translation.

\begin{remark}
\label{rem:FiltrageInvTrans}
Un filtrage $\phi$ \emph{invariant par translation} est enti�rement
d�termin� par sa composition avec sa composition avec la projection canonique
$\Pi_0$,
voir~(\ref{eq:projectioncanoniquesingle}). En effet, notons
$\phi_0=\Pi_0\circ\phi$. Alors pour tout $s\in \Zset$, $\Pi_s\circ\phi=
\Pi_0\circ S^{s}\circ\phi=\Pi_0\circ\phi\circ S^{s}$. Il suffit enfin
d'observer que pour tout $x\in E^T$, $\phi(x)$ est la suite
$(\pi_s\circ\phi)_{s\in T}$.
\end{remark}









%%% Local Variables:
%%% mode: latex
%%% ispell-local-dictionary: "francais"
%%% TeX-master: "../monographie-serietemporelle"
%%% End:


\section{Processus du second ordre}
\label{sec:stat-second-ordre}


\begin{definition}[Processus du second ordre]
Le processus $X=(X_t)_{t \in T}$ \`a valeurs dans $\Cset^d$ est dit
du second ordre, si $\PE {|X_t|^2} < \infty$ pour tout $t\in T$, o\`u $|x|$
est la norme hermitienne de $x\in \Cset^d$.
\end{definition}
Notons que la \emph{fonction moyenne} d\'efinie sur $T$ par $\mu(t)= \PE{X_t}$
est \`a valeurs dans $\cset^d$ et que la \emph{fonction d'autocovariance}
d\'efinie sur $T\times T$ par
\[
\Gamma(s,t)
   = \cov(X_s,X_t)
   = \PE{(X_s - \mu(s))(X_t-\mu(t))^H}\;.
\]
Elle prend ses valeurs dans l'espace des matrices de dimension $d\times d$. Pour tout $s\in T$, $\Gamma(s,s)$ est une matrice
d'autocovariance. C'est donc une matrice hermitienne positive. Plus
g\'en\'eralement, toute fonction d'autocovariance v\'erifie les propri\'et\'es
suivantes.

\begin{proposition}
 \label{prop:positifcovgene}
 Soit $\Gamma$ la fonction d'autocovariance d'un processus du second ordre
 index\'e par $T$ \`a valeurs dans $\Cset^d$. Elle v\'erifie alors les propri\'et\'es
 suivantes.
\begin{enumerate}
\item
Sym\'etrie hermitienne: pour tout $s,t \in T$,
\begin{equation}\label{eq:gamma_hermitienne}
\Gamma(s,t)= \Gamma(t,s)^H
\end{equation}
\item Type positif \index{Fonction d'autocovariance
\subitem{positivit\'e}}:
pour tout $n\geq1$, pour tout $t_1,\dots,t_n\in T$ et pour tout
$a_1,\cdots,a_n\in\cset^d$,
\begin{equation}
\label{eq:typenonnegatif} \sum_{1 \leq k,m \leq n} a_k^H
 \Gamma(t_k,t_m)a_m \geq 0
\end{equation}
\end{enumerate}
\end{proposition}
\begin{proof}\smartqed
La propri\'et\'e~(\ref{eq:gamma_hermitienne}) est imm\'ediate par d\'efinition de la
covariance.  Pour monntrer~(\ref{eq:typenonnegatif}),
formons la combinaison lin\'eaire $Y= \sum_{k=1}^n a^H_k X_{t_k}$.
$Y$ est une variable al\'eatoire complexe. % Sa variance, qui est
En utilisant les propri\'et\'es de forme hermitienne de la covariance, on obtient
\[
 \Var{Y}= \sum_{1 \leq k,m \leq n} a_k^H
\Gamma(t_k,t_m)a_m
\]
ce qui \'etablit~(\ref{eq:typenonnegatif}).

\end{proof}
Dans le cas scalaire ($d = 1$), on note en g\'en\'eral $\gamma(s,t)$
la covariance, en r\'eservant la notation $\Gamma(s,t)$ au cas des
processus vectoriels ($d > 1$).
%======================================================
%======================================================

\section{Covariance d'un processus stationnaire au second
ordre}

%==================================================
%==================================================
Dor\'enavant, dans ce chapitre, on prend $T=\zset$.  On d\'efinit la stationnarit\'e
au second ordre en ne retenant que les propri\'et\'es du second ordre (moyenne et
covariance) d'un processus stationnaire au sens strict index\'e par $\Zset$.  En
effet, soit $X=(X_{t})_{t \in \Zset}$ un processus stationnaire au sens strict
\`a valeurs dans $\Cset^d$. Supposons de plus qu'il est du second ordre. Alors sa
fonction moyenne est constante puisque la loi marginale l'est, et sa fonction
d'autocovariance $\Gamma$ v\'erifie $\Gamma(s,t)=\Gamma(s-t,0)$ pour tout
$s,t\in\zset$ puisque les lois bi-dimensionnelles sont invariantes par
translation.  Cela donne la d\'efinition suivante.
\begin{definition}[Stationnarit\'e au second ordre]\label{def:statio_sec_ordre}
  Soit $\mu\in\Cset^d$ et $\Gamma:\Zset\to\Cset^{d\times d}$.  Un processus
  $(X_{t})_{t \in \Zset}$ \`a valeurs dans $\Cset^d$ est dit \emph{stationnaire
    au second ordre} (ou \emph{faiblement stationnaire}) de moyenne $\mu$ et de
  \emph{fonction d'auto-covariance} $\Gamma$ si~:
\begin{enumerate}[label=(\alph*)]
\item $X$ est un processus du second ordre, i.e.
$\PE{|X_{t}|^2}<+\infty$,
\item pour tout $t \in \Zset$, $\PE{ X_{t}
}=\mu$,
\item\label{eq:diffgamma} pour tout couple $(s,t) \in \Zset \times \Zset$, $\cov(X_s,X_t)= \Gamma(s-t)$.
\end{enumerate}
\end{definition}

Par convention la fonction d'autocovariance d'un processus stationnaire au
second ordre index\'e par $T$ est d\'efinie sur $T$ au lieu de $T\times T$ pour le
cas g\'en\'eral.

Comme expliqu\'e en pr\'eambule de la d\'efinition, un processus du second ordre
stationnaire au sens strict est stationnaire au second ordre.  L'implication
inverse est vraie pour la classe des processus gaussiens d\'efinies au
paragraphe~\ref{sec:proc-gauss-reels} d'apr\`es la
proposition~\ref{prop:vect_gaussiens}.


On remarque qu'un processus $(X_{t})_{t \in \Zset}$ \`a valeurs dans $\Cset^d$
est stationnaire au second ordre de moyenne $\mu$ et de \emph{fonction
  d'auto-covariance} $\Gamma$ si et seulement si pour tout $\lambda\in\Cset^d$,
le processus $(\lambda^HX_{t})_{t \in \Zset}$ \`a valeurs dans $\Cset$ est
stationnaire au second ordre de moyenne $\lambda^H\mu$ et de \emph{fonction
  d'auto-covariance} $\lambda^H\Gamma\lambda$.  L'\'etude des processus
stationnaires au second ordre peut donc se restreindre au cas $d=1$ sans grande
perte de g\'en\'eralit\'e.





%=========================================================
%=========================================================
%=========================================================
\subsection{Propri\'et\'es}
Les propri\'et\'es de la proposition~\ref{prop:positifcovgene} se d\'eclinent pour un
processus stationnaire au second ordre de la fa\c{c}on suivante.
\begin{proposition}
 \label{prop:stat2}
 La fonction d'autocovariance $\gamma: \Zset \rightarrow \Cset$ d'un processus
 stationnaire au second ordre \`a valeurs complexes v\'erifie les propri\'et\'es
 suivantes qui sont une cons\'equence directe de la proposition
 \ref{prop:positifcovgene}.
\begin{enumerate}
\item Sym\'etrie hermitienne~: \index{Fonction d'autocovariance
\subitem{sym\'etrie hermitienne}} Pour tout $s\in\zset$,
\[
\gamma(-s)= \overline{\gamma(s)}
\]
\item\label{item:type_positif} Type positif~: \index{Fonction
    d'autocovariance \subitem{positivit\'e}} Pour tout entier $n\geq1$ et tout
  vecteur $(a_1, \cdots, a_n)$ de valeurs complexes,
\begin{eqnarray*}
   \sum_{s=1}^{n}\sum_{t=1}^{n}
\overline{a_s} \gamma(s-t) a_t \geq 0
\end{eqnarray*}
\end{enumerate}
\end{proposition}
La matrice de covariance de $n$ valeurs cons\'ecutives $X_1,\dots,X_n$ du
processus poss\`ede de plus une structure particuli\`ere, dite de \emph{Toeplitz},
caract\'eris\'ee par le fait que $(\Gamma_n)_{ij} = \gamma(i-j)$.  On obtient une
matrice de la forme
 \begin{align}
 \Gamma_n \nonumber
      &=\cov([X_1\,\,\dots\,\, X_n]^T) \\
      &=
     \left [
     \begin{matrix}
      \gamma(0)&\gamma(-1)&\cdots&\gamma(1-n)\\
      \gamma(1)&\gamma(0)&\cdots&\gamma(2-n)\\
        \vdots\\
      \gamma(n-1)&\gamma(n-2)&\cdots&\gamma(0)
     \end{matrix}
     \right ]
 \label{eq:matcov}
\end{align}
Lorsque $\gamma(0)$ est non-nul il peut \^{e}tre pratique de normaliser la fonction
d'autocovariance. On obtient la d\'efinition suivante.
\begin{definition}[Fonction d'autocorr\'elation]
  Pour un processus stationnaire au second ordre de variance non nulle, on
  appelle fonction d'autocorr\'elation $\rho$ la fonction d\'efinie sur $s\in\zset$
  par $\rho(s)= \gamma(s)/\gamma(0)$. Il s'agit d'une quantit\'e normalis\'ee dans
  le sens o\`u $\rho(0) = 1$ et $|\rho(s)| \leq 1$ pour tout $s\in\zset$.
\end{definition}
En effet, l'in\'egalit\'e de Cauchy-Schwarz  appliqu\'ee \`a $\gamma$ implique
\[
 |\gamma(s)|
  = \left|\cov(X_{s},X_0)\right| \leq
    \sqrt{\Var{X_{s}}\Var{X_0}} = \gamma(0)
\]
la derni\`ere in\'egalit\'e d\'ecoulant de l'hypoth\`ese de
stationnarit\'e.%  Attention, certaines r\'ef\'erences (livres et
% publications), en g\'en\'eral anciennes, utilisent (incorrectement) le
% terme de ``fonction d'autocorr\'elation'' pour $\gamma(h)$.
% Dans la
% suite de ce document, le terme autocorr\'elation est r\'eserv\'ee \`a la
% quantit\'e normalis\'ee $\rho(h)$.
\begin{example}[Retournement du temps (suite)]
  \label{exe:stat_retourne} Soit $(X_t)_{t\in\zset}$ un processus al\'eatoire
  stationnaire au second ordre \`a valeurs r\'eelles de moyenne $\mu_X$ et de
  fonction d'autocovariance $\gamma_X$. On note, pour tout $t\in\zset$,
  $Y_t=X_{-t}$ le processus {\it retourn\'e}, comme dans
  l'exemple~\ref{exple:time_reversion}. Alors $Y_t$ est un processus
  stationnaire au second ordre de m\^{e}me moyenne et de m\^{e}me fonction
  d'autocovariance que le processus $X_t$. En effet on a~:
 \begin{align*}
  &\PE{Y_t}= \PE{X_{-t}}=\mu_X\\
  &\cov(Y_{t+h},Y_t)=\cov(X_{-t-h},X_{-t})=\gamma_X(-h)=\gamma_X(h)
 \end{align*}
\end{example}
\begin{definition}[Bruit blanc faible]\index{Bruit blanc!faible}
  On appelle bruit blanc faible un processus al\'eatoire stationnaire au second
  ordre \`a valeurs complexes ou r\'eelles, centr\'e, de fonction d'autocovariance
  $\gamma$ d\'efinie par $\gamma(0)= \sigma^2>0$ et $\gamma(s)=0$  pour tout
  $s\neq0$. On le notera $ (X_{t}) \sim  \BB(0,\sigma^2)$.
\end{definition}
\begin{definition}[Bruit blanc fort]\index{Bruit blanc!fort}
On appelle bruit blanc fort une suite de variables
al\'eatoires $(X_{t})$, centr\'ees, ind\'ependantes et identiquement
distribu\'ees (i.i.d.) de variance $\PE{X_{t}^2} = \sigma^2 <
\infty$. On le notera $ (X_{t}) \sim \BBF(0,\sigma^2)$.
\end{definition}
Par d\'efinition un bruit blanc fort est un bruit blanc faible. La
structure de bruit blanc fort est clairement plus contraignante
que celle du  bruit blanc faible. % En g\'en\'eral, il est tout \`a fait
% inutile de faire un telle hypoth\`ese lorsque l'on s'int\'eresse \`a
% des processus  stationnaires au second ordre.
% Il arrivera cependant dans la suite que nous adoptions cette
% hypoth\`ese plus forte afin de simplifier les d\'eveloppements
% math\'ematiques.
Notons que, de m\^{e}me que la stationnarit\'e stricte  d'un processus
gaussien d\'ecoule de la stationnarit\'e
faible, un bruit blanc faible gaussien est un bruit blanc fort.

\begin{example}[Processus MA(1)]
 \label{exe:MA1covth}
Soit $(X_{t})$ le processus stationnaire au second ordre
d\'efini par~:
\begin{equation}
\label{eq:recurrenceMA1}
 X_{t}= Z_{t} + \theta Z_{t-1} \; ,
\end{equation} o\`u $(Z_{t}) \sim \BB(0,\sigma^2)$ r\'eel et $\theta \in
\Rset$. On v\'erifie ais\'ement que $\PE{X_{t}}= 0$ et que sa fonction
d'autocovariance est d\'efinie par
\begin{equation}
  \label{eq:ma1-cov}
\gamma(s)=
\begin{cases}
  \sigma^2(1+\theta^2) & \text{si $s=0$,} \\
  \sigma^2 \theta & \text{si $s=\pm 1$,} \\
  0 & \text{sinon.}
\end{cases}
\end{equation}
Le processus $(X_{t})$ est donc bien stationnaire au second ordre.
Un tel processus est appel\'e \emph{processus \`a moyenne ajust\'ee d'ordre 1}.
Cette propri\'et\'e se g\'en\'eralise, sans difficult\'e, \`a un processus
MA($q$). Nous reviendrons plus en d\'etail, paragraphe
\ref{s:procARMA}, sur la d\'efinition et les propri\'et\'es de ces
processus.
\end{example}
\begin{example}[Processus harmonique r\'eel]
  \label{ex:processusharmonique} Soient $(A_k)_{1 \leq k \leq N}$ $N$
  v.a. r\'eelles de variance finie. On note $\sigma_k^2=\Var{A_k}$. Soient
  $(\Phi_k)_{1 \leq k \leq N}$, $N$ variables al\'eatoires ind\'ependantes et
  identiquement distribu\'ees (i.i.d), de loi uniforme sur $[-\pi,\pi]$, et
  ind\'ependantes de $(A_k)_{1 \leq k \leq N}$. On d\'efinit~:
\begin{equation}
   X_{t} = \sum_{k=1}^N A_k \cos(\lambda_k t + \Phi_k ) \;,
\end{equation}
o\`u $(\lambda_k)_{1 \leq k \leq N}\in [- \pi,\pi]$ sont $N$ pulsations. Le
processus $(X_{t})$ est appel\'e processus harmonique. On v\'erifie ais\'ement que
$\PE {X_{t}}= 0$ et que, pour tout $s,t\in\zset$,
\[
\PE{X_{s}X_{t}} = \frac{1}{2}
     \sum_{k=1}^N \sigma_k^2 \cos ( \lambda_k (s-t) ) \;.
\]
Le processus harmonique est donc stationnaire au second ordre.
\end{example}
\begin{example}[Marche al\'eatoire]
\label{ex:marche_aleatoire} Soit $(S_{t})$ le processus d\'efini sur
$t \in \Nset$ par $S_{t}= X_{0} + X_{1} + \cdots + X_{t}$, o\`u
$(X_{t})$ est un bruit blanc fort r\'eel. Un tel processus est appel\'e
\emph{une marche al\'eatoire}. On en d\'eduit que $\PE{S_{t}}= 0$,
$\PE{S_t^2}=t\sigma^2$ et, pour $s\leq t\in\nset$, on a~:
\[
\PE{S_sS_t} = \PE{ (S_{s} + X_{s+1} + \cdots + X_{t})S_{s}}
    = s \; \sigma^2
\]
Le processus $(S_{t})$ n'est donc pas stationnaire au second
ordre.
\end{example}
\begin{example}
 \label{exe:testposivite1}
Nous allons montrer que la fonction $\chi$ d\'efinie sur $\zset$, par
\begin{equation}
  \label{eq:covma1chi}
 \chi(s)=
 \begin{cases}
  1 & \text{si $s=0$,} \\
  \rho & \text{si $s=\pm 1$,} \\
  0 & \text{sinon.}
 \end{cases}
\end{equation}
est la fonction d'autocovariance d'un processus stationnaire au
second ordre r\'eel si et seulement si $\rho \in [-1/2,1/2]$. Nous avons d\'ej\`a
montr\'e exemple \ref{exe:MA1covth} que la fonction d'autocovariance
$\gamma$ d'un processus MA(1) est donn\'ee par~(\ref{eq:ma1-cov}).
La fonction $\chi$ est donc la fonction d'autocovariance d'un
processus MA(1) si et seulement si $\sigma^2(1+\theta^2)= 1$ et
$\sigma^2 \theta = \rho$. Lorsque $|\rho| \leq 1/2$, ce
syst\`eme d'\'equations admet comme solution~:
\[
 \theta = (2 \rho)^{-1}( 1 \pm \sqrt{1 - 4 \rho^2})
 \quad\mbox{et}\quad
 \sigma^2= (1+ \theta^2)^{-1}\;.
\]
Lorsque $|\rho| > 1/2$, ce syst\`eme d'\'equations n'admet pas de solution r\'eelles
et la fonction $\chi$ n'est donc pas la fonction d'autocovariance d'un
processus MA(1). Plus g\'en\'eralement, si $|\rho|>1/2$, alors $\chi$ n'est en
fait pas de type positif et n'est donc pas une fonction de covariance,
voir Proposition~\ref{prop:stat2}.
\end{example}

\subsection{Interpr\'etation de la fonction d'autocovariance}
\label{sec:interp_cov} Dans les exemples pr\'ec\'edents, nous avons
\'et\'e amen\'es \`a \'evaluer la fonction d'autocovariance de processus pour
quelques exemples simples de s\'eries temporelles. Dans la plupart
des probl\`emes d'int\'er\^{e}t pratique, nous ne
partons pas de mod\`eles de s\'erie temporelle d\'efinis \emph{a
priori}, mais d'\emph{observations}, $\{ x_1, \dots, x_n\}$
associ\'ees \`a une \emph{r\'ealisation} du processus. Afin de
comprendre la structure de d\'ependance entre les diff\'erentes
observations, nous serons amen\'es \`a
\emph{estimer} la loi du processus, ou du moins des caract\'eristiques de ces lois.
Pour un processus stationnaire au second ordre, nous pourrons, \`a titre d'exemple, estimer
sa moyenne par la \emph{moyenne empirique}~:
\[
 \hat\mu_n = n^{-1} \sum_{k=1}^n x_k
\]
et les fonctions d'autocovariance et d'autocorr\'elation par les
fonctions d'autocorr\'elation et d'autocovariance \emph{empiriques}
\[
 \hat{\gamma}(h) = n^{-1} \sum_{k=1}^{n - |h|} (x_k - \hat\mu_n)(x_{k+|h|} - \hat\mu_n)
 \quad\mbox{et}\quad
 \hat{\rho}(h) = \hat{\gamma}(h) / \hat{\gamma}(0)\;.
\]
Lorsqu'il est \emph{a priori} raisonnable de penser que la s\'erie
consid\'er\'ee est stationnaire au second ordre, la moyenne empirique,
la fonction d'autocovariance empirique et la fonction
d'autocorr\'elation empirique sont des estimateurs consistants de la fonction d'autocovariance
et de la fonction d'autocorr\'elation.

L'analyse de la fonction d'autocovariance empirique est un \'el\'ement
permettant de guider le choix d'un mod\`ele appropri\'e pour les
observations. Par exemple, le fait que la fonction
d'autocovariance empirique soit \emph{proche} de z\'ero pour tout
$h \ne 0$ (proximit\'e qu'il faudra d\'efinir dans un sens statistique
pr\'ecis) indique par exemple qu'un bruit blanc est un mod\`ele
ad\'equat pour les donn\'ees. La figure \ref{fig:xcorrhr} repr\'esente
les $100$ premi\`eres valeurs de la fonction d'autocorr\'elation
empirique de la s\'erie des battements cardiaques repr\'esent\'ee figure
\ref{fig:figcard1}. On observe que cette s\'erie est
\emph{positivement corr\'el\'ee} c'est-\`a-dire que les fonctions
coefficients d'autocorr\'elation sont positifs et significativement
non nuls. Nous avons, \`a titre de comparaison, repr\'esent\'e aussi la
fonction d'autocorr\'elation empirique d'une trajectoire de m\^{e}me
longueur d'un bruit blanc gaussien.
 %=========== FIGURE =====================
\begin{figure}
  \centering
  % Requires \usepackage{graphicx}
  \includegraphics[width=0.6\textwidth]{Figures/corrHR11839}\\
  \caption{Courbe de gauche~: fonction d'autocorr\'elation empirique de la s\'erie
 des battements cardiaques (figure \ref{fig:figcard1}). Courbe de droite~:
 fonction d'autocorr\'elation
 empirique d'une trajectoire de m\^{e}me longueur d'un bruit blanc
 gaussien.}\label{fig:xcorrhr}
\end{figure}
 La figure~\ref{fig:xcov} montre que le fait que $\hat{\rho}(1)
 = 0.966$ pour la s\'erie des battements cardiaques se traduit par une forte
 pr\'edictabilit\'e de $X_{t+1}$ en fonction de $X_t$ (les couples de points
 successifs s'alignent quasiment sur une droite). Nous montrerons au
 chapitre~\ref{chap:Prediction}, que dans un tel contexte,
 $\PE{(X_{t+1}-\mu)-\rho(1)(X_t-\mu)} = (1-\rho^2) \cov(X_t)$, c'est-\`a-dire,
 compte tenu de la valeur estim\'ee pour $\rho(1)$, que la variance de ``l'erreur
 de pr\'ediction'' $X_{t+1}-[\mu+\rho(1)(X_t-\mu)]$ est 15 fois plus faible que
 celle du signal original.
 %=========== FIGURE =====================
\begin{figure}
  \centering
  % Requires \usepackage{graphicx}
  \includegraphics[width=0.6\textwidth]{Figures/cov_hr11839}\\
  \caption{$X_{t+1}$ en fonction de $X_t$ pour la s\'erie
 des battements cardiaques de la figure~\ref{fig:figcard1}. Les tirets
 repr\'esentent la meilleure droite de r\'egression lin\'eaire de $X_{t+1}$ sur $X_t$.}
 \label{fig:xcov}
\end{figure}

\begin{figure}
  \centering
  % Requires \usepackage{graphicx}
  \includegraphics[width=0.6\textwidth]{Figures/logretourSP}\\
  \caption{Log-Retours de la s\'erie S\&P 500}\label{fig:sp-logretour}
\end{figure}
 L'indice S\&P$500$ trac\'e (fig.~\ref{fig:SP}) pr\'esente un cas de figure plus
 difficile, d'une part parce que la s\'erie n'est clairement pas
 stationnaire~; d'autre part, parce que selon le choix de la transformation des
 donn\'ees consid\'er\'ees, la s\'erie transform\'ee pr\'esente ou non des effets de
 corr\'elation. On d\'efinit tout d'abord les \emph{log-retours} de l'indice
 S\&P$500$ comme les diff\'erences des logarithmes de l'indice \`a deux dates
 successives~:
\[
 X_{t} = \log( S_{t}) - \log(S_{t-1})
      = \log \left( 1 + \frac{S_{t}-S_{t-1}}{S_{t-1}} \right)
\]
La s\'erie des log-retours de la s\'erie S\&P 500 est repr\'esent\'ee dans la
figure \ref{fig:sp-logretour}.
 %=========== FIGURE =====================

\begin{figure}
  \centering
  % Requires \usepackage{graphicx}
  \includegraphics[width=0.6\textwidth]{Figures/corrsplogretour}\\
  \caption{Fonction d'autocorr\'elation empirique de la s\'erie des log-retours
 de l'indice S\&P 500.}\label{fig:sp-xcorr}
\end{figure}
  Les coefficients d'autocorr\'elation
empiriques de la s\'erie des log-retours sont repr\'esent\'es dans la figure
\ref{fig:sp-xcorr}. On remarque qu'ils sont approximativement nuls
pour $h \ne 0$ ce qui sugg\`ere de mod\'eliser la s\'erie des
log-retours par un bruit blanc faible.
  %=========== FIGURE =====================
\begin{figure}
  \centering
  % Requires \usepackage{graphicx}
  \includegraphics[width=0.6\textwidth]{Figures/corrabssplogretour}\\
  \caption{Fonction d'autocorr\'elation empirique de la s\'erie des valeurs absolues des
 log-retours de l'indice S\&P 500.}\label{fig:sp-abs-xcorr}
\end{figure}
Il est int\'eressant d'\'etudier aussi la s\'erie des log-retours
absolus, $A(t) = |X_{t}|$. On peut, de la m\^{e}me fa\c{c}on,
d\'eterminer la suite des coefficients d'autocorr\'elation empirique
de cette s\'erie, qui est repr\'esent\'ee dans la figure
\ref{fig:sp-abs-xcorr}. On voit, qu'\`a l'inverse de la s\'erie des
log-retours, la s\'erie des valeurs absolues des log-retours est
positivement corr\'el\'ee, les valeurs d'autocorr\'elation \'etant
significativement non nulles pour $|h| \leq 100$. On en d\'eduit, en
particulier, que la suite des log-retours peut \^{e}tre mod\'elis\'ee
comme un bruit blanc, mais pas un bruit blanc fort\,: en effet,
pour un bruit blanc fort $X_{t}$, nous avons, pour toute fonction
$f$ telle que $\PE{f(X_{t})^2} = \sigma_f^2 < \infty$,
$\cov(f(X_{t+h}),f(X_{t}))= 0$ pour $h\neq 0$ (les variables
$f(X_{t+h})$ et $f(X_{t})$ \'etant ind\'ependantes, elles sont a
fortiori non corr\'el\'ees).
%==================================================
%==================================================
\section{Mesure spectrale d'un processus stationnaire}
%==================================================
Dans toute la suite, $\tore$ d\'esigne le tore $\ocint{-\pi,\pi}$ et
$\btore$ la tribu bor\'elienne associ\'ee. Le th\'eor\`eme
d'Herglotz ci dessous \'etablit l'\'equivalence entre la fonction
d'autocovariance et une mesure finie d\'efinie sur
$(\tore,\btore)$. Cette mesure, appel\'ee \emph{mesure
  spectrale du processus}, joue un r\^ole analogue \`a celui de la transformation
de Fourier pour les fonctions de carr\'e int\'egrable.
%================== HERGLOTZ =======================
\begin{theorem}[Herglotz]
\label{theo:herglotz}
 Une suite $(\gamma(h))_{h \in
\Zset}$ est de type positif si et seulement si il existe une unique
mesure positive $\nu$ sur $(\tore,\btore)$ telle que~:
\begin{eqnarray}
\label{eq:herglotz}
 \gamma(h) = \int_{\tore} \rme^{\rmi h\lambda} \nu(\rmd\lambda),\; \forall h\in\Zset\;.
\end{eqnarray}
%  Si la suite $(\gamma(h))$ est de carr\'e sommable
% (\textit{i.e.} $\sum_{h\in \zset} \gamma^2(h)<\infty$),
% la mesure $\nu$ poss\`ede une densit\'e $f$ (\textbf{fonction
% positive}) par rapport \`a la mesure de Lebesgue sur
% $(\tore,\btore)$ et s'\'ecrit donc
% $$
% \gamma(h) = \int_{\tore} \rme^{\rmi h\lambda} f(\lambda)\rmd\lambda\; ,
% $$
% o\`u $f$ est donn\'ee par la s\'erie de Fourier (convergente dans $\ltwo(\tore,\lleb)$ \footnote{voir Th\'eor\`eme~\ref{theo:convergence-series-fourier}})
% \[
%  f(\lambda)= \frac{1}{2\pi} \sum_{k \in \Zset} \gamma(k) \rme^{-\rmi k \lambda}\; .
% \]
\end{theorem}
Lorsque $\gamma$ est la fonction d'autocovariance d'un processus
stationnaire au second ordre, on sait d'apr\`es la proposition
\ref{prop:stat2} que $\{\gamma(h)\}_{h \in
\Zset}$ est de type positif. Les hypoth\`eses du th\'eor\`eme de Herglotz
sont donc v\'erifi\'ees et dans ce cas
la mesure $\nu$ est appel\'ee la
\emph{mesure spectrale} du processus.
Si la mesure $\nu$ poss\`ede une densit\'e $f$  par rapport \`a la mesure de Lebesgue sur
 $(\tore,\btore)$ alors  $f$  est
appel\'ee la \emph{densit\'e spectrale de puissance}
du processus.\index{Densit\'e spectrale}

\begin{proof}
 Si $\gamma(n)$ a la
repr\'esentation~(\ref{eq:herglotz}), montrons que $\gamma(n)$
est de type positif. En effet, pour tout $n$ et toute suite
$\{a_k\in\mathbb{C}\}_{1\leq k\leq n}$,
$$
 \sum_{k,m} a_k\overline{a_m} \gamma(k-m)=
 \int_{\tore} \sum_{k,m} a_k\overline{a_m} \rme^{\rmi k \lambda}\rme^{-im \lambda} \nu(\rmd\lambda)=
 \int_{\tore}\left| \sum_{k} a_k \rme^{\rmi k \lambda}\right|^2 \nu(\rmd\lambda)\geq 0\;.
 $$
 R\'eciproquement, supposons que $\gamma(n)$ soit une suite de type
 positif et consid\'erons la suite de fonctions index\'ee par $n$~:
\[
 f_n(\lambda)
 = \frac{1}{2\pi n} \sum_{k=1}^n \sum_{m=1}^n \gamma(k-m) \rme^{-\rmi k \lambda} \rme^{\rmi m\lambda}
 = \frac{1}{2\pi}\sum_{k=-(n-1)}^{n-1} \left( 1 - \frac{|k|}{n} \right)
                    \gamma(k) \rme^{-\rmi k \lambda}\; .
% = \frac{1}{2\pi}\sum_{k=-\infty}^{\infty}\gamma_n(k)\rme^{-\rmi k \lambda}
\]
$\gamma$ \'etant de type positif, $f_n(\lambda)\geq 0$, pour tout $\lambda\in \tore.$
Notons
$\nu_n$ la mesure (positive) de densit\'e $f_n$ par rapport \`a la mesure de
Lebesgue sur $\tore$. On a alors
\begin{multline}\label{eq:herglotz_1}
\int_{\tore}\rme^{\rmi h \lambda}\nu_n(\rmd\lambda)
=\int_{\tore}\rme^{\rmi h \lambda}f_n(\lambda)\rmd\lambda
=\frac{1}{2\pi}\sum_{k=-(n-1)}^{n-1} \left( 1 - \frac{|k|}{n} \right)
                    \gamma(k) \int_{\tore}\rme^{\rmi (h-k)
                      \lambda}\rmd\lambda\\
=
\left\lbrace
\begin{array}{cc}
\left(1-\frac{|h|}{n}\right)\gamma(h),&\textrm{ si }|h|<n\; ,\\
0,&\textrm{ sinon}\; .
\end{array}
\right.
\end{multline}
% \emnote{Attention \`a ce genre de citations avec des noms.. donc il faut mettre
%   une r\'ef\'erence explicite, dans un ouvrage si possible r\'ecent et diffus\'e, avec
%   des noms. Je ne trouve pas idiot de donner un \'enonc\'e pr\'ecis du th\'eor\`eme. Il
%   manque \`a mon sens dans l'\'enonc\'e et dans la preuve, l'unicit\'e de la limite. Il
%   me semble que le th\'eor\`eme de Prohorov est en g\'en\'eral formul\'e pour des mesures
%   de probabilit\'es, donc des mesures dont la masse totale est constante... du
%   coup, on donne ce th\'eor\`eme dans notre chapitre}
Quitte \`a renormaliser $\nu_n$ pour en faire une mesure de probabilit\'e,
le th\'eor\`eme de Prohorov implique qu'il existe une mesure positive $\nu$ et une sous-suite $\nu_{n_k}$ de
$\nu_n$ telle que
$$
\int_{\tore}\rme^{\rmi h\lambda}\nu_{n_k}(\rmd\lambda)
\longrightarrow \int_{\tore}\rme^{\rmi h\lambda}\nu(\rmd\lambda),
\textrm{ lorsque }k\to\infty\;.
$$
En rempla\c{c}ant $n$ par $n_k$ dans \eqref{eq:herglotz_1} et en faisant
tendre $k$ vers l'infini, on a
$$
\gamma(h)=\int_{\tore}\rme^{\rmi h \lambda}\nu(\rmd\lambda),\; \forall h\in\Zset\;.
$$
Montrons \`a pr\'esent que $\nu$ est unique. En effet, s'il existait une autre mesure
$\mu$ telle que pour tout $h\in\Zset$ : $\int_{\tore}\rme^{\rmi h
  \lambda}\nu(\rmd\lambda)=\int_{\tore}\rme^{\rmi h
  \lambda}\mu(\rmd\lambda)$
alors d'apr\`es le lemme \ref{lem:approxLinfini-convol-noyau},
$\int_{\tore}
g(\lambda)\nu(\rmd\lambda)=\int_{\tore}g(\lambda)\mu(\rmd\lambda)$
pour toute fonction continue $g$ telle que $g(\pi)=g(-\pi)$.
On en d\'eduit donc que $\nu=\mu$.

\end{proof}

\begin{corollary}[Corollaire du th\'eor\`eme d'Herglotz]
\label{prop:testpositif}
 Une suite $(\gamma(h))_{h \in \Zset}$ \`a valeurs complexes telle que
 $\sum_{h\in \zset} |\gamma(h)|^2<\infty$ est de type
positif si et seulement si la fonction d\'efinie par
$$
 f(\lambda)=\frac{1}{2\pi}\sum_{h\in\mathbb{Z}} \gamma(h)\rme^{-\rmi h \lambda}
$$
est positive pour tout $\lambda \in \mathbb{T}$.
\end{corollary}

\begin{proof}\smartqed
D'apr\`es le th\'eor\`eme de Herglotz (Th\'eor\`eme \ref{theo:herglotz}),
$(\gamma(h))_{h \in \Zset}$ est de type
positif si et seulement si il existe une mesure positive
 $\nu$ sur $(\tore,\btore)$ telle que~:
\begin{eqnarray*}
 \gamma(h) = \int_{\tore} \rme^{\rmi h\lambda} \nu(\rmd\lambda)\; .
\end{eqnarray*}
D'apr\`es le th\'eor\`eme~\ref{theo:convergence-series-fourier} et le
corollaire \ref{cor:completude-base-l2}, comme $\sum_{h\in \zset} |\gamma(h)|^2<\infty$, on peut consid\'erer
la s\'erie de Fourier associ\'ee convergente dans
$\ltwo(\tore,\lleb)$ :
$(2\pi)^{-1} \sum_{k \in \Zset} \gamma(k) \rme^{-\rmi k
  \lambda}\eqdef f(\lambda)$.
Ainsi, $\gamma(h) = \int_{\tore} \rme^{\rmi h\lambda} f(\lambda)\rmd\lambda$
et donc la positivit\'e de $\nu$ revient \`a la positivit\'e de $f$, ce qui
conclut la preuve.


% Supposons tout d'abord que $\gamma$ est absolument sommable et
% montrons que si $f(\lambda)$ d\'efinie dans la proposition est positive
% sur $\tore$ alors $\{\gamma(h)\}$ est de type positif. D'apr\`es
% \eqref{eq:herglotz_2},
% $\gamma(h)=\int_\tore \rme^{\rmi h\lambda}f(\lambda)\rmd\lambda$.
% Comme $f(\lambda)\geq 0$, $f(\lambda)\rmd\lambda$ d\'efinit bien une
% mesure positive sur $\tore$ et donc d'apr\`es le th\'eor\`eme de
% Herglotz, $\{\gamma(h)\}$ est de type positif.

% Supposons \`a pr\'esent que $\{\gamma(h)\}$ est de type positif et
% absolument sommable et montrons que $f(\lambda)$ d\'efinie dans la proposition est positive
% sur $\tore$. On a
% \emnote{pourquoi on refait la preuve ? si $\gamma(h)$ est de type positif et absolument sommable, Herglotz montre que sa mesure
% spectrale a une densit\'e par rapport \`a la mesure de Lebesgue, non ?}
% \begin{multline*}
% 0\leq f_n(\lambda)=\frac{1}{2\pi n}\sum_{1\leq r,s\leq n}
% \rme^{-\rmi r\lambda}\gamma(r-s)\rme^{\rmi s\lambda}\\
% =\frac{1}{ 2\pi}\sum_{|m|<n}\left(1-\frac{|m|}{n}\right)\rme^{-\rmi m\lambda}\gamma(m)
% \to \frac{1}{2\pi}\sum_{m\in\mathbb{Z}}\gamma(m)\rme^{-\rmi m\lambda}=f(\lambda),\textrm{ lorsque }n\to\infty\;,
% \end{multline*}
% d'apr\`es le th\'eor\`eme de convergence domin\'ee que l'on peut appliquer
% puisque $\sum_{h}|\gamma(h)|<\infty$.
% Ainsi $f(\lambda)\geq 0$ comme limite de fonctions positives.

\end{proof}
\begin{example}
En reprenant l'exemple~\ref{exe:testposivite1}, on v\'erifie
imm\'ediatement que $(\chi(h))$ est de module sommable et que\,:
$$
 f(\lambda)=\frac{1}{2\pi}\sum_h \chi(h)\rme^{-\rmi h \lambda}
     =\frac{1}{2\pi}(1+2\rho\cos( \lambda))
     $$
     et donc que la s\'equence est une fonction d'autocovariance uniquement
     lorsque $|\rho|\leq 1/2$.
\end{example}
\begin{example}[Densit\'e spectrale de puissance du bruit blanc]
  La fonction d'autocovariance d'un bruit blanc est donn\'ee par $\gamma(h)=
  \sigma^2 \delta(h)$, d'o\`u l'expression de la densit\'e spectrale correspondante
\[
 f(\lambda) = \frac{\sigma^2}{2\pi}
\]
La densit\'e spectrale d'un bruit blanc est donc constante. Cette
propri\'et\'e est \`a l'origine de la terminologie ``bruit blanc'' qui
provient de l'analogie avec le spectre de la lumi\`ere blanche
constant dans toute la bande de fr\'equences visibles.
\end{example}
\begin{example}[Densit\'e spectrale de puissance du processus MA(1)]
  \label{ex:MA1dsp}
  Le processus MA(1) introduit dans l'exemple~\ref{exe:MA1covth} poss\`ede une
  s\'equence d'autocovariance donn\'ee par $\gamma(0) = \sigma^2(1+\theta^2)$,
  $\gamma(1) = \gamma(-1) = \sigma^2 \theta$ et $\gamma(h) = 0$ sinon (cf.
  exemple~\ref{exe:MA1covth}). D'o\`u l'expression de sa densit\'e spectrale~:
\[
 f(\lambda)= \frac{\sigma^2}{2 \pi} (2 \theta \cos(\lambda) + (1+ \theta^2))
     = \frac{\sigma^2}{2 \pi} \left |1+ \theta \rme^{-\rmi\lambda}\right |^2
\]
La densit\'e spectrale d'un tel processus est repr\'esent\'ee figure
\ref{fig:dspthMA1} pour $\theta = -0.9$ et $\sigma^2=1$ avec une
\'echelle logarithmique (dB).
\end{example}
 %================ FIGURE
\begin{figure}
  \centering
  % Requires \usepackage{graphicx}
  \includegraphics[width=0.6\textwidth]{Figures/dspthMA1}\\
  \caption{Densit\'e spectrale (en dB) d'un processus MA-1, d\'efini par l'\'equation
~(\ref{eq:recurrenceMA1}) pour $\sigma=1$ et $\theta=-0.9$.}\label{fig:dspthMA1}
\end{figure}

\begin{example}[Mesure spectrale du processus harmonique]
La fonction d'autocovariance du processus harmonique $X_{t} =
\sum_{k=1}^N A_k \cos(\lambda_k t + \Phi_k )$ (voir exemple
\ref{ex:processusharmonique}) est donn\'ee par~:
\begin{equation}
 \label{eq:cov_harm}
 \gamma(h) = \frac{1}{2} \sum_{k=1}^N \sigma_k^2
  \cos ( \lambda_k h)
\end{equation}
o\`u $\sigma_k^2=\PE{A_k^2}$. Cette suite de coefficients
d'autocovariance n'est pas sommable et la mesure spectrale n'admet
pas de densit\'e. En notant cependant que~:
\[
 \cos(\lambda_k h)
 = \frac{1}{2} \int_{- \pi}^{\pi} \rme^{\rmi  h \lambda}
 (\delta_{\lambda_k}(\rmd\lambda) + \delta_{-\lambda_k}(\rmd\lambda))
\]
o\`u $\delta_{x_0}(\rmd\lambda)$ d\'esigne la mesure de Dirac au
point $x_0$ (cette mesure associe la valeur $1$ \`a tout bor\'elien de
$[-\pi,\pi]$ contenant $x_0$ et la valeur $0$ sinon), la mesure
spectrale du processus harmonique peut s'\'ecrire~:
\[
 \nu(\rmd\lambda)=
    \frac{1}{4} \sum_{k=1}^N \sigma_k^2 \delta_{\lambda_k}(\rmd\lambda)
    +
    \frac{1}{4} \sum_{k=1}^N \sigma_k^2 \delta_{-\lambda_k}(\rmd\lambda)
\]
Elle appara\^{i}t donc comme une somme de mesures de Dirac, dont
les masses $\sigma_k^2$ sont localis\'ees aux pulsations des
diff\'erentes composantes harmoniques.
\end{example}
Contrairement aux autres exemples
\'etudi\'es, le processus harmonique poss\`ede une fonction d'autocovariance, donn\'ee
par~\ref{eq:cov_harm}, non absolument sommable ($\gamma(h)$ ne
tend pas m\^{e}me vers 0 pour les grandes valeurs de $h$). Par suite, il admet une mesure spectrale mais pas une densit\'e
spectrale. La propri\'et\'e suivante, \`a d\'emontrer \`a titre d'exercice,
implique que le processus harmonique est en fait enti\`erement
pr\'edictible \`a partir de quelques-unes de ses valeurs pass\'ees.
\begin{proposition}
  S'il existe un rang $n$ pour lequel la matrice de covariance $\Gamma_n$
  d\'efinie en (\ref{eq:matcov}) est non inversible, le processus correspondant
  $X_t$ est pr\'edictible dans le sens o\`u il existe une combinaison lin\'eaire
  $a_1, \dots a_{l}$ avec $l \leq n-1$ telle que $X_t = \sum_{k=1}^l a_k
  X_{t-k}$, l'\'egalit\'e ayant lieu presque s\^urement.
\end{proposition}
L'expression de la fonction d'autocovariance, obtenue
en~(\ref{eq:cov_harm}) pour le processus harmonique, montre que
les matrices de covariances associ\'ees s'\'ecrivent comme la somme de
$2 N$ matrices complexes de rang 1. Par cons\'equent, les matrices
$\Gamma_n$ ne sont pas inversibles d\`es que $n > 2N$, ce qui
implique que le processus harmonique est pr\'edictible d\`es lors
que l'on en a observ\'e $2N$ valeurs. Ce r\'esultat est sans surprise
compte tenu du fait que les trajectoires de ce processus sont des
sommes de sinuso\"{i}des de fr\'equences $\lambda_1,\dots,
\lambda_N$ dont seules les amplitudes et les phases sont
al\'eatoires. La propri\'et\'e suivante donne une condition suffisante
simple pour \'eviter ce type de comportements ``extr\^{e}mes''.
Cette propri\'et\'e implique en particulier que, pour une fonction
d'autocovariance absolument sommable (tous les exemples vus
ci-dessus en dehors du processus harmoniques), les valeurs futures
du processus correspondant ne sont pas pr\'edictibles sans erreur \`a
partir d'un ensemble fini de valeurs pass\'ees du processus. Nous
reviendrons en d\'etail sur ces probl\`emes de pr\'ediction au
chapitre~\ref{chap:Prediction}.
\begin{proposition}
 \label{prop:Gammanrangplein}
Soit $\gamma(h)$ la fonction d'autocovariance d'un processus
stationnaire au second ordre. On suppose que $\gamma(0)>0$ et que
$\gamma(h)\rightarrow 0$ quand $h\rightarrow \infty$. Alors, quel
que soit $n$, la matrice de covariance d\'efinie
en~(\ref{eq:matcov}) est de rang plein et donc inversible.
\end{proposition}
\begin{proof}\smartqed
% autre demo pp 167 du brockwell
 Supposons qu'il existe une suite de valeurs complexes $(a_1,\dots,a_n)$
non toutes nulles, telle que $\sum_{k=1}^n\sum_{m=1}^n a_k \overline{a_m}
\gamma(k-m)=0$. En notant $\nu_X$ la mesure spectrale de $X_t$, on
peut \'ecrire~:
\begin{eqnarray*}
  0
    =\sum_{k=1}^n\sum_{m=1}^n a_k \overline{a_m}
\int_{\tore} \rme^{\rmi(k-m)\lambda}\nu_X(\rmd\lambda)
    =\int_{\tore} \left| \sum_{k=1}^n a_k \rme^{\rmi k \lambda}\right|^2 \nu_X(\rmd\lambda)
\end{eqnarray*}
Ce qui implique que $| \sum_{k=1}^n a_k \rme^{\rmi k \lambda}|^2=0$ $\nu_X$ presque partout, c'est \`a dire que $$
\nu_X(\{\lambda
: \left| \sum_{k=1}^n a_k \rme^{\rmi k \lambda}\right|^2\neq 0\})=\nu_X(\tore-Z)=0
$$
o\`u
$Z=\{\lambda_1,\dots,\lambda_M\,: \sum_{k=1}^n a_k \rme^{\rmi k \lambda_m}
= 0\}$ d\'esigne l'ensemble \emph{fini} ($M<n$) des racines $x\in \tore$
du polyn\^ome trigonom\'etrique $\sum_{k=1}^n a_k \rme^{\rmi k \lambda}$. Par
cons\'equent, les seuls \'el\'ements de $\btore$, qui peuvent \^{e}tre
de mesure non nulle pour $\nu_X$, sont les singletons
$\{\lambda_m\}$. Ce qui implique que $\nu_X=\sum_{m=1}^M a_m
\delta_{\lambda_m}$ (o\`u $a_m\geq 0$ ne peuvent \^{e}tre tous
nuls si $\gamma(0)\neq 0$). Mais, dans ce cas,
$\gamma(h)=\sum_{m=1}^M a_m \rme^{\rmi h\lambda_m}$, ce qui contredit
l'hypoth\`ese que $\gamma(h)$ tend vers $0$ quand $n$ tend vers
l'infini.
\end{proof}




%%% Local Variables:
%%% mode: latex
%%% ispell-local-dictionary: "francais"
%%% TeX-master: "../monographie-serietemporelle"
%%% End:

\chapter{Filtrage des signaux al\'eatoires \`a temps-discret}

Dans ce chapitre nous nous int\'eressons \`a une classe tr\`es importante
de processus du second ordre, les processus autor\'egressifs \`a
moyenne ajust\'ee ou processus ARMA. Afin de pouvoir \'etudier
leurs propri\'et\'es, nous allons tout d'abord \'etablir les propri\'et\'es
des processus obtenus par un filtrage lin\'eaire de processus stationnaires au
second ordre.


\section{Filtrages lin\'eaires de processus au second ordre}


On s'int\'eresse dans ce paragraphe aux propri\'et\'es du processus
$(Y_t)$ obtenu comme image du processus $(X_t)$ par le filtre
lin\'eaire suivant :
\begin{equation}\label{eq:filtrage_1}
Y_t=\sum_{k\in\mathbb{Z}}\psi_k X_{t-k}\;,
\end{equation}
o\`u $(\psi_k)$ est une suite de nombres complexes.
Lorsqu'il n'y a qu'un nombre fini de $\psi_k$ non nuls,
la somme (\ref{eq:filtrage_1}) est bien d\'efinie. On dit dans ce cas-l\`a
que le filtre est \`a r\'eponse impusionnelle finie.
La question devient plus d\'elicate lorsque l'on consid\`ere des
filtres \`a r\'eponse impulsionnelle infinie c'est \`a dire lorsque
le nombre de $\psi_k$ non nuls est infini. En effet, $Y_t$ d\'efini par
(\ref{eq:filtrage_1}) est la limite dans un sens \`a pr\'eciser, d'une
suite de variables al\'eatoires. Le th\'eor\`eme \ref{theo:filtragepassl}
donne un sens pr\'ecis \`a cette limite.

% Pour les processus stationnaires au second ordre, on consid\`ere des filtrages
% lin\'eaires \`a valeurs dans l'espace engendr\'e par toutes les combinaisons
% lin\'eaires du processus, \'etendues \`a toutes leurs limites $L^2$.

% \begin{definition}
%   Soit $X=\{X_t, t\in \Zset\}$ un processus du second ordre. On note
%   $\cH^X_\infty$ la fermeture dans $L^2(\Omega)$ du sous-espace engendr\'e par
%   les v.a. $\{X_t,\,t\in \Zset\}$,
% $$
% \cH^X_\infty = \cspan{X_t,\,t\in \Zset}\;.
% $$
% \end{definition}
% Cet ensemble est alors le sous-espace de $L^2(\Omega)$ contenant toute v.a. $Y$
% pour lesquelles il existe une suite d'\'el\'ements $(Y_n)_{n\geq1}$ de
% $\lspan{X_t,\,t\in \Zset}$ (l'espace des combinaisons lin\'eaires finies form\'ees
% d'\'el\'ements de $\{X_t, t\in \Zset\}$) qui converge vers $Y$ au sens $L^2$ quand
% $n\to\infty$, \textit{i.e.}
% $$
% \lim_{n\to\infty}\esp{\left|Y-Y_n\right|^2}\to 0 \; .
% $$


% On introduit l'{\em op\'erateur de retard} qui facilitera l'\'ecriture
% des filtrages lin\'eaires \`a valeurs dans $\cH^X_\infty$.

% \begin{definition}[Op\'erateur de retard]
%   Soit $\{X_t,\,t\in\Zset\}$ d\'efini sur $(\Omega,\cF,\PP)$ un processus du
%   second ordre. On d\'efinit l'op\'erateur de retard $B$ (comme {\em backshift} en
%   anglais) comme l'op\'erateur de l'espace $\cH^X_\infty$ dans lui-m\^{e}me
%   d\'efini par $B(X_t)=X_{t-1}$. (l'extension \`a $\cH^X_\infty$ tout entier est
%   obtenu en compl\'etant par lin\'earit\'e et densit\'e.)
% \end{definition}


% On a le r\'esultat suivant dont la preuve \'el\'ementaire est omise.
% \begin{proposition}
%   Soit $X=\{X_t,\,t\in\Zset\}$ un processus du second ordre. Supposons que $X$
%   soit de moyenne constante, pour tout $t$,
%   $\esp[X_t]=\mu$. Alors $X$ est stationnaire au second ordre si et seulement
%   si $B$ est une isom\'etrie de $\cH^X_\infty$ dans lui-m\^{e}me.
% \end{proposition}

% On note $B^k = B \circ B^{k-1}$ pour $k \geq 1$ les compositions successives de
% l'op\'erateur $B$.  Pour $k<0$, $B^k$ est d\'efini comme l'op\'erateur inverse de
% $B^{-k}$.

% \begin{remark}
%   On remarque que l'op\'erateur $B$ est tr\`es li\'e \`a l'op\'erateur $S^{-1}$. Une
%   diff\'erence essentielle est qu'il op\`ere sur un espace de v.a. (l'espace
%   $\cH^X_\infty$) alors que $S^{-1}$ op\`ere sur un espace de trajectoires
%   (l'espace $E^\Zset$). Cette relation est formellement donn\'ee par l'\'egalit\'e des
%   2 v.a.  $B(\Pi_t\circ X)=\Pi_t\circ S^{-1}\circ X$.
% \end{remark}

% Cette remarque permet imm\'ediatement d'adapter le cas des filtres RIF de
% l'exemple~\ref{exple:rif} au contexte d\'ecrit ci-dessus.
% Notons $\phi(B)=\sum_k\phi_kB^k$, o\`u $(\phi_k)_{k\in\Zset}$ est une
% suite \`a support fini. Alors, pour tout $t\in\Zset$,
% $$
% Y_t=\phi(B)(X_t)= \sum_k\phi_k X_{t-k}
% $$
% est un \'el\'ement de $\cH^X_\infty$. Il est ais\'e de calculer la moyenne et la
% covariance du processus obtenu $Y=\{Y_t,\,t\in\Zset\}$ en utilisant les
% propri\'et\'es de lin\'earit\'e et de sesquilin\'earit\'e de la covariance. Le th\'eor\`eme
% suivant traite un cas un peu plus g\'en\'eral, qui montre que ces calculs passent
% ais\'ement \`a la limite dans le cas d'une somme infinie, pourvu que
% $\sum_k|\phi_k|<\infty$. Cette condition ne permet cependant pas la description
% la plus g\'en\'erale du filtrage lin\'eaire \`a valeurs dans $\cH^X_\infty$, comme nous
% le verrons au paragraphe~\ref{sec:filtr-des-proc-theo-generale}.

\begin{theorem}
 \label{theo:filtragepassl}
 Soit $(\psi_k)_{k\in\Zset}$ une suite absolument sommable, \ie\ $\sum_{k= -\infty}^\infty |\psi_k| < \infty$
et soit $(X_{t})$ un processus al\'eatoire tel que $\sup_{t \in \Zset} \PE {|X_{t}|} < \infty$.
Alors, pour tout $t \in \Zset$, la suite~:
\[
Y_{n,t} = \sum_{k=-n}^n \psi_k X_{t-k}
\]
converge presque s\^urement, quand $n$ tend vers l'infini, vers
une limite $Y_t$ que nous notons
$$
Y_{t} = \sum_{k=-\infty}^\infty \psi_s X_{t-s}\eqsp.
$$
De plus, la variable al\'eatoire $Y_t$ est int\'egrable, \ie\ $\PE{|Y_t|} < \infty$ et la suite $(Y_{n,t})_{n \geq 0}$
converge vers $Y_t$ dans  $L^1\espaceproba$, \textit{i.e.}
$$
\limn \PE{ |Y_{n,t} - Y_{t}|} = 0 \eqsp.
$$
Supposons que $\sup_{t \in \Zset} \PE{|X_t|^2} < \infty$ alors
$\PE{|Y_{t}|^2} < \infty$ et la suite $(Y_{n,t})_{n \geq 0}$ converge en moyenne quadratique vers la variable
al\'eatoire $Y_{t}$, \textit{i.e.}
$$
\limn \PE {|Y_{n,t} - Y_{t}|^2} = 0 \eqsp.
$$
\end{theorem}
\begin{proof}\smartqed
Notons pour tout $t \in \Zset$ et $n \in \Nset$, $U_{n,t} = \sum_{k=-n}^{n} |\psi_k| |X_{t-k}|$. La suite
$(U_{n,t})_{n \geq 0}$ est une suite de variables al\'eatoires int\'egrables. Puisque $\limn \uparrow U_{n,t} = \sum_{k\in\Zset}|\psi_k| |X_{t-k}|$, on en d\'eduit que (th\'eor\`eme de Beppo-Levi)
$$
\lim_{n \to \infty} \uparrow \PE{U_{n,t}} = \PE{\sum_{k\in\zset}|\psi_k| |X_{t-k}|}\;,
$$
o\`u $\lim_{n \to \infty} \uparrow$ signifie qu'il s'agit d'une limite croissante.
Comme
$$
\PE{U_{n,t}}\leq \sum_{k=-n}^{n} |\psi_k| \PE{|X_{t-k}|} \leq \sup_{t \in \Zset} \PE{|X_t|} \, \sum_{k\in\Zset} |\psi_s|
<\infty\eqsp,
$$
on en d\'eduit que
\[
\PE {\sum_{k\in\Zset} |\psi_k| |X_{t-k}| }< \infty \eqsp.
\]
Par cons\'equent, il existe un ensemble $\Omega_0 \in \cF$ tel que $\PP(\Omega_0) = 1$
et tel que, pour tout $\omega \in \Omega_0$,
\[
\sum_{k\in\Zset} |\psi_k| | X_{t-k}(\omega) | < \infty\; .
\]
Donc pour tout $\omega \in \Omega_0$,
$$
|Y_{n,t}(\omega)-Y_t(\omega)|\leq\sum_{|k|>n}|\psi_k||X_{t-k}(\omega)|\to 0\;,\;
\textrm{lorsque } n\to\infty\;.
$$
Ainsi, pour tout $\omega \in \Omega_0$, $Y_{n,t}(\omega)$ est convergente et converge vers
$Y_t(\omega)$, ce qui montre que $\limn Y_{n,t}= Y_t$ \ps\ . Le lemme de Fatou montre que
\[
\PE{|Y_t|}= \PE{\liminf_n |Y_{n,t}|} \leq \liminf_n \PE{Y_{n,t}} \leq \sup_t \PE{|X_t|} \sum_{j=-\infty}^\infty |\psi_j| < \infty \eqsp,
\]
et donc que $Y_t \in \lone\espaceproba$. Comme $|Y_{n,t} - Y_t| \leq \sum_{k \in \Zset} |\psi_k| |X_{t-k}|$,
le th\'eor\`eme de convergence domin\'ee montre que $\lim_n \PE{|Y_{n,t}-Y_t|} = 0$ et donc que la suite
$\{ Y_{n,t} \}$ converge vers $Y_t$ dans $\lone\espaceproba$.

Consid\'erons maintenant le cas o\`u $\sup_{t \in \Zset}
\PE{|X_{t}|^2} < \infty$. Remarquons tout d'abord que $\PE{
|X_{t}|} \leq (\PE {|X_{t}|^2})^{1/2}$ et donc que cette condition
implique que $\sup_{t \in \Zset} \PE {|X_{t}|} < \infty$. La
suite $(Y_{m,t})_{m\geq 0}$ est une suite de Cauchy dans
$L^2\espaceproba$. En effet, pour $p \geq q$,
nous avons en notant $\| X \|_2 = \left( \PE{|X|^2} \right)^{1/2}$
\begin{multline*}
\|Y_{p,t} - Y_{q,t} \|_2 = \left\| \sum_{|k|=q+1}^p \psi_k X_{t-k} \right\|_2
\leq \sup_t \|X_t\| \sum_{|k|=q+1}^p |\psi_k|  \underset{q,p \rightarrow \infty}{\longrightarrow} 0\;.
\end{multline*}
Comme $L^2\espaceproba$ est complet, la suite $(Y_{n,t})$ converge vers une variable $Y_t^\star$.
En utilisant le lemme de Fatou, nous avons
\[
\PE{|Y_t - Y_t^\star|^2} = \PE{\liminf_n |Y_{n,t} - Y_t^{\star}|^2} \leq \liminf_n \PE{|Y_{n,t}- Y_t^{\star}|^2}= 0 \eqsp,
\]
ce qui montre que les limites \ps\ $Y_t$ et $\ltwo\espaceproba$, $Y_t^\star$ co\"{i}ncident \ps\ .

\end{proof}


Le r\'esultat suivant \'etablit que le processus $(Y_t)$ obtenu par filtrage
lin\'eaire d'un processus stationnaire au second ordre $(X_t)$ via
l'\'equation (\ref{eq:filtrage_1}) est
lui-m\^{e}me stationnaire au second ordre, \`a condition que la
suite des $(\psi_k)$ soit absolument sommable
\textit{i.e.} $\sum_{k\in\mathbb{Z}}|\psi_k|<\infty$.

\begin{theorem}[Filtrage des processus stationnaires au second ordre]
 \label{theo:filtragepassl_stat}
 Soit $(\psi_k)$ une suite
telle que $\sum_{k= - \infty}^\infty |\psi_k| < \infty$ et soit
$(X_{t})$ un processus stationnaire au second ordre de moyenne
$\mu_X= \PE{X_{t}}$ et de fonction d'autocovariance $\gamma_X(h)=
\cov(X_{t+h},X_{t})$ alors le processus $Y_{t} =
\sum_{k=-\infty}^\infty \psi_k X_{t-k}$ est stationnaire au second
ordre de moyenne~:
\begin{equation}
\label{eq:moyennefiltre}
 \mu_Y = \mu_X \sum_{k=-\infty}^\infty \psi_k\; ,
\end{equation}
de fonction d'autocovariance~:
\begin{equation}
\label{eq:facfiltrage}
 \gamma_Y(h)= \sum_{j=-\infty}^\infty
    \sum_{k=-\infty}^\infty \psi_j \bar{\psi}_k \gamma_X(h+k-j)\; ,
\end{equation}
et de mesure spectrale~:
\begin{equation}
\label{eq:dspfiltrage}
  \nu_Y(\rmd\lambda) = |\psi(\rme^{-i\lambda})|^2 \nu_X(\rmd\lambda)\; ,
\end{equation}
o\`u $\psi(\rme^{-\rmi \lambda}) = \sum_{k\in \zset} \psi_k \rme^{-\rmi k \lambda}$.
%est
%la transform\'ee de Fourier \`a temps discret de la suite $\{ \psi_k \}_{k \in \Zset}$.
%Enfin l'intercovariance entre
%les processus $Y_{t}$ et $X_{t}$ a pour expression~:
%\begin{equation}
% \label{eq:intercovfiltrage}
% \gamma_{YX}(h)=\PE{(Y_{t+h}-\mu_Y)(X_{t}-\mu_X)}
% = \sum_{k=-\infty}^\infty \psi_k \gamma_X(h-k)
% \end{equation}
\end{theorem}
\begin{proof}\smartqed
D'apr\`es la continuit\'e du produit scalaire dans $L^2\espaceproba$, voir th\'eor\`eme \ref{theo:cont_prod_int}, on a
\begin{multline*}
\PE{\sum_{k\in\Zset}\psi_k X_{t-k}}=\PE{\lim_{n\to\infty}\sum_{k=-n}^n \psi_k X_{t-k}}
=\lim_{n\to\infty}\PE{\sum_{k=-n}^n \psi_k X_{t-k}}\\
=\mu_X \left(\lim_{n\to\infty}\sum_{k=-n}^n \psi_k \right)
=\mu_X \sum_{k\in\Zset} \psi_k\; .
\end{multline*}

Montrons \`a pr\'esent le r\'esultat sur la fonction d'auto-covariance.
D'apr\`es la continuit\'e du produit scalaire dans $L^2\espaceproba$, voir th\'eor\`eme \ref{theo:cont_prod_int}, on a
\begin{equation*}
\Cov{Y_t}{Y_{t+h}} =\lim_{n\to\infty}\sum_{k,j=-n}^n\psi_k \bar{\psi}_j\Cov{X_{t-k}}{X_{t+h-j}}
= \sum_{k\in\Zset}\psi_k \bar{\psi}_j\gamma_X(h+k-j)\; ,
\end{equation*}
ce qui montre Eq.~\eqref{eq:facfiltrage}
% Pour la fonction
% d'autocovariance, notons tout d'abord que, pour tout $n$, le
% processus $Y_{n,t} = \sum_{s=-n}^n \psi_s X_{t-s}$ est
% stationnaire au second ordre et que nous avons~
% \[
%  \cov(Y_{n,t},Y_{n,t+h})
%   = \sum_{j=-n}^n \sum_{k=-n}^n \psi_j \psi_k
%           \gamma_X(h+k-j)
% \]
% Remarquons ensuite que
% \begin{align*}
%  \cov( Y_{t},Y_{t+h})
%  &= \cov( Y_{n,t} + (Y_{t}-Y_{n,t}), Y_{n,t+h} + (Y_{t+h}-Y_{n,t+h})) \\
%  &= \cov( Y_{n,t}, Y_{n,t+h}) + \cov( Y_{t}- Y_{n,t},Y_{n,t+h}) \\
%     &+ \cov(Y_{n,t}, Y_{t+h}-Y_{n,t+h}) +
%     \cov(Y_{t}-Y_{n,t},Y_{t+h}-Y_{n,t+h})\\
%  &=A+B+C+D
% \end{align*}
% L'in\'egalit\'e~:
% \[
% \Var{Y_{n,t}-Y_{t}}
%    = \lim_{p \rightarrow \infty} \Var(Y_{n,t} - Y_{p,t})
%    \leq \left(\sum_{j=n+1}^\infty |\psi_j| \right)^2 \gamma_X(0)
% \]
% permet ensuite de d\'eduire, quand $n$ tend vers l'infini, les
% limites suivantes
% \begin{align*}
% &|B|
%   \leq (\Var{Y_{t}- Y_{n,t}})^{1/2} (\Var{Y_{n,t+h}} )^{1/2} \rightarrow 0 \\
% &|C|
%   \leq (\Var{Y_{t+h}- Y_{n,t+h}})^{1/2} (\Var{Y_{n,t}} )^{1/2} \rightarrow 0 \\
% &|D|
%   \leq (\Var{Y_{t+h}-Y_{n,t+h}})^{1/2} (\Var{Y_{t}- Y_{n,t}} )^{1/2} \rightarrow 0
% \end{align*}
% et donc $\cov( Y_{t},Y_{t+h}) = \limn\cov( Y_{n,t},Y_{n,t+h})$, ce
% qui d\'emontre l'expression~(\ref{eq:facfiltrage})
% ~\footnote{Nous
% venons ici de d\'emontrer directement la propri\'et\'e de continuit\'e de
% la covariance dans $L^2$ que nous verrons comme une cons\'equence de
% la structure d'espace de Hilbert au chapitre~\ref{chap:Prediction}.}.

D'apr\`es le th\'eor\`eme \ref{theo:herglotz},
$\gamma_X(h)=\int_\tore
\rme^{\rmi h \lambda}\nu_X(\rmd\lambda)$ o\`u $\nu_X$ d\'esigne la mesure
spectrale du processus $(X_{t})$.
En reportant cette expression de $\gamma_X(h)$ dans
\eqref{eq:facfiltrage}, nous obtenons
\begin{equation}\label{eq:gam_Y_spec}
 \gamma_Y(h)= \sum_{j=-\infty}^\infty \sum_{k=-\infty}^\infty
          \psi_j \bar{\psi}_k \int_\tore \rme^{\rmi (h+k-j)\lambda} \nu_X(\rmd\lambda)\;.
\end{equation}
Puisque
\[
\sum_{j=-\infty}^\infty \sum_{k=-\infty}^\infty
  \int_\tore |\psi_j| |\psi_k| \nu_X(\rmd\lambda)
  \leq \gamma_X(0) \left( \sum_{j=-\infty}^\infty |\psi_j| \right)^2<\infty\;,
\]
on peut appliquer le th\'eor\`eme de Fubini et permuter les
signes somme et int\'egrale dans \eqref{eq:gam_Y_spec}. Ce
qui donne~:
\[
\gamma_Y(h)
   = \int_\tore \rme^{\rmi h \lambda} \sum_{j=-\infty}^\infty \sum_{k=-\infty}^\infty
            \psi_j \bar{\psi}_k \rme^{\rmi k \lambda}\rme^{-\rmi j\lambda}
   = \int_\tore \rme^{\rmi h \lambda} |\psi(\rme^{-\rmi \lambda})|^2 \nu_X(\rmd\lambda)\;,
\]
o\`u $\psi(\rme^{-\rmi \lambda})=\sum_{k\in\Zset} \psi_k \rme^{-\rmi k \lambda}$.
On en d\'eduit, d'apr\`es le th\'eor\`eme \ref{theo:herglotz}
que $\nu_Y(\rmd\lambda) = |\psi(\rme^{-\rmi \lambda})|^2\nu_X(\rmd\lambda)$.
%  Pour d\'eterminer l'expression de
% l'intercovariance entre les processus entre les processus $Y_{t}$
% et $X_{t}$, il suffit de noter $|\cov(Y_{t+h},X_t)|^2 \leq
% \gamma_Y(0)\gamma_X(0)<+\infty$ et que~:
% \begin{align*}
%  \PE{(Y_{t+h}-\mu_Y)(X_t-\mu_X)}
%          &=\limn \cov (Y_{n,t+h},X_t)
%          =\limn \sum_{k=-n}^n \psi_k \cov(X_{t+h-k}X_t)\\
%          &=\sum_{k=-\infty}^{\infty} \psi_k \gamma_X(h-k)
% \end{align*}
% Ce qui conclut la preuve.

\end{proof}




% La relation~(\ref{eq:dspfiltrage}) qui donne la mesure spectrale
% du processus filtr\'e en fonction de la fonction de transfert du
% filtre et de la mesure spectrale du processus de d\'epart est
% particuli\`erement simple.
% \emnote{$\alpha(B)$ et $\beta(B)$ ne sont pas bien d\'efinis; je trouve que l'on devrait faire un \'enonc\'e de ce r\'esultat que l'on utilise dans la suite fr\'equemment}
% Elle montre par exemple que la mise
% en s\'erie de deux filtres $\alpha(B)$, $\beta(B)$ de r\'eponses
% impulsionnelles absolument sommables conduit \`aune mesure
% spectrale $|\alpha(\rme^{-\rmi\lambda})|^2 |\beta(\rme^{-\rmi\lambda})|^2
% \nu_X(\rmd\lambda)$ pour le processus filtr\'e, ce qui montre au
% passage que l'ordre d'application des filtres est indiff\'erent.

Nous d\'efinissons \`a pr\'esent une classe tr\`es importante de processus
obtenus par filtrage : les \emph{processus lin\'eaires} qui sont obtenus
en filtrant un bruit blanc.
\index{Processus!linaire}
\begin{definition}[Processus lin\'eaire]
\label{def:proc_lin}
Nous dirons que $(X_{t})$ est un \emph{processus lin\'eaire} s'il existe
un bruit blanc $Z_{t} \sim \BB(0,\sigma^2)$ et une suite de coefficients
$(\psi_k)_{k \in \Zset}$ absolument sommable telle que\,:
\begin{equation}
\label{eq:representationlineaire}
 X_{t}= \mu+\sum_{k=-\infty}^\infty \psi_k Z_{t-k},\quad t\in\zset\;,
\end{equation}
o\`u $\mu$ est un nombre complexe. On dira que $(X_t)_{t\in\zset}$ est un
processus lin\'eaire \emph{causal} par rapport \`a $(Z_t)_{t\in\zset}$
si~(\ref{eq:representationlineaire}) est v\'erifi\'ee avec $\psi_k=0$ pour tout
$k<0$.  On dira que $(X_t)_{t\in\zset}$ est un processus lin\'eaire
\emph{inversible} par rapport \`a $(Z_t)_{t\in\zset}$
si~(\ref{eq:representationlineaire}) est v\'erifi\'ee et qu'il existe de plus une
suite $(\pi_k)_{k\geq0}$ absolument sommable telle que
\begin{equation}
\label{eq:representationlineaireinversible}
 Z_{t}= \sum_{k=0}^\infty \pi_k \; (X_{t-k}-\mu),\quad t\in\Zset\;.
\end{equation}
\end{definition}
D'apr\`es le th\'eor\`eme \ref{theo:filtragepassl_stat}, un processus
lin\'eaire est stationnaire au second ordre de moyenne
$\mu$, de fonction d'autocovariance :
\begin{equation}\label{eq:gamm_proc_lin}
  \gamma_X(h)= \sigma^2 \sum_{j=-\infty}^\infty \psi_j \bar{\psi}_{j+h}= \sigma^2 \sum_{\ell=-\infty}^\infty \psi_{\ell-h} \bar{\psi}_{\ell}\eqsp,
\end{equation}
et dont la mesure spectrale admet une densit\'e  donn\'ee par :
\begin{equation}
  \label{eq:dps_modlin}
  f_X(\lambda) = \frac{\sigma^2}{2 \pi} |\psi(\rme^{-\rmi\lambda})|^2\; ,
\end{equation}
o\`u $\psi(\rme^{-\rmi\lambda}) = \sum_{k\in \zset} \psi_k \rme^{-\rmi k \lambda}$.
%==================================================
%==================================================
%==================================================



\section{Processus ARMA}
\label{s:procARMA}
%==================================================
% Dans ce paragraphe nous nous int\'eressons \`a une classe importante
% de processus du second ordre, les processus autor\'egressifs \`a
% moyenne ajust\'ee ou processus ARMA. Il s'agit de restreindre la
% classe des processus lin\'eaires en ne consid\'erant que les filtres
% dont la fonction de transfert est rationnelle.
%L'int\'er\^{e}t de cette param\'etrisation est qu'\`a partir d'un nombre fini de
%param\`etres, on peut approcher avec une pr\'ecision arbitraire toute densit\'e spectrale
%suffisamment r\'eguli\`ere.
%=====================================

Avant de passer au cas g\'en\'eral des processus ARMA, nous nous
int\'eressons \`a deux classes de processus ARMA particuliers :
les processus \`a moyenne ajust\'ee (MA) et les processus autor\'egressifs (AR).

\subsection{Processus MA$(q)$}
%=====================================
\begin{definition}[Processus MA($q$)]
On dit que le processus $(X_t)$ est \`a moyenne ajust\'ee d'ordre $q$ (ou MA($q$)) si $X_t$ est donn\'e par\,:
\begin{equation}
 \label{eq:recurrenceMAq}
 X_t= Z_t + \theta_1 Z_{t-1} + \cdots + \theta_q Z_{t-q}
\end{equation}
o\`u $Z_t \sim \BB(0,\sigma^2)$ et les $\theta_i$ sont des nombres complexes.
\end{definition}
Le terme ``moyenne ajust\'ee'' est la traduction assez malheureuse
du nom anglo-saxon ``moving average'' (moyenne mobile). Observons
que $X_t=\sum_{k=0}^q \theta_k Z_{t-k}$, avec la convention $\theta_0=1$.
En utilisant les r\'esultats du th\'eor\`eme~\ref{theo:filtragepassl_stat}, on obtient $\PE{X_t}=
0$, et
\begin{equation}
\label{eq:autocovariance-MA}
\gamma_X(h)=
\begin{cases}
\sigma^2 \sum_{t=0}^{q-h} \theta_k \bar{\theta}_{k+h}, & \text{si $0
  \leq h \leq q$}\;,\\
\sigma^2 \sum_{t=0}^{q+h} \bar{\theta}_k \theta_{k-h}, & \text{si $-q
  \leq h \leq 0$}\;,\\
 0, &\text{sinon}\;.
\end{cases}
\end{equation}
Enfin, d'apr\`es la formule~(\ref{eq:dps_modlin}), le processus
admet une densit\'e spectrale dont l'expression est :
$$
 f_X(\lambda)=\frac{\sigma^2}{2\pi}
     \left |1 + \sum_{k=1}^q \theta_k \rme^{-\rmi k\lambda} \right
     |^2\; .
$$
Un exemple de densit\'e spectrale pour le processus MA$(1)$ est
repr\'esent\'e sur la figure~\ref{fig:dspthMA1}.
%=====================================
\subsection{Processus AR$(p)$}
%=====================================
\begin{definition}[Processus AR($p$)]
On dit que le processus $\{ X_t \}$ est un processus autor\'egressif d'ordre $p$ (ou AR($p$)) si
$\{X_t \}$ est un processus stationnaire au second ordre et s'il est solution de l'\'equation de r\'ecurrence\,:
\begin{equation}
 \label{eq:recurrenceARp}
 X_t = \phi_1 X_{t-1} + \cdots + \phi_p X_{t-p} + Z_t\; ,
\end{equation}
o\`u $Z_t \sim \BB(0,\sigma^2)$ est un bruit blanc et les $\phi_k$ sont des nombres
complexes.
\end{definition}
Le terme ``autor\'egressif'' provient de la forme de l'\'equation~(\ref{eq:recurrenceARp}) dans
laquelle la valeur courante du processus s'exprime sous la forme
d'une r\'egression des
$p$ valeurs pr\'ec\'edentes du processus plus un bruit additif.

L'existence et l'unicit\'e d'une solution stationnaire au second ordre de
l'\'equation~(\ref{eq:recurrenceARp}) sont des questions d\'elicates (qui ne se posaient pas
lorsque nous avions d\'efini les mod\`eles MA). Nous d\'etaillons ci-dessous la r\'eponse \`a cette question
dans le cas $p=1$.

%======================================
\subsubsection{Cas~: $|\phi_1| < 1$}
%======================================
L'\'equation de r\'ecurrence (\ref{eq:recurrenceARp}) s'\'ecrit dans le cas $p=1$ :
\begin{equation}
\label{eq:recurrenceAR1}
        X_{t} = \phi_1 X_{t-1} + Z_{t}\;,
\end{equation}
o\`u $(Z_t)\sim\BB(0,\sigma^2)$. En it\'erant (\ref{eq:recurrenceAR1}),
on obtient :
\begin{align*}
X_t&=\phi_1(\phi_1 X_{t-2}+Z_{t-1})+Z_t=\phi_1^2 X_{t-2}+\phi_1 Z_{t-1}+Z_t\\
&=\phi_1^{k+1} X_{t-k-1}+\phi_1^k Z_{t-k}+\dots+\phi_1^2 Z_{t-2}+\phi_1 Z_{t-1}
+Z_t\;.
\end{align*}
En prenant la limite quand $k \to \infty$, on en d\'eduit que
\begin{equation}\label{AR1sol}
X_t=\sum_{j=0}^{\infty}\phi_1^j Z_{t-j}\;,
\end{equation}
la s\'erie convergeant dans $\ltwo\espaceproba$ et \ps\ .
En effet, si on suppose que $X_t$ une solution stationnaire,
$$
\PE{\left|X_t-\sum_{j=0}^{k}\phi_1^j Z_{t-j}\right|^2}
=|\phi_1|^{2k+2} \PE{|X_{t-k-1}|^2}=|\phi_1|^{2k+2} \PE{|X_0|^2}\to 0,\;
k\to\infty\;,
$$
puisque $|\phi_1| < 1$. De plus, d'apr\`es la d\'efinition \ref{def:proc_lin}, $(X_t)$
defini par \eqref{AR1sol} est un processus lin\'eaire et est donc stationnaire
au second ordre. On peut v\'erifier que $(X_t)$
defini par \eqref{AR1sol} est bien solution de \eqref{eq:recurrenceAR1}
en notant que\,:
\[
  X_t = Z_t + \phi_1 \sum_{k=0}^{+\infty} \phi_1^k Z_{t-1-k}
      = Z_t + \phi_1 X_{t-1}\;.
\]
Remarquons que la solution donn\'ee par \eqref{AR1sol} peut \^{e}tre obtenu
en utilisant le d\'eveloppement la fraction rationnelle $\psi(z)=(1-\phi_1
z^{-1} )^{-1}$ en s\'erie enti\`ere
\[
  \psi(z)=\frac{1}{1-\phi_1 z^{-1}} = \sum_{k=0}^{+\infty} \phi_1^k z^{-k}
\]
convergeant sur le disque $\ensemble{z \in \Cset}{|\phi_1| < |z|}$.
Ce lien n'a rien de fortuit, comme nous le verrons  dans le \Cref{sec:cas_general}.
\begin{figure}
\centering
  \includegraphics[width=0.6\textwidth]{Figures/chronoar1}\\
  \caption{Trajectoires de longueur $500$ d'un processus AR$(1)$ gaussien.
 Courbe du haut~: $\phi_1=-0.7$.
 Courbe du milieu~: $\phi_1=0.5$.
 Courbe du bas~: $\phi_1=0.9$}
 \label{fig:figar1}
\end{figure}

La fonction d'autocovariance de $(X_{t})$ solution stationnaire
de~(\ref{eq:recurrenceAR1}) est donn\'ee par la
formule~(\ref{eq:gamm_proc_lin}) qui s'\'ecrit\,;
\begin{align}
\label{eq:autocov:AR1}
\gamma_X(h)&= \sigma^2 \sum_{k=0}^\infty \phi_1^k \bar{\phi_1}\;^{k+h}
= \sigma^2 \frac{\bar{\phi_1}\;^{h}}{1 - |\phi_1|^2}\;,\;
\textrm{ si }h\geq 0\;,\\
&=\overline{\gamma(-h)}\;,\;\textrm{ sinon.}
\end{align}

Lorsque $\phi_1$ est un r\'eel strictement positif, le processus $(X_{t})$ est positivement
corr\'el\'e, dans le sens o\`u tous ses coefficients
d'auto-covariance sont positifs. Les exemples de trajectoires
repr\'esent\'ees sur la figure~\ref{fig:figar1} montrent que des valeurs
de $\phi_1$ proches de 1 correspondent \`a des trajectoires
``persistantes''. Inversement, des
valeurs de $\phi_1$ r\'eelles et n\'egatives conduisent \`a des trajectoires o\`u une valeur positive a tendance \`a \^{e}tre
suivie par une valeur n\'egative.
%====== FIGURE
\begin{figure}
\centering
  \includegraphics[width=0.6\textwidth]{Figures/dspthAR1}
  \caption{Densit\'e spectrale d'un processus AR(1), d\'efini
  par~(\ref{eq:recurrenceAR1}) pour $\sigma=1$ et $\phi_1=0.7$.}
  \label{fig:dspthAR1}
\end{figure}
la densit\'e spectrale de $(X_t)$ est donn\'ee par
\begin{equation}
\label{eq:dsp:AR1}
 f_X(\lambda)
 = \frac{\sigma^2}{2 \pi} \left| \sum_{k=0}^\infty \phi_1^k \rme^{-\rmi k\lambda} \right|^2
 = \frac{\sigma^2}{2 \pi} \frac{1}{|1 - \phi_1 \rme^{-\rmi \lambda}
   |^2}\; .
\end{equation}
%===============================================
\subsubsection{Cas $|\phi_1| > 1$}
%===============================================
Dans ce cas-l\`a, $\PE{|X_t-\sum_{j=0}^{k}\phi_1^j
Z_{t-j}|^2} = |\phi_1|^{2k+2} \PE{X_{t-k-1}^2}$ diverge
lorsque $k$ tend vers l'infini. Par contre, on peut
r\'e\'ecrire l'\'equation d\'efinissant $X_t$ en fonction de $Z_t$ comme suit
$$
X_t=-\phi_1^{-1} Z_{t+1}+\phi_1^{-1} X_{t+1}.
$$
En it\'erant l'\'equation pr\'ec\'edente, on obtient
\begin{eqnarray*}
X_t&=&-\phi_1^{-1} Z_{t+1}-\phi_1^{-2} Z_{t+2}+\phi_1^{-2} X_{t+2}=\dots\\
&=&-\phi_1^{-1} Z_{t+1}-\phi_1^{-2} Z_{t+2}-\dots-\phi_1^{-k-1}
Z_{t+k+1}+\phi_1^{-k-1} X_{t+k+1}\; .
\end{eqnarray*}
En utilisant exactement les m\^{e}mes arguments que ceux employ\'es
pr\'ec\'edemment, on d\'eduit que la solution stationnaire dans ce cas vaut
\begin{equation}
\label{eq:ar1_sol>1}
X_t=-\sum_{j\geq 1} \phi_1^{-j} Z_{t+j}\; .
\end{equation}
Cette solution est  \textbf{non causale} : elle d\'epend
uniquement du ``futur'' du processus $(Z_t)$.

Remarquons que, comme pr\'ec\'edemment, la solution donn\'ee par \eqref{eq:ar1_sol>1} est obtenu
en choisissant le d\'eveloppement fraction rationnelle $\psi(z)=(1-\phi_1
z^{-1})^{-1}$
\[
  \psi(z)=\frac{1}{1-\phi_1 z^{-1}} = \frac{-(\phi_1 z^{-1})^{-1}}{1-(\phi_1
    z^{-1})^{-1}}=-(\phi_1 z^{-1})^{-1}\sum_{k=0}^{+\infty} (\phi_1 z^{-1})^{-k}
=-\sum_{k\geq 1}\phi_1^{-k} z^{k}\;,
\]
qui converge sur le disque $ \ensemble{ z \in \Cset}{|z| < |\phi_1|}$. Nous remarquons que
nous avons choisi dans les deux cas $|\phi_1| < 1$ et $|\phi_1| > 1$ les d\'eveloppements convergeant
dans des domaines incluant le cercle unit\'e, $\ensemble{ z \in \Cset}{ |z|=1 }$.
Ce choix est justifié pr\'ecis\'ement dans le paragraphe \ref{sec:cas_general}.

%===============================================
\subsubsection{Cas $|\phi_1| = 1$}
%===============================================
Supposons qu'il existe une solution stationnaire dans ce cas alors,
par stationnarit\'e de $X_t$,
$$
\PE{\left|X_t - \sum_{j=0}^{k-1} \phi_1^j Z_{t-j}\right|^2} = |\phi_1|^{2k} \
\PE{|X_{t-k}|^2} = |\phi_1|^{2k}\ \PE{|X_{t}|^2}=\PE{|X_{t}|^2}\;.
$$
Or, le terme de gauche est aussi \'egal \`a
$$
\PE{|X_{t}|^2} + \PE{\left|\sum_{j=0}^{k-1} \phi_1^j Z_{t-j}\right|^2}
- 2 \PE{\bar{X}_{t} \sum_{j=0}^{k-1} \phi_1^j Z_{t-j}}\eqsp.
$$
Ainsi, $\PE{|\sum_{j=0}^{k-1} \phi_1^j Z_{t-j}|^2}
=2 \PE{\bar{X}_{t} \sum_{j=0}^{k-1} \phi_1^j Z_{t-j}}$.
De plus, $\PE{|\sum_{j=0}^{k-1} \phi_1^j Z_{t-j}|^2} = \sum_{j=0}^{k-1}
|\phi_1|^{2j} \sigma^2 = k \sigma^2 $. D'o\`u, en utilisant
l'in\'egalit\'e de Cauchy-Schwarz,
$$
k\sigma^2 \leq 2 \PE{|X_t|^2}^{1/2} \PE{\left|\sum_{j=0}^{k-1}
\phi_1^j Z_{t-j}\right|^2}^{1/2} \leq 2 (\gamma_X(0)+|\mu_X|^2)^{1/2} \
k^{1/2} \sigma \eqsp,
$$
ce qui est impossible pour $k$ grand.
Donc, dans ce cas, \textbf{il n'existe pas de solution stationnaire}.

\subsubsection{Conclusion}

Nous avons donc montr\'e, dans le cas $p=1$, que l'\'equation de  r\'ecurrence
(\ref{eq:recurrenceARp}) n'admettait pas de solution stationnaire
lorsque $|\phi_1|=1$ et qu'elle admettait une solution stationnaire
lorsque $|\phi_1|\neq 1$, donn\'ee par :
$$
X_t=\sum_{j\geq 0}\phi_1^j Z_{t-j}\;, \textrm{ si } |\phi_1|<1\;,
$$
et
$$
X_t=-\sum_{j\geq 1} \phi_1^{-j} Z_{t+j}\;, \textrm{ si } |\phi_1|>1\;.
$$

%
%\begin{definition}[Op\'erateur de retard]
%\label{def:operateur-retard}
%\index{Op\'erateur de retard}
%  Soit $(X_t)_{t\in\Zset}$ d\'efini sur $\espaceproba$ un processus stationnaire au
%  second ordre et soit $\cH^X_\infty$ son enveloppe lin\'eaire.
%D'apr\`es le th\'eor\`eme \ref{theo:prolongement-isometrie}, il existe une unique
%isom\'etrie $B^X$ de  $\cH^X_\infty$ dans $L^2\espaceproba$ telle que
%$B^X(X_t)=X_{t-1}$, pour tout $t\in\Zset$. De plus, $B^X(\cH^X_\infty)=\cH^X_\infty$.
% On appellera $B^X$ l'op\'erateur de retard de $X$ (abr\'eviation de {\em backward} en
%  anglais).
%\end{definition}
%
%On note $(B^X)^k = B^X \circ (B^X)^{k-1}$ pour $k \geq 1$ les compositions successives de
%l'op\'erateur $B^X$.  Pour $k<0$, $(B^X)^k$ est d\'efini comme l'op\'erateur inverse de
%$(B^X)^{-k}$. Cette notation permet d'\'ecrire les \'equations d\'efinissant un processus
%MA et un processus AR d'une fa\c{c}on plus condens\'ee : (\ref{eq:recurrenceMAq})
%se r\'e\'ecrit en : $X_t=\theta(B^Z)Z_t$ et (\ref{eq:recurrenceARp}) se r\'e\'ecrit
%en : $\phi(B^X)X_t=Z_t$ o\`u $\theta(z)=1+\sum_{k=1}^q \theta_k z^k$ et
%$\phi(z)=1-\sum_{k=1}^p \phi_k z^k.$
%%%%%%%%%%%%%%%%%%%%%%%%%%%%%%%%%%%%%%%%%%%%%%%%%%%%%%%%%%%%%%%%%%%%%%%%%%%%%%%%
%%%%%%%%%%%%%%%%%%%%%%%%%%%%%%%%%%%%%%%%%%%%%%%%%%%%%%%%%%%%%%%%%%%%%%%%%%%%%%%%


\subsection{Cas g\'en\'eral}\label{sec:cas_general}
% La notion de processus ARMA g\'en\'eralise les notions de processus MA
% et AR.
Avant d'\'enoncer le th\'eor\`eme \ref{theo:ARMApq} qui donne
une condition n\'ecessaire et suffisante d'existence d'une solution stationnaire
\`a l'\'equation r\'ecurrente \eqref{eq:recurrenceARMApq}
d\'efinissant un processus ARMA($p,q$), nous introduisons un nouvel op\'erateur
qui sera utile dans la preuve du th\'eor\`eme \ref{theo:ARMApq}.

Soit $\cS\espaceproba$ l'ensemble des processus index\'es par $\zset$
stationnaires au second ordre et \`a valeurs complexes. A toute suite de
coefficients complexes $(\alpha_k)$ v\'erifiant :
$\sum_{k\in\Zset}|\alpha_k|<\infty$, on associe un op\'erateur
$\operatorname{F}_\alpha$ qui \`a $X\in\cS\espaceproba$ associe le processus $Y$
d\'efini par :
$$
\operatorname{F}_\alpha:X\mapsto Y=(Y_t)_{t\in\Zset}=\left(\sum_{k\in\Zset}\alpha_k X_{t-k}\right)_{t\in\Zset}\; .
$$
D'apr\`es le th\'eor\`eme \ref{theo:filtragepassl_stat}, $Y$ est aussi dans $\cS\espaceproba$.

Le lemme \ref{lem:composition} montre comment composer deux op\'erateurs
de type $\operatorname{F}_\alpha$.

\begin{lemma}\label{lem:composition}
  Soient $(\alpha_k)$ et $(\beta_k)$ des suites de coefficients complexes
  telles que : $\sum_{k\in\Zset}|\alpha_k|<\infty$ et
  $\sum_{k\in\Zset}|\beta_k|<\infty$.  Si $X\in\cS\espaceproba$ alors
$$
\filtop{\alpha} \circ \filtop{\beta}X = \filtop{\alpha * \beta} X\;, \textrm{
  dans } L^2\espaceproba\;,\;
$$
o\`u $(\alpha*\beta)_k=\sum_{j\in\mathbb{Z}}\alpha_j\beta_{k-j}$ est la convolution discrète des suites $\alpha$ et $\beta$.
\end{lemma}

 \begin{proof}
 Soit $Y=\filtop{\beta}X$. D'apr\`es le th\'eor\`eme \ref{theo:filtragepassl_stat},
puisque $\sum_k|\beta_k|<\infty$, $Y$ est dans $\cS\espaceproba$.
Pour les m\^{e}mes raisons, $\filtop{\alpha}Y$ est lui aussi dans $\cS\espaceproba$.
Soient $Z=\filtop{\alpha}[\filtop{\beta}X]$ et $W=[\filtop{\alpha\beta}]X$, on a alors, pour tout $t\in\Zset$,
$Z_t=\sum_{j\in\Zset}\alpha_j Y_{t-j}$,
o\`u $Y_t=\sum_{k\in\Zset}\beta_k X_{t-k}$ et $W_t=\sum_{k\in\Zset}(\sum_{j\in\mathbb{Z}}\alpha_j\beta_{k-j})X_{t-k}$. Ainsi,
$Z_t=\sum_{j\in\Zset}\alpha_j(\sum_{k\in\Zset}\beta_k X_{t-j-k})$.

D\'efinissons $Z_{t,m,n}$ et $W_{t,m,n}$ par :
$Z_{t,m,n}=\sum_{j=-m}^m\alpha_j(\sum_{k=-n}^n\beta_k X_{t-j-k})$
et $W_{t,m,n}=\sum_{k=-m}^m(\sum_{j=-n}^n\alpha_j\beta_{k-j})X_{t-k}$.
En posant $\ell=j+k$, on en d\'eduit que
\begin{equation}\label{eq:chang_ind}
Z_{t,m,n}=\sum_{\ell=-(m+n)}^{m+n}(\sum_{j=-m}^m\alpha_j\beta_{\ell-j})X_{t-\ell}
=W_{t,m+n,m}\;.
\end{equation}
En notant $\| X \|_2 = \left( \PE{|X|^2} \right)^{1/2}$, nous pouvons
\'ecrire en utilisant l'in\'egalit\'e triangulaire que :
\begin{multline}\label{eq:dec_filtre1}
\|Z_t-W_t\|_2\leq \|Z_t-Z_{t,m,n}\|_2+\|Z_{t,m,n}-W_{t,m+n,m}\|_2
+\|W_{t,m+n,m}-W_t\|_2\;,
\end{multline}
le deuxi\`eme terme du membre de droite de (\ref{eq:dec_filtre1}) \'etant
nul d'apr\`es (\ref{eq:chang_ind}).
D'autre part, avec :
$Z_{t,m}=\sum_{j=-m}^m\alpha_j(\sum_{k\in\Zset}\beta_k X_{t-j-k})
=\sum_{j=-m}^m\alpha_j Y_{t-j}$, on a :
\begin{equation}\label{eq:dec_filtre2}
\|Z_t-Z_{t,m,n}\|_2\leq\|Z_t-Z_{t,m}\|_2+\|Z_{t,m}-Z_{t,m,n}\|_2\;.
\end{equation}
En utilisant l'in\'egalit\'e de Cauchy-Schwarz, le fait que
$Y$ est dans $\cS\espaceproba$ et $\sum_{k\in\Zset}|\alpha_k|<\infty$,  on a
\begin{multline}\label{eq:dec_filtre3}
\|Z_t-Z_{t,m}\|_2^2=\left\|\sum_{|j|>m}\alpha_j Y_{t-j}\right\|_2^2
\leq\PE{\sum_{|j|>m,|j'|>m}|\alpha_j| |\alpha_{j'}| |Y_{t-j}| |Y_{t-j'}|}\\
\leq \PE{Y_0^2}\left(\sum_{|j|>m} |\alpha_j|\right)^2\to 0\;,\; m\to\infty\;.
\end{multline}
D'autre part, en utilisant l'in\'egalit\'e de Cauchy-Schwarz, le fait que
$X$ est dans $\cS\espaceproba$ et l'absolue sommabilit\'e de $(\alpha_k)$
et $(\beta_k)$ :
\begin{align}\label{eq:dec_filtre4}
\|Z_{t,m}-Z_{t,m,n}\|_2^2&=\left\|\sum_{|j|\leq
  m}\alpha_j(\sum_{|k|>n}\beta_k X_{t-j-k})\right\|_2^2\\
&\leq \PE{X_0^2}\left(\sum_{-m\leq j,j'\leq m}|\alpha_j| |\alpha_{j'}|\right)
\left(\sum_{|k|,|k'|>n} |\beta_k||\beta_{k'}|\right)\\
&\leq \PE{X_0^2} \left(\sum_{j\in\Zset}|\alpha_j|\right)^2
\left(\sum_{|k|>n}|\beta_k|\right)^2\to 0\;,\; m,n\to\infty\;.
\end{align}
En utilisant (\ref{eq:dec_filtre2}), (\ref{eq:dec_filtre3}) et
(\ref{eq:dec_filtre4}), on obtient que le premier terme du membre de
droite de (\ref{eq:dec_filtre1}) tend vers 0 lorsque $m$ et $n$
tendent vers l'infini.
On peut montrer en utilisant le m\^{e}me type d'arguments que
$\|W_{t,m+n,m}-W_t\|_2$
tend vers 0 lorsque $m$ et $n$ tendent vers l'infini ce qui conclut la
preuve avec (\ref{eq:dec_filtre1}).

\end{proof}


\begin{theorem}[Existence et unicit\'e des processus ARMA$(p,q)$]
\label{theo:ARMApq} Soit l'\'equation r\'ecurrente~:
\begin{equation}
 \label{eq:recurrenceARMApq}
  X_t - \phi_1 X_{t-1} - \cdots - \phi_p X_{t-p}
  =
  Z_t + \theta_1 Z_{t-1} + \cdots + \theta_q Z_{t-q}\;,
\end{equation} o\`u $Z_t \sim \BB(0,\sigma^2)$ et les
$\phi_j$ et les $\theta_j$ sont des nombres complexes. On note $\phi(z)$ et $\theta(z)$ les transformées en $z$
\begin{align}
\label{eq:tz-phi}
\phi(z)&= 1 - \phi_1 z^{-1} - \dots - \phi_p z^{-p} \\
\label{eq:tz-theta}
\theta(z)&= 1 + \theta_1 z^{-1} + \dots + \theta_q z^{-q}  \eqsp.
\end{align}
On suppose que $\phi(z)$ et $\theta(z)$ n'ont pas de z\'eros communs. Alors l'\'equation
(\ref{eq:recurrenceARMApq}) admet une solution stationnaire au
second ordre si et seulement si le polyn\^ome $\phi(z) \neq 0$ pour
$|z| = 1$. Cette solution est unique et a pour expression~:
\begin{equation}
 \label{eq:solutionARMApq}
 X_t = \sum_{k=-\infty}^{\infty} \psi_k Z_{t-k}\;,
\end{equation}
o\`u les $(\psi_k)$ sont donn\'es par les coefficients du d\'eveloppement
\begin{equation}\label{eq:dev_laurent_statio}
\frac{\theta(z)}{\phi(z)}=\sum_{k\in\Zset}\psi_k z^{-k}\;,
\end{equation}
convergeant dans la couronne
\begin{equation}
\label{eq:RC-psi}
\ensemble{ z\in\cset}{\delta_1<|z|<\delta_2} \eqsp,
\end{equation}
o\`u $\delta_1<1$ et $\delta_2>1$ sont d\'efinis par
\begin{equation}
\label{eq:definition-delta1}
\delta_1=\max\{ z \in \cset, | z | < 1, \phi(z) =0 \}
\end{equation}
et
\begin{equation}
\label{eq:definition-delta2}
\delta_2=\min \{ z \in \cset, |z| > 1, \phi(z) = 0 \} \eqsp,
\end{equation}
\end{theorem}
avec la convention $\max \emptyset = 0$ et $\min \emptyset = \infty$.
\begin{proof}
Nous commen\c{c}ons par \'enoncer et prouver un lemme utile pour la preuve
du th\'eor\`eme \ref{theo:ARMApq}.
%\emnote{il faut se faire une religion ici sur la fa\c{c}on de noter les polynomes, pour le moment on est plut\^ot
%avec des petites lettres grecques, non.. ?}
\begin{lemma}\label{lem:dev_laurent}
  Soient $\theta$ et $\phi$ deux polyn\^omes \`a coefficients complexes tels que
  $\phi(z) \neq 0$ pour $|z|=1$ et $\phi(0)=1$ alors la fraction rationnelle
  $\theta(z)/\phi(z)$ est d\'eveloppable en s\'erie de Laurent, c'est-\`a-dire
$$
\frac{\theta(z)}{\phi(z)}=\sum_{k\in\Zset} c_k z^{-k}\;,
$$
o\`u la s\'erie $\sum_{k \in \zset} c_k z^{-k}$ est uniform\'ement convergente dans la
couronne d\'efinie par
$\left\{ z\in\cset \eqsp, r_1<|z|<r_2 \right\}$, o\`u
\begin{align*}
r_1&=\max\{ |z|~:~ z \in \cset, | z | < 1, \,\phi(z) =0 \}\\
r_2&=\min\{ |z|~:~ z  \in \cset, |z| > 1, \,\phi(z) = 0 \}\; .
\end{align*}
avec la convention $\max(\emptyset)=0$ et $\min(\emptyset)=\infty$.

Le cas $r_1=0$ correspond \`a $\phi(z)\neq0$ pour tout complexe $z$ tel que
$|z|\leq1$. Dans ce cas on a $c_k=0$ pour tout $k>0$.
Sinon, pour tout $\eta\in(0,r_1)$, $c_k=O(\eta^{-k})$ quand $k\to +\infty$.

Le cas $r_2=\infty$ correspond \`a $\phi(z)\neq0$ pour tout complexe $z$ tel que
$|z|\geq1$. Dans ce cas on a $c_k=0$ pour tout $k>\max(-1,\mathrm{deg}(\theta)
- \mathrm{deg}(\phi))$.  Sinon, pour tout $\eta\in(0,1/r_2)$, $c_k=O(\eta^{k})$
quand $k\to\infty$.
\end{lemma}
\begin{proof}\smartqed
  La d\'ecomposition en \'el\'ements simples de la fraction rationnelle
  $\theta(z)/\phi(z)$ s'\'ecrit comme la somme d'un polyn\^ome de degr\'e
  $\mathrm{deg}(\theta) - \mathrm{deg}(\phi)$ (avec la convention que tout
  polyn\^ome de degr\'e strictement n\'egatif est le polyn\^ome nul) et de termes de la forme~:
  $a/(z-z_0)^r$, o\`u $z_0$ est une racine de $\phi$ de multiplicit\'e sup\'erieure
  ou \'egale \`a $r$ et $a$ est une constante.  On \'ecrit :
\begin{align*}
\textrm{si }|z_0|<1,& \;
\frac{1}{(1-z_0 z^{-1})^r}=\frac{z^{-r}}{(1-z_0/z)^r},\;
\textrm{lorsque }|z_0|<|z|\;,\\
\textrm{si }|z_0|>1,& \;
\frac{1}{(z-z_0)^r}=\frac{(-z_0)^{-r}}{(1-z/z_0)^r},\;
\textrm{lorsque }|z|<|z_0|\;.
\end{align*}
%\emnote{ce n'est que le DL de $(1-u)^\alpha$ on n'a pas besoin de trop disserter dessus}
On utilise que :
\begin{multline*}
(1-u)^{-r}=\frac{(-1)^{r-1}}{(r-1)!}\sum_{k\geq r-1}\frac{k!}{(k-r+1)!}
u^{k-r+1}\\
=\frac{(-1)^{r-1}}{(r-1)!}\sum_{k\geq 0}\frac{(k+r-1)!}{k!}
u^{k}\;,\textrm{ lorsque }|u|<1\;,
\end{multline*}
Ainsi,
\begin{align*}
&\textrm{si }|z_0|<1, \;
\frac{1}{(z-z_0)^r}=\frac{z^{-r}}{(1-z_0/z)^r}
=z^{-r}\frac{(-1)^{r-1}}{(r-1)!}\sum_{k\geq 0}\frac{(k+r-1)!}{k!}(z_0/z)^k,\; \\
&\textrm{qui converge si }|z_0|<|z|\;,\\
&\textrm{si }|z_0|>1, \;
\frac{1}{(z-z_0)^r}=\frac{(-z_0)^{-r}}{(1-z/z_0)^r}
=-\frac{z_0^{-r}}{(r-1)!}\sum_{k\geq 0}\frac{(k+r-1)!}{k!}
(z/z_0)^k,\; \\
&\textrm{qui converge si }|z|<|z_0|\;.
\end{align*}
En majorant $(k+r-1)!/k!$ par $k^{r-1}$, on en d\'eduit  que
\begin{align*}
&\textrm{si }|z_0|<1, \;
\frac{1}{(z-z_0)^r}=\sum_{k\leq -r} v_k z^k,\; \textrm{qui converge si }|z|>|z_0|\;,\\
&\textrm{si }|z_0|>1, \;
\frac{1}{(z-z_0)^r}=\sum_{k\geq 0} w_k z^k,\;\textrm{qui converge si }|z|<|z_0|\;,
\end{align*}
o\`u $|v_k|$ et $|w_k|$ sont major\'es par $C \eta^{|k|}$, $C$ \'etant une constante
strictement positive pour tout $\eta$ choisi dans
$(0,r_1)$ ou $(0,1/r_2)$, respectivement.

\end{proof}

\textbf{Retour \`a la preuve du th\'eor\`eme \ref{theo:ARMApq}}\\

Supposons que $\phi(z)\neq 0$ pour $|z|= 1$, alors d'apr\`es
le \Cref{lem:dev_laurent} il existe $r_1<1$ et $r_2>1$ tels que
\begin{equation}\label{eq:psi_arma}
\psi(z)=\frac{\theta(z)}{\phi(z)}=\sum_{k=-\infty}^{\infty} \psi_k z^{-k},\; r_1<|z|<r_2\;,
\end{equation}
o\`u la suite $(\psi_k)_{k\in\Zset}$ v\'erifie $\sum_k |\psi_k|<\infty$.
V\'erifions que le processus $(X_t)$ d\'efini par : $X_t=\sum_{k\in\Zset} \psi_k Z_{t-k}=(\filtop{\psi} Z)_t$, pour tout
$t\in\Zset$ est une solution stationnaire de (\ref{eq:recurrenceARMApq}). D'apr\`es la d\'efinition
\ref{def:proc_lin}, $(X_t)$ est stationnaire. De plus, d'apr\`es le \Cref{lem:composition},
$$
\filtop{\phi} \circ \filtop{\psi} Z =\filtop{\phi*\psi}Z=\filtop{\theta}Z \eqsp,
$$
ce qui montre l'existence d'une solution
stationnaire \`a \eqref{eq:recurrenceARMApq}.

D'autre part, si $X$ est un processus stationnaire au second ordre solution de
(\ref{eq:recurrenceARMApq}) alors $X$ v\'erifie :
\begin{equation}\label{eq:arma_F}
\filtop{\phi} X=\filtop{\theta}Z\;.
\end{equation}
Comme $\phi(z)\neq 0$ pour $|z|= 1$, alors d'apr\`es le
\Cref{lem:dev_laurent} il existe $r_1<1$ et $r_2>1$ tels que :
$$
\xi(z)=\frac{1}{\phi(z)}=\sum_{k\in\Zset} \xi_k z^{k},\; r_1<|z|<r_2\;,
$$
o\`u la suite $(\xi_k)_{k\in\Zset}$ v\'erifie $\sum_k |\xi_k|<\infty$.
On peut donc appliquer l'op\'erateur $\filtop{\xi}$ aux deux membres de l'\'equation
(\ref{eq:arma_F}) d'o\`u l'on d\'eduit en utilisant le lemme \ref{lem:composition}
que $X=\filtop{\xi\theta}Z=\filtop\psi Z$ o\`u $(\psi_k)$ est d\'efinie dans (\ref{eq:psi_arma}).
Donc $X_t=\sum_{k\in\Zset} \psi_k Z_{t-k}=(\filtop{\psi} Z)_t$, pour tout
$t\in\Zset$, ce qui assure l'unicit\'e de la solution.

R\'eciproquement, si $(X_t)$ est un processus stationnaire solution de (\ref{eq:recurrenceARMApq}) de la forme
$X_t=\sum_{k\in\Zset}\eta_k Z_{t-k}$ o\`u $\sum_k |\eta_k|<\infty$, montrons que
$\phi(z)\neq 0$ pour $|z|= 1$. En effet, puisque $X$ est solution de (\ref{eq:arma_F})
alors : $\filtop{\phi} X=\filtop{\phi}[\filtop{\eta} Z]=\filtop{\theta} Z$. D'apr\`es le lemme \ref{lem:composition},
$\filtop{\phi\eta}Z=\filtop{\theta} Z$. Posons
$\zeta_k=\sum_{j\in\Zset}\phi_j \eta_{k-j}$. On a alors, pour tout $t\in\Zset$,
$\sum_{k\in\Zset}\zeta_k Z_{t-k}=\sum_{j=1}^q\theta_j Z_{t-j}.$
En multipliant les deux membres de cette \'equation par $Z_{t-\ell}$
et en prenant l'esp\'erance, on d\'eduit que $\zeta_\ell=\theta_\ell$,
$\ell=0,\dots,q$ et $\zeta_\ell=0$, sinon. Ainsi,
$\theta(z)=\phi(z)\eta(z)$, $|z|=1$. Puisque $\theta$ et $\phi$
n'ont pas de racines communes et que $|\eta(z)|\leq\sum_{k\in\Zset}|\eta_k|<\infty$,
si $|z|=1$, $\phi(z)$ ne s'annule pas sur le cercle unit\'e :
$\{z,\; |z|=1\}$, ce qui conclut la preuve du th\'eor\`eme \ref{theo:ARMApq}.

% Supposons que $\phi(z)\neq 0$ pour $|z|= 1$, alors d'apr\`es le lemme
% \ref{lem:dev_laurent} il existe $r_1<1$ et $r_2>1$ tels que
% $$
% \psi(z)=\frac{\theta(z)}{\phi(z)}=\sum_{k=-\infty}^{\infty} \psi_k z^{k},\; r_1<|z|<r_2\;,
% $$
% o\`u la suite $(\psi_k)_{k\in\Zset}$ v\'erifie $\sum_k |\psi_k|<\infty$.
% V\'erifions que $X_t=\sum_{k\in\Zset} \psi_k Z_{t-k}=\psi(B^Z)Z_t$ pour tout
% $t\in\Zset$ est une solution stationnaire de (\ref{eq:recurrenceARMApq}). D'apr\`es la d\'efinition
% \ref{def:proc_lin}, $(X_t)$ est stationnaire. De plus, par d\'efinition, $\cH^X_\infty\subset\cH^Z_\infty$ et
% d'apr\`es le lemme \ref{lem:composition},
% $\phi(B^Z)X_t=\phi(B^Z)\psi(B^Z)Z_t=(\phi\psi)(B^Z)Z_t=\theta(B^Z)Z_t.$
% Montrons que cette solution est unique. Pour cela, notons En effet, soit $(Y_t)$ un autre processus stationnaire
% solution de (\ref{eq:recurrenceARMApq}) alors :
% $(X_t-Y_t)-\sum_{k=1}^p\phi_k(X_{t-k}-Y_{t-k})=0$


% R\'eciproquement, si $X_t$ est une solution stationnaire de (\ref{eq:recurrenceARMApq}) de la forme
% $X_t=\sum_{k\in\Zset}\eta_k Z_{t-k}$ o\`u $\sum_k |\eta_k|<\infty$
% alors : $\phi(B^X)X_t=\phi(B^Z)\eta(B^Z)Z_t=\theta(B^Z)Z_t$. D'apr\`es le lemme \ref{lem:composition},
% $(\phi\eta)(B^Z)Z_t=\theta(B^Z)Z_t$. Posons
% $\zeta(z)=\phi(z)\eta(z)=\sum_{k\in\Zset}\zeta_k z^k$, lorsque $|z|=1$. On a alors,
% $\sum_{k\in\Zset}\zeta_k Z_{t-k}=\sum_{j=1}^q\theta_j Z_{t-j}.$
% En multipliant les deux membres de cette \'equation par $Z_{t-\ell}$
% et en prenant l'esp\'erance, on d\'eduit que $\zeta_\ell=\theta_\ell$,
% $\ell=0,\dots,q$ et $\zeta_\ell=0$, sinon. Ainsi,
% $\theta(z)=\phi(z)\eta(z)$, $|z|=1$. Puisque $\theta$ et $\phi$
% n'ont pas de racines communes et que $|\eta(z)|\leq\sum_{k\in\Zset}|\eta_k|<\infty$,
% si $|z|=1$, $\phi(z)$ ne s'annule pas sur le cercle unit\'e :
% $\{z,\; |z|=1\}.$

\end{proof}
Dans le cas o\`u $\phi(z)$ et $\theta(z)$ ont des z\'eros communs,
deux configurations sont possibles~:
\begin{enumerate}[label=(\alph*)]
\item Les z\'eros communs ne sont pas sur le cercle unit\'e. Dans ce
cas on se
  ram\`ene au cas sans z\'ero commun en annulant les facteurs communs.
\item Certains des z\'eros communs se trouvent sur le cercle unit\'e.
  L'\'equation~(\ref{eq:recurrenceARMApq}) admet une infinit\'e de solutions
  stationnaires au second ordre.
\end{enumerate}
Du point de vue de la mod\'elisation, la pr\'esence de z\'eros communs
ne pr\'esente aucun int\'er\^{e}t puisqu'elle est sans influence sur
la densit\'e spectrale de puissance. Elle conduit de plus \`a une
ambigu\"{i}t\'e sur l'ordre r\'eel des parties AR et MA.
%=========================================================
\subsubsection{ARMA$(p,q)$ causal}
%  Comme dans le cas d'un processus AR($p$), on peut
% distinguer trois cas, suivant que les z\'eros de $\phi(z)$ sont \`a
% l'ext\'erieur, \`a l'int\'erieur ou de part et d'autre du cercle unit\'e.
% Dans le cas o\`u les z\'eros de $\phi(z)$ sont \`a l'ext\'erieur du
% cercle unit\'e, la suite $\xi_k$ est causale ($\xi_k=0$ pour $k<0$)
% et donc $\psi_k=\xi_k+\sum_{j=1}^q \theta_j\xi_{k-j}$ est aussi
% causale. Par cons\'equent le processus $X_t$ s'exprime causalement
% en fonction de $Z_t$.
Le th\'eor\`eme \ref{theo:ARMApq_causal} donne une condition n\'ecessaire
et suffisante d'existence d'une solution causale \`a l'\'equation
(\ref{eq:recurrenceARMApq}).

\begin{definition}[Repr\'esentation ARMA causale]
  Sous les hypoth\`eses du th\'eor\`eme~\ref{theo:ARMApq}, on dira que
  l'\'equation~(\ref{eq:recurrenceARMApq}) fournit une repr\'esentation causale de
  la solution stationnaire au second ordre $(X_t)$ si $(X_t)_{t\in\zset}$ est
  un processus lin\'eaire causal par rapport \`a $(Z_t)_{t\in\zset}$.
\end{definition}

\begin{theorem}[ARMA$(p,q)$ causal]
\label{theo:ARMApq_causal}
Soit l'\'equation r\'ecurrente~:
\begin{equation}
 \label{eq:recurrenceARMApqcausal}
  X_t - \phi_1 X_{t-1} - \cdots - \phi_p X_{t-p}
  =
  Z_t + \theta_1 Z_{t-1} + \cdots + \theta_q Z_{t-q}
\end{equation} o\`u $Z_t \sim \BB(0,\sigma^2)$ et
$\{ \phi_j \}_{j = 1}^p$ et $\{ \theta_j \}_{j=1}^q$ sont des nombres complexes. On pose
$\phi(z)= 1 - \phi_1 z^{-1} - \dots - \phi_p z^{-p}$ et $\theta(z)= 1 +
\theta_1 z^{-1} + \dots + \theta_p z^{-p}$. On suppose que $\phi(z)$ et
$\theta(z)$ n'ont pas de z\'eros communs. Alors l'\'equation
(\ref{eq:recurrenceARMApqcausal})
fournit une repr\'esentation causale de la solution stationnaire au second ordre
si et seulement si le polyn\^ome $\phi(z)
\neq 0$ pour $|z| \geq 1$. Cette solution a pour
expression~:
\begin{equation}
 \label{eq:solutionARMApq_causal}
  X_t = \sum_{k\geq 0} \psi_k Z_{t-k}
\end{equation}
o\`u la suite $(\psi_k)$ est donn\'ee par les coefficients du d\'eveloppement
$$
\frac{\theta(z)}{\phi(z)}=\sum_{k=0}^{\infty} \psi_k z^{-k}
$$
qui converge dans la couronne $\{z\in\cset\;,\;|z|\geq 1\}.$
\end{theorem}
\begin{proof}\smartqed
Le th\'eor\`eme~\ref{theo:ARMApq} montre l'existence et l'unicit\'e
de la solution de l'\'equation~(\ref{eq:recurrenceARMApqcausal}) et v\'erifie
\begin{equation*}
  X_t = \sum_{k\in\zset} \psi_k Z_{t-k}
\end{equation*}
o\`u la suite $(\psi_k)$ est caract\'eris\'ee par l'\'equation
$$
\frac{\theta(z)}{\phi(z)}=\sum_{k=0}^{\infty} \psi_k z^{-k},\quad z\in\cset,\,|z|=1\;.
$$
Si maintenant $\phi(z)\neq 0$ pour $|z|\leq 1$, alors, d'apr\`es le
 lemme \ref{lem:dev_laurent}, comme $r_1=0$, on a $\psi_k=0$ pour $k<0$
 et~(\ref{eq:solutionARMApq_causal}) suit.


R\'eciproquement, si $X$ est une solution stationnaire de
(\ref{eq:recurrenceARMApqcausal}) de la forme $X_t=\sum_{k\geq 0}\eta_k
Z_{t-k}$ o\`u $\sum_k |\eta_k|<\infty$, montrons que $\phi(z)\neq 0$ pour
$|z|\leq 1$. En effet, puisque $X$ est solution
de~(\ref{eq:recurrenceARMApqcausal}), alors : $\filtop{\phi}
X=\filtop{\phi}[\filtop{\eta} Z]=\filtop{\theta} Z$. D'apr\`es le lemme
\ref{lem:composition}, $\filtop{\phi\eta}Z=\filtop{\theta} Z$. Posons, pour
tout $k\in\mathbb{N}$, $\zeta_k=\sum_{j\geq 0}\phi_j \eta_{k-j}$, o\`u par
convention, $\eta_\ell=0$, si $\ell<0$.  On a alors, pour tout $t\in\Zset$,
$\sum_{k\geq 0}\zeta_k Z_{t-k}=\sum_{j=0}^q\theta_j Z_{t-j},$ avec la
convention $\theta_0=1$.  En multipliant les deux membres de cette \'equation par
$Z_{t-\ell}$ et en prenant l'esp\'erance, on d\'eduit que $\zeta_\ell=\theta_\ell$,
$\ell=0,\dots,q$ et $\zeta_\ell=0$, sinon. Ainsi, $\theta(z)=\phi(z)\eta(z)$,
$|z|\leq 1$. Puisque $\theta$ et $\phi$ n'ont pas de racines communes et que
$|\eta(z)|\leq\sum_{k\geq 0}|\eta_k|<\infty$, si $|z|\leq 1$, $\phi(z)$ ne
s'annule pas sur le disque unit\'e : $\{z,\; |z|\leq 1\}$, ce qui conclut la
preuve du th\'eor\`eme~\ref{theo:ARMApq_causal}.

\end{proof}
%==========================================


\begin{definition}[Repr\'esentation ARMA inversible]
  Sous les hypoth\`eses du th\'eor\`eme~\ref{theo:ARMApq}, on dira que
  l'\'equation~(\ref{eq:recurrenceARMApq}) fournit une repr\'esentation inversible
  de la solution stationnaire au second ordre $(X_t)$ si $(X_t)_{t\in\zset}$
  est un processus lin\'eaire inversible par rapport \`a $(Z_t)_{t\in\zset}$.
\end{definition}

\begin{theorem}[ARMA$(p,q)$ inversible]
 \label{theo:ARMAinversible}
Soit l'\'equation r\'ecurrente~:
\begin{equation}
 \label{eq:recurrenceARMApqinversible}
  X_t - \phi_1 X_{t-1} - \cdots - \phi_p X_{t-p}
  =
  Z_t + \theta_1 Z_{t-1} + \cdots + \theta_q Z_{t-q}
\end{equation} o\`u $Z_t \sim \BB(0,\sigma^2)$ et les
$\phi_j$ et les $\theta_j$ sont des nombres complexes. On pose
$\phi(z)= 1 - \phi_1 z - \dots - \phi_p z^p$ et $\theta(z)= 1 +
\theta_1 z + \dots + \theta_p z^p$. On suppose que $\phi(z)$ et
$\theta(z)$ n'ont pas de z\'eros communs. Alors l'\'equation
(\ref{eq:recurrenceARMApqinversible}) fournit une repr\'esentation inversible
de la solution stationnaire au second ordre si et seulement si le polyn\^ome $\theta(z)
\neq 0$ pour $|z| \leq 1$. Cette solution est unique et a pour
expression~:
\begin{equation}
 \label{eq:solutionARMApq_inversible}
  Z_t = \sum_{k\geq 0} \pi_k X_{t-k}
\end{equation}
o\`u la suite $(\pi_k)$ est donn\'ee par les coefficients du d\'eveloppement
$$
\frac{\phi(z)}{\theta(z)}=\sum_{k=0}^{\infty} \pi_k z^{-k}
$$
qui converge dans $\{z\in\cset\;,\;|z|\leq 1\}.$
% Soit $X_t$ un processus ARMA$(p,q)$. On suppose que $\phi(z)$ et
% $\theta(z)$ n'ont pas de z\'eros communs. Alors il existe une suite
% $\{\pi_k\}$ causale absolument sommable telle que~:
% \begin{equation}
%  \label{eq:analyseARMApq}
%  Z_t=\sum_{k=0}^{\infty} \pi_k X_{t-k}
% \end{equation}
% si et seulement si $\theta(z)\neq 0$ pour $z\leq 1$. On dit alors
% que le mod\`ele ARMA$(p,q)$ est inversible. La suite $\pi_k$ est
% la suite des coefficients du d\'eveloppement en s\'erie de
% $\phi(z)/\theta(z)$ dans le disque $\{z: |z|\leq 1\}$.
\end{theorem}
La preuve de ce th\'eor\`eme est tout \`a fait analogue \`a celle du
th\'eor\`eme~\ref{theo:ARMApq_causal} et n'est donc pas d\'etaill\'ee ici.
% Remarquons que la notion d'inversibilit\'e, comme celle de causalit\'e, est bien relative au
% mod\`ele ARMA$(p,q)$ lui-m\^{e}me et pas uniquement au processus
% $X_t$.
%\begin{exercice}
%  Soit $X_t$ un processus stationnaire au second ordre solution de l'\'equation
%  de r\'ecurrence~(\ref{eq:recurrenceARMApqcausal}) o\`u le mod\`ele ARMA$(p,q)$
%  correspondant est suppos\'e sans z\'ero commun mais pas n\'ecessairement
%  inversible. Montrer qu'il existe un bruit blanc $\tilde{Z}_t$ tel que $X_t$
%  soit solution de
%  \[
%    \phi(B) X_t = \tilde{\theta}(B) \tilde{Z}_t
%  \]
%  o\`u le mod\`ele ARMA$(p,q)$ d\'efini par $\phi_1, \dots \phi_p$ et
%  $\tilde{\theta}_1, \dots \tilde{\theta}_q$ est inversible
%  (indication\,: consid\'erer des facteurs passe-tout).
%\end{exercice}

Un mod\`ele ARMA$(p,q)$ est causal et inversible lorsque
les racines des polyn\^omes $\phi(z)$ et $\theta(z)$ sont toutes
situ\'ees \`a l'ext\'erieur du disque unit\'e. Dans ce cas, $X_t$ et $Z_t$
se d\'eduisent mutuellement l'un de l'autre par des op\'erations de
filtrage causal.
%, la r\'eponse impulsionnelle de chacun de ces
%filtres \'etant {\em \`a phase minimale} (c'est \`a dire inversible
%causalement).

\subsubsection{Calcul des covariances d'un processus ARMA$(p,q)$ causal}
Une premi\`ere m\'ethode consiste \`a utiliser l'expression
(\ref{eq:gamm_proc_lin}) % qui s'\'ecrit, compte tenu du fait que
% $\{Z_t\}$ est un bruit blanc,
% $$
%  \gamma(h)=
%    \sigma^2 \sum_{k=0}^\infty\psi_k \psi_{k+|h|}
%    $$
   o\`u la suite $(\psi_k)$ se d\'etermine de fa\c{c}on r\'ecurrente \`a partir de
   l'\'egalit\'e $\psi(z)\theta(z)=\phi(z)$ par identification du terme en $z^k$.
   Pour les premiers termes on trouve~:
\begin{eqnarray*}
 &&\psi_0=1\\
 &&\psi_1=\theta_1+\psi_0\phi_1\\
 &&\psi_2=\theta_2+\psi_0\phi_2+\psi_1\phi_1\\
 &&\cdots
\end{eqnarray*}
La seconde m\'ethode utilise une formule de r\'ecurrence, v\'erifi\'ee par
la fonction d'autocovariance d'un processus ARMA$(p,q)$, qui
s'obtient en multipliant les deux membres de
(\ref{eq:recurrenceARMApq}) par $\bar{X}_{t-k}$ et en prenant
l'esp\'erance. On obtient~:
\begin{align}
 \label{eq:recurrencegamma1}
  &\gamma(k)-\phi_1\gamma(k-1)-\cdots-\phi_p\gamma(k-p)
  =
  \sigma^2\sum_{k\leq j\leq q}\theta_j\bar{\psi}_{j-k}\;,\;
0\leq k < \max(p,q+1)
  \\
 \label{eq:recurrencegamma2}
  &\gamma(k)-\phi_1\gamma(k-1)-\cdots-\phi_p\gamma(k-p)
  =
  0\;,\; k \geq \max(p,q+1)
\end{align}
o\`u nous avons utilis\'e la causalit\'e du processus pour \'ecrire
que $\PE{Z_t \bar{X}_{t-k}}=0$ pour tout $k\geq 1$. Le calcul de la
suite $\{\psi_k\}$ pour $k=1,\dots, p$ se fait comme pr\'ec\'edemment.
En reportant ces valeurs dans~(\ref{eq:recurrencegamma1}) pour
$0\leq k \leq p$, on obtient $(p+1)$ \'equations lin\'eaires aux
$(p+1)$ inconnues $(\gamma(0),\dots,\gamma(p))$ que l'on peut
r\'esoudre. Pour d\'eterminer les valeurs suivantes on utilise
l'expression (\ref{eq:recurrencegamma2}).
\subsubsection{Densit\'e spectrale d'un processus ARMA$(p,q)$}
\begin{theorem}[Densit\'e spectrale d'un processus ARMA$(p,q)$]
Soit $(X_t)$ un processus ARMA$(p,q)$ (pas n\'ecessairement causal ou
inversible) \textit{i.e.} la solution stationnaire de l'\'equation
(\ref{eq:recurrenceARMApq}) o\`u les polyn\^omes $\theta(z)$ et $\phi(z)$ sont des
polyn\^omes de degr\'e $q$ et $p$ n'ayant pas de z\'eros communs. Alors
$(X_t)$ poss\`ede une densit\'e spectrale qui a pour expression~:
\begin{equation}
 \label{eq:dspARMApq}
 f(\lambda)=\frac{\sigma^2}{2\pi}
    \frac{\left| 1+\sum_{k=1}^q \theta_k \rme^{-\rmi k \lambda}\right|^2}
         {\left| 1-\sum_{k=1}^p \phi_k \rme^{-\rmi k \lambda}\right|^2}\;,\; -\pi\leq\lambda\leq\pi\;.
\end{equation}
\end{theorem}

\begin{remark}
D'apr\`es le th\'eor\`eme \ref{theo:ARMApq}, l'expression de $f$ est bien
d\'efinie puisque $\phi$ ne s'annule pas sur le cercle unit\'e.
\end{remark}





%%% Local Variables:
%%% mode: latex
%%% ispell-local-dictionary: "francais"
%%% TeX-master: "../monographie-serietemporelle"
%%% End:

\chapter{Pr\'ediction des signaux al\'eatoires \`a temps-discret}
% \section{th\'eor\`eme de Projection}
%     \subsection{Espace de Hilbert}
%     \subsection{Bases orthonormales}
%     \subsection{th\'eor\`eme de projection}
% \section{Algorithmes de Levinson-Durbin}
%================================================
%================================================
\section{Pr\'ediction lin\'eaire de processus stationnaires}
Soit $(X_t)_{t\in\Zset}$ un processus stationnaire au
second ordre \`a valeurs r\'eelles, \textbf{d'esp\'erance nulle} et de fonction
d'autocovariance $\gamma(h)= \cov(X_h,X_0)$. On cherche \`a
\emph{pr\'edire} la valeur du processus \`a la date $t$ \`a partir
d'une combinaison lin\'eaire des $p$ derniers \'echantillons du pass\'e
$X_{t-1}, \dots, X_{t-p}$. La meilleure combinaison lin\'eaire
(\textit{i.e.} le pr\'edicteur lin\'eaire optimal)
est la projection orthogonale de $X_t$ sur
$\cH_{t-1,p}$  not\'ee $\proj{X_t}{\cH_{t-1,p}}$, o\`u $\cH_{t-1,p}$ est
d\'efini par :
\begin{equation}
\label{eq:cht}
\cH_{t-1,p} = \lspan{X_{t-1}, X_{t-2}, \cdots, X_{t-p}}\;.
\end{equation}
Les indices dans la notation $\cH_{t-1,p}$ doivent \^{e}tre compris ainsi
: $\cH_{t-1,p}$ est le sous-espace vectoriel engendr\'e par les
$p$ observations pr\'ec\'edant $X_{t-1}$ \`a savoir
$X_{t-1}, \dots, X_{t-p}$.
D'apr\`es le th\'eor\`eme \ref{theo:projection},
\begin{equation}\label{eq:def_phi_kp}
\proj{X_t}{\cH_{t-1,p}}=\sum_{k=1}^p \phi_{k,p} X_{t-k}\;,
\end{equation}
o\`u les coefficients $(\phi_{k,p})_{1\leq k\leq p}$ satisfont
\begin{equation}\label{eq:scal_1}
\left\langle X_t-\sum_{k=1}^p \phi_{k,p} X_{t-k},X_{t-j}\right\rangle=0\;,\;
j=1,\dots,p\;,
\end{equation}
la notation $\left\langle \cdot,\cdot\right\rangle$ correspondant au
produit scalaire dans $\ltwo\espaceproba$ d\'efini pour $X$ et $Y$
dans $\ltwo\espaceproba$ par  $\left\langle X,Y\right\rangle=\PE{XY}$.
L'\'equation (\ref{eq:scal_1}) se r\'e\'ecrit encore sous la forme
\begin{equation}\label{eq:scal_2}
\left\langle X_t,X_{t-j}\right\rangle=\sum_{k=1}^p \phi_{k,p}
\left\langle  X_{t-k},X_{t-j}\right\rangle\;,\;
j=1,\dots,p\;,
\end{equation}
soit encore
\begin{equation}\label{eq:scal_3}
\sum_{k=1}^p \phi_{k,p}\gamma(k-j)=\gamma(j)\;,\;
j=1,\dots,p\;.
\end{equation}
En posant $\Gamma_p$ la matrice de covariance
du vecteur $(X_{t-1},\dots, X_{t-p})$ d\'efinie par
\begin{equation*}
\Gamma_p =
 \left[
 \begin{matrix}
  \gamma(0)  & \gamma(1)  &  \cdots    &            & \gamma(p-1) \\
  \gamma(1)  & \gamma(0)  &  \gamma(1) &            & \vdots \\
  \vdots     &\ddots      &  \ddots    &  \ddots    &     \\
  \vdots     &            &            &            &\gamma(1)  \\
  \gamma(p-1)& \gamma(p-2)&  \cdots    & \gamma(1)  & \gamma(0)
 \end{matrix}
 \right]\;,
\end{equation*}
on peut r\'e\'ecrire (\ref{eq:scal_3}) comme suit :
\begin{equation}\label{eq:YW1}
 \Gamma_p \bfphi_p = \bfgamma_p\;,
\end{equation}
o\`u $\bfphi_p=(\phi_{1,p},\dots,\phi_{p,p})^T$ et
$\bfgamma_p=(\gamma(1), \gamma(2), \cdots, \gamma(p))^T$.

\begin{definition}
\index{Innovation!partielle}
Nous appellerons dans la suite
\emph{erreur de pr\'ediction directe} d'ordre $p$ ou
\emph{innovation partielle} d'ordre $p$ le processus\,:
\begin{equation}
\label{eq:deferreurforward}
 \epsilon_{t,p}^+
 = X_t - \proj{X_t}{\cH_{t-1,p}}
 = X_t - \sum_{k=1}^{p} \phi_{k,p} X_{t-k}\;.
\end{equation}
La variance de l'erreur de pr\'ediction directe d'ordre $p$ est not\'ee
$\sigma_p^2$ et d\'efinie par
\begin{equation}\label{eq:var_pred_dir}
\sigma_p^2=\|X_t - \proj{X_t}{\cH_{t-1,p}}\|^2=\PE{|X_t -
  \proj{X_t}{\cH_{t-1,p}}|^2}\;.
\end{equation}
\end{definition}
D'apr\`es (\ref{eq:def_phi_kp}) et la proposition \ref{prop:projecteur},
la variance de l'erreur de pr\'ediction directe d'ordre $p$ a pour expression :
\begin{equation}
 \label{eq:YW2}
 \sigma_p^2 = \pscal{X_t}{X_t - \proj{X_t}{\cH_{t-1,p}}}
            =\gamma(0)- \sum_{k=1}^p \phi_{k,p}\gamma(k)
            =\gamma(0)-\bfphi_p^T\bfgamma_p\;.
\end{equation}
Les \'equations \eqref{eq:YW1} et \eqref{eq:YW2} sont appel\'ees les
\emph{\'equations de Yule-Walker}.
% Notons la propri\'et\'e importante
% suivante~: pour $p$ fix\'e, la suite des coefficients
% $\{\phi_{k,p}\}_{1\leq k \leq p}$ du pr\'edicteur lin\'eaire optimal
% et la variance de l'erreur minimale de pr\'ediction {\em ne
% d\'ependent pas de $t$}.

Notons que (\ref{eq:YW1}) a une unique solution si et seulement si
la matrice $\Gamma_p$ est inversible auquel cas la solution vaut :
\begin{equation}\label{eq:sol_unique}
\bfphi_p=\Gamma_p^{-1} \bfgamma_p\; .
\end{equation}
La proposition \ref{prop:Gammanrangplein} fournit les conditions suffisantes assurant
que $\Gamma_p$ est inversible pour tout $p$.  On a ainsi des conditions
sous lesquelles on peut calculer le pr\'edicteur de $X_t$ \`a partir de
$X_{t-1},\dots,X_{t-p}$.



% Ce probl\`eme est bien entendu
% un cas particulier du probl\`eme pr\'ec\'edent o\`u nous avons $X=
% X_t$ et $Y_k = X_{t-k}$, pour $k \in \{1, \dots, p \}$ et
% o\`u\,:
% \begin{equation}
% \label{eq:cht}
%   \cH_{t-1,p} = \lspan{1, X_{t-1}, X_{t-2}, \cdots, X_{t-p}}
% \end{equation}
% Formons la matrice de covariance $\Gamma_p$ du vecteur $[X_{t-1},
% \cdots, X_{t-p}]$:
% \begin{equation}
% \Gamma_p =
%  \left[
%  \begin{matrix}
%   \gamma(0)  & \gamma(1)  &  \cdots    &            & \gamma(p-1) \\
%   \gamma(1)  & \gamma(0)  &  \gamma(1) &            & \vdots \\
%   \vdots     &\ddots      &  \ddots    &  \ddots    &     \\
%   \vdots     &            &            &            &\gamma(1)  \\
%   \gamma(p-1)& \gamma(p-2)&  \cdots    & \gamma(1)  & \gamma(0)
%  \end{matrix}
%  \right]
% \end{equation}
% Cette matrice est dite de Toeplitz, ses \'el\'ements \'etant \'egaux
% le long de ses diagonales. Notons $\bfgamma_p$ le vecteur
% $[\gamma(1), \gamma(2), \cdots, \gamma(p)]^T$ le vecteur des
% coefficients de corr\'elation. D'apr\`es l'\'equation
% \eqref{eq:Eqsnormales}, les coefficients $\{\phi_{k,p}\}_{1\leq
% k\leq p}$ du pr\'edicteur lin\'eaire optimal d\'efini par\,:
% \begin{equation}
%  \label{eq:formegenedeXtsurH}
%  \proj{X_t}{\cH_{t-1,p}} - \mu=\sum_{k=1}^p \phi_{k,p} (X_{t-k}-\mu)
% \end{equation}
% sont solutions du syst\`eme d'\'equations\,:
% \begin{equation}
%  \label{eq:YW1}
%  \Gamma_p \bfphi_p = \bfgamma_p \hspace{1cm}
% \end{equation}
% D'autre part l'erreur de pr\'ediction minimale a pour expression\,:
% \begin{eqnarray}
%  \label{eq:YW2}
%  &\sigma_p^2 &= \| X_t - \proj{X_t}{\cH_{t-1,p}}\|^2
%             = \pscal{X_t-\mu}{X_t - \proj{X_t}{\cH_{t-1,p}}} \nonumber \\
%             &&=\gamma(0)- \sum_{k=1}^p \phi_{k,p}\gamma(k)
%             =\gamma(0)-\bfphi_p^T\bfgamma_p
% \end{eqnarray}
% Les \'equations \eqref{eq:YW1} et \eqref{eq:YW2} sont appel\'ees
% \emph{\'equations de Yule-Walker}. Notons la propri\'et\'e importante
% suivante~: pour $p$ fix\'e, la suite des coefficients
% $\{\phi_{k,p}\}_{1\leq k \leq p}$ du pr\'edicteur lin\'eaire optimal
% et la variance de l'erreur minimale de pr\'ediction {\em ne
% d\'ependent pas de $t$}.

% Les \'equations \eqref{eq:YW1} et
% \eqref{eq:YW2} peuvent encore \^{e}tre r\'e\'ecrites \`a partir des
% coefficients de corr\'elation $\rho(h)=\gamma(h)/\gamma(0)$. Il
% vient\,:
% \begin{equation}
%  \label{eq:YWcorrelation}
%  \left[
%  \begin{matrix}
%   \rho(0)  & \rho(1)  &  \cdots    &            & \rho(p-1) \\
%   \rho(1)  & \rho(0)  &  \rho(1) &            & \vdots \\
%   \vdots     &\ddots      &  \ddots    &  \ddots    &     \\
%   \vdots     &            &            &            &\rho(1)  \\
%   \rho(p-1)& \rho(p-2)&  \cdots    & \rho(1)  & \rho(0)
%  \end{matrix}
%  \right]
%  \left[
%  \begin{matrix}
%   \phi_{1,p}\\
%   \phi_{2,p}\\
%   \vdots\\
%   \vdots \\
%   \phi_{p,p}
%  \end{matrix}
%  \right]
%  =
%   \left[
%  \begin{matrix}
%   \rho(1)\\
%   \rho(2)\\
%   \vdots\\
%   \vdots \\
%   \rho(p)
%  \end{matrix}
%  \right]
% \end{equation}
%========================================================
\begin{example}[Cas d'un processus AR$(m)$ causal]
Soit $(X_t)$ le processus AR$(m)$ causal solution de
l'\'equation r\'ecurrente\,:
\begin{equation}\label{eq:def_ar}
  X_t=\phi_1X_{t-1}+\cdots+\phi_m X_{t-m}+Z_t\;,
\end{equation}
o\`u $Z_t\sim \BB(0,\sigma^2)$ et o\`u
$\phi(z)=1-\sum_{k=1}^m\phi_k z^{k}\neq 0$ lorsque $|z|\leq 1$.
Dans ce cas, pour tout $p\geq m$ :
$$
 \phi_{k,p}=
 \begin{cases}
    \phi_k,&\text{lorsque }  1\leq k\leq m\;,\\
    0,& \text{lorsque }  m< k\leq p \;.
 \end{cases}
$$
En effet, $(X_t)$ \'etant causal on a, pour tout $h\geq 1$,
$\PE{Z_tX_{t-h}}=0$ et donc, d'apr\`es (\ref{eq:def_ar}),
$\PE{(X_t-\sum_{k=1}^m\phi_kX_{t-k})X_{t-h}}=0$.
Ainsi, pour tout $p\geq m$, $\sum_{k=1}^m\phi_kX_{t-k} \in
\cH_{t-1,p}$ et $(X_t-\sum_{k=1}^m\phi_kX_{t-k})\perp \cH_{t-1,p}$
et donc, d'apr\`es le th\'eor\`eme \ref{theo:projection}, pour tout $p\geq m$,
$$
\sum_{k=1}^m\phi_kX_{t-k}=\proj{X_t}{\cH_{t-1,p}}\;.
$$
% La projection orthogonale d'un AR$(m)$ causal sur son pass\'e de longueur $p\geq m$ co\"{i}ncide avec la projection
% orthogonale sur les $m$ derni\`eres valeurs et les coefficients
% de pr\'ediction sont pr\'ecis\'ement les coefficients de l'\'equation
% r\'ecurrente.
\end{example}
% Dans le cas o\`u la matrice de covariance $\Gamma_p$, suppos\'ee
% \emph{connue}, est inversible, le probl\`eme de la
% d\'etermination des coefficients de pr\'ediction $\bfphi_p$ et de la
% variance de l'erreur de pr\'ediction $\sigma_p^2$ a une solution
% unique. Rappelons que, d'apr\`es la propri\'et\'e
% \ref{prop:Gammanrangplein}, si $\gamma(0)>0$ et si $\limn
% \gamma(n)=0$, alors la matrice $\Gamma_p$ est inversible \`a
% tout ordre.

% Il est facile de d\'emontrer que\,:
% \begin{multline}
%  \label{eq:onpeutcenter}
%  \proj{X_t}{\lspan{1,X_{t-1},\dots,X_{t-p}}}
% \\ =\mu+
%  \proj{X_t-\mu}{\lspan{X_{t-1}-\mu,\dots,X_{t-p}-\mu}} \eqsp.
% \end{multline}
% Par cons\'equent, dans le probl\`eme de la pr\'ediction,
% il n'y a aucune perte de g\'en\'eralit\'e \`a consid\'erer que le
% processus est centr\'e. S'il ne l'\'etait pas, il suffirait,
% d'apr\`es l'\'equation \eqref{eq:onpeutcenter}, d'effectuer le
% calcul des pr\'edicteurs sur le processus centr\'e $X_t^c=X_t-\mu$
% puis d'ajouter $\mu$. \emph{Dans la suite, sauf indication
% contraire, les processus sont suppos\'es centr\'es}.

Les
coefficients de pr\'ediction d'un processus stationnaire au second
ordre fournissent une d\'ecomposition particuli\`ere de la matrice
de covariance $\Gamma_{p+1}$ sous la forme d'un produit de matrices
triangulaires explicit\'ee dans le th\'eor\`eme \ref{theo:choleski}.
\begin{theorem}
 \label{theo:choleski} Soit $(X_t)$ un processus stationnaire au second
ordre, centr\'e, de fonction d'autocovariance $\gamma(h)$. On
note\,:
\[
A_{p+1} = \left[
\begin{matrix}
   1            & 0             & \cdots & \cdots      & 0 \\
   - \phi_{1,1} & 1             & \ddots &             & \vdots \\
   \vdots       &               & \ddots & \ddots      & \vdots \\
   \vdots       &               &        & \ddots      & 0 \\
   - \phi_{p,p} & - \phi_{p-1,p}& \cdots &- \phi_{1,p} &1
\end{matrix}
\right] \text{ et } D_{p+1} = \left[
\begin{matrix}
\sigma^2_0 & 0          & \cdots & 0 \\
0          & \sigma_1^2 & \cdots & 0 \\
\vdots     &            &        & \vdots \\
0          &            & \cdots & \sigma_p^2
\end{matrix}
\right]\;,
\]
o\`u les coefficients $(\phi_{k,p})_{1\leq k\leq p}$ et
$(\sigma_k^2)_{1\leq k\leq p}$ sont respectivement d\'efinis dans
(\ref{eq:def_phi_kp}) et (\ref{eq:var_pred_dir}).
On a alors\,:
\begin{equation}
 \label{eq:decompocholeski}
 \Gamma_{p+1} = A_{p+1}^{-1} D_{p+1} (A_{p+1}^T)^{-1}\;.
\end{equation}
\end{theorem}
\begin{proof}\smartqed
Pour simplifier les notations, posons $\cH_k =\cH_{k,k}= \lspan{X_k, \cdots, X_1}$ et montrons tout
d'abord que, pour $k \neq \ell$, nous avons\,:
\begin{equation}
 \label{eq:erreurblanche}
 \pscal{X_k - \proj{X_k}{\cH_{k-1}}}{X_{\ell} - \proj{X_{\ell}}{\cH_{{\ell}-1}}} = 0 \eqsp.
\end{equation}
En effet, pour $k < \ell$, on a $X_{k} - \proj{X_{k}}{\cH_{k-1}} \in
\cH_k\subseteq\cH_{\ell-1}$ et $X_{\ell} -
\proj{X_{\ell}}{\cH_{\ell-1}} \perp \cH_{\ell-1}$.
D'autre part, si on note ${\bf X}_{p+1}$ le vecteur :
$(X_1,\dots,X_{p+1})^T$, alors,
par d\'efinition des coefficients de pr\'ediction (\ref{eq:def_phi_kp}), on peut \'ecrire :
\[
A_{p+1} {\bf X}_{p+1} = \left[
\begin{array}{llll}
1            & 0             & \cdots & 0 \\
- \phi_{1,1} & 1             & \cdots & 0 \\
\vdots       &               &        & \vdots \\
- \phi_{p,p} & - \phi_{p-1,p}& \cdots & 1
\end{array}
\right] \left[
\begin{array}{l}
X_{1} \\
X_{2} \\
\vdots \\
X_{p+1}
\end{array}
\right] = \left [
\begin{array}{l}
X_1 \\
X_2 - \proj{X_2}{\cH_1} \\
\vdots \\
X_{p+1} - \proj{X_{p+1}}{\cH_{p}}
\end{array}
\right ]\;,
\]
qui donne\,:
$$
 \PE{A_{p+1} {\bf X}_{p+1} {\bf X}_{p+1}^T A_{p+1}^T}
 =D_{p+1}\;,
$$
d'apr\`es \eqref{eq:erreurblanche} et (\ref{eq:var_pred_dir}).
Par ailleurs,
$$
\PE{A_{p+1} {\bf X}_{p+1} {\bf X}_{p+1}^T A_{p+1}^T}
= A_{p+1} \Gamma_{p+1}A_{p+1}^T\;,
$$
ce qui d\'emontre \eqref{eq:decompocholeski} puisque la
matrice $A_{p+1}$ est inversible, son d\'eterminant \'etant \'egal \`a
$1$.

\end{proof}
% Dans la suite nous notons
% $\cH_{t-1,p}=\lspan{X_{t-1},\dots,X_{t-p}}$ et nous appelons
% \emph{erreur de pr\'ediction directe} d'ordre $p$ ou
% \emph{innovation partielle} d'ordre $p$ le processus\,:
% \begin{equation}
% \label{eq:deferreurforward}
%  \epsilon_{t,p}^+
%  = X_t - \proj{X_t}{\cH_{t-1,p}}
%  = X_t - \sum_{k=1}^{p} \phi_{k,p} X_{t-k}
% \end{equation}
D'apr\`es l'\'equation \eqref{eq:decompocholeski} lorsque la
matrice $\Gamma_{p+1}$ est inversible, la variance
$\sigma_p^2=\|\epsilon_{t,p}^+\|^2$ est strictement positive.
D'autre part, la suite $\sigma_p^2$ est
d\'ecroissante. En effet, par d\'efinition de $\cH_{t-1,p}$,
$\cH_{t-1,p}$ est inclus dans $\cH_{t-1,p+1}$ donc
$\proj{X_{p+1}}{\cH_{t-1,p}}$ est dans $\cH_{t-1,p+1}$.
On d\'eduit donc du th\'eor\`eme \ref{theo:projection} que $\sigma_{p+1}^2\leq\sigma_{p}^2$.
La suite $(\sigma_p^2)$ \'etant d\'ecroissante et minor\'ee, elle poss\`ede
donc une limite quand $p$ tend vers l'infini. Cela conduit \`a la d\'efinition suivante,
dont nous verrons au paragraphe \ref{s:wold} qu'elle joue un r\^ole
fondamental dans la d\'ecomposition des processus stationnaires au
second ordre.

\begin{definition}[Processus r\'egulier/d\'eterministe]
 \label{def:paregulier}
 Soit $(X_t)_{t \in \Zset}$ un processus al\'eatoire stationnaire au second
ordre. On note $\sigma^2=\lim_{p\to\infty}\sigma_p^2$ o\`u
$\sigma_p^2$ est la variance de l'innovation partielle
d'ordre $p$. On dit que le processus $(X_t)$ est \emph{r\'egulier} si
$\sigma^2 > 0$ et \emph{d\'eterministe} si $\sigma^2=0$.
\end{definition}


Par ailleurs, nous pouvons remarquer que le probl\`eme de la recherche des coefficients de pr\'ediction pour
un processus stationnaire au second ordre se ram\`ene \`a
celui de la minimisation de l'int\'egrale\,:
\[
%\inf_{\psi \in \cP_p}
   \frac{1}{2\pi}
             \int_{-\pi}^{\pi} |\psi(\rme^{-\rmi\lambda})|^2 \nu_X(\rmd\lambda)
%           = \frac{1}{2\pi}
%             \int_{-\pi}^{\pi} |\phi_p(e^{\rmi x})|^2 \mu_X(dx)
%           =\sigma_p^2
\]
sur l'ensemble $\cP_p$ des polyn\^omes \`a coefficients r\'eels
de degr\'e $p$ de la forme $\psi(z) = 1 + \psi_1 z + \cdots + \psi_p
z^p$. En effet, en utilisant la relation \eqref{eq:dspfiltrage} de
filtrage des mesures spectrales, on peut \'ecrire que la variance de
$ \| \epsilon_{t,p}^+ \|^2$, qui minimise l'erreur de
pr\'ediction, a pour expression\,:
\begin{equation}
\label{eq:variance:innovation:partielle}
% \| \epsilon_{t,p}^+ \|^2
 \sigma_p^2
 = \frac{1}{2 \pi} \int_{-\pi}^{\pi} |
    \phi_p(\rme^{-\rmi\lambda})|^2 \nu_X(\rmd\lambda)
\end{equation}
o\`u\,:
\[
  \phi_p(z) = 1 - \sum_{k=1}^p \phi_{k,p} z^{-k}
\]
d\'esigne le \emph{polyn\^ome pr\'edicteur d'ordre $p$}.
\begin{theorem}
 \label{theo:procregulpredicstable}
 Si $\{ X_t \}$ est un processus r\'egulier, alors, pour
tout $p$, $\phi_p(z) \ne 0$ pour $|z|\leq 1$. Tous les z\'eros des
polyn\^omes pr\'edicteurs sont \`a l'ext\'erieur du cercle unit\'e.
\end{theorem}

\begin{proof}[Preuve du th\'eor\`eme~\ref{theo:procregulpredicstable}]
\smartqed
Nous allons tout d'abord montrer que le pr\'edicteur optimal n'a pas
de racines sur le cercle unit\'e. Raisonnons par contradiction.
Supposons que le polyn\^ome $\phi_p(z)$ ait deux racines
complexes conjugu\'ees, de la forme $\exp(\pm i\theta)$, sur le
cercle unit\'e (on traite de fa\c{c}on similaire le cas de racines
r\'eelles, $\theta= 0$ ou $\pi$). Nous pouvons \'ecrire\,:
\[
 \phi_p(z) = \phi^*_p(z) (1 - 2 \cos(\theta) z + z^2)
\]
On note $\bar{\nu}_X(d \lambda) = \nu_X(d \lambda) |\phi^*_p(\rme^{-i
\lambda})|^2$. $\bar{\nu}_X$ est une mesure positive sur
$[-\pi,\pi]$ de masse finie. On note:
\[
 \bar{\gamma}(\tau) = \frac{1}{2 \pi} \int_{- \pi}^{\pi}
 \rme^{\rmi \tau \lambda} \bar{\nu}_X(\rmd \lambda)\;.
\]
Nous avons donc\,:
\begin{multline*}
 \sigma_p^2 = \frac{1}{2 \pi} \int_{-\pi}^\pi
                  (1 - 2 \cos(\theta) \rme^{-i \lambda} + \rme^{-2 i \lambda})
                  \bar{\nu}_X(\rmd \lambda)\\
           = \inf_{\psi \in \cP_2} \frac{1}{2 \pi}
           \int_{-\pi}^\pi
           |1 + \psi_1 \rme^{-\rmi\lambda} + \psi_2 \rme^{-2 \rmi \lambda}|^2
             \bar{\nu}_X(\rmd \lambda) \eqsp.
\end{multline*}
La minimisation de $\sigma_p^2$ est équivalente à
\`a la r\'esolution des \'equations de Yule-Walker \`a
l'ordre $p=2$ pour la suite des covariances $\bar{\gamma}(h)$.
Par cons\'equent la suite des coefficients $\{1,-2\cos(\theta),1\}$
doit v\'erifier l'\'equation\,:
\[
\left[
\begin{array}{ccc}
\bar{\gamma}(0) & \bar{\gamma}(1) & \bar{\gamma}(2) \\
\bar{\gamma}(1) & \bar{\gamma}(0) & \bar{\gamma}(1) \\
\bar{\gamma}(2) & \bar{\gamma}(1) & \bar{\gamma}(0) \\
\end{array}
\right] \left[
\begin{array}{c}
1 \\
-2 \cos(\theta) \\
1
\end{array}
\right] = \left[
\begin{array}{c}
\sigma_p^2 \\
0 \\
0
\end{array}
\right]
\]
De cette \'equation il s'en suit (les premi\`ere et troisi\`eme
lignes sont \'egales) que $\sigma_p^2=0$, ce qui est contraire \`a
l'hypoth\`ese que le processus est r\'egulier.

D\'emontrons maintenant que les racines des polyn\^omes
pr\'edicteurs sont toutes \emph{strictement \`a  l'int\'erieur du
cercle unit\'e}. Raisonnons encore par l'absurde. Supposons que le
polyn\^ome pr\'edicteur \`a l'ordre $p$ ait $m$ racines $\{a_k,
|a_k| < 1, 1 \le k \leq m \}$ \`a l'int\'erieur du cercle unit\'e et
$(p-m)$ racines $\{ b_{\ell}, |b_{\ell}| > 1, 1 \leq \ell \leq p-m
\}$ \`a l'ext\'erieur du cercle unit\'e. Le polyn\^ome pr\'edicteur
\`a l'ordre $p$ s'\'ecrit donc\,:
\[
 \phi_p(z)=
 \prod_{k=1}^m ( 1 - a_k z^{-1}) \prod_{\ell=1}^{p-m} (1 - b_{\ell} z^{-1})\;.
\]
Consid\'erons alors le polyn\^ome\,:
\[
 \bar{\phi}_p(z) =
 \prod_{k=1}^m (1 - a_k z^{-1}) \prod_{\ell=1}^{p-m} (1 - (1/b_{\ell}^{*}) z^{-1})\;.
\]
Il a d'une part toutes ses racines strictement \`a l'int\'erieur
du cercle unit\'e et d'autre part il v\'erifie
$|\bar{\phi}_p(\rme^{-\rmi\lambda })|^2< |\phi_p(\rme^{-\rmi\lambda })|^2$. On a
en effet $| 1 - (1/b_\ell^*) \rme^{-\rmi\lambda } | = |b_\ell| |1 - b_\ell \rme^{-\rmi \lambda }|$ et donc
$|\bar{\phi}_p(\rme^{-\rmi \lambda })|^2 = \left (\prod_{\ell=1}^{p-m} |b_\ell|^{-2} \right )|\phi_p(\rme^{-\rmi\lambda })|^2$, ce qui d\'emontre le r\'esultat
annonc\'e puisque $|b_\ell|<1$. On en d\'eduit alors
que\,:
\begin{gather*}
\frac{1}{2 \pi} \int_{-\pi}^{\pi} | \bar{\phi}_p(\rme^{-i\lambda
})|^2 \nu_X(\rmd\lambda )   < \sigma_p^2\;,
\end{gather*}
ce qui contredit que $\phi_p(z)=\inf_{\psi \in \cP_p}(2\pi)^{-1}
\int_{-\pi}^\pi |\psi(\rme^{-i\lambda })|^2 \nu_X(\rmd\lambda )$.

\end{proof}
Une cons\'equence directe du th\'eor\`eme
\ref{theo:procregulpredicstable} est qu'\`a toute matrice de
covariance de type d\'efini positif, de dimension $(p+1)\times
(p+1)$, on peut associer un processus AR$(p)$ causal dont les
$(p+1)$ premiers coefficients de covariance sont pr\'ecis\'ement la
premi\`ere ligne de cette matrice. Ce r\'esultat n'est pas g\'en\'eral.
Ainsi il existe bien un processus AR$(2)$ causal ayant
$\gamma(0)=1$ et $\gamma(1)=\rho$, comme premiers coefficients de
covariance, \`a condition toutefois que la matrice de covariance
soit positive c'est-\`a-dire que $|\rho|<1$, tandis qu'il n'existe
pas, pour cette m\^{e}me matrice de processus MA$(1)$. Il faut en
effet, en plus du caract\`ere positif, que $|\rho|\leq 1/2$
(voir exemple \ref{exe:testposivite1}).
%============================================================================
%============================================================================
%============================================================================
\section{Algorithme de Levinson-Durbin}
\label{sec:algorithme-levinson-durbin}
%============================================================================
La solution directe du syst\`eme des \'equations de Yule-Walker
requiert de l'ordre de $p^3$ op\'erations~: la r\'esolution classique
de ce syst\`eme implique en effet la d\'ecomposition de la matrice
$\Gamma_p$ sous la forme du produit d'une matrice triangulaire
inf\'erieure et de sa transpos\'ee, $\Gamma_p = L_p L_p^T$
(d\'ecomposition de Choleski) et la r\'esolution par substitution de
deux syst\`emes triangulaires. Cette proc\'edure peut s'av\'erer
co\^uteuse lorsque l'ordre de pr\'ediction est grand (on utilise
g\'en\'eralement des ordres de pr\'ediction de l'ordre de quelques
dizaines \`a quelques centaines), ou lorsque, \`a des fins de
mod\'elisation, on est amen\'e \`a \'evaluer la qualit\'e de pr\'ediction
pour diff\'erents horizons de pr\'ediction. L'algorithme de
Levinson-Durbin exploite la structure g\'eom\'etrique particuli\`ere
des processus stationnaires au second ordre pour \'etablir une
formule de r\'ecurrence donnant les coefficients de pr\'ediction \`a
l'ordre $(p+1)$ \`a partir des coefficients de pr\'ediction
obtenus \`a l'ordre $p$. Il fournit \'egalement une relation de r\'ecurrence
entre l'erreur de pr\'ediction directe \`a l'ordre $p+1$
et l'erreur de pr\'ediction directe \`a l'ordre $p$.

On supposera dans toute cette partie que {\boldmath$\Gamma_p$} \textbf{est inversible
pour tout} {\boldmath $p\geq 1$}.

Supposons que les
coefficients de pr\'ediction lin\'eaire et la variance de l'erreur de
pr\'ediction directe \`a l'ordre $p$, pour $p \geq 0$, sont connus :
\begin{gather*}
%\label{eq:definition:predictionretrograde}
   \proj{X_t}{\cH_{t-1,p}} =
   \sum_{k=1}^{p} \phi_{k,p} X_{t-k}
   \quad\mbox{et}\quad
   \sigma_{p}^2 = \| X_t - \proj{X_t}{\cH_{t-1,p}} \|^2\;,
\end{gather*}
et d\'eterminons, \`a partir de la
projection \`a l'ordre $p$ de $X_t$, la projection de $X_t$ \`a
l'ordre $p+1$ sur le sous-espace $\cH_{t-1,p+1} =
\lspan{X_{t-1}, \cdots, X_{t-p-1}}$.

Pour cela, on d\'ecompose cet espace
en somme orthogonale de la fa\c{c}on suivante\,:
\begin{multline*}
\cH_{t-1,p+1} = \cH_{t-1,p}  \oplusperp \lspan{ X_{t-p-1} -\proj{X_{t-p-1}}{\cH_{t-1,p}}}
\\= \cH_{t-1,p} \oplusperp \lspan{\epsilon_{t-p-1,p}^-}\;,
\end{multline*}
o\`u, de fa\c{c}on g\'en\'erale, $\epsilon_{t,p}^-$ correspond \`a
l'\emph{erreur de pr\'ediction
r\'etrograde \`a l'ordre $p$} d\'efinie par :
\[
\epsilon_{t,p}^-
     = X_t - \proj{X_t}{\cH_{t+p,p}}
     = X_t - \proj{X_t}{\lspan{X_{t+p}, \cdots, X_{t+1}}}\;.
\]
Elle repr\'esente la diff\'erence entre la valeur \`a l'instant courant $X_t$ et
la projection orthogonale de $X_t$ sur les $p$ \'echantillons
\emph{qui suivent} l'instant courant $\{X_{t+1}, \cdots, X_{t+p}\}$.
Le qualificatif \emph{r\'etrograde} est clair\,: il
traduit le fait que l'on cherche \`a pr\'edire la valeur courante
en fonction des valeurs futures.

D'apr\`es la proposition \ref{prop:projecteur},
\begin{multline*}
%\label{eq:decomposition}
  \proj{X_t}{\cH_{t-1,p+1}}
  = \proj{X_t}{\cH_{t-1,p}} + \proj{X_t}{\lspan{\epsilon_{t-p-1,p}^-}}\;,
\end{multline*}
o\`u d'apr\`es l'exemple \ref{exe:proj1vecteur} :
$$
\proj{X_t}{\lspan{\epsilon_{t-p-1,p}^-}} = \alpha \epsilon_{t-p-1,p}^-
 \quad \mbox{avec} \quad
 \alpha=\pscal{X_t}{\epsilon_{t-p-1,p}^-}/\|\epsilon_{t-p-1,p}^-\|^2\;.
$$
On en d\'eduit donc que :
\begin{multline}
\label{eq:decomposition}
  \proj{X_t}{\cH_{t-1,p+1}}
  = \proj{X_t}{\cH_{t-1,p}} \\
+ k_{p+1} \left[X_{t-p-1} - \proj{X_{t-p-1}}{\cH_{t-1,p}} \right] \eqsp,
\end{multline}
o\`u
\begin{equation}\label{eq:k_{p+1}}
k_{p+1}=\frac{\pscal{X_t}{\epsilon_{t-p-1,p}^-}}{\|\epsilon_{t-p-1,p}^-\|^2}\;.
\end{equation}
Montrons \`a pr\'esent que les coefficients de pr\'ediction r\'etrograde
co\"{i}ncident avec les coefficients de pr\'ediction directe.
Plus pr\'ecis\'ement, si
\begin{equation}
\label{eq:devtdirectretro1}
 \proj{X_t}{\cH_{t-1,p}} = \sum_{k=1}^{p} \phi_{k,p} X_{t-k}\;,
\end{equation}
alors
\begin{equation}
\label{eq:devtdirectretro2}
 \proj{X_{t-p-1}}{\cH_{t-1,p}} =\sum_{k=1}^{p} \phi_{k,p} X_{t-p-1+k} =\sum_{k=1}^{p} \phi_{p+1-k,p} X_{t-k}\;.
\end{equation}
En effet, les coefficients des deux d\'eveloppements
(\ref{eq:devtdirectretro1}) et (\ref{eq:devtdirectretro2}) sont
tous les deux donn\'es par (\ref{eq:sol_unique}).
En utilisant \eqref{eq:devtdirectretro1} et
\eqref{eq:devtdirectretro2} dans
\eqref{eq:decomposition}, on a :
\begin{multline*}
\proj{X_{t}}{\cH_{t-1,p+1}}
  = \sum_{k=1}^{p+1} \phi_{k,p+1}X_{t-k}\\
  = \sum_{k=1}^{p} (\phi_{k,p} - k_{p+1} \phi_{p+1-k,p} ) X_{t-k} + k_{p+1} X_{t-p-1}\;.
\end{multline*}
On en d\'eduit, par unicit\'e, les formules de r\'ecurrence donnant les coefficients
de pr\'ediction \`a l'ordre $p+1$ \`a partir de ceux \`a
l'ordre $p$\,:
\begin{equation}
 \label{eq:recursionLevinson}
\begin{cases}
\phi_{k,p+1} = \phi_{k,p} - k_{p+1} \phi_{p+1-k,p}\;, & \quad\mbox{pour}\quad k \in \{1, \cdots, p \}\;, \\
\phi_{p+1,p+1} = k_{p+1}\;. &
\end{cases}
\end{equation}
Explicitons \`a pr\'esent la relation (\ref{eq:k_{p+1}}) d\'efinissant
$k_{p+1}$. En utilisant \eqref{eq:devtdirectretro2}, on a :
\begin{multline*}
\pscal{X_t}{\epsilon_{t-p-1,p}^-}
            = \pscal{X_{t}}{X_{t-p-1} - \proj{X_{t-p-1}}{\cH_{t-1,p}}}\\
            = \gamma(p+1) - \pscal{X_t}{\sum_{k=1}^{p} \phi_{k,p} X_{t-p-1+k}}
            = \gamma(p+1) - \sum_{k=1}^{p} \phi_{k,p} \gamma(p+1-k)\;.
\end{multline*}
D'autre part,
\begin{multline}\label{eq:norme2_epsretro}
\|\epsilon_{t-p-1,p}^-\|^2=\pscal{X_{t-p-1}}{X_{t-p-1}-\sum_{k=1}^{p}
  \phi_{k,p} X_{t-p-1+k}}\\
=\gamma(0)-\sum_{k=1}^{p} \phi_{k,p}\gamma(k)
=\sigma_p^2=\|\epsilon_{t,p}^+\|^2\;,
\end{multline}
ce qui donne
\begin{equation}\label{eq:def:k_{p+1}}
  k_{p+1}=
   \frac{\gamma(p+1) - \sum_{k=1}^{p} \phi_{k,p} \gamma(p+1-k)}
   {\sigma_p^2}\;.
 \end{equation}
 Il nous reste maintenant \`a d\'eterminer l'erreur de pr\'ediction
${\sigma_{p+1}^2}$ \`a l'ordre $(p+1)$ en fonction de $\sigma_p^2$.
En utilisant l'\'equation \eqref{eq:decomposition}, on a
\begin{multline*}
 \epsilon_{t,p+1}^+ =X_{t} - \proj{X_{t}}{\cH_{t-1,p+1}} \\
 = X_{t} - \proj{X_{t}}{\cH_{t-1,p}} - k_{p+1}
[X_{t-p-1} - \proj{X_{t-p-1}}{\cH_{t-1,p}}]\\
=X_{t} - \proj{X_{t}}{\cH_{t-1,p}} - k_{p+1}\epsilon_{t-p-1,p}^-\;,
\end{multline*}
dont on d\'eduit d'apr\`es \eqref{eq:norme2_epsretro}\,:
\begin{multline*}
  \sigma_{p+1}^2=\|\epsilon_{t,p+1}^+\|^2
=\sigma_p^2+ k_{p+1}^2 \sigma_p^2-2k_{p+1}\pscal{X_{t} - \proj{X_{t}}{\cH_{t-1,p}}}{\epsilon_{t-p-1,p}^-}\;.
      % = \sigma_p^2 + k_{p+1}^2 \sigma_p^2
      % - 2 k_{p+1} \pscal{X_{t} - \proj{X_{t}}{\cH_{t-1,p}}}{X_{t-p-1} - \proj{X_{t-p-1}}{\cH_{t-1,p}}} \\
      %= \sigma_p^2 (1 - k_{p+1}^2)\;,
\end{multline*}
En utilisant que $\proj{X_{t}}{\cH_{t-1,p}}$ et $\epsilon_{t-p-1,p}^-$
sont orthogonaux, \eqref{eq:k_{p+1}} et \eqref{eq:norme2_epsretro}, on
obtient
\begin{equation}\label{eq:recursion_sigma_p+1}
 \sigma_{p+1}^2=\sigma_p^2 (1 - k_{p+1}^2)\;.
\end{equation}
A partir de ces r\'ecursions, nous allons \`a pr\'esent d\'ecrire l'algorithme
de Levinson-Durbin.



% Indiquons que l'erreur r\'etrograde
% joue un r\^ole absolument essentiel dans tous les algorithmes
% rapides de r\'esolution des \'equations de Yule-Walker.

% Remarquons tout
% d'abord que les coefficients de pr\'ediction r\'etrograde
% co\"{i}ncident avec les coefficients de pr\'ediction directe :
% Cette
% propri\'et\'e, que nous avons rencontr\'ee exemple \ref{exe:predAVAR},
% est fondamentalement due \`a la {\em propri\'et\'e de r\'eversibilit\'e}
% des processus stationnaires au second ordre. En effet, si $Y_t=
% X_{-t}$, alors $Y_t$ a m\^{e}me moyenne et m\^{e}me fonction de
% covariance que $X_t$ (voir exemple \ref{exe:stat_retourne}
% chapitre \ref{chap:passl}) et par cons\'equent, en utilisant aussi
% l'hypoth\`ese de stationnarit\'e, on a simultan\'ement pour tout
% $u,v\in\Zset$\,:
% \begin{multline*}
%   \proj{X_{t+u}}{\cH_{t+u-1,p}}
%          = \sum_{k=1}^{p} \phi_{k,p} X_{t+u-k}
%   \quad \text{et} \\
%   \proj{X_{t+v}}{\cH_{t+v+p,p}} = \sum_{k=1}^p \phi_{k,p} X_{t+v+k}
% \end{multline*}
% ainsi que\,:
% \begin{gather}
%  \label{eq:eplusemoins}
%  \sigma_p^2
%  =
%  \|\epsilon_{t+u,p}^+\|^2  %%=\| X_{t+u} - (X_{t+u} | \cH_{t+u-1,p}) \|^2
%  =
%  \|\epsilon_{t+v,p}^-\|^2 %%= \| X_{t+v-p-1,p} - (X_{t+v-p-1,p} | \cH_{t+v-1,p}) \|^2
% \end{gather}
% En particulier on a\,:
% \begin{equation}
%  \label{eq:devtdirectretro}
%  \begin{cases}
%  \proj{X_t}{\cH_{t-1,p}} = \sum_{k=1}^{p} \phi_{k,p} X_{t-k}\;, \\
%  \proj{X_{t-p-1}}{\cH_{t-1,p}} =\sum_{k=1}^{p} \phi_{k,p} X_{t-p-1+k} =\sum_{k=1}^{p} \phi_{p+1-k,p} X_{t-k}\;.
%  \end{cases}
% \end{equation}



% Cherchons maintenant \`a d\'eterminer, \`a partir de la
% projection \`a l'ordre $p$, la projection de $X_t$ \`a
% l'ordre $p+1$ sur le sous-espace $\cH_{t-1,p+1} =
% \lspan{X_{t-1}, \cdots, X_{t-p-1}}$. Pour cela d\'ecomposons cet espace
% en somme orthogonale de la fa\c{c}on suivante\,:
% \begin{multline*}
% \cH_{t-1,p+1} = \cH_{t-1,p}  \oplusperp \lspan{ X_{t-p-1} -\proj{X_{t-p-1}}{\cH_{t-1,p}}}
% \\= \cH_{t-1,p} \oplusperp \lspan{\epsilon_{t-p-1,p}^-}
% \end{multline*}
% Un calcul simple montre (voir exemple \ref{exe:proj1vecteur}) que
% $$
% \proj{X_t}{\epsilon_{t-p-1,p}^-} = \alpha \epsilon_{t-p-1,p}^-
%  \quad \mbox{avec} \quad
%  \alpha=(X_t,\epsilon_{t-p-1,p}^-)/\|\epsilon_{t-p-1,p}^-\|^2
% $$
% et donc que
% \begin{equation}
% \label{eq:decomposition}
%   \proj{X_t}{\cH_{t-1,p+1}}
%   = \proj{X_t}{\cH_{t-1,p}} + k_{p+1} \left[X_{t-p-1} - \proj{X_{t-p-1}}{\cH_{t-1,p}} \right] \eqsp,
% \end{equation}
% o\`u, en utilisant aussi \eqref{eq:eplusemoins}, on peut
% \'ecrire\,:
% \begin{equation}
% \label{eq:defkpP1}
%   k_{p+1}=\frac{\pscal{X_t}{\epsilon_{t-p-1,p}^-}}{\sigma_p^2}=
%    \frac{\pscal{X_t}{\epsilon_{t-p-1,p}^-}}{\|\epsilon_{t+u,p}^+\| \|\epsilon_{t+v,p}^-\|} \eqsp.
% \end{equation}
% En portant \`a pr\'esent \eqref{eq:devtdirectretro} dans
% \eqref{eq:decomposition}, on obtient l'expression\,:
% $$
% \proj{X_{t}}{\cH_{t-1,p+1}}
%   = \sum_{k=1}^{p+1} \phi_{k,p+1}X_{t-k}
%   = \sum_{k=1}^{p} (\phi_{k,p} - k_{p+1} \phi_{p+1-k,p} ) X_{t-k} + k_{p+1} X_{t-p-1}
% $$
% On en d\'eduit les formules de r\'ecurrence donnant les coefficients
% de pr\'ediction \`a l'ordre $p+1$ \`a partir de ceux \`a
% l'ordre $p$\,:
% \begin{equation}
%  \label{eq:recursionLevinson}
% \begin{cases}
% \phi_{k,p+1} = \phi_{k,p} - k_{p+1} \phi_{p+1-k,p} & \quad\mbox{pour}\quad k \in \{1, \cdots, p \} \\
% \phi_{p+1,p+1} = k_{p+1} &
% \end{cases}
% \end{equation}
% D\'eterminons maintenant la formule de r\'ecurrence donnant $k_{p+1}$.
% En utilisant encore \eqref{eq:devtdirectretro} et
% \eqref{eq:decomposition}, on obtient\,:
% $$
% \pscal{X_t}{\proj{X_{t-p-1}}{\cH_{t-1,p}}}
%      =\sum_{k=1}^{p} \phi_{k,p} \PE{X_t X_{t-p-1+k}}
%      =\sum_{k=1}^{p} \phi_{k,p} \gamma(p+1-k)
% $$
% Partant de l'expression de $\pscal{X_t}{\epsilon_{t-p-1,p}^-}$ on en
% d\'eduit que\,:
% \begin{multline*}
% \pscal{X_t}{\epsilon_{t-p-1,p}^-}
%             = \pscal{X_{t}}{X_{t-p-1} - \proj{X_{t-p-1}}{\cH_{t-1,p}}}\\
%             = \gamma(p+1) - \sum_{k=1}^{p} \phi_{k,p} \gamma(p+1-k)
% \end{multline*}
% et donc d'apr\`es \eqref{eq:defkpP1}\,:
% $$
%   k_{p+1}=
%    \frac{\gamma(p+1) - \sum_{k=1}^{p} \phi_{k,p} \gamma(p+1-k)}
%    {\sigma_p^2}
% $$
% Il nous reste maintenant \`a d\'eterminer l'erreur de pr\'ediction
% ${\sigma_{p+1}^2}$ \`a l'ordre $(p+1)$. En utilisant l'\'equation
% \eqref{eq:decomposition}, on a
% \begin{multline*}
%  \epsilon_{t,p+1}^+ =X_{t} - \proj{X_{t}}{\cH_{t-1,p+1}} \\
%  = X_{t} - \proj{X_{t}}{\cH_{t-1,p}} - k_{p+1}
% (X_{t-p-1} - \proj{X_{t-p-1}}{\cH_{t-1,p}})
% \end{multline*}
% dont on d\'eduit d'apr\`es \eqref{eq:defkpP1}\,:
% \begin{multline*}
%   \sigma_{p+1}^2=\|\epsilon_{t,p+1}^+\|^2
%       = \sigma_p^2 + k_{p+1}^2 \sigma_p^2
%       - 2 k_{p+1} \pscal{X_{t} - \proj{X_{t}}{\cH_{t-1,p}}}{X_{t-p-1} - \proj{X_{t-p-1}}{\cH_{t-1,p}}} \\
%       = \sigma_p^2 (1 - k_{p+1}^2)
% \end{multline*}

Pour initialiser l'algorithme, nous nous int\'eressons au cas $p=0$.
Dans ce cas, la meilleure pr\'ediction de $X_t$ est $\PE{X_t}=0$ et la variance de
l'erreur de pr\'ediction est donn\'ee par $\sigma_{0}^2
=\PE{(X_t-0)^2}=\gamma(0)$. Au pas suivant on a
$k_1=\gamma(1)/\gamma(0)$, en posant $p=0$ dans (\ref{eq:def:k_{p+1}}),
$\phi_{1,1}=\gamma(1)/\gamma(0)$, en posant $p=0$ dans (\ref{eq:recursionLevinson}) et
$\sigma_1^2=\gamma(0)(1-k_1^2)$, en posant  $p=0$ dans \eqref{eq:recursion_sigma_p+1}.

L'algorithme de \emph{Levinson-Durbin} qui permet de d\'eterminer les coefficients de pr\'ediction
$\{\phi_{m,p}\}_{1\leq m\leq p,1\leq p\leq K}$ \`a partir de
$\gamma(0),\dots,\gamma(K)$ s'\'ecrit alors de la fa\c{c}on suivante :


% Partant d'une suite de $(K+1)$ coefficients de covariance $\gamma(0),\dots,\gamma(K)$, l'\emph{algorithme de Levinson-Durbin}
% permet de d\'eterminer les coefficients de pr\'ediction
% $\{\phi_{m,p}\}_{1\leq m\leq p,1\leq p\leq K}$\,:

\begin{algorithm}[Levinson-Durbin]
\item[Initialisation] $k_1=\gamma(1)/\gamma(0)$, $\phi_{1,1}=\gamma(1)/\gamma(0)$ et $\sigma_1^2=\gamma(0)(1-k_1^2)$
\item[R\'ecursion] Pour $p=\{2,\dots,K\}$ r\'ep\'eter\,:
\begin{enumerate}[label=(\alph*)]
\item Calculer
\begin{align*}
&k_p=\sigma_{p-1}^{-2} \left( \gamma(p) - \sum_{k=1}^{p-1} \phi_{k,p-1} \gamma(p-k) \right) \\
&\phi_{p,p}=k_p \\
&\sigma_{p}^2=\sigma_{p-1}^2(1-k_p^2)
\end{align*}
\item Pour $m\in \{1,\cdots,p-1\}$ calculer\,:
\[
\phi_{m,p}=\phi_{m,p-1}-k_p\phi_{p-m,p-1}
\]
\end{enumerate}
\end{algorithm}

\begin{proposition}
Soit $(X_t)$ un processus stationnaire au second ordre de fonction d'autocovariance
$\gamma(h)$.
%telle que $\gamma(0)>0$ et $\gamma(h)\to 0$ lorsque $h$
%tend vers l'infini.
Le coefficient $k_{p+1}$ d\'efini par
(\ref{eq:k_{p+1}}) v\'erifie, pour tout $p\geq 0$ :
\begin{equation}
 \label{eq:defkp}
 k_{p+1}=\frac{\pscal{\epsilon_{t,p}^+}{\epsilon_{t-p-1,p}^-}}
        {\|\epsilon_{t,p}^+ \| \,\, \|\epsilon_{t-p-1,p}^- \|}\;,
\end{equation}
et
\begin{equation}
\label{eq:maj:k_p+1}
|k_{p+1}|\leq 1\;.
\end{equation}
\end{proposition}
\begin{proof}\smartqed
En utilisant que $\proj{X_{t}}{\cH_{t-1,p}}$ est orthogonal \`a
$\epsilon_{t-p-1,p}^-$, (\ref{eq:k_{p+1}}) et (\ref{eq:deferreurforward}), on a
$$
k_{p+1}=\frac{\pscal{\epsilon_{t,p}^+}{\epsilon_{t-p-1,p}^-}}{\|\epsilon_{t-p-1,p}^- \|^2}\;.
$$
Or, d'apr\`es (\ref{eq:norme2_epsretro}), $\|\epsilon_{t-p-1,p}^-
\|^2=\sigma_p^2=\|\epsilon_{t,p}^+ \|^2$, la derni\`ere \'egalit\'e venant
de (\ref{eq:var_pred_dir}), d'o\`u l'on d\'eduit \eqref{eq:defkp}.
L'in\'egalit\'e \eqref{eq:maj:k_p+1} se d\'eduit alors de \eqref{eq:defkp}
en utilisant l'in\'egalit\'e de Cauchy-Schwarz.

\end{proof}

%\begin{proof}\smartqed
%  Notons tout d'abord que
% $\proj{X_{t}}{\cH_{t-1,p}}\perp \epsilon_{t-p-1,p}^-$ puisque
% $\proj{X_{t}}{\cH_{t-1,p}}\in\cH_{t-1,p}$ et que
% $\epsilon_{t-p-1,p}^-\perp \cH_{t-1,p}$. Partant de
% \eqref{eq:defkpP1} on peut \'ecrire que\,:
% \begin{equation}
%  \label{eq:defkp}
%  k_{p+1}
%   =\frac{\pscal{X_{t} - \proj{X_{t}}{\cH_{t-1,p}}}{X_{t-p-1} - \proj{X_{t-p-1}}{\cH_{t-1,p}}}}
%     {\|\epsilon_{t,p}^+ \| \,\, \|\epsilon_{t-p-1,p}^- \|}
%   =\frac{\pscal{\epsilon_{t,p}^+}{\epsilon_{t-p-1,p}^-}}
%         {\|\epsilon_{t,p}^+ \| \,\, \|\epsilon_{t-p-1,p}^- \|}
% \end{equation}
% En utilisant l'in\'egalit\'e de Schwarz, on montre que $|k_{p+1}| \leq
% 1$.
%
%\end{proof}
%Dans la litt\'erature, $k_p$ est appel\'e coefficient d'autocorr\'elation partielle.
\begin{definition}[Fonction d'autocorr\'elation partielle]
 \label{def:corrpar}
 Soit $(X_t)$ un processus stationnaire au second ordre de fonction d'autocovariance
$\gamma(h)$. On appelle \emph{fonction d'autocorr\'elation partielle} la
suite des coefficients d'autocorr\'elation partielle  $(k_p)_{p \geq 1}$
d\'efinie par :
\begin{equation}
 \label{eq:defcorrpart}
 k_p = \corr(X_t,X_{t-1})= \frac{\pscal{X_t}{X_{t-1}}}{\|X_t\| \,\,
   \|X_{t-1}\|},\textrm{ si } p=1\;,
\end{equation}
et
\begin{multline}
k_p=\corr(\epsilon_{t,p-1}^+,\epsilon_{t-p,p-1}^-)\\
           = \dfrac{\pscal{X_{t}-\proj{X_t}{\cH_{t-1,p-1}}}{X_{t-p}-\proj{X_{t-p}}{\cH_{t-1,p-1}}}}
            {\|X_{t}-\proj{X_t}{\cH_{t-1,p-1}}\| \,\,
              \|X_{t-p}-\proj{X_{t-p}}{\cH_{t-1,p-1}}\|},
\textrm{ si } p\geq 2\;.
\end{multline}
\end{definition}

\begin{remark}
Dans \eqref{eq:defcorrpart}, l'expression pour $p=1$ est en accord
avec celle pour $p\geq 2$ dans la mesure o\`u on peut noter que
$\epsilon_{t,0}^+=X_t$ et que $\epsilon_{t-1,0}^-=X_{t-1}$. Notons
aussi que, dans l'expression de $k_p$, $X_t$ et $X_{t-p}$ sont
projet\'es sur le m\^{e}me sous-espace
$\lspan{X_{t-1},\dots,X_{t-p+1}}$. Le r\'esultat remarquable est
que la suite des coefficients de corr\'elation partielle est donn\'ee
par\,:
\begin{equation}
 \label{eq:parcoretcoeffpredic}
  k_p=\phi_{p,p}
\end{equation} o\`u $\phi_{p,p}$ est d\'efini au moyen des \'equations de
Yule-Walker~(\ref{eq:YW1}).
\end{remark}
Dans le cas particulier d'un
processus AR$(m)$ causal, on a alors\,:
$$
 k_p=\left\{
   \begin{matrix}
     \phi_{p,p}&\mbox{pour}& 1\leq p < m\;,\\
     \phi_m&\mbox{pour}& p = m\;,\\
     0&\mbox{pour}& p > m\;.
   \end{matrix}
   \right.
$$
%==================================================================
%==================================================================
%==================================================================
% \section{Algorithme de Schur}
% %==================================================================
% %==================================================================


% Partant des coefficients d'autocorr\'elation, l'algorithme de
% Levinson-Durbin \'evalue \`a la fois les coefficients des
% pr\'edicteurs lin\'eaires optimaux et les coefficients
% d'autocorr\'elation partielle. Dans certains cas, seuls les
% coefficients d'autocorr\'elation partielle sont n\'ecessaires. Il en
% est ainsi, par exemple, lorsque l'on cherche \`a calculer les
% erreurs de pr\'ediction directe et r\'etrograde \`a partir du
% processus $X_t$. Montrons, en effet, que les erreurs de pr\'ediction
% \`a l'ordre $(p+1)$ s'expriment, en fonction des erreurs de
% pr\'edictions \`a l'ordre $p$, \`a l'aide d'une formule de
% r\'ecurrence ne faisant intervenir que la valeur du coefficient de
% corr\'elation partielle\,:
% \begin{equation}
%  \label{eq:celluleanalyse}
% \begin{cases}
% \epsilon_{t,p+1}^+= \epsilon_{t,p}^+ -k_{p+1}\epsilon_{(t-1)-p,p}^- \\
% \epsilon_{t-(p+1),p+1}^-=\epsilon_{(t-1)-p,p}^- -k_{p+1}\epsilon^+_{t,p}
% \end{cases}
% \end{equation}
% Reprenons les expressions de l'erreur de pr\'ediction directe et de
% l'erreur de pr\'ediction r\'etrograde\,:
% \begin{eqnarray*}
%  \epsilon_{t,p}^+ = X_t - \sum_{k=1}^p \phi_{k,p} X_{t-k}
%  &\mbox{et}&
%  \epsilon_{t-p-1,p}^- = X_{t-p-1} - \sum_{k=1}^p \phi_{k,p} X_{t-p-1+k}
% \end{eqnarray*}
% En utilisant directement la r\'ecursion de Levinson-Durbin,
% \'equations \eqref{eq:recursionLevinson}, dans l'expression de
% l'erreur de pr\'ediction directe \`a l'ordre $p+1$, nous
% obtenons\,:
% \begin{align}
% \label{eq:predictiondirecte}
%  \epsilon_{t,p+1}^+
%   &=  X_t - \sum_{k=1}^{p+1} \phi_{k,p+1} X_{t-k}
%   \nonumber
%   \\&
%   = \left( X_t - \sum_{k=1}^p \phi_{k,p} X_{t-k} \right)
%      - k_{p+1}
%      \left( X_{t-p-1} - \sum_{k=1}^p \phi_{k,p} X_{t-p-1+k} \right)
%   \nonumber
%   \\
%   &= \epsilon_{t,p}^+ - k_{p+1} \epsilon_{t-p-1,p}^-
% \end{align}
% De fa\c{c}on similaire, nous avons\,:
% \begin{align}
% \label{eq:predictionretrograde}
%  \epsilon_{t-p-1,p+1}^-
%   &=  X_{t-p-1} - \sum_{k=1}^{p+1} \phi_{k,p+1} X_{t-p-1+k}
%   \nonumber
%   \\&
%   = \left( X_{t-p-1} - \sum_{k=1}^p \phi_{k,p} X_{t-p-1+k} \right)
%      - k_{p+1}
%      \left( X_{t} - \sum_{k=1}^p \phi_{k,p} X_{t-k} \right)
%   \nonumber
%   \\
%    &= \epsilon_{t-p-1,p}^- - k_{p+1} \epsilon_{t,p}^+
% \end{align}
% Partant de la suite des autocorr\'elations, l'algorithme de Schur
% calcule r\'ecursivement les coefficients de corr\'elation partielle,
% sans avoir \`a d\'eterminer les valeurs des coefficients de
% pr\'ediction. Historiquement, l'algorithme de Schur a \'et\'e introduit
% pour tester le caract\`ere d\'efini positif d'une suite (ou de
% fa\c{c}on \'equivalente, la positivit\'e des matrices de Toeplitz
% construites \`a partir de cette suite). En effet, comme nous
% l'avons montr\'e ci-dessus, une suite de coefficients de covariance
% est d\'efinie positive si et seulement si les coefficients de
% corr\'elation partielle sont de module strictement inf\'erieur \`a
% $1$. D\'eterminons \`a pr\'esent cet algorithme. En faisant $t=0$
% dans l'\'equation \eqref{eq:predictiondirecte}, en multipliant \`a
% gauche par $X_m$ et en utilisant la stationnarit\'e, il vient\,:
% \begin{equation}
%  \label{eq:kpplus}
%  \pscal{X_m}{\epsilon_{0,p+1}^+}
%  =
%  \pscal{X_m}{\epsilon_{0,p}^+} - k_{p+1} \pscal{X_{m}}{\epsilon_{-p-1,p}^-}
%  =
%  \pscal{X_m}{\epsilon_{0,p}^+} - k_{p+1} \pscal{X_{m+p+1}}{\epsilon_{0,p}^-} \eqsp.
% \end{equation}
% En faisant $t=p+1$ dans l'\'equation
% \eqref{eq:predictionretrograde}, en multipliant \`a gauche par
% $X_{m+p+1}$ et en utilisant la stationnarit\'e, il vient\,:
% \begin{multline}
%  \label{eq:kpmoins}
%  \pscal{X_{m+p+1}}{\epsilon_{0,p+1}^-}
%  =
%  \pscal{X_{m+p+1}}{\epsilon_{0,p}^-} - k_{p+1}\pscal{X_{m+p+1}}{\epsilon_{p+1,p}^+}\\
%  =
%  \pscal{X_{m+p+1}}{\epsilon_{0,p}^-} - k_{p+1}\pscal{X_{m}}{\epsilon_{0,p}^+}  \eqsp.
% \end{multline}
% En faisant $m=0$  dans \eqref{eq:kpmoins}, il vient\,:
% \begin{equation}
%   \label{eq:kpmoinsen0}
%  \pscal{X_{p+1}}{\epsilon_{0,p+1}^-}
%  =
%  \pscal{X_{p+1}}{\epsilon_{0,p}^-} - k_{p+1} \pscal{X_{p+1}}{\epsilon_{p+1,p}^+}
%  = \pscal{X_{p+1}}{\epsilon_{0,p}^-} - k_{p+1} \pscal{X_0}{\epsilon_{0,p}^+} \eqsp.
% \end{equation}
% Mais on a aussi\,:
% \[
% \pscal{X_{p+1}}{\epsilon_{0,p+1}^-}
%  = \pscal{X_{p+1}}{ X_0 - \proj{X_0}{\lspan{X_1, \cdots, X_{p+1}}}} = 0 \eqsp.
% \]
% Nous pouvons donc d\'eduire de l'\'equation \eqref{eq:kpmoinsen0}\,:
% \begin{equation}
% \label{eq:kpshur}
%   k_{p+1} = \frac{\pscal{X_{p+1}}{ \epsilon_{0,p}^-}}{\pscal{X_0}{\epsilon_{0,p}^+}}
% \end{equation}
% En couplant les \'equations \eqref{eq:kpplus}, \eqref{eq:kpmoins} et
% \eqref{eq:kpshur} et en partant des conditions initiales\,:
% \[
%   \pscal{X_m}{\epsilon_{0,0}^+}= \gamma(m)
%   \quad\mbox{et}\quad
%   \pscal{X_{m+1}}{\epsilon_{0,0}^-}=\gamma(m+1) \eqsp.
% \]
% on peut d\'eterminer les coefficients de corr\'elation partielle
% directement, sans avoir \`a \'evaluer explicitement les
% coefficients de pr\'ediction.

% On note $u(m,p)=\pscal{X_m}{\epsilon_{0,p}^+}$ et $v(m,p)=\pscal{X_{m+p+1}}{\epsilon_{0,p}^-}$.
% Partant des $(K+1)$ coefficients de covariance
% $\{\gamma(0),\dots,\gamma(K)\}$, l'{\em algorithme de Schur}
% calcule les $K$ premiers coefficients de corr\'elation partielle\,:
% \begin{description}
% \item[Initialisation] Pour $m=\{0,\dots,K-1\}$\,:
% \begin{align*}
% &u(m,0)=\gamma(m) \\
% &v(m,0)=\gamma(m+1)
% \end{align*}
% \item[R\'ecursion]
% \begin{enumerate}[label=(\alph*)]
% \item Pour $p=\{1,\dots,K\}$, calculer
% \[k_p = \frac{v(0,p-1)}{u(0,p-1)} \]
% \item Pour $m=\{0,\dots,K-p-1\}$ calculer\,:
%   $$
%   \begin{cases}
%   u(m,p)=u(m,p-1)-k_pv(m,p-1) \\
%   v(m,p)=v(m+1,p-1)-k_pu(m+1,p-1)
%   \end{cases}\eqsp.
%   $$
% \end{enumerate}
% \end{description}
% La complexit\'e de l'algorithme de Schur est \'equivalente \`a
% l'algorithme de Levinson.
% %==========================================================
% \subsubsection{Filtres en treillis}
% %==========================================================
% En notant $e(t,p)=[\epsilon_{t,p}^+\quad \epsilon_{t-p,p}^-]^T$ et
% en utilisant l'op\'erateur de retard $B$, les expressions
% \eqref{eq:celluleanalyse} peuvent se mettre sous la forme
% matricielle\,:
% $$
%  e(t,p+1)=
%  \left [
%  \begin{matrix}
%    1&-k_{p+1}B \cr -k_{p+1}B&1
%  \end{matrix}
%  \right ]
%  e(t,p)
% $$
% Les erreurs initiales ($p=0$) sont $e(t,0)=[X_t\quad X_t]^T$. Ces
% \'equations d\'ebouchent sur une structure de filtrage dite en
% treillis qui calcule, au moyen des coefficients de corr\'elation
% partielle, les erreurs de pr\'ediction directe et r\'etrograde \`a
% partir du processus $\{X_t, t \in \Zset\}$. Ce filtre d'analyse est repr\'esent\'e figure
% \ref{fig:anatreillis}.
%  %================= FIGURE
%  %====== FIGURE
%  \figtit{\FIGPREDIC treillisanalyse}
%  {Filtre d'analyse en treillis. Ce filtre permet de construire les erreurs de
%  pr\'ediction directe et r\'etrograde \`a partir du processus et de la donn\'ee
%  des coefficients de corr\'elation partielle.}
%  {fig:anatreillis}
% Les \'equations \eqref{eq:celluleanalyse} peuvent encore s'\'ecrire\,:
% $$
% \begin{cases}
% \epsilon_{t,p}^+=\epsilon_{t,p+1}^+ +k_{p+1}\epsilon_{(t-1)-p,p}^- \\
% \epsilon_{t-(p+1),p+1}^-=\epsilon_{(t-1)-p,p}^-k_{p+1}\epsilon_{t,p}^+
% \end{cases}
% $$
% qui donne le sch\'ema de filtrage de la figure
% \ref{fig:syntreillis}.
%  %================= FIGURE
%  %====== FIGURE
%  \figtit{\FIGPREDIC treillissynthese}
%  {Filtre de synth\`ese en treillis. Ce filtre permet de reconstruire
%  le processus \`a partir de la suite des erreurs de
%  pr\'ediction directe et de la donn\'ee
%  des coefficients de corr\'elation partielle.}
%  {fig:syntreillis}
% %==========================================================
% %==========================================================
\section{Algorithme des innovations}
\label{sec:algorithmes-des-innovations}
L'algorithme des innovations est une application directe de la m\'ethode de Gram-Schmidt et est, \`a cet
\'egard, plus \'el\'ementaire que l'algorithme de Levinson-Durbin. De plus, il ne suppose pas que le processus
$(X_t)_{t \in \Zset}$ soit stationnaire. L'esp\'erance de $X_t$ \'etant
suppos\'ee nulle dans ce chapitre, nous notons
\[
\kappa(i,j)= \pscal{X_i}{X_j}= \PE{X_iX_j} \eqsp,
\]
la fonction d'autocovariance de ce processus.
% Nous supposerons dans tout ce paragraphe que
% la matrice $[\kappa(i,j)]_{i,j=1}^n$ est inversible pour tout $n \geq
% 1$.
Notons, pour $n \geq 1$,
$$\cH_n= \lspan{X_1,\dots,X_n} \textrm{ et }
\sigma_n^2= \| X_{n+1} - \proj{X_{n+1}}{\cH_n}\|^2\;.
$$
La proc\'edure d'orthogonalisation de Gram-Schmidt permet alors d'\'ecrire
pour tout $n \geq 1$ :
\[
\cH_n= \lspan{X_1, X_2 - \proj{X_2}{X_1}, \dots, X_n - \proj{X_n}{\cH_{n-1}}} \eqsp,
\]
o\`u on utilise la convention suivante : $\proj{X_1}{\cH_{0}}=0$.
On a alors :
\begin{equation}
\label{eq:definition-projecteur}
\proj{X_{n+1}}{\cH_n}= \sum_{j=1}^n \theta_{n,j} \left( X_{n+1-j} - \proj{X_{n+1-j}}{\cH_{n-j}}\right) \eqsp.
\end{equation}
L'algorithme des innovations d\'ecrit dans la proposition suivante
fournit une m\'ethode r\'ecursive permettant de calculer
$(\theta_{n,j})_{1\leq j\leq n}$ et $\sigma_n^2$ pour $n\geq 1$.

\begin{proposition}
Soit $(X_t)$ un processus \`a moyenne nulle tel que la matrice
$[\kappa(i,j)]_{1\leq i,j\leq n}$ soit inversible pour tout $n\geq 1$
alors
$$
\proj{X_{n+1}}{\cH_n}=
\begin{cases}
0\;, \textrm{ si } n=0\;,\\
 \sum_{j=1}^n \theta_{n,j} \left( X_{n+1-j} -
   \proj{X_{n+1-j}}{\cH_{n-j}}\right)\;, \textrm{ si } n\geq 1\;,
\end{cases}
$$
o\`u
$$
\begin{cases}
\sigma_0^2=\kappa(1,1)\;,\\
\theta_{n,n-k}= \sigma_{k}^{-2} \left[ \kappa(n+1,k+1) -
  \sum_{j=0}^{k-1} \theta_{k,k-j} \theta_{n,n-j} \sigma_{j}^2
\right]\;,\; 0\leq k\leq n-1\;,\\
\sigma_{n}^2= \kappa(n+1,n+1) - \sum_{j=0}^{n-1} \theta^2_{n,n-j}
\sigma_{j}^2\;,\; n\geq 1 \;.
\end{cases}
$$
\end{proposition}

\begin{proof}\smartqed
Remarquons tout d'abord que les vecteurs $(X_i -
\proj{X_i}{\cH_{i-1}})_{i \geq 1}$ sont orthogonaux. En effet,
pour $i < j$, $X_i - \proj{X_i}{\cH_{i-1}} \in \cH_{j-1}$ et $X_j -
\proj{X_j}{\cH_{j-1}} \perp \cH_{j-1}$.
On en d\'eduit, en faisant le produit scalaire de
\eqref{eq:definition-projecteur} par $X_{k+1}-\proj{X_{k+1}}{\cH_k}$
que, pour $0 \leq k < n$ :
\[
\pscal{\proj{X_{n+1}}{\cH_n}}{X_{k+1}-\proj{X_{k+1}}{\cH_k}}= \theta_{n,n-k} \sigma_{k}^2 \eqsp.
\]
Puisque $\pscal{X_{n+1}-\proj{X_{n+1}}{\cH_n}}{X_{k+1}-\proj{X_{k+1}}{\cH_k}}=0$, les coefficients $\theta_{n,n-k}$,
$k=0,\dots,n-1$ sont donn\'es par
\begin{equation}\label{eq:theta_n_n-k}
\theta_{n,n-k}= \sigma_{k}^{-2} \pscal{X_{n+1}}{X_{k+1}-\proj{X_{k+1}}{\cH_k}} \eqsp.
\end{equation}
En utilisant la repr\'esentation \eqref{eq:definition-projecteur},
\begin{multline*}
\proj{X_{k+1}}{\cH_k}=\sum_{j=1}^k \theta_{k,j} \left( X_{k+1-j} -
  \proj{X_{k+1-j}}{\cH_{k-j}}\right) \\
=\sum_{j=0}^{k-1} \theta_{k,k-j} \left( X_{j+1} -
  \proj{X_{j+1}}{\cH_{j}}\right) \eqsp,
\end{multline*}
d'o\`u l'on d\'eduit que
\[
\theta_{n,n-k}= \sigma_{k}^{-2} \left( \kappa(n+1,k+1) - \sum_{j=0}^{k-1} \theta_{k,k-j} \pscal{X_{n+1}}{X_{j+1}-\proj{X_{j+1}}{\cH_j}}\right) \eqsp.
\]
D'apr\`es \eqref{eq:theta_n_n-k},
 $\pscal{X_{n+1}}{X_{j+1}-\proj{X_{j+1}}{\cH_j}}= \sigma_{j}^{2} \theta_{n,n-j}$ pour $0 \leq j < n$, nous avons donc pour
$k \in \{1,\dots,n-1\}$,
\begin{equation}
\label{eq:mise-a-jour-theta}
\theta_{n,n-k}= \sigma_{k}^{-2} \left( \kappa(n+1,k+1) - \sum_{j=0}^{k-1} \theta_{k,k-j} \theta_{n,n-j} \sigma_{j}^2 \right) \eqsp.
\end{equation}
L'\'equation \eqref{eq:mise-a-jour-theta} est encore valable lorsque
$k=0$ en utilisant la convention que la somme sur $j$ dans le membre
de droite est nulle dans ce cas.
Par ailleurs, la proposition \ref{prop:projecteur} (Pythagore) implique que
\begin{multline}
\label{eq:mise-a-jour-sigma}
\sigma_{n}^2= \| X_{n+1} - \proj{X_{n+1}}{\cH_n} \|^2= \| X_{n+1} \|^2 - \| \proj{X_{n+1}}{\cH_n} \|^2 \\
= \kappa(n+1,n+1) - \sum_{k=0}^{n-1} \theta^2_{n,n-k} \sigma_{k}^2 \eqsp.
\end{multline}

\end{proof}

Alors que l'algorithme de Levinson-Durbin permet de d\'eterminer
les coefficients du d\'eveloppement de $\proj{X_{n+1}}{\cH_n}$ sur
$X_1,\dots,X_n$ donn\'es par $\proj{X_{n+1}}{\cH_n}= \sum_{j=1}^n \phi_{n,j} X_{n+1-j}$,
l'algorithme des innovations calcule les coefficients du d\'eveloppement
de
$\proj{X_{n+1}}{\cH_n}$ sur $X_1$, $X_2 -
\proj{X_2}{X_1}$,
$\dots$, $X_{n} - \proj{X_n}{\cH_{n-1}}$.



\begin{example}[Pr\'ediction d'un processus MA(1)]
Consid\'erons le processus $X_t = Z_t + \theta Z_{t-1}$ o\`u $(Z_t) \sim \BB(0,\sigma^2)$. Nous avons donc $\kappa(i,j)= 0$ pour $|i-j| > 1$,
$\kappa(i,i)= \sigma^2(1+\theta^2)$ et $\kappa(i,i+1)= \theta
\sigma^2$. Dans ce cas, nous avons

$$
\begin{cases}
 \theta_{n,j}= 0\;,\; 2 \leq j \leq n \;, \\
 \theta_{n,1}= \sigma_{n-1}^{-2} \theta \sigma^2 \;,\\
\end{cases}
$$
et les variances des innovations qui sont donn\'ees par
$$
\begin{cases}
 \sigma_0^{2} = (1+\theta^2) \sigma^2 \eqsp, \\
\sigma_{n}^{2}= [1 + \theta^2 - \sigma_{n-1}^{-2} \theta^2 \sigma^2] \sigma^2 \eqsp.\\
\end{cases}
$$
Si nous posons $r_n = \sigma_n^2/\sigma^2$, nous avons
\[
\proj{X_{n+1}}{\cH_n}= \theta \left(X_n - \proj{X_n}{\cH_{n-1}}\right)/r_{n-1} \eqsp,
\]
avec $r_0=1+\theta^2$, et pour $n \geq 1$, $r_{n+1}= 1+\theta^2-\theta^2/r_n$.
\end{example}

%======================================================================
%======================================================================
%======================================================================



%%% Local Variables:
%%% mode: latex
%%% ispell-local-dictionary: "francais"
%%% TeX-master: "../monographie-serietemporelle"
%%% End:

%\chapter{Filtrage de Wiener}
\label{wiener-chap}

Les donn\'ees $Y[n]$ que l'on cherche
s'obtiennent souvent par une estimation bas\'ee sur 
des mesures bruit\'ees $X[n]$
corr\'el\'ees avec $Y[n]$.
La r\'eduction de bruit et la 
pr\'ediction sont des examples d'estimation.
Nous introduisons le 
filtrage de Wiener qui 
calcule la meilleure estimation lin\'eaire
d'un processus sationnaire 
$Y[n]$ \'etant donn\'ees $N$ mesures pass\'ees 
$\{X[n-k]\}_{1 \leq k \leq N}$, en minimisant l'esp\'erance
quadratique de l'erreur.
Le filtrage de Wiener permet entre autres de faire des calculs de
pr\'ediction lin\'eaire, de r\'eduire les bruits additifs, et
d'optimiser la mod\'elisation autor\'egressive
de processus stationnaires.


\section{Filtrage de Wiener}
\label{wiener}

Un filtrage d'ordre $N$ calcule une estimation
d'un processus $Y[n]$ en fonction 
de $N+1$ mesures $X[n]$, ...., $X[n-N]$, o\`u $X$ est un processus
corr\'el\'e avec $Y$.
On sait \cite{neveu} que l'estimateur optimal $\tilde Y[n]$ 
de $Y[n]$ sachant $\{X[n], ...., X[n-N]\}$, qui minimise
l'esp\'erance quadratique de l'erreur
\begin{equation}
\label{erreur}
E \{ (Y[n] - \tilde Y[n])^2 \} 
\end{equation}
est l'esp\'erance conditionnelle
\begin{equation}
\label{condit}
\tilde Y [n] = E \{Y[n]~\BS~ X[n], X[n-1],..., X[n-N]\} .
\end{equation}
Dans le cas particulier o\`u
les variables $Y[n]$ et $\{ X[n-k]\}_\ZkN$ sont
conjointement gaussiennes, cette esp\'erance conditionnelle
est une fonction lin\'eaire des donn\'ees $\{ X[n-k]\}_\ZkN$.
Cependant, dans les cas non gaussiens,
l'esp\'erance conditionnelle (\ref{condit}) est souvent
une fonction
non-lin\'eaire compliqu\'ee de $\{X[k]\}_\ZkN$,
qu'il est tr\`es difficile de calculer. 
Le filtrage de Wiener
simplifie le probl\`eme en calculant la combinaison lin\'eaire
$\tilde Y[n]$ 
des variables al\'eatoires $\{X[k]\}_\ZkN$ 
qui minimise l'erreur quadratique (\ref{erreur}).

\subsection{R\'egression lin\'eaire optimale}

Le probl\`eme d'estimation lin\'eaire optimale peut \^etre r\'esolu
de fa\c{c}on g\'eom\'etrique en se 
pla\c{c}ant dans l'espace $\LDP$ 
des variables al\'eatoires
d\'efinies sur $(\Omega , \cal A , \rm P)$, dont l'esp\'erance
quadratique est finie. On rappelle que le
produit scalaire de deux variables al\'eatoires est
\begin{equation}
<A,B>_\LDP = E\{A B\} .
\end{equation}
\\
\noindent{\bf Projection orthogonale}
On suppose que $E\{Y[k]^2\} < +\infty$ 
et que $E\{X[k]^2\} < +\infty$ pour tout $k \in \Z$.
L'espace $\cX_N$ des variables al\'eatoires 
qui sont des combinaisons
lin\'eaires des $\{ X[n-k] \}_\ZkN$ est un sous espace de $\LDP$.
L'estimateur lin\'eaire
\begin{equation}
\label{est}
\tilde Y[n] = \sum_{l=0}^{N} h[n,n-l] X[n-l] .
\end{equation}
qui minimise $\| Y[n] - \tilde Y[n] \|_\LDP$ est
donc la projection orthogonale de $Y[n]$ sur $\cX_N$
que l'on note
\[
\tilde Y[n] = \E \{Y[n] \BS \cX_N \}
\]
L'erreur de pr\'ediction $W [n] = Y[n] - \tilde Y[n]$
est orthogonale \`a $\cX_N$ et donc \`a chacune des donn\'ees $X[n-k]$
pour $0 \leq k \leq N$
\begin{equation}
\label{orth}
<W [n] , X[n-k] > _\LDP = 
E\{Y[n] X[n-k]\} - 
\sum_{l=0}^{N} h[n,n-l] E \{ X[n-l] X[n-k] \} = 0 .
\end{equation}
Pour simplifier l'explication, nous supposons pour l'instant
que $X$ et $Y$ sont des processus de moyenne nulle.
Les \'equations (\ref{orth}) d\'efinissent
un syst\`eme lin\'eaire sur les covariances
$R_{YX} [n,m] = E\{Y[n] X[m]\}$ et
$R_{X} [n,m] = E\{X[n] X[m]\}$ 
\begin{equation}
\label{system}
\sum_{l=0}^{N} h[n,n-l] R_X [n-l,n-k] = R_{YX} [n,n-k] .
\end{equation}
\\
\noindent{\bf Equations de Wiener-Hopf}
Si l'on suppose que les processus $X$ et $Y$ sont conjointement
stationnaires au sens large, alors 
\[
R_{YX} [n,m]= R_{YX} [n-m] ~~~\mbox{et}~~~
R_{X} [n,m]= R_{X} [n-m] .
\]
En ins\'erant ces \'equations dans (\ref{system}), 
on s'aper\c{c}oit que les constantes
$h[n,n-l]$ satisfont un syst\`eme d'\'equations ind\'ependant
de $n$ et peuvent donc s'\'ecrire $h[n,n-l] = h[l]$.
On obtient alors le syst\`eme d'\'equations de Wiener-Hopf
\begin{equation}
\sum_{l=0}^{N} h[l] R_X [k-l] = R_{YX} [k] .
\end{equation}
Les coefficients de r\'egression $h[l]$ se calculent en
r\'esolvant ce syst\`eme lin\'eaire.
En rempla\c{c}ant de m\^eme $h[n,n-l]$ par $h[l]$ dans (\ref{est}), 
l'estimateur lin\'eaire optimal s'\'ecrit comme une
convolution de $X[n]$ avec le filtre causal de r\'eponse 
impulsionnel $h[l]$
\begin{equation}
\label{filtre-wiener}
\tilde Y[n] = \sum_{l=0}^{N} h[l] X[n-l] = h \star X [n] .
\end{equation}

Il est souvent utile d'exprimer le syst\`eme de
Wiener-Hopf sous forme matricielle 
\begin{equation}
\label{wiener-hopf}
{ \bR_X} \bh = \bR_{YX} ,
\end{equation}
en utilisant la matrice sym\'etrique de Toeplitz
\[
{{ \bR_X}} = 
\left( \begin{array}{ccccc}
R_X[0] &R_X[1] & ...&R_X[N-1] &R_X[N] \\
R_X[-1]&R_X[0]& ...&R_X[N-2]&R_X[N-1]\\
. &. &...&. &. \\
.&.&...&.&.\\
.&.&...&.&.\\
R_X[-N-1]&R_X[-N+2]&...&R_X[0]&R_X[1]\\
R_X[-N]&R_X[-N+1]&...&R_X[-1]&R_X[0]\\
\end{array}
\right)
\]
et les vecteurs
\[
{{ \bR_{YX}}} = 
\left( \begin{array}{c}
R_{YX}[0] \\
R_{YX}[1] \\
. \\
. \\
. \\
R_{YX}[N-1] \\
R_{YX}[N]
\end{array}
\right)
~~~
{{ \bh}} = 
\left( \begin{array}{c}
h[0] \\
h[1] \\
. \\
. \\
. \\
h[N-1] \\
h[N]
\end{array}
\right) .
\]
Si 
on suppose que les variables al\'eatoires $X[0]$, $X[1]$, ...,$X[N]$
sont lin\'eairement ind\'ependantes, ce qui veut dire que la
matrice ${ \bR_X}$ est inversible, alors
\begin{equation}
\label{bh}
\bh = { \bR_X}^{-1} \bR_{YX} .
\end{equation}
Si elles ne sont pas ind\'ependantes, on diminue $N$
jusqu'\`a ce qu'elles soient ind\'ependantes, pour minimiser l'ordre
de la r\'egression lin\'eaire.
\\
\\
\noindent{\bf Erreur d'estimation}
L'erreur d'estimation $W [n] = Y[n] - \tilde Y[n]$
\'etant orthogonale \`a $\tilde Y[n]$, le th\'eor\`eme de Pythagore
prouve que sa variance est 
\[
E\{W [n]^2\} = E\{Y [n]^2\}  - E\{\tilde Y [n]^2\} .
\]
Par ailleurs, l'orthogonalit\'e implique que 
$E\{\tilde Y[n] \tilde Y[n]\} = E\{ \tilde Y[n] Y[n]\}$.
En ins\'erant (\ref{filtre-wiener}) on obtient
\begin{equation}
\label{erreur-reg}
E\{W [n]^2\} = E\{Y [n]^2\}  - 
\sum_{l=0}^{N} h[l] R_{YX} [l] .
\end{equation}
\\
\noindent{\bf Moyenne non-nulle}
Lorsque les processus $X[n]$ et $Y[n]$ sont de moyenne non-nulle,
$E\{Y[n]\} = \mu_Y$ et $E\{X[n]\} = \mu_X$,
on se ram\`ene \`a des processus de moyenne nulle en soustrayant
leur moyenne. La formule d'estimation optimale 
(\ref{filtre-wiener}) s'applique alors pour ces processus 
\[
\tilde Y[n] = \mu_Y +  \sum_{l=0}^{N} h[l] (X[n-l] - \mu_X) .
\]
Le syst\`eme d'\'equations de Wiener-Hopf qui d\'efinit
les param\`etres $h[l]$ reste alors le m\^eme.


\subsection{Pr\'ediction lin\'eaire}
\label{predict-lin-se}

La pr\'ediction progressive lin\'eaire
est un cas particulier important de r\'egression
qui se r\'esoud par filtrage de Wiener.
On veut estimer $Y[n]= X[n]$ \`a partir des $N$ 
donn\'ees pr\'ec\'edentes
$\{X[n-k]\}_\UkN$.
L'estimateur lin\'eaire est donn\'e par la projection orthogonale
\begin{equation}
\label{est_prog}
\tilde Y[n] = X^p_N [n]
= \E \{ X[n] \BS X[n-1] , ..., X[n-N] \} = 
\sum_{k=1}^N a_N [k] X[n-k] .
\end{equation}
Les coefficients du filtre optimal $a_N[k]$ sont calcul\'es
de fa\c{c}on \`a ce que l'erreur de pr\'ediction
\begin{equation}
\label{error_formula}
W^p_N [n] = X[n] - X^p_N [n] = X[n] - \sum_{k=1}^N a_N [k] X[n-k] 
\end{equation}
soit orthogonale aux variables al\'eatoires 
$\{X[n-k]\}_\UkN$.
Le syst\`eme d'\'equations de Wiener-Hopf fait donc
intervenir l'autocovariance $R_X[l]$ des variables
$\{X[n-k]\}_\UkN$ et la covariance
\[
R_{YX} [k ] = E\{ X[n] X[n-k]\} = R_X[k] .
\]
On obtient
\begin{equation}
\label{WH_pred}
\left\{
\begin{array}{ccccl}
R_X[0]  a_N[1] & + R_X[1] a_N [2] &+  ...& + R_X[N-1] a_N [N] & = R_X[1]\\
R_X[1]  a_N[1] & + R_X[0] a_N [2] &+  ...& + R_X[N-2] a_N [N] & = R_X[2]\\
. & . & . & . & = . \\
R_X[N-1]  a_N[1] & + R_X[N-2] a_N [2] &+ ... & +  R_X[0] a_N [N] & = R_X[N]\\
\end{array}
\right.
\end{equation}

La variance de l'erreur d'estimation se d\'eduit de 
(\ref{erreur-reg}) avec $E[Y[n]^2 ] = E[X[n]^2 ] = R_X[0]$
\begin{equation}
\label{error_pred}
\epsilon_N = E\{|W [n]|^2\} = 
R_X[0]  - \sum_{k=1}^N a_N[k] R_X[k] .
\end{equation}
\\
\\
{\bf Mod\'elisation Autor\'egressive}
Les \'equations de
Wiener-Hopf (\ref{WH_pred}) pour une pr\'ediction lin\'eaire
sont identiques aux
\'equations de Yule-Walker (\ref{yule-walker})
qui caract\'erisent l'autocovariance
d'un processus AR 
(\`a un signe pr\`es d\^u \`a un changement de signe dans la
d\'efinition des coefficients $a_N [k]$).
Cela prouve que l'\'equation de diff\'erences d'un processus AR est
une r\'egression lin\'eaire optimale.

La variance de l'erreur d'estimation (\ref{error_pred})
est la m\^eme que celle
obtenue en (\ref{variance-bruit}) 
pour un processus AR (en changeant le signe des
coefficients de r\'egression).
Lorsque
$X[n]$ n'est pas un processus AR d'ordre $N$,
l'erreur $W [n]$ n'est pas un bruit blanc.
L'existence de corr\'elations dans l'erreur d'approximation montre
que le processus est d'un ordre sup\'erieur. 

\section{Algorithme de Levinson-Durbin}

La r\'esolution du syst\`eme lin\'eaire (\ref{wiener-hopf}) de $N+1$
\'equations \`a $N+1$ inconnues demande en g\'en\'eral
$O(N^3)$ op\'erations. 
En utilisant la structure Toeplitz de la matrice
de covariance $\R_X$,
l'algorithme de Levinson-Durbin 
factorise $\R_X$ avec $O(N^2)$ op\'erations sous forme
\begin{equation}
\label{factor}
\D = \L \R_X \L^t ,
\end{equation}
o\`u $\D$ est une matrice diagonale et $\L$ est une
matrice triangulaire inf\'erieure. 
Les $N+1$ coefficients du filtre de Wiener se calculent
alors par r\'esolution de l'\'equation de Wiener-Hopf
\[
\bh = \bR_X^{-1} \bR_{YX} = \bf L^t \bf D^{-1} \bf L \bR_{YX} 
\]
avec un total de $O(N^2)$ op\'erations.

L'algorithme de Levinson-Durbin calcule la factorisation
(\ref{factor}) par une orthogonalisation de
Gram-Schmidt de la famille $\{X[n+k]\}_\ZkN$. Cette 
orthogonalisation est \'equivalente \`a une pr\'ediction lin\'eaire.
Nous pr\'esentons un algorithme rapide qui calcule les
coefficients de pr\'ediction par r\'ecurrence en utilisant
les propri\'et\'es de la pr\'ediction progressive et r\'etrograde.

\subsection{Pr\'ediction r\'etrograde}
\label{levions-pred}

La pr\'ediction lin\'eaire r\'etrograde estime $X[n]$
\`a partir de son futur $\{X[n+k]\}_\UkN$
\[
X^r_N [n] = \sum_{k=1}^N a_N[k] X[n+k] = \E \{ X[n] \BS
X[n+1], ..., X[n+N] \} .
\]
Les facteurs $a_N [k]$ de pr\'ediction r\'etrograde sont les
m\^emes que pour la
pr\'ediction progressive (\ref{est_prog}) car 
les syst\`emes d'\'equations de Wiener-Hopf sont les m\^emes 
dans les
deux cas.
Cela se v\'erifie facilement en observant que
$R_X[l] = R_X[-l]$.

L'erreur de pr\'ediction r\'etrograde est
\begin{equation}
\label{error-retro}
W_N^r [n] = X[n] - X^r_N [n] = X[n] - \sum_{k=1}^N a_N[k] X[n+k] .
\end{equation}
Les variances des erreurs d'estimation progressive et r\'etrograde, 
calcul\'ees avec (\ref{erreur-reg}), sont aussi les m\^emes
\[
\epsilon_N = E\{W_N^p [n]^2 \}  = 
E\{W_N^r [n]^2 \} = R_X[0]  - \sum_{k=1}^N a_N[k] R_X[k] .
\]
Le th\'eor\`eme suivant \'etablit deux relations de r\'ecurrence
entre ces erreurs d'estimation, qui permettent de les calculer
rapidement.

\begin{theorem}
\label{levinson}
\begin{equation}
\label{erreur-rec}
\left\{ \begin{array}{l}
W_N^p [n] = W_{N-1}^p [n] - 
K_N ~W^r_{N-1} [n-N] \\
W_N^r [n-N] = W_{N-1}^r [n-N] - 
K_N ~W^p_{N-1} [n] 
\end{array}
\right.
\end{equation}
Si $\epsilon_{N-1} = 0$ alors $K_N = 0$, sinon 
\begin{equation}
\label{reflection}
K_N = \frac {R_X [N] - \sum_{k=1}^{N-1} a_{N-1} [k] R_X[N-k]}
{\epsilon_{N-1}} \leq 1
\end{equation}
et
\begin{equation}
\label{energie-kn}
\epsilon_N = e _{N-1} (1 - K_N^ 2 )  .
\end{equation}
\end{theorem}

\noindent{\bf D\'emonstration de (\ref{erreur-rec})}\\
L'erreur 
\[
W^r_{N-1}[n-N] = X[n-N] - 
\E \{ X[n-N] \BS X[n-N+1] , ... , X[n-1] \} 
\]
est le compl\'ement orthogonal de $X[n-N]$ par rapport
\`a $X[n-1]$, ... , $X[n-N+1]$. 
On peut donc d\'ecomposer la projection
\[
X^p_N [n] =\E \{ X[n] \BS X[n-1] , ... , X[n-N] \} 
\]
comme somme de deux projecteurs
\[
X^p_N [n] 
=\E \{ X[n] \BS X[n-1] , ... , X[n-N+1]\}
+ \E\{ X[n] \BS W^r_{N-1}[n-N] \} ,
\]
ce qui prouve que
\begin{equation}
\label{dcom}
X^p_N [n] 
= X^p_{N-1}[n] 
+ \E\{ X[n] \BS W^r_{N-1}[n-N] \} .
\end{equation}
Comme $W^r_{N-1}[n-N]$ est orthogonale \`a
$X[n-1]$, ... , $X[n-N+1]$, la projection de $X[n]$ sur
$W^r_{N-1}[n-N]$ est \'egale \`a la projection
du compl\'ement de $X[n]$ par rapport 
$X[n-1]$, ... , $X[n-N+1]$ sur 
$W^r_{N-1}[n-N]$ et donc
\begin{equation}
\label{Kn}
\E\{ X[n] \BS W^r_{N-1}[n-N] \} = 
\E\{ W^p_{N-1}[n] \BS W^r_{N-1}[n-N] \} 
= K_N W^r_{N-1}[n-N] .
\end{equation}
En ins\'erant cette \'equation dans (\ref{dcom}) on obtient 
\begin{equation}
X_N^p [n] = X_{N-1}^p [n] +
K_N W^r_{N-1} [n-N] .
\end{equation}
Soustraire chaque c\^ot\'e de cette \'egalit\'e \`a $X[n]$ donne
\[
W_N^p [n] = W_{N-1}^p [n] -  
K_N W^r_{N-1} [n-N] .
\]
La d\'emonstration de la seconde \'equation
se fait exactement de la m\^eme mani\`ere
en inversant les erreurs r\'etrogrades et progressives dans
ces \'equations.

\noindent{\bf D\'emonstration de (\ref{reflection})}\\
Si $\epsilon_{N-1} = 0$ et donc $W^r_{N-1}[n-N] = W^p_{N-1}[n-N] = 
0$, et comme
$W^r_{N}[n-N] = W^p_{N}[n-N] = 0$ 
on peut choisir $K_N = 0$.
Dans le cas o\`u
\[
\epsilon_{N-1} = E \{ W_{N-1}^p [n]^2 \} = 
E\{ W_{N-1}^r [n-N]^2 \} \neq 0 ,
\]
la constante de projection $K_N$ 
d\'efinie en (\ref{Kn}) est alors \'egale au 
produit scalaire normalis\'e des deux vecteurs de la projection
\begin{equation}
\label{express}
K_N = \frac  {E \{ W_{N-1}^p [n]W_{N-1}^r [n-N] \}}
{E\{ W_{N-1}^r [n-N]^2 \}} = 
\frac  {E \{ W_{N-1}^p [n]W_{N-1}^r [n-N] \}} 
{\epsilon_{N-1}} \leq 1 .
\end{equation}
Par ailleurs
\begin{equation}
\label{intermed}
{E \{ W_{N-1}^p [n]W_{N-1}^r [n-N] \}} = 
{E \{ W_{N-1}^p [n]X [n-N] \}} .
\end{equation}
En effet
$W^r_{N-1} [n-N]$ est la composante de $X[n-N]$
qui est orthogonale \`a $X[n-1], X[n-2], ... , X[n-N+1]$,
et $W_{N-1}^p [n]$ est orthogonale \`a ces m\^emes variables 
al\'eatoires.
En rempla\c{c}ant $W_{N-1}^p [n]$ par son expression
(\ref{error_formula}) on d\'eduit de (\ref{express}-\ref{intermed}) que
\[
K_N = \frac {R_X[N] - \sum_{k=1}^{N-1} a_{N-1} [k] R_X[N-k]} 
{\epsilon_{N-1}} .
\]

\noindent {\bf D\'emonstration de (\ref{energie-kn})}\\
Nous avons prouv\'e que
\begin{equation}
\label{error-rec}
W_{N-1}^p [n] = W_N^p [n] + 
K_N W^r_{N-1} [n-N] .
\end{equation}
Comme $W_N^p [n]$ et $W^r_{N-1} [n-N]$
sont deux variables al\'eatoires orthogonales et que
\[
\epsilon_{N-1} = E\{W_{N-1}^p [n]^2\} = 
E\{W_{N-1}^r [n-N]^2\} , 
\]
le th\'eor\`eme de Pythagore appliqu\'e \`a (\ref{error-rec}) 
nous donne
\[
\epsilon_{N-1} = \epsilon_N +  K_N^2 \epsilon_{N-1} . ~~~~\Box
\]
\\
\\
{\bf Coefficients de r\'egression}
Pour calculer les coefficients de r\'egression $a_N[k]$,
on \'etablit une relation de r\'ecurrence qui est une cons\'equence
des
\'equations (\ref{erreur-rec}) du Th\'eor\`eme 
\ref{levinson}.
Rappellons que
\[
W^p_{N}[n] = X[n] - \sum_{k=1}^N a_N[k] X[n-k] ~~
\]
et 
\[
W^r_{N}[n-N] = X[n-N] - 
\sum_{k=1}^{N} a_{N}[k] X[n-N+k] .
\]
Si $\epsilon_{N-1} \neq 0$ alors 
les variables al\'eatoires $(X[n-k])_\ZkN$
sont lin\'eairement
ind\'ependantes, et (\ref{erreur-rec}) se traduit par une
\'egalit\'e sur les facteurs devant chaque variable al\'eatoire.
On obtient 
\begin{equation}
\label{rec_ak}
\left\{ \begin{array}{l}
a_N[k] =  a_{N-1} [k] - K_N a_{N-1} [N-k] ~~,~~
1 \leq k \leq N-1 \\
a_N[N] =  K_N
\end{array}
\right.
\end{equation}
L'algorithme de Levinson-Durbin utilise les \'equations 
(\ref{rec_ak}-\ref{reflection}-\ref{energie-kn})
pour calculer 
$\epsilon_N$, $K_N$ et $a_N$ par r\'ecurrence \`a partir
de l'autocovariance $R_X$.
\vspace{5mm}

\noindent {\bf Algorithme de Levinson-Durbin}\\
{\it Initialisation:}  $\epsilon_{0} = R_X[0] $.\\
{\it Boucle:} Pour $m$ allant de $1$ \`a $N$\\
Calcul de $K_m$\\
\indent Si $\epsilon_{m-1} > 0$\\
\begin{equation}
K_m = \frac {R_X [m] - \sum_{k=1}^{m-1} a_{m-1} [k] R_X[m-k]} 
{\epsilon_{m-1} } 
\end{equation}
\indent Sinon
\[
K_m = 0 .
\]
Calcul de $a_m$\\
\[
\left\{ \begin{array}{l}
a_m[k] =  a_{m-1} [k] - K_m a_{m-1} [m-k] ~~,~~
1 \leq k \leq m-1 \\
a_m[m] =  K_m
\end{array}
\right.
\]
Calcul de $\epsilon_m$\\
\[
\epsilon_m = \epsilon_{m-1} (1 - K_m^2 ) .
\]
\\
Cet algorithme
demande $O(N^2 )$ op\'erations
pour calculer $a_{N}[k]$ pour $1 \leq k \leq N$.


\subsection{Filtrage en treillis}

Au-del\`a du probl\`eme de pr\'ediction, l'algorithme de
Levinson-Durbin permet de r\'esoudre en $O(N^2)$ op\'erations les
\'equations de Wiener-Hopf pour l'estimation de tout
processus $Y[n]$ conjointement stationnaire avec
$\{X [n] , X[n-1] , ..., X[n-N] \}$.
\\
\\
{\bf Orthogonalisation rapide}
Les erreurs de pr\'ediction r\'etrogrades 
\[
\{W^r_0 [n] , W^r_1 [n] , ..., W^r_N [n] \}
\]
peuvent s'interpr\'eter comme
une orthogonalisation de Gram-Schmidt des
variables al\'eatoires
$\{X [n] , X[n-1] , ..., X[n-N] \}$
\[
\begin{array}{l}
W^r_0 [n] = X[n] \\
W^r_1 [n-1] = X[n-1] - \E \{ X[n-1] \BS X[n] \} \\
...\\
W^r_N [n-N] = 
X[n-N] - \E \{ X[n-N] \BS X[n],..., X[n-N+1] \} \\
\end{array}
\]

L'algorithme de Levinson-Durbin effectue un calcul rapide
de cette orthogonalisation gr\^ace aux \'equations r\'ecurrentes
\begin{equation}
\label{erreur-rec2}
\left\{ \begin{array}{l}
W_k^p [n] = W_{k-1}^p [n] - 
K_k ~W^r_{k-1} [n-k] \\
W_k^r [n-k] = W_{k-1}^r [n-k] - 
K_k ~W^p_{k-1} [n] 
\end{array}
\right.
\end{equation}
Ces \'equations s'impl\'ementent 
par un filtrage en treillis illustr\'e
par la figure \ref{bloque-treillis}.

\begin{figure}[bhtp]
%\centerline{
%	\leavevmode\epsfbox{/home/mallat/X/TREX/figures/SigFig/FIG5.3.EPS.txt}}
\vspace{5cm}
\caption{Bloque \'el\'ementaire d'un filtrage 
en treillis pour calculer les erreurs de pr\'ediction progressives
et r\'etrogrades.}
\label{bloque-treillis}
\end{figure}
\\
\\
{\bf Treillis}
Plut\^ot que d'exprimer l'estimation $\tilde Y [n]$ 
de $Y[n]$ comme combinaison
lin\'eaire des variables 
$\{X [n] , X[n-1] , ..., X[n-N] \}$ par le
filtre de Wiener $\bh$, il est souvent
plus efficace de d\'ecomposer cette estimation sur les
variables orthogonales
$\{ W^r_0[n], ..., W_N^r[n] \}$ qui g\'en\`erent le
m\^eme espace et que l'on
calcule efficacement par un filtre en treillis.
\[
\tilde Y [n] = \E \{ Y[n] \BS X[n], ..., X[n-N] \} = 
\E \{ Y[n] \BS W^r_0[n], ..., W_N^r[n-N] \} .
\]
Comme ces variables al\'eatoires sont orthogonales
\begin{equation}
\label{estim-some}
\tilde Y [n] = 
\sum_{k=0}^N \E \{ Y[n] \BS W_k^r[n-k] \} =
\sum_{k=0}^N c[k] W_k^r[n-k] ,
\end{equation}
avec
\[
c[k] = \frac {E \{Y[n]  W_k^r[n-k] \} } 
{E \{ W_k^r[n-k]^2 \} } .
\]
En rempla\c{c}ant $W_k^r[n-k]$ par son expression (\ref{error-retro})
on obtient
\[
c[k] = \frac {R_{YX} [0] - \sum_{i=1}^k a_k[i] R_{YX} [k-i]\} } 
{\epsilon_k } .
\]
Si l'on a d\'ej\`a calcul\'e les coefficients $a_k[n]$ avec
l'algorithme de Levinson-Durbin, le
calcule de $c[k]$ pour $0 \leq k \leq N$ demande
$O(N^2)$ op\'erations.

Les \'equations de r\'ecurrence (\ref{erreur-rec}) 
du th\'eor\`eme \ref{levinson} montrent
que $W_k^r[n-k]$ se calcule par une cascade
de filtres en treillis illustr\'e en figure \ref{fitre-treillis}.
La somme (\ref{estim-some}) de l'estimation $\tilde Y [n]$ 
s'obtient donc par cette architecture en treillis.\\

\begin{figure}[bhtp]
%\centerline{
%	\epsfxsize=14cm	
%	\leavevmode\epsfbox{/home/mallat/X/TREX/figures/SigFig/FIG5.1.EPS.txt}}
\vspace{8cm}
\caption{Calcul par filtrage en treillis de
l'estimateur lin\'eaire optimal $\tilde Y[n]$ de $Y[n]$ sachant
$X[n] ... X[n-N]$}
\label{fitre-treillis}
\end{figure}


\chapter{Traitement de la Parole}
\label{parole-chap}

Le traitement de la parole n\'ecessite une analyse
physiologique des m\'ecanismes
de production, ce qui permet de comprendre les propri\'et\'es
particuli\`eres de ce type de signal.
Cela motive notament l'utilisation de mod\`eles autor\'egressifs.
L'identification des param\`etres de ces mod\`eles se fait
par r\'egression lin\'eaire. On peut ainsi
effectuer un codage compacte des signaux de parole pour
la t\'el\'ephonie cellulaire.


\section{Mod\'elisation du signal de parole}
\label{modelisation-paro}

\subsection{Production}


La production de la parole se fait en trois \'etapes.
Les poumons compressent de l'air qui est envoy\'e
\`a travers la trach\'ee. Cet air passe par le larynx qui est
compos\'e d'un syst\`eme de cartilages et de muscles incluant les
cordes vocales. Le larynx produit alors un signal d'excitation qui
se propage \`a travers le conduit vocal. C'est la
d\'eformation du conduit vocal qui produit l'articulation de la
parole. Les \'el\'ements principaux de cette articulation sont la
langue, les l\`evres et la machoire inf\'erieure. La figure
\ref{parole-physio} illustre l'appareil phonatoire.
\begin{figure}
%(voire Genat/Karar p. 10)
\vspace{7cm}
\caption{Syst\`eme phonatoire \protect \cite{karar}.}
\label{parole-physio}
\end{figure}
\\
\\
{\bf Excitation}
Le larynx peut produire des signaux d'excitation diff\'erents.
Les sons vois\'es tels que les voyelles sont produits par
vibration des cordes vocales. L'air est forc\'e \` a travers les
cordes vocales qui vibrent comme les l\`evres d'un trompettiste. Cela
produit un train de quasi-impulsions, illustr\'e en
figure \ref{quasi-impulsions},
qui est envoy\'e dans le conduit vocal.
La fr\'equence des r\'ep\'etitions, appel\'ee
``pitch'', est essentiellement contr\^ol\'ee par la tension des cordes
vocales. Elle correspond \`a la fr\'equence fondamentale (hauteur) du son.
Dans le cas de la voix parl\'ee, le pitch est typiquement
entre 100Hz et 300Hz. Une soprano peut cependant augmenter cette
fr\'equence jusqu'\`a 3600Hz.

\begin{figure}
%(voire fig. 3-6 p. 65, Parson)
\vspace{2cm}
\caption{Train de quasi-impulsions \'emises par les cordes
vocales}.
\label{quasi-impulsions}
\end{figure}

Pour un chuchotement, les cordes vocales ne vibrent pas
mais laissent un passage
\'etroit entre les cartilage du larynx, qui envoie un
air turbulent dans le conduit vocal.
Cet air turbulent peut \^etre mod\'elis\'e par un bruit Gaussien, dont
la puissance spectrale a un large support fr\'equentiel.
\\
\\
{\bf Articulation}
Le conduit vocal donne l'articulation au son qui caract\'erise chaque
phon\`eme. Nous avons vu que
pour un son vois\'e, le larynx \'emet un train d'onde
riche en harmoniques qui est filtr\'e par le conduit
vocal. Ce conduit vocal a des r\'esonances appel\'ees ``formants''.
La d\'eformation du conduit vocal d\'eplace les fr\'equences de
r\'esonance, ce qui permet de former
toutes les voyelles et certaines consonnes.
Pour des sons non-vois\'es, le conduit vocal peut aussi effectuer
des constrictions qui produit des fricatives ou des sons
chuint\'es telles que le
[s] et le [ch].
Les sons plosifs sont produits par une fermeture du conduit
vocal ce qui cr\'ee une occlusion. Le rel\^achement brutal de
cette occlusion produit alors une plosive telle que [p] ou [t].
L'articulation du son est aussi affect\'ee par
l'ouverture de la cavit\'e nasale.
Des cat\'egorisations tr\`es d\'etaill\'ees des
diff\'erents sons de parole ont \'et\'e faites
par les phon\'eticiens.


\subsection{Conduit vocal}
\label{conduit-sec}
Le conduit vocal peut se mod\'eliser comme la juxtaposition
de plusieurs cylindres de m\^eme longueur \'egale \`a
$\Delta$ mais de diam\`etres variables,
comme l'illustre la figure \ref{conduit-voc}.
Chaque cylindre est un syst\`eme lin\'eaire avec en entr\'ee
une onde directe $f_{n-1} (t)$ mesurant le d\'ebit du flot d'air
qui passe \`a l'entr\'ee du cylindre par unit\'e de temps,
et une onde inverse
$b_{n-1} (t)$. En sortie, on a onde directe $f_n (t)$
et une onde inverse $b_n (t)$ qui est reli\'ee \`a l'entr\'ee
par une matrice $T_n$ qui d\'epend de l'imp\'edance acoustique
des cylindres $n-1$ et $n$
\[
\left(
\begin{array} {l}
f_n (t)\\
b_n (t)
\end{array}
\right) = T_n
\left(
\begin{array} {l}
f_{n-1} (t)\\
b_{n-1} (t)
\end{array}
\right)
\]
Si l'on cascade la r\'eponse de chaque cylindre, on obtient
\[
\left(
\begin{array} {l}
f_p (t)\\
b_p (t)
\end{array}
\right) = T
\left(
\begin{array} {l}
f_{0} (t)\\
b_{0} (t)
\end{array}
\right)
\]
avec
\[
T = \prod_{n=1}^{p} T_n .
\]


\begin{figure}
% (voire p. 109 Parson)
\vspace{3.5cm}
\caption{Le conduit vocal peut se mod\'eliser comme une
succession de cylindres de m\^eme longueur $\Delta$
ayant des diam\`etres
diff\'erents.}
\label{conduit-voc}
\end{figure}

On discr\'etise ce syst\`eme avec un pas d'\'echantillonnage de
$\frac \Delta c$ qui est le temps de propagation de l'onde
acoustique dans chaque
cylindre. Le signal d'entr\'ee est le signal discret
\[
f_0 [n] = f_0 (\frac {n \Delta} c ) - b_0 (\frac {n \Delta} c ) ,
\]
tandis que le signal de sortie est
\[
f_p [n] = f_p (\frac {n \Delta} c ) - b_p (\frac {n \Delta} c ) .
\]
En \'ecrivant les \'equations de propagation des ondes et
les conservations de d\'ebit et de pression au travers des
jonctions des cylindres, on peut calculer la fonction de
transfert qui relie la transform\'ee en z de $f_0 [n]$ et de
$f_p [n]$ \`a partir des diam\`etres de chacun des cylindres
\[
\frac {\hat f_p (z)} {\hat f_0 (z)} = \hat h(z) .
\]
En l'absence de perte le long du syst\`eme, on peut montrer que
$\hat h(z)$ a $p$ p\^oles mais n'a pas de z\'eros.
C'est donc un filtre autor\'egressif qui peut s'\'ecrire
\[
\hat h (z) = \frac 1 {a[0] + a[1] z^{-1}+ ... + a[p] z^{-p}} .
\]
Cette condition reste valable tant que le conduit vocal peut
\^etre repr\'esent\'e par un seul tube sans embranchement. On n\'eglige
donc l'influence introduite par le conduit nasal ainsi que
les pertes d'\'energie dues aux vibrations
des parois du conduit vocal et aux frictions.
\\
\\
{\bf Formants}
Ce mod\`ele simplifi\'e montre que le conduit vocal peut \^etre repr\'esent\'e
par un filtre autor\'egressif dont les param\`etres $a[k]$ d\'ependent
de la configuration du conduit vocal.
Ce filtre \' etant causal et stable, nous savons que les p\^oles de
$\hat h (z)$ sont tous de module plus petit que 1.
Les p\^oles de $\hat h(z)$ qui sont proches du cercle unit\'e
produisent un pic dans le module de la r\'eponse fr\'equentielle
$|\hat h (\eom)|$.
Ces pics de fr\'equence sont appel\'es formants. Ils ont une
importance particuli\`ere pour la reconnaissance des sons produits.
Plus le z\'ero est proche du cercle unit\'e, plus le formant est
prononc\'e. Les formants de plus grande amplitude
sont g\'en\'eralement les
2 premiers, apparaissant aux plus basses fr\'equences.
La figure \ref{autoreg}
donne un exemple de filtre autor\'egressif ayant 8 p\^oles dont
la position est indiqu\'ee dans
le plan complexe. La r\'eponse fr\'equentielle
$|\hat h (\eom)|_\db$ est donn\'ee \`a droite.

\begin{figure}
% voire p.23-25 genat/karar
\vspace{8cm}
\caption{La figure de gauche donne la position des 8 p\^oles
d'un filtre autor\'egressif tandis que la figure de droite
donne le module $|\hat h (\eom)|_\db$.}
\label{autoreg}
\end{figure}


\subsection {Excitation}

{\bf Sons vois\'es}
En mode vibratoire, les cordes vocales \'emettent un train
d'onde qui peut s'\'ecrire
\[
f(t) = \sum_{n=-\infty}^{+\infty} g(t - n T) =
g(t) \star \sum_{n=-\infty}^{+\infty} \delta (t - n T) .
\]
o\`u $g(t)$ est une onde \'el\'ementaire dont le support est
petit devant $T$ (voire figure \ref{quasi-impulsions}).
On peut supposer que cette
onde est ind\'ependante de la forme du conduit vocal.
En appliquant la formule de Poisson (\ref{poisson}), on calcule
la transform\'ee de Fourier de $f(t)$
\[
\hat f (\om) = \hat g(\om)
\frac {2 \pi} T \sum_{n=-\infty}^{+\infty}
\delta (\om - \frac {2 n \pi} T ) =
\frac {2 \pi} T \sum_{n=-\infty}^{+\infty}
\hat g(\frac {2 n \pi} T ) \delta (\om - \frac {2 n \pi} T ) .
\]
C'est une succession d'harmoniques dont l'enveloppe est \'egale
a $\hat g (\om)$.

On a vu que le conduit vocal est \'equivalent \`a un filtre lin\'eaire
de fonction de transfert $\hat h (\om)$. La transform\'ee
de Fourier du son \'emis est donc
\[
\hat f(\om) \hat h (\om ) =
\hat h (\om ) \hat g(\om)
\frac {2 \pi} T \sum_{n=-\infty}^{+\infty}\delta (\om - \frac {2 n \pi} T ) .
\]
Cette r\'eponse peut \^etre mod\'elis\'ee comme un train d'impulsions
de Diracs qui passe \`a travers un filtre dont la fonction
de transfert est sp\'ecifi\'ee par l'enveloppe totale
$\hat h (\om) \hat g (\om)$.

On sait que la discr\'etisation de l'onde se propageant dans le
conduit vocal peut se mod\'eliser par un filtrage autor\'egressif.
La discr\'etisation de $g(t)$
peut de m\^eme \^etre approxim\'ee par un filtre autor\'egressif.
Un signal de parole vois\'e discr\'etis\'e se mod\'elise donc par
un train d'impulsion de Diracs discrets
$\sum_{k=-\infty}^{+\infty} \delta [n-kT]$ filtr\'e par
un filtre autor\'egressif qui d\'epend \`a la fois du conduit
vocal et de la forme de l'impulsion $g(t)$ produite par
les cordes vocales.
\\
\\
{\bf Sons non vois\'es}
Les sons non vois\'es sont produits par un signal turbulent
\'emis par le larynx qui est ensuite modifi\'e par le conduit vocal.
La discr\'etisation de ce signal turbulent
peut \^etre mod\'elis\'ee par un processus Gaussien
stationnaire $Y[n]$ dont la puissance spectrale
$\hat R_{Y} (\eom)$
\`a une \'energie qui est r\'epartie sur
une large bande de fr\'equence. Un tel processus peut
aussi s'ecrire comme
un bruit blanc Gaussien $W[n]$ filtr\'e par un filtre $g[n]$
\[
Y[n] = W \star g [n].
\]
Le th\'eor\`eme \ref{cov_conv_th}  prouve que la puissance spectrale
de $Y[n]$ est reli\'ee a $\hat g (\eom)$ par
\[
\hat R_{Y} (\eom) = |\hat g (\eom)|^2 .
\]

Nous avons vu que le conduit vocal se comporte
comme un filtre AR de r\'eponse impulsionelle $h [n]$.
Le son produit est alors mod\'elis\'e par le processus
\[
Y \star h [n] = W \star g \star h [n] .
\]
Un tel signal discr\'etis\'e se mod\'elise donc par un bruit blanc discret
$W[n]$ filtr\'e par $g \star h [n]$ dont la fonction de transfert
est $\hat g (z) \hat h (z)$.
En g\'en\'eral, $\hat g (z)$
peut avoir des z\'eros. Ces z\'eros sont neglig\'es car leur importance
perceptuelle est secondaire \`a c\^ote des p\^oles. On mod\'elise donc
$\hat g (z) \hat h (z)$ par un seul filtre autor\'egressif.
\\
\\
{\bf Mod\`ele synth\'etique}
Suivant que le son est vois\'e ou non, le signal de parole peut
se mod\'eliser comme un train d'impulsion ou
comme la r\'ealisation d'un bruit
blanc filtr\'e par un filtre autor\'egressif
dont les caract\'eristiques d\'ependent
du son prononc\'e. Dans le cas d'un son vois\'e, la p\'eriode des
impulsions (pitch) est un param\`etre qui doit \^etre d\'etermin\'e.
Ce mod\`ele est illustr\'e par la
figure \ref{modele-parole}.


\begin{figure}[bhtp]
%(p.55 Genat/Kara)
\centerline{
	\leavevmode\epsfbox{/home/mallat/X/TREX/figures/TrexFig/MALLATFIG7.7-eps}}
\caption{Mod\'elisation d'un son de parole par une excitation
p\'eriodique ou al\'eatoire, filtr\'ee par un filtre autor\'egressif.}
\label{modele-parole}
\end{figure}
\\
\\
\noindent
{\bf Stationnarit\'e}
Le conduit vocal ne se comporte comme un filtre lin\'eaire
homog\`ene que sur des intervalles de temps relativement petits.
Sur une dur\'ee plus longue,
un signal de parole est non stationnaire
puisque les sons changent au cours du temps. La taille des  intervalles
sur lesquels le son peut \^etre approxim\'e par une excitation
modul\'ee par un filtre homog\`ene d\'epend de la nature du son.
Pour une voyelle, cette approximation est valable sur environ
$10^{-2}$ seconde alors que beaucoup de sons de
consonnes telles que les plosives
ne restent pas stationnaires sur cette dur\'ee.
La variation de ces intervalles de stationnarit\'e est l'une des
difficult\'es du traitement de la parole.

\section{Estimation d'un mod\`ele de parole}
\label{est-par-sec}

Nous avons expliqu\'e qu'un signal de parole peut localement
\^etre approxim\'e par une excitation $e[n]$
filtr\'ee par un filtre AR dont
les param\`etres d\'ependent du son prononc\'e. Pour des applications
de reconnaissance et de codage, on veut identifier l'excitation
ainsi que les param\`etres du filtre.
On isole une portion d'un signal de parole $f[n]$
en le multipliant avec une fen\^etre $w[n]$ de taille $P$
centr\'ee en un instant $pP$
\[
x[n] = f[n] w[n-pP] ,
\]
comme le montre la figure \ref{fenetrage}.
On prend $P$
suffisamment petit pour que le son isol\'e puisse \^etre consid\'er\'e
comme stationnaire. Par exemple, la fen\^etre de
Hamming est d\'efinie par
\[
w[n] =
\left\{
\begin{array}{ll}
0.54 + 0.46 \cos (\frac {2\pi n} P ) &
\mbox{si $|n| < \frac P 2$}\\
0 & \mbox{sinon}
\end{array}
\right.
\]
Le signal r\'esultant $x[n]$ poss\`ede
au plus $P$ coefficients non-nuls.
Dans un mod\`ele de parole,
$x[n]$ est produit par une excitation
$e[n]$ filtr\'ee par un filtre AR dont la fonction de transfert
est renormalis\'ee pour s'\'ecrire
\[
\hat h (z) = \frac 1 {1 - a[1] z^{-1} - ... - a[N] z^{-N}} .
\]

\begin{figure}
% (p. 55 Genat/Karar)
\vspace{4cm}
\caption{Des portions du signal de parole sont isol\'ees par
des fen\^etres, qui couvrent des intervalles de temps o\`u
le signal peut \^etre consid\'er\'e comme stationnaire.}
\label{fenetrage}
\end{figure}
\\
\\
\noindent
{\bf Calcul de l'excitation} De nombreuses techniques
ont \'et\'e d\'evelopp\'ees
pour d\'eterminer le voisement et le pitch d'un son.
Une approche particuli\`erement simple est bas\'ee sur
la somme des diff\'erences du signal \`a intervalles $k$ variables
\begin{equation}
\label{AMDF}
D[k] = \frac 1 P \sum_{n} |x[n] - x[n-k]| .
\end{equation}
Si le signal est vois\'e et que la fr\'equence des impulsions est
$T$ alors $D[k]$ a un minimum a $k = T$. Lorsque le minimum
de $D[k]$ n'est pas suffisamment bas,
on en d\'eduit que le son est non vois\'e.
Cet algorithme a l'avantage de ne n\'ecessiter aucune multiplication
et donc de s'impl\'ementer tr\`es rapidement.


\subsection{R\'egression lin\'eaire}

Pour identifier tous les param\`etres d'un son, il nous faut
estimer les param\`etres $a[k]$ du filtre AR qui sp\'ecifient
les propri\'et\'es du conduit vocal.
Le signal $x[n]$ satisfait l'\'equation r\'ecurrente
\[
x[n] - \sum_{k=1}^N a[k] x[n-k] = e[n] .
\]
On peut interpr\'eter
\begin{equation}
\label{estim-dex}
\tilde x [n] =  \sum_{k=1}^N a[k] x[n-k]
\end{equation}
comme une estimation de $x[n]$ \`a partir de $N$ valeurs pass\'ees,
auquel cas l'erreur d'estimation n'est autre que l'excitation
\[
x [n] - \tilde x[n] = e[n] .
\]
Si $e[n]$ est la r\'ealisation d'un bruit blanc, donc non corr\'el\'ee
d'un \'echantillon \`a l'autre, chaque $e[n]$ peut \^etre consid\'er\'e
comme l'innovation apport\'ee par l'excitation relativement \`a la
pr\'ediction de $x[n]$ par son pass\'e. Ce point de vue permet de poser
l'identification des param\`etres  $a[k]$
du filtre AR comme un probl\`eme
de pr\'ediction lin\'eaire.
Etant donn\'e un signal $x[n]$, on veut calculer les
coefficients de r\'egression tels que l'estimation (\ref{estim-dex})
g\'en\`ere une erreur
\begin{equation}
\label{err-prod}
\sum_{n=-\infty}^{+\infty} |x[n] - \tilde x[n]|^2 =
\sum_{n=-\infty}^{+\infty} |e[n]|^2
\end{equation}
qui est minimum.
Pour effectuer ce calcul on introduit
l'autocorr\'elation empirique du signal $x[n]$
\begin{equation}
\label{autocro-x}
r_x[k] = \sum_{n=-\infty}^{+\infty}  x[n-k] x[n] .
\end{equation}
Cette somme s'\'etend seulement sur $P$ valeurs car $x[n]$ a au plus
$P$ valeurs cons\'ecutives non nulles.
Le th\'eor\`eme suivant caract\'erise les coefficients de r\'egression
de $\tilde x[n]$.

\begin{theorem} [Pr\'ediction lin\'eaire]
Les coefficients de r\'egression $\{a[k]\}_\UkN$ du
signal $\tilde x [n] =  \sum_{k=1}^N a[k] x[n-k]$ qui
minimise
\[
\epsilon_N = \sum_{n=-\infty}^{+\infty} |x[n] - \tilde x[n]|^2
\]
sont solutions du syst\`eme
\begin{equation}
\label{yule-walker2}
\left( \begin{array}{cccc}
r_x[0] &r_x[1] & ...&r_x[N] \\
r_x[-1]&r_x[0]& ...&r_x[N-1]\\
. &. &...&.  \\
.&.&...&.\\
.&.&...&.\\
r_x[-N]&r_x[-N+1]&...&r_x[0]\\
\end{array}
\right)
\left( \begin{array}{c}
a[1]\\
a[2]\\
.\\
.\\
.\\
a[N]\\
\end{array}
\right)
=
\left( \begin{array}{c}
r_x[1]\\
r_x[2]\\
.\\
.\\
.\\
r_x[N]\\
\end{array}
\right) .
\end{equation}
L'erreur r\'esultante est
\begin{equation}
\label{erro-regre}
\epsilon_N = r_x [0] - \sum_{p=1}^N a[p] r_x[p] .
\end{equation}
\end{theorem}

{\bf D\'emonstration} La d\'emonstration se fait par une
interpr\'etation g\'eom\'etrique du probl\`eme de minimisation
Le signal $x[n]$ a une \'energie finie et donc appartient
\`a l'espace $\lD$ muni de la
norme
\[
\| x[n] \|^2 = \sum_{n=-\infty}^{+\infty} |x[n]|^2
\]
et du produit scalaire
\[
<x[n],y[n]> = \sum_{n=-\infty}^{+\infty} x[n] y[n] .
\]
Le vecteur $\tilde x[n]$ est une combinaison lin\'eaire des
vecteurs $\{x_k [n] = x[n-k]\}_{1 \leq k \leq N}$
et appartient donc \`a l'espace $\V$
g\'en\'er\'e par ces $N$ vecteurs.
Minimiser l'erreur de pr\'ediction (\ref{err-prod}) revient
donc \`a trouver un vecteur $\tilde x[n]$ de $\V$ qui minimise
$\|x - \tilde x \|^2$.
Le th\'eor\`eme de projection prouve que ce vecteur
est la projection orthogonale de $\tilde x$ dans $\V$.
C'est donc un vecteur tel que $x - \tilde x$ est orthogonal \`a
tous les vecteurs de $\V$ et en particulier aux vecteurs
$\{ x[n-k] \}_\UkN$
\[
<x[n] - \tilde x [n] , x[n-k]> =
\sum_{n=-\infty}^{+\infty} (x[n] - \tilde x [n])  x[n-k] =  0 .
\]
En ins\'erant (\ref{estim-dex})
ces \'equations se r\'e\'ecrivent
\begin{equation}
\label{proj-equai}
\sum_{n=-\infty}^{+\infty} x[n] x[n-k] -
\sum_{1=0}^N a[p] \sum_{n=-\infty}^{+\infty} x[n-p] x[n-k] = 0 .
\end{equation}
En ins\'erant (\ref{autocro-x}) on obtient
\[
\sum_{p=1}^N a[p] r_x[k-p] = r_x [k]~~,~~\mbox{pour $1 \leq k \leq N$} ,
\]
qui correspond au syst\`eme de $N$ \'equations
\`a $N$ inconnues (\ref{yule-walker2}).

Comme $x[n] - \tilde x[n]$ est orthogonal
\`a tout vecteur dans
$\V$ et donc \`a $\tilde x[n]$, l'\'energie de l'erreur
peut s'\'ecrire
\[
\epsilon_N =  <x[n] - \tilde x[n] , x[n] - \tilde x[n]> =
<x[n] - \tilde x[n] , x[n]>.
\]
En rempla\c{c}ant $\tilde x[n]$
par son expression (\ref{estim-dex}), on obtient
\[
\epsilon_N = \sum_{n=-\infty}^{+\infty}  x[n] x[n] -
\sum_{p=1}^N a[p] \sum_{n=-\infty}^{+\infty}  x[n-p] x[n]
\]
et donc (\ref{erro-regre})
$\Box$
\\
\\
{\bf Filtre AR pour sons non-vois\'es}
Un son non-vois\'e est modelis\'e par bruit blanc traversant
un filtre AR.
Le paragraphe \ref{proc-autoregre-se} montre qu'un
filtre AR excit\'e par un bruit blanc $W[n]$ produit
un processus autor\'egressif $X[n]$ dont l'autocorr\'elation
$R_X [k]$ satisfait les \'equations de
Yule-Walker (\ref{yule-walker}).
En comparant le syst\`eme de Yule-Walker
(\ref{yule-walker}) et le syst\`eme (\ref{yule-walker2}),
on s'aper\c{c}oit que ces \'equations sont identiques lorsque l'on
remplace l'autocorr\'elation $R_X [k]$ du processus
$X[n]$ par l'autocorr\'elation empirique
$r_x [k]$ du signal $x[n]$.
Si $e[n]$ est une r\'ealisation du bruit blanc $W[n]$ alors $x[n]$
est une r\'ealisation du processus autor\'egressif $X[n]$.
L'autocorr\'elation empirique
$r_x [k]$ peut donc s'interpr\'eter
comme une estimation de la v\'eritable
autocorr\'elation $R_X [k]$ du processus.
Les \'equations de pr\'edictions lin\'eaires
(\ref{yule-walker2}) sont donc une
approximation des \'equations de Yule-Walker (\ref{yule-walker}),
qui permettent d'estimer les coefficients $a[k]$ du filtre AR.
La r\'esolution du syst\'eme (\ref{yule-walker2}) peut
se faire en utilisant l'algorithme rapide
de Levinson-Durbin qui n\'ecessite $O(N^2)$ op\'erations.
\\
\\
{\bf Filtre AR pour son vois\'e}
Lorsque le son est vois\'e, l'algorithme de r\'egression lin\'eaire
donne aussi une bonne estimation du filtre AR. La justification
est cependant plus compliqu\'ee et nous n'en donnons qu'une explication
superficielle.
Le signal de parole $f[n]$ est construit en filtrant
un train d'impulsion
\[
e[n] = \sum_{k=-\infty}^{+\infty} \delta [n -k T]
\]
avec un filtre AR $h[n]$ et $x[n]$ est obtenu
en multipliant $f[n]$ par la fen\^etre $w[n - p P]$
\[
x[n] = w[n-pP] \times  (e \star h)[n] .
\]
On peut en d\'eduire que le spectre $\hat x (\eom)$ est la
somme de composantes harmoniques situ\'ees autour des fr\'equences
$\om = \frac {2 k \pi} T$ dont l'amplitude est proportionnelle
a $\hat h (e ^{\frac {2 k \pi} T})$.
Pour v\'erifier que le filtre obtenu par r\'egression lin\'eaire
est proche du filtre $h[n]$ on montre que l'optimisation des
coefficients $a[k]$ de la r\'egression lin\'eaire calcule un filtre
autor\'egressif qui interpole approximativement
les valeurs $\hat h (e ^{\frac {2 k \pi} T})$.
Comme le filtre $\hat h (\eom)$ est lui-m\^eme autor\'egressif,
on en d\'eduit que
la r\'egression lin\'eaire optimale calcule une approximation
de ce filtre.
%
%La transform\'ee de Fourier de $e \star h[n]$ est
%\[\
%hat e (\eom) \hat h(\eom)
%= \sum_{k=-\infty}^{+\infty} H(e^{i \frac{2 k \pi} T})
%\delta (\om - \frac{2 k \pi} T) ,
%\]
%d'o\`u
%\begin{eqnarray*}
%\hat x(\eom) &=& \hat e (\eom) \hat h(\eom) \star \hat w (\eom)
%e^{-i pT \om} \\
%&=&
%\frac 1 {2 \pi} e^{-i pT \om}
%\sum_{k=-\infty}^{+\infty} \hat h(e^{i \frac{2 k \pi} T})
%\hat w (e^I{\om - \frac{2 k \pi} T}) .
%\end{eqnarray*}
%C'est un train d'ondes \'el\'ementaires qui ressemblent \`a des
%impulsions, qui sont les transform\'ees de Fourier translat\'ees de
%la fen\^etre $w[n]$. Comme ces impulsions ont des supports qui ne
%se recouvrent pratiquement pas,
%\[
%|X(\eom)|^2 \approx
%\frac 1 {4 \pi^2}
%\sum_{k=-\infty}^{+\infty} |H(e^{i \frac{2 k \pi} T}) |^2
%|W(e^{i(\om - \frac{2 k \pi} T)})|^2 .
%\]
%La valeur de $|X(\eom)|^2$ est non n\'egligeable qu'aux voisinage
%des frequences $\frac{2 k \pi} T$ et l'on a
%\[
%|X(e^{i\frac{2 k \pi} T})|^2 \approx |W(e^0)|^2
%|H(e^{i \frac{2 k \pi} T}) |^2 .
%\]
%La fonction de transfert du filtre AR de
%coefficients $a[k]$ est
%\[
%H(\eom) = \frac 1 {\sum_{k=0}^P a[k] e^{-i k\om}} .
%\]
%Puisque $x [n] = h \star \epsilon [n]$, leur transform\'ee de Fourier
%satisfait
%\[
%E(\eom) = \frac 1 {H(\eom)} X(\eom) .
%\]
%En appliquant la formule de Parseval (?) on peut calculer
%l'\'energie de cette excitation
%\begin{eqnarray*}
%E &=& \|e[n]\|^2 = \frac 1 {2 \pi} \int_{-\pi}^{\pi}
%|E(\eom)|^2 d\om \\
%&=& \frac 1 {2 \pi} \int_{-\pi}^{\pi} \frac {|X(\eom)|^2}
%{|H(\eom)|^2} d\om .
%\end{eqnarray*}
%Le filtre $H(\eom)$ est le filtre AR qui minimise cette expression.
%Notons que l'on a la contrainte suivante
%\[
%\int_{-\pi}^{\pi} \frac 1 {H(\eom)} d\om = a[0] = 1.
%\]
%Lorsque $|X(\eom)|^2$ la valeur de $|H(\eom)|^2$ influence tr\`es
%peu l'erreur E. Cette valeur n'est importante que lorsque
%$|X(\eom)|$ est grand. Nous calculons donc \`a pr\'esent la valeur
%du spectre de $x[n]$.
%
%
%Pour minimiser l'erreur $E$ on verifie alors que le filtre fait
%une interpolation des valeurs $H(e^{i \frac{2 k \pi} T})$
%\[
%H(e^{i \frac{2 k \pi} T}) \approx \lambda
%H(e^{i \frac{2 k \pi} T}) .
%\]
%
%Si l'on excite ce filtre avec le m\^eme train d'impulsion
%$e[n]$ et que l'on multiplie le signal r\'esultant par la m\^eme
%fen\^etre $w[n]$, on obtient alors un signal $x_1 [n]$ dont
%la transform\'ee de Fourier est
%\[
%|X_1 (\eom)|^2 =
%\frac 1 {4 \pi^2}
%\sum_{k=-\infty}^{+\infty} |H(e^{i \frac{2 k \pi} T}) |^2
%|W(e^{i(\om - \frac{2 k \pi} T)})|^2 .
%\]
%On obtient donc un signal dont l'\'energie en Fourier est similaire
%\`a celle du signal original. La phase de ce signal est par contre
%modifi\'ee mais cela est peut perceptible d'un point de vue auditif.


%\subsection{Filtrage en Treillis}
%
%Le syst\'eme de pr\'ediction lin\'eaire d\'eterministe
%(\ref{yule-walker2}) est identique au syst\'eme de pr\'ediction
%lin\'eaire de Wiener-Hopf (\ref{WH_pred}) si l'on remplace
%l'autocorr\'elation $R_X [k]$ du processus $X[n]$ par
%l'autocorr\'elation empirique $r_x [k]$ de $x[n]$.
%La r\'esolution du syst\'eme (\ref{yule-walker2}) peut donc
%se faire en utilisant le m\^eme algorithme rapide
%de Levinson-Durbin qui n\'ecessite $O(N^2)$ op\'erations.
%\\
%\\
%\noindent {\bf Algorithme de Levinson-Durbin}\\
%{\it Initialisation:}  $\epsilon_{0} = r_x[0] $.\\
%{\it Boucle:} Pour $m$ allant de $1$ \`a $N$\\
%Calcul de $K_m$\\
%\indent Si $\epsilon_{m-1} > 0$\\
%\begin{equation}
%K_m = \frac {r_X [m] - \sum_{k=1}^{m-1} a_{m-1} [k] r_x[m-k]}
%{\epsilon_{m-1} }
%\end{equation}
%\indent Sinon
%\[
%K_m = 0 .
%\]
%Calcul de $a_m$\\
%\[
%\left\{ \begin{array}{l}
%a_m[k] =  a_{m-1} [k] - K_m a_{m-1} [m-k] ~~,~~
%1 \leq k \leq m-1 \\
%a_m[m] =  K_m
%\end{array}
%\right.
%\]
%Calcul de $\epsilon_m$\\
%\[
%\epsilon_m = \epsilon_{m-1} (1 - K_m^2 ) .
%\]
%\\
%\\
%{\bf Impl\'ementation en treillis}
%L'algorithme de Levinson-Durbin est bas\'e sur le calcul
%d'erreurs de pr\'ediction progressives et r\'etrogrades,
%d\'efinis par des formules identiques \`a
%(\ref{error_formula}) et (\ref{error-retro})
%\begin{equation}
%\label{error_formula-det}
%w^p_m [n+1] = x[n] - \sum_{k=1}^m a_m [k] x[n-k] ,
%\end{equation}
%\begin{equation}
%\label{error-retro-det}
%w_m^r [n] = x[n] - \sum_{k=1}^m a_m[k] x[n+k] ,
%\end{equation}
%avec
%\[
%w_N^p [n] = e[n]
%\]
%et
%\[
%a_N [k] = a[k] .
%\]
%Cet algorithme
%revient \`a effectuer une
%orthogonalization de Gram-Schmidt rapide gr\^ace aux \'equations
%r\'ecurrentes d\'emontr\'ees par le
%Th\'eore\`eme \ref{levinson}
%\begin{equation}
%\label{erreur-rec-det}
%\left\{ \begin{array}{l}
%w_m^p [n] = w_{m-1}^p [n] -
%K_m ~w^r_{m-1} [n-N] \\
%w_m^r [n-N] = w_{m-1}^r [n-N] -
%K_m ~w^p_{m-1} [n]
%\end{array}
%\right.
%\end{equation}
%Ces \'equations
%s'impl\'ementent pas le filtrage en treillis
%illustr\'e en figure \ref{treilli-directe}.
%Ce filtre en treillis a une fonction de
%transfert \'egale \`a $1 - \sum_{k=1}^N a [k] z^{-k}$.
%Il est \'equivalent au filtre en
%\'echelle illustr\'e par la figure \ref{filtre-echelle}.
%\begin{figure}[bhtp]
%%\centerline{
%%	\leavevmode\epsfbox{/home/mallat/X/TREX/figures/SigFig/FIG6.1.EPS.txt}}
%\vspace{6cm}
%\caption{Impl\'ementation en \'echelle d'un filtre dont la fonction de
%transfert est $1 - \sum_{k=1}^N a [k] z^{-k}$.}
%\label{filtre-echelle}
%\end{figure}
%
%
%\begin{figure}[bhtp]
%%\centerline{
%%
%%\leavevmode\epsfbox{/home/mallat/X/TREX/figures/SigFig/FIG6.3.EPS.txt}}
%\vspace{5cm}
%\caption{Impl\'ementation en treillis d'un filtre dont la fonction de
%transfert est $1 - \sum_{k=1}^N a [k] z^{-k}$.}
%\label{treilli-directe}
%\end{figure}
%\\
%\\
%{\bf Filtre pr\'edicteur inverse}
%Pour synth\'etiser $x[n]$ \`a partir de l'excitation $e[n]$
%on utilise le filtre inverse
%\[
%H(z) = \frac {1} {1 - \sum_{k=1}^N a [k] z^{-k}}.
%\]
%Ce filtre peut s'impl\'ementer avec la r\'ealization classique
%en \'echelle illustr\'ee par la figure \ref{AR-echele}
%ou en inversant le
%filtre en treillis comme l'illustre la figure \ref{treilli-inv}
%L'inversion de ce filtre en treillis est bas\'e sur le
%syst\`eme d'\'equations d\'eduit de (\ref{erreur-rec-det})
%\begin{equation}
%\label{erreur-rec-det-inv}
%\left\{ \begin{array}{l}
%w_{m-1}^p [n] = w_m^p [n] + K_m ~w^r_{m-1} [n-N] \\
%w_m^r [n-N] = w_{m-1}^r [n-N] -
%K_m ~w^p_{m-1} [n]
%\end{array}
%\right.
%\end{equation}
%
%Le filtrage en treillis de la figure \ref{treilli-inv}
%peut \^etre reli\'e aux \'equations
%de propagation dans une succession de cylindres
%de m\^eme taille $\Delta$, qui
%mod\'elisent le conduit vocal, comme nous l'avons vu
%dans le paragraphe \ref{conduit-sec}, et illustr\'e par la
%figure \ref{conduit-voc}.
%Les constantes $K_m$
%s'interpr\`etent comme des constantes
%de r\'eflexion pour les ondes directes et inverses,
%\`a l'interface de deux cylindres d'imp\'edence diff\'erentes.
%\begin{figure}[bhtp]
%%\centerline{
%%	\leavevmode\epsfbox{/home/mallat/X/TREX/figures/SigFig/FIG6.2.EPS.txt}}
%\vspace{5cm}
%\caption{Impl\'ementation en \'echelle d'un filtre autor\'egressif.}
%\label{AR-echele}
%\end{figure}
%
%
%\begin{figure}[bhtp]
%%\centerline{
%%	\leavevmode\epsfbox{/home/mallat/X/TREX/figures/SigFig/FIG6.4.EPS.txt}}
%\vspace{5cm}
%\caption{Impl\'ementation en treillis d'un filtre autor\'egressif.}
%\label{treilli-inv}
%\end{figure}
%\\
%\\
%{\bf Stabilit\'e}
%Lors de la synth\`ese d'un son par filtrage AR, il faut
%s'assurer que le filtre est stable, ce qui revient
%\`a montrer que les p\^oles de $H(z)$ ont tous un module
%strictement plus petit que $1$.
%Pour une application de compression, les
%coefficients du filtre $a[k]$ doivent \^etre quantifi\'es afin
%de pouvoir les stocker avec un nombre r\'eduit de bits.
%De petites modifications de ces coefficients peut bouger
%de fa\c{c}on importante les racines du polyn\^ome
%${1 - \sum_{k=1}^N a [k] z^{-k}}$ et introduire des racines
%\`a l'ext\'erieur du cercle unit\'e.
%
%La r\'ealization par filtrage en treillis permet de s'assurer
%simplement que le filtre reste stable.
%Nous avons vu lors du calcul des constantes $K_m$ par
%l'agorithme de Levinson-Durbin que n\'ecessairement
%$K_m \leq 1$. On peut d\'emontrer que
%le filtre AR impl\'ement\'e par le
%treillis inverse de la figure \ref{treilli-inv}
%est strictement stable si
%et seulement si $K_m < 1$ pour $1 \leq m \leq N$.


\subsection{Compression par pr\'ediction lin\'eaire}
\label{compre-LPC}

Pour la t\'el\'ephonie et en particulier les t\'el\'ephones cellulaires,
le d\'ebit d'information est limit\'e par la gamme de fr\'equences
utilisable pour la transmission. Au contraire, la demande augmente
constamment, ce qui n\'ecessite de transmettre toujours plus
de conversations. Une solution est de comprimer le signal de parole
pour augmenter le nombre de conversations sous contrainte
d'un d\'ebit fixe.
La qualit\'e du signal de parole peut \^etre
d\'egrad\'ee mais le codage doit maintenir une bonne intelligibilit\'e
des sons prononc\'es.
Pour des forts taux de compression,
le codage par pr\'ediction lin\'eaire est actuellement la
technique la plus efficace.

Le standard LPC-10 demande 2400bits/s pour coder un
signal de parole \'echantillonn\'e \`a 8kHz. Le signal de parole est
divis\'e sur des fen\^etres de $P = 180$ \'echantillons.
Un filtre AR d'ordre $N=10$ est calcul\'e pour chaque fen\^etre, \`a partir de l'autocorr\'elation du signal, par r\'egression
lin\'eaire. Le voisement et le
pitch sont d\'etermin\'es en
testant l'amplitude des diff\'erences (\ref{AMDF}) \`a intervalles
variables.
Pour les signaux vois\'es, on code aussi l'intervalle $T$ du
pitch, en quantifiant uniform\'ement $\log T$.
Les algorithmes de quantification sont pr\'esent\'es
dans le paragraphe \ref{scalar-quant-sec}.

La quantificaton des coefficients $a[k]$ va d\'eplacer
les p\^oles du filtre AR, qui risquent de sortir du cercle
unit\'e. Le filtre r\'esultant est alors instable.
Pour guarantir la stabilit\'e du filtre AR, on quantifie
plut\^ot les coefficients de r\'eflexion
$\{K_m\}_{1 \leq m \leq N}$,
calcul\'es par l'algorithme de Levinson-Durbin.
Ces coefficients charact\'erisent les valeurs des $a[k]$
pour $1 \leq k \leq N$ et on peut v\'erifier que le
filtre AR est stable si et seulement si $|K_m| < 1$
pour ${1 \leq m \leq N}$. Il suffit donc de s'assurer que les
valeurs quantifi\'ees des $K_m$ restent plus petites que $1$
pour obtenir un filter AR stable.

A la r\'eception, on restaure un signal
dans chaque fen\^etre de 180 \'echantillons en utilisant les param\`etres
du code. Si l'excitation est cod\'ee comme \'etant un bruit blanc,
elle est reproduite avec un
g\'en\'erateur de nombres al\'eatoires. Sinon, on g\'en\`ere
un train d'impulsions
s\'epar\'ees par un intervalle $T$ dont la valeur a \'et\'e cod\'ee.
Cette excitation est ensuite filtr\'ee par le filtre AR
dont les coefficients de r\'eflection ont \'et\'e transmis.

La qualit\'e de ce code peut \^etre am\'elior\'ee en reproduisant
plus fid\`element l'excitation $e[n]$. Au lieu de coder cette
excitation comme un bruit blanc ou un train d'impulsions,
des techniques de quantifications vectorielles
permettent de restaurer des propri\'et\'es importantes de cette
excitation de fa\c{c}on a synth\'etiser des voix de meilleure  qualit\'e.
Ces codes sont appel\'es ``Coded Excited Linear Predictive Filters''
(CELP).

%\chapter{Analyse Temps-Fr\'equence}
\label{chap-ana-temp-fre}

En \'{e}coutant de la musique, 
nous percevons clairement les variations temporelles
des ``fr\'{e}quences'' sonores. 
On met en \'{e}vidence les propri\'{e}t\'{e}s temporelles
et fr\'equentielles
des sons gr\^{a}ce \`{a} 
la transform\'{e}e de Fourier \`a fen\^etre,
qui d\'{e}compose les 
signaux en fonctions \'{e}l\'{e}mentaires bien concentr\'{e}es en 
temps et en fr\'{e}quence. La 
mesure des variations temporelles des
``fr\'{e}quences instantan\'{e}es'' est une application
importante, qui illustre les limitations impos\'ees par
le principe d'incertitude de Heisenberg.

\section{Transform\'ee de Fourier \`a Fen\^etre}
\label{temp-fre-se}

On peut classifier les sons suivant leurs propri\'et\'es 
fr\'equentielles. 
Par exemple, les fr\'equences de r\'esonance du conduit
vocale produisent des ``formants'' qui caract\'erisent les voyelles.
La transform\'ee de Fourier ne peut pas \^etre utilis\'ee car
\[
\hat f (\om) = \int_{-\infty}^{+\infty} f(t) \Exp^{-i\om t} dt
\]
d\'epend des valeurs de $f(t)$ \`a tout instant. Pour diff\'erencier
des sons produits successivement, on d\'efinit une  
transform\'ee de Fourier \`a fen\^etre qui 
s\'epare les diff\'erentes
composantes du signal gr\^ace \`a une fen\^etre translat\'ee.

La fen\^etre $g(t)$
est une fonction paire dont le support est concentr\'e au
voisinage de $0$. 
La transform\'ee de Fourier \`a fen\^etre
au voisinage de $u$, \`a la fr\'equence $\xi$ est d\'efinie par
\[
Sf(u,\xi) = \int_{-\infty}^{+\infty} f(t) \,g(t-u) \,\Exp^{-i\om t} dt .
\]
\\
\\
{\bf Localisation temps-fr\'equence}
On d\'efinit
\[
g_\uxi (t) = g(t-u) \,\Exp^{i\xi t}
\] 
qui peut \^etre interpr\'et\'e comme une ``note de musique''
localis\'ee au voisinage de $t =u$, et autour de 
la fr\'equence $\xi$.
La transform\'ee de Fourier \`a fen\^etre mesure la corr\'elation
entre le signal $f(t)$ et cette note \'el\'ementaire
\begin{equation}
\label{temp-fre-at}
Sf(u,\xi) = \int_{-\infty}^{+\infty} f(t) g_\uxi^*(t) dt .
\end{equation}
Comme $g$ est paire, $g_{u,\xi} (t ) = \Exp^{i \xi t}	g(t-u)$ en 
centr\'{e} sur $u$. 
Le produit $f\, g_\uxi$ isole donc les composantes de
$f$ au voisinage de $u$, et $Sf(u,\xi)$ ne d\'epend que
des propri\'et\'es de $f$ dans ce voisinage.
L'\'{e}talement en temps de $g_\uxi$ est 
ind\'{e}pendant de $u$ et de $\xi$:
\begin{equation}
\label{sigma-t}
\sigma^2_t = \int_{-\infty}^{+\infty} (t-u) ^2 \,
|g_\uxi (t)|^2\, dt = 
\int_{-\infty}^{+\infty} t ^2 \,|g(t)|^2\, dt .
\end{equation}

Si l'on applique la formule de Parseval (\ref{parseval})
\`a (\ref{temp-fre-at}), on obtient une int\'egrale fr\'equentielle
\[
Sf(u,\xi) = \frac 1 {2 \pi} 
\int_{-\infty}^{+\infty} \hat f(\om) \,\hat g_\uxi^* (\om)\, d\om .
\]
La valeur $Sf(u,\xi)$
ne d\'epend donc que du comportement de $\hat f (\om)$ dans
le domaine fr\'equentiel o\`u $\hat g_\uxi^* (\om)$ n'est pas
n\'egligeable.
La transform\'{e}e de Fourier $\hat g$ de $g$ est r\'{e}elle et 
sym\'{e}trique car $g$ est r\'{e}elle et sym\'{e}trique. 
En utilisant les propri\'et\'es (\ref{trans}) et (\ref{modul}),
on montre que
la transform\'ee de Fourier de $g_\uxi (t) = g(t-u) e^{i\xi t}$
peut s'\'ecrire
\[
\hat g_\uxi (\om) = \Exp^{-iu(\om-\xi)} \hat g(\om - \xi) .
\] 
C'est donc une fonction centr\'ee \`a la fr\'equence $\om = \xi$.
Son \'{e}talement fr\'{e}quentiel autour de $\xi$ vaut
\begin{equation}
\label{sigma-om}
\sigma^2_\om = \frac 1 {2 \pi}
\int_{-\infty}^{+\infty} (\om-\xi) ^2\, |\hat g_\uxi (\om)|\, d\om = 
\frac 1 {2 \pi} \int_{-\infty}^{+\infty} \om ^2\, |\hat g(\om)|\, d\om .
\end{equation}
Il est ind\'{e}pendant de $u$ et de $\xi$. 
Dans un plan temps-fr\'equence $(t,\om)$, 
on repr\'esente $g_{\uxi}$ par une 
{\it bo\^{\i}te de Heisenberg} 
de taille $\sigma_t \times \sigma_\om$, 
centr\'{e}e en $(u,\xi)$, comme on peut le voir sur la figure 
\ref{4.2}. 
La taille de cette bo\^{\i}te ne d\'{e}pend pas de $(u,\xi)$, 
ce qui veut dire que la r\'{e}solution de la transform\'{e}e de 
Fourier fen\^{e}tr\'{e}e reste constante sur tout le plan 
temps-fr\'{e}quence.

\begin{figure}[bhtp]
\centerline{
        \epsfxsize=6cm
        \leavevmode\epsfbox{figures/chap1/wf-tfloc.eps}}
\caption{
Les bo\^{\i}tes de Heisenberg de deux atomes de Fourier fen\^{e}tr\'{e}s
$g_\uxi$ and $g_{\nu,\gamma}$.}
\label{4.2}
\end{figure}
\\
\\
\noindent
{\bf Incertitude de Heisenberg}
Pour mesurer les composantes de $f$ et de $\hat f$ dans
des petits voisinages de $u$ et $\xi$ il faut construire
une fen\^etre $g(t)$ qui est 
bien localis\'{e}e dans le temps, et dont l'\'{e}nergie de la 
transform\'{e}e de Fourier est concentr\'{e}e dans un petit domaine 
fr\'{e}quentiel. 
Le Dirac $g(t) = \delta(t)$ a un support ponctuel $t=0$, 
mais sa transform\'{e}e de Fourier  $\hat \delta (\om) = 1$ a une 
\'{e}nergie qui est distribu\'{e}e uniform\'{e}ment sur toutes les 
fr\'{e}quences. On sait que $|\hat g(\om)|$ d\'{e}croit rapidement 
dans les hautes fr\'{e}quences seulement si $g(t)$ est une
fonction qui varie r\'eguli\`erement.
L'\'{e}nergie de $g$ est donc n\'ecessairement 
r\'{e}partie sur un domaine temporel relativement large.

Pour r\'{e}duire l'\'{e}talement temporel de $g$, 
on peut op\'{e}rer un 
changement d'\'{e}chelle de temps d'un facteur $s<1$ sans changer son 
\'{e}nergie totale, soit
\[
g_s (t) = \frac 1 {\sqrt s} g( \frac t s ) ~~~\mbox{avec}~~~
\| g_s \|^2 = \|g \|^2.
\]
La transform\'{e}e de Fourier $\hat g_s ( \om ) = \sqrt s\, \hat	g( s \om )$
est dilat\'{e}e d'un facteur $1/s$, et on perd donc en fr\'{e}quentiel 
ce qu'on a gagn\'{e} en temporel. On voit appara\^{\i}tre un compromis 
entre la localisation en temps et celle en fr\'{e}quence.

Les concentrations en temps en en fr\'{e}quences sont limit\'{e}es par 
le principe d'incertitude d'Heisenberg\index{Principe 
d'incertitude}\index{Heisenberg!principe d'incertitude}. Ce principe 
d'incertitude \`{a} une interpr\'{e}tation, particuli\`{e}rement 
importante en m\'{e}canique quantique, comme une incertitude sur la 
position et l'impulsion d'une particule libre.
Plus $\sigma_t$ et $\sigma_\om$ sont grandes, plus on a 
d'incertitude 
sur la position et l'impulsion de la particule libre.
Le th\'eor\`eme suivant montre que le produit
$\sigma_t \times \sigma_\om$ ne peut \^etre arbitrairement
petit.

\begin{theorem}
[Incertitude de Heisenberg]
\label{uncert}
On suppose que $g \in \LD$ est une fonction centr\'ee en
$0$ et dont la transform\'ee de Fourier est aussi centr\'ee en
$0$:
\[
\int_{-\infty}^{+\infty} t \,|g(t)|^2\, dt = 
\int_{-\infty}^{+\infty} \om \,|\hat g(\om)|^2\, d\om = 0 ~.
\]
Alors les variances definies en (\ref{sigma-t},\ref{sigma-om})
satisfont
\begin{equation}
\label{uncertainty}
\sigma^2_t\, \sigma^2_\om \geq \frac 1 4 .
\end{equation}

Cette in\'{e}galit\'{e} est une \'{e}galit\'{e} si et seulement si il 
existe $(a,b) \in \C^2$  tel que
\begin{equation}
\label{Gabor}
g(t) =  a \, \Exp^{-b t^2} .
\end{equation}
\end{theorem}

{\bf D\'emonstration}
La preuve suivante, due \`{a} Weyl, suppose que
$\lim _{|t| \rightarrow + \infty} \sqrt t g(t) = 0$,
mais le th\'{e}or\`{e}me est vrai pour tout $g \in \LD$. 
Remarquons que
\begin{equation}
\sigma^2_t\, \sigma^2_\om = \frac 1 {2 \pi \|g\|^4}
{\int_{- \infty}^{+ \infty} |t\, g(t)|^2\, dt} \,
{\int_{- \infty}^{+ \infty} |\om\, \hat g( \om )|^2 \,d\om} .
\end{equation}
Comme $i	\om	\hat g(\om)$ est la transform\'{e}e de Fourier de $g'(t)$,
l'identit\'{e} de Plancherel (\ref{plancherel}) appliqu\'{e}e \`{a} 
$i	\om	\hat g(	\om	)$ donne
\begin{equation}
\label{tobeplisch}
\sigma^2_t \sigma^2_\om = \frac 1 {\|g\|^4}
{\int_{- \infty}^{+ \infty} |t \,g(t)|^2 \,dt} \,
{\int_{- \infty}^{+ \infty} |g'( t )|^2\, dt} .
\end{equation}
L'in\'{e}galit\'{e} de Schwarz implique
\begin{eqnarray*}
\sigma^2_t \sigma^2_\om & \geq & \frac 1 {\|g\|^4}
\left[ {\int_{- \infty}^{+ \infty} |t\, g'(t) \,g^* (t)|\, dt} \right]^2\\
& \geq & \frac 1 {\|g\|^4} \left[
{\int_{- \infty}^{+ \infty} \frac t 2\, [ g'(t)\, g^* (t) 
+ {g'}^*(t)\, g(t)]\, dt}
\right]^2\\  
&\geq &\frac 1 { 4 \|g\|^4}
\left[ {\int_{- \infty}^{+ \infty} t \,(|g(t)|^2)'\, dt} \right]^2 .
\end{eqnarray*}
Comme
$\lim _{|t| \rightarrow + \infty} \sqrt t \,g(t) = 0$,
on obtient, apr\`{e}s int\'{e}gration par parties
\begin{equation}
\sigma^2_t \sigma^2_\om \geq \frac 1 { 4 \|g\|^4}
\left[ {\int_{- \infty}^{+ \infty} |g(t)|^2 \,dt} \right]^2 
= \frac 1 4 .
\end{equation}
Pour atteindre l'\'{e}galit\'{e}, il faut que l'in\'{e}galit\'{e} de 
Schwarz appliqu\'{e}e \`{a} (\ref{tobeplisch}) soit elle-m\^{e}me une 
\'{e}galit\'{e}. Cela implique qu'il existe $b \in \C$ tel que
\begin{equation}
g'(t) = -2\,b\, t\, g(t) .
\end{equation}
Il existe donc $a \in \C$ tel que
$g(t) = a \,\Exp^{-b t^2}$.
Les in\'{e}galit\'{e}s suivantes dans la preuve sont alors des 
\'{e}galit\'{e}s, ce qui fait qu'on atteint effectivement le minorant.
$\Box$


En m\'{e}canique quantique, ce th\'{e}or\`{e}me montre qu'on ne peut 
arbitrairement r\'{e}duire l'incertitude \`{a} la fois sur la position 
et sur l'impulsion d'une particule libre. Les gaussiennes 
(\ref{Gabor}) ont une localisation minimale \`{a} la fois en 
temps et en fr\'{e}quences.
\\
\\
{\bf Spectrogramme}
On peut associer \`a la transfrom\'ee de Fourier \`a fen\^etre
une densit\'{e} 
d'\'{e}nergie qu'on appelle {\it spectrogramme}, et qu'on note $P_S$:
\begin{equation}
\label{SpectroDef}
P_S f (u,\xi) = |Sf(u,\xi)|^2 
= \left|\int_{- \infty}^{+ \infty} f(t)\, g(t-u)\, \Exp^{- i \xi t} \,dt 
\right|^2 .
\end{equation}
Il mesure l'\'{e}nergie de $f$ et de $\hat f$ dans le voisinage 
temps-fr\'{e}quence ou l'\'energie de $g_\uxi$ est concentr\'ee.

\begin{Examples} 
\item 
Une sinuso\"{\i}de $f(t)=	\Exp^{i	\xi_0 t}$, dont la transform\'{e}e 
de Fourier est le Dirac $\hat f(\om) = 2 \pi \delta	(\om - \xi_0)$ a 
pour transform\'{e}e de Fourier \`a fen\^{e}tr{e}
\[
Sf(u,\xi) = \Exp^{-iu(\xi - \xi_0)} \, \hat g(\xi - \xi_0). 
\]
Son \'{e}nergie est r\'{e}partie sur l'intervalle fr\'{e}quentiel
$[\xi_0 - \frac {\sigma_\om} 2 , \xi_0 + \frac {\sigma_\om} 2 ]$.

\item 
La transform\'{e}e de Fourier fen\^{e}tr\'{e}e d'un Dirac
$f(t) = \delta	(t - u_0)$ vaut
\[
Sf(u,\xi) = \Exp^{-i \xi u_0} \,g(u_0 - u) .
\]
Son \'{e}nergie est localis\'{e}e dans l'intervalle temporel
$[u_0 - \frac {\sigma_t} 2 , u_0 + \frac {\sigma_t} 2 ]$.

\item 
Dans la figure \ref{WFTChirps}, on voit le spectrogramme 
d'un signal qui a une composantes dont la ``fr\'equence
instantan\'ee''
augmente lin\'eairement dans le temps (chirp lin\'eaire)
et une seconde composante
dont la fr\'equence d\'ecroit de fa\c{c}on quadratique 
dans le temps (chirp quadratique). S'ajoute \`a cela 
deux gaussiennes modul\'{e}es. 
On a calcul\'{e} le spectrogramme avec une 
fen\^{e}tre gaussienne dilat\'{e}e d'un facteur $s = 50$. 
Le chirp lin\'{e}aire a des coefficients de grande amplitude 
le long de la trajectoire de sa fr\'{e}quence instantan\'ee.
Le chirp quadratique donne des grands coefficients le long 
d'une parabole. Les deux gaussiennes modul\'{e}es donnent deux taches 
fr\'{e}quentielles \`{a} haute et basse fr\'{e}quence, en
 $u =	512$ et $u	= 896$.

\item La figure \ref{spectrogram-parole} 
montre le spectrogramme du son ``greasy'' dont le graphique
est donn\'e au-dessus. L'amplitude de $|Sf(u,\xi)|^2$ est
d'autant plus grande que l'image du spectrogramme est sombre.
Le ``ea'' tout comme le ``y'' sont
des sons vois\'es dont les formants apparaissent clairement sur
le spectrogramme. Le ``s'' est un son non-vois\'e dont l'\'energie
est diffus\'e en hautes fr\'equences.
\end{Examples}

\vspace{0cm}\setlength{\tabcolsep}{0cm} % Separation entre les lignes d'images
\setlength{\fboxsep}{0cm} % Separation entre la boite et l'image
\avecboite = 0
\begin{figtab}
\begin{figrow}{4}
{{\label{WFTChirps}
Le signal comprend un chirp lin\'{e}aire de fr\'{e}quence croissante, 
un chirp quadratique de fr\'{e}quence d\'{e}croissante, et deux 
gaussiennes modul\'{e}es situ\'{e}es en $t=512$ et $t=896$.
(a) Spectrogramme $P_S	f(u,\xi)$. Les axes horizontaux et
verticaux correspondent respectivement au temps $u$ et \`a la
fr\'equence $\xi$. Les points sombres correspondent
\`{a} des coefficients de grande amplitude.
(b) Phase complexe de $Sf(u,\xi)$ dans les r\'{e}gions o\`{u} le 
module de $P_S	f (u,\xi)$ est non nul.}}\\
\figentry{8cm}{0cm}{figures/chap4/WFTChirps.eps}\\

\centry{(a)}\\

\figentry{8cm}{0cm}{figures/chap4/WFTPhaseChirps.eps}\\

\centry{(b)}
\end{figrow} 
\end{figtab}

\setlength{\tabcolsep}{0cm} % Separation entre les lignes d'images
\setlength{\fboxsep}{0cm} % Separation entre la boite et l'image
\avecboite = 0
\begin{figtab}
\begin{figrow}{2} {{Le graphique du dessus correspond au son ``greasy''
enregistr\'e \`a 8kHz. Son spectrogramme $|Sf(u,\xi)|^2$
est montr\'e au-dessous, dans le plan temps-fr\'equence.}
\label{spectrogram-parole}}\\

\figentry{10cm}{0cm}{/home/mallat/X/TREX/figures/gribonval/greasy.ps} \\
\avecboite = 1
\figentry{8cm}{10cm}{/home/mallat/X/TREX/figures/gribonval/sonog1.ps} 

\end{figrow} 
\end{figtab} 
\\
\\
\noindent
{\bf Compl\'etude et stabilit\'e}
Lorsque les coordonn\'{e}es temps-fr\'{e}quence $(u,\xi)$ parcourent 
$\R^2$, les bo\^{\i}tes de Heisenberg des atomes $g_\uxi$ recouvrent 
tout le plan temps-fr\'{e}quence. On peut donc s'attendre \`{a} 
pouvoir reconstituer $f$ \`{a} partir de sa transform\'{e}e de 
Fourier \`a fen\^{e}tre $Sf(u,\xi)$. Le th\'{e}or\`{e}me suivant nous fournit 
une formule de reconstruction et montre qu'on a conservation de 
l'\'{e}nergie.

\begin{theorem}
\label{window-four-form}
\label{formual-WF}
Si $f \in \LD$ 
alors
\begin{equation}
\label{inverse-WF}
f(t) = \frac 1 {2 \pi} \int_{-\infty}^{+\infty}  \int_{-\infty}^{+\infty} 
Sf(u,\xi) \,g(t-u) \,\Exp^{i \xi t} \, d\xi \, du 
\end{equation}
et 
\begin{equation}
\label{energy-WF}
\int_{-\infty}^{+\infty} |f(t)|^2 \,dt = 
\frac 1 {2 \pi} \int_{-\infty}^{+\infty}  \int_{-\infty}^{+\infty} 
|S f(u,\xi)|^2 \, d\xi \, du .
\end{equation}
\end{theorem}

{\bf D\'emonstration\ }
On commence par la preuve de la formule de reconstruction 
(\ref{inverse-WF}). Appliquons la formule de Fourier Parseval 
(\ref{parseval}) \`{a} l'int\'{e}grale (\ref{inverse-WF}) en la 
variable $u$. On calcule la transform\'{e}e de Fourier de 
$f_\xi (u) = Sf(u,\xi)$ en $u$ en remarquant que
\begin{equation}
\label{wind-filt}
Sf(u,\xi) = \Exp^{-iu\xi} 
\int_{- \infty}^{+ \infty} f(t) \,g(t-u)\, \Exp^{i \xi (u-t)}\, dt
= \Exp^{-iu\xi} \,f \star g_\xi (u) ,
\end{equation}
avec $g_\xi	(t)	= g(t) \Exp^{i \xi t}$, car $g(t) =	g(-t)$. Sa 
transform\'{e}e de Fourier vaut donc
\[
\hat f_\xi (\om) = \hat f(\om + \xi) \,\hat g_\xi (\om + \xi) = 
\hat f(\om + \xi)\, \hat g( \om ) .
\]
La transform\'{e}e de Fourier de $g(t-u)$ en $u$ vaut 
$\hat	g(\om) \Exp^{-it\om}$. On a donc
\begin{eqnarray*}
\frac 1 {2 \pi} \left( \int_{-\infty}^{+\infty}  
\int_{-\infty}^{+\infty} Sf(u,\xi) \,g(t-u) \,
\Exp^{i \xi t}\,  du \right) d\xi & = & \\
\frac 1 {2 \pi} \int_{-\infty}^{+\infty}  \left(
\frac 1 {2 \pi} \int_{-\infty}^{+\infty}  
\hat f(\om+\xi)\, |\hat g(\om)|^2\, \Exp^{it(\om+\xi)}\, d \om \right)  d\xi &  . &
\end{eqnarray*}
Si  on peut appliquer le th\'{e}or\`{e}me de Fubini pour 
changer l'ordre d'int\'{e}gration. Le th\'{e}or\`{e}me de 
transform\'{e}e de Fourier inverse montre que
\[
\frac 1 {2 \pi} \int_{-\infty}^{+\infty}  
\hat f(\om + \xi) \,\Exp^{it(\om+\xi)}\,  d\xi = f(t) .
\]
Comme
$\frac 1 {2 \pi} \int_{-\infty}^{+\infty}  
|\hat g(\om)|^2\,  d\om  = 1,$
on en d\'{e}duit (\ref{inverse-WF}). Si $\hat	f \nnin	\,\LU$,
on d\'{e}montre la formule \`{a} partir de l\`{a} gr\^{a}ce \`{a}
un argument de densit\'{e}.

Occupons nous maintenant de la conservation de l'\'{e}nergie 
(\ref{energy-WF}). Comme la transform\'{e}e de Fourier en $u$ de 
$Sf(u,\xi)$ est $\hat	f(\om +	\xi) \,\hat	g(\om )$,
on obtient, en appliquant la formule de Plancherel au membre de 
droite de (\ref{energy-WF}):
\[
\frac 1 {2 \pi} \int_{-\infty}^{+\infty}  \int_{-\infty}^{+\infty} 
|Sf(u,\xi)|^2 \,du\, d\xi = 
\frac 1 {2 \pi} \int_{-\infty}^{+\infty}  
\frac 1 {2 \pi} \int_{-\infty}^{+\infty}  
|\hat f(\om + \xi) \,\hat g(\om ) |^2 \,d\om\, d\xi .
\]
On peut appliquer le th\'{e}or\`{e}me de Fubini, et la formule de 
Plancherel montre que
\[
\frac 1 {2 \pi} \int_{-\infty}^{+\infty}  
|\hat f(\om + \xi) |^2\,  d\xi = \|f \|^2,
\]
ce qui implique (\ref{energy-WF}). $\Box$


On peut r\'{e}\'{e}crire la formule de reconstruction 
(\ref{inverse-WF}) sous la forme
\begin{equation}
f(t) = \frac 1 {2 \pi} \int_{-\infty}^{+\infty}  \int_{-\infty}^{+\infty} 
\lb f,g_\uxi\rb \, g_\uxi (t)\,  d\xi\, du .
\end{equation}
Cette formule ressemble \`{a} celle d'une d\'{e}composition sur une base 
orthogonale, mais ce n'est pas le cas, car la famille $\{g_\uxi \}_\uxiR$ 
est largement redondante dans $\LD$. La seconde identit\'{e} 
(\ref{energy-WF})  justifie qu'on interpr\`{e}te le spectrogramme 
$P_S f(u,\xi)=|Sf(u,\xi)|^2$ comme une densit\'{e} d'\'{e}nergie, car 
son int\'{e}grale en temps-fr\'{e}quence est \'{e}gale \`{a} 
l'\'{e}nergie du signal.
\\
\\
{\bf Discr\'{e}tisation}
La discr\'{e}tisation et le calcul rapide de la transform\'{e}e de 
Fourier \`a fen\^{e}tre rel\`{e}ve des m\^{e}mes id\'{e}es que la 
discr\'{e}tisation de la transform\'{e}e de Fourier classique, 
d\'{e}crite pr\'{e}c\'edemment dans le paragraphe \ref{finite-sig}. On
consid\`{e}re des signaux discrets de p\'{e}riode $N$. On prend comme 
fen\^{e}tre $g[n]$ un signal discret sym\'{e}trique et de p\'{e}riode 
$N$ et de norme unit\'{e} $\|g \| =	1.$ On d\'{e}finit les atomes de 
Fourier fen\^{e}tr\'{e}s discrets
\index{Transform\'{e}e de Fourier!fen\^{e}tr\'{e}e!discr\`{e}te}
\index{Fourier!transform\'{e}e de!fen\^{e}tr\'{e}e discr\`{e}te}
par
\[
g_\ml [n] = g[n-m] \, \Exp^{\frac {i 2 \pi  ln} N}. 
\]
La transform\'{e}e de Fourier discr\`{e}te $g_\ml$ a pour valeurs
\[
\hat g_\ml [k] = \hat g[k-l] \, \Exp^{\frac {-i 2 \pi  m (k-l)} N}. 
\]
La transform\'{e}e de Fourier fen\^{e}tr\'{e}e discr\`{e}te d'un 
signal de p\'{e}riode $N$ est
\begin{equation}
\label{DisWdinFTa}
Sf[m,l] = \lb f,g_\ml\rb  = \sum_{n=0}^{N-1} 
f[n]\, g[n-m]\, \Exp^{\frac {-i2\pi ln} N} ,
\end{equation}
Pour chaque $0 \leq	m <	N$, on calcule $Sf[m,l]$ pour $0 \leq l < N$ 
par transform\'{e}e de Fourier discr\`{e}te sur $f[n] g[n-m]$. On 
r\'{e}alise ce calcul au moyen de $N$ FFT de taille $N$, ce qui donne 
un total de $O(N^2 \log_2 N)$ op\'{e}rations. C'est cet algorithme qui 
a servi pour le calcul des figures \ref{WFTChirps} et 
\ref{spectrogram-parole}.


\section{Fr\'{e}quence instantan\'{e}e}
\label{inst-fre-sec}

Dans un morceau de musique, on distingue plusieurs fr\'{e}quences 
variant dans le temps. Il reste \`{a} d\'{e}finire la notion de 
fr\'{e}quence instantan\'{e}e.  Afin d'estimer plusieurs 
fr\'{e}quences instantan\'{e}es, on s\'{e}pare les composantes 
fr\'{e}quentielles \`{a} l'aide d'une transform\'{e}e de 
r\'{e}solution suffisante en fr\'{e}quence, mais \'{e}galement en 
temps, de mani\`{e}re \`{a} pouvoir r\'{e}aliser des mesures variables 
dans le temps. On \'{e}tudie la mesure des fr\'{e}quences 
instantan\'{e}es par des transform\'{e}es de Fourier 
fen\^{e}tr\'{e}es.
\\
\\
{\bf 
Fr\'{e}quence instantan\'{e}e analytique\ } 
Un cosinus modul\'{e}
\[
f(t) = a \cos (w_0 t + \phi_0 ) = a \cos \phi(t)
\]
a une fr\'{e}quence $\om_0$ \'{e}gale \`{a} la d\'{e}riv\'{e}e de la 
phase $\phi(t) =	w_0	t +	\phi_0$. Afin de g\'{e}n\'{e}raliser cette 
notion, on \'{e}crit les signaux r\'{e}els $f$ comme ayant une 
amplitude $a$ et un phase $\phi$ variant dans le temps:
\begin{equation}
\label{phase-fre}
f(t) = a(t)\, \cos \phi(t) ~~\mbox{
avec}~~ a(t) \geq 0~.
\end{equation}
On d\'{e}finit {\it la fr\'{e}quence instantan\'{e}e} comme la 
d\'{e}riv\'{e}e de la phase:\index{Analytique!signal}
\[
\om(t) = \phi' (t) \geq 0 ~.
\]
On peut se ramener \`{a} une d\'{e}riv\'{e}e positive en jouant sur le 
signe de $\phi (t)$. Il convient n\'{e}anmoins d'\^{e}tre prudent car
il existe de nombreuses valeurs pour $a(t)$ et de  $\phi (t)$,
et $\om (t)$ n'est donc pas d\'{e}fini de mani\`{e}re unique pour un 
$f$ donn\'{e}.

On obtient une 
d\'{e}compostion particuli\`{e}re de type (\ref{phase-fre}) 
en calculant la partie analytique $f_a $ de $f$, 
dont la transform\'{e}e de Fourier est d\'efinie par
\begin{equation}
\label{analyt-par-f}
\hat f_a (\om) = \left\{
\begin{array}{ll}
2\, \hat f(\om) & \mbox{si $\om \geq 0$}\\
0 & \mbox{si $\om < 0$}
\end{array} 
\right. .
\end{equation}
On dit que le signal complexe $f_a (t)$ est analytique car
on peut d\'emontrer qu'il a une extension analytique sur
le demi plan complexe sup\'erieur.
Par ailleurs on peut verifier que
$f = \Real [f_a ]$ car
$\hat f (\om) = \frac 1 2 (\hat f_a (\om) + \hat f_a^* (-\om))$.

On peut repr\'{e}senter $f_a$ en s\'{e}parant le module de la phase 
complexe
\[
f_a (t) = a (t)\, \Exp^{i \phi (t)}~.
\]
On en d\'{e}duit
\[
f(t) = a(t) \,\cos \phi(t) .
\]
On dira que $a(t)$ est l'amplitude {\it analytique} de $f(t)$, et 
$\phi'(t) $ sa fr\'{e}quence analytique
instantan\'{e}e; elles sont d\'{e}finies 
de mani\`{e}re unique.

\begin{Example}
\item Si $f(t) = a(t)\, \cos (\om_0 t + \phi_0)$, alors
\[
\hat f (\om) =  \frac 1 2
\left(
\Exp^{i \phi_0} \, \hat a(\om - \om_0) +
\Exp^{-i \phi_0} \, \hat a(\om +\om_0)
\right).
\]
Si les variations de $a(t)$ sont lentes en comparaison de la 
p\'{e}riode $\frac {2 \pi} {\om_0}$, ce qu'on peut obtenir en 
for\c{c}ant le support de
$\hat a (\om)$ \`{a} \^{e}tre dans $[-\om_0,\om_0]$, alors
\[
\hat f_a (\om) = \Exp^{i \phi_0} \, \hat a(\om - \om_0)
\]
d'o\`{u} $f_a (t) = a(t)\, \Exp^{i (\om_0 t + \phi_0)}$.
\end{Example}

Si $f$ est un  signal constitu\'{e} de la somme de deux 
sinuso\"{\i}des:
\[
f (t) =  a \cos({ \om_1 t}) + a \cos({ \om_2 t}) ,
\]
alors
\[
f_a (t) = a \,\Exp^{i \om_1 t} + a \,\Exp^{i \om_2 t} =
 a \, \cos \left( \frac {\om_1 - \om_2} 2\, t \right)
\,\Exp^{i \frac {\om_1 + \om_2} 2 t}.  
\]
La fr\'{e}quence instantan\'{e}e vaut
$\phi'(t) = \frac {\om_1 + \om_2} 2$
et l'amplitude instan\'ee est
\[
a(t) = a \, \left| \cos \left(\frac {\om_1 - \om_2} 2 \,t\right)\right| .
\]
Ce r\'{e}sultat n'est pas satisfaisant parce qu'il ne montre pas que 
le signal est compos\'{e} de deux sinuso\"{\i}des de m\^{e}me 
amplitude. On a obtenu une fr\'{e}quence moyenne. Nous allons 
expliquer dans la section qui suit comment mesurer les 
fr\'{e}quences instantan\'{e}es de plusieurs composantes spectrales en 
les s\'{e}parant \`{a} l'aide de transform\'{e}es de Fourier 
\`a fen\^{e}tre.
\\
\\
{\bf Modulation de fr\'{e}quence\ }
En communications, on peut transmettre l'information \`{a} travers 
son amplitude $a(t)$ (modulation d'amplitude) ou sa fr\'{e}quence 
instantan\'{e}e $\phi'(t)$ (modulation de fr\'{e}quence).
La modulation de fr\'{e}quence est plus robuste en pr\'{e}sence de 
bruits blancs gaussiens additifs. De plus, elle r\'{e}siste 
mieux aux 
interf\'{e}rences entre chemins 
multiples, qui d\'{e}truisent l'information d'amplitude. Une 
modulation de fr\'{e}quence envoie un message $m(t)$ \`{a} 
travers un signal
\[
f(t) = a \cos \phi (t)~~~\mbox{with}~~~ 
\phi'(t) = \om_0 + k \,m(t) .
\]
La largeur de bande de $f$ est proportionnelle \`{a} $k$. On 
ajuste cette constante en fonction des bruits de transmission et de la 
bande passante disponible. A la r\'{e}ception, on r\'{e}cup\`{e}re le 
message $m(t)$ gr\^{a}ce \`{a} une d\'{e}modulation de fr\'{e}quence 
qui calcule la fr\'{e}quence instantan\'{e}e $\phi'(t)$.
\\
\\
{\bf Mod\`{e}les de sons additifs\ }
On peux mod\'{e}liser les sons musicaux et les phon\`{e}mes comme des somme de
{\it partielles} sinuso\"{\i}dales: \index{Partielle}
\begin{equation}
\label{NewSoundMOd}
f(t) = \sum_{k=1}^K f_k (t) =  \sum_{k=1}^K a_k (t) \cos \phi_k (t)~ ,
\end{equation}
o\`{u} $a_k$ et $\phi^\prime_k$	sont lentement variables.
De telles d\'{e}compositions sont 
utilis\'{e}es pour reconna\^{i}tre des formes et modifier des sons.
Le paragraphe \ref{sec-wind-four-ridges} explique
comment calculer les $a_k$ et les fr\'{e}quences instantan\'{e}es 
$\phi_k'$ de chaque partielle, 
dont on d\'{e}duit la phase $\phi_k$ par 
int\'{e}gration.

Pour comprimer de son $f$ d'un facteur $\alpha$ dans le temps, et sans 
modifier les valeurs des $\phi_k'$ et des $a_k$, on synth\'{e}tise
\begin{equation}
\label{Accelerate-sound}
g (t) =  \sum_{k=1}^K a_k (\alpha\, t)\,
\cos \Bigl( \frac 1 \alpha \,\phi_k (\alpha\, t) \Bigr) .
\end{equation}
Les partielles de $g$ en $t = \alpha \, t_0$ et les partielles de $f$ 
en $t = t_0$ ont les m\^{e}mes amplitudes et les m\^{e}mes 
fr\'{e}quences instantan\'{e}es. Pour $\alpha >	1$, le son $g$ est 
plus court que $f$ tout en \'{e}tant per\c{c}u comme ayant le 
m\^{e}me ``contenu fr\'{e}quentiel'' que $f$.

On op\`{e}re un d\'{e}placement fr\'{e}quentiel en multipliant chaque 
phase par une constante $\alpha$:
\begin{equation}
\label{Transpose-sound}
g (t) =  \sum_{k=1}^K b_k (t)\,
\cos \Bigl( \alpha\, \phi_k (t ) \Bigr) .
\end{equation}
La fr\'{e}quence instantan\'{e}e de chaque partielle vaut maintenant 
$\alpha \, \phi_k'(t)$. Pour calculer les nouvelles amplitudes $b_k (t)$, 
on utilise un mod\`{e}le de r\'{e}sonnance, qui suppose que ces 
amplitudes sont les \'{e}chantillons d'une enveloppe fr\'{e}quentielle 
lisse $F(t,	\om)$:
\[
a_k (t) = F\Bigl(t, \phi_k'(t) \Bigr) ~~\mbox{et}~~
b_k (t) = F\Bigl(t, \alpha\, \phi_k'(t) \Bigr) ~.
\]
En traitement de la parole, cette enveloppe est compos\'e de plusieurs
{formants\/}. 
Il est fonction du type de phon\`{e}me qui a \'{e}t\'{e} 
prononc\'{e}. Comme $F(t,\om)$ est une fonction r\'{e}guli\`{e}re\index{Formant}
de $\om$, on calcule son amplitude en $\om	= \alpha\, \phi_k'(t)$ par 
interpolation des valeurs $a_k (t)$	correspondant \`{a} $\om = \phi_k'(t)$.

\subsection{Cr\^{e}tes de transform\'ee de Fourier \`a fen\^{e}tre}
\label{sec-wind-four-ridges}

Le spectrogramme $P_S f(u,\xi)= |Sf(u,\xi)|^2$
mesure l'\'{e}nergie de $f$ dans un voisinage temps-fr\'{e}quence de $(u,\xi)$.
L'algorithme de cr\^{e}tes calcule les fr\'{e}quences 
instantan\'{e}es \`{a} partir des maxima locaux de $P_S	f(u,\xi)$
\cite{torresani}.

On calcule la transform\'{e}e de Fourier \`a fen\^{e}tre \`{a} 
l'aide d'une fen\^{e}tre sym\'{e}trique $g(t) =	g(-t)$ de support
$[-\frac	1 2	, \frac	1 2]$. La transform\'{e}e de Fourier $\hat g$ de 
$g$ est une fonction sym\'{e}trique r\'{e}elle avec
$|\hat g	(\om)| \leq	\hat g (0)$	pour tout	$\om \in \R$.
Son maximum $\hat g(0) =	\int_{1/2}^{-1/2} g(t)\,dt$ est de l'ordre de 
1. 
On normalise la fen\^{e}tre $g$ de mani\`{e}re 
\`{a} avoir $\|g\| = 1$. 
On d\'efinit la largeur de bande $\Delta \om$ de
$\hat g$ par
\begin{equation}
\label{Band-widt-g}
|\hat g(\om)| \ll 1 ~~~
\mbox{
pour}~~~ |\om| \geq \Delta \om.
\end{equation}

A une \'{e}chelle donn\'{e}e $s$,
$g_s	(t)	= \frac	1 {\sqrt s}	g(\frac	t s)$ a un support de taille $s$ 
et est de norme unit\'{e}. Les atomes de Fourier correspondants sont
\[
g_{s,u,\xi} (t) = g_s (t-u) \, \Exp^{i \xi t} .
\]
La transform\'{e}e de Fourier \`a fen\^{e}tre est alors
\begin{equation}
\label{scale-wind-Four}
Sf(u,\xi) \,=\, \lb f,g_\suxi \rb \, =\,
\int_{-\infty}^{+\infty} f(t)\, g_s (t-u)\, \Exp^{-i\xi t} \,dt.
\end{equation}
Le th\'{e}or\`{e}me suivant exprime $Sf(u,\xi)$ en fonction de la 
fr\'{e}quence instantan\'{e}e de $f$.

\begin{theorem}
\label{approx-window}
Soit $f(t) = a(t)\, \cos \phi (t)$.
Si les variations de $a(t)$ et $\phi'(t)$ sont
n\'egligeables sur le support $[u-s/2,u+s/2]$
de $g_s(t-u)$ et que $\phi'(u) \geq s^{-1} \, \Delta \om$
alors pour tout $\xi \geq 0$ on a
\begin{equation}
\label{wt-est}
Sf(u,\xi) 
\approx \frac {\sqrt s} 2  \,a(u)\,\Exp^{ i (\phi(u) - \xi u)}
\hat g\Bigl(s[\xi - \phi'(u)]\Bigr) .
\end{equation}
\end{theorem}

{\bf D\'emonstration\ }
Remarquons que
\begin{eqnarray*}
\lb f,g_\suxi \rb  &=&
\int_{-\infty}^{+\infty} a(t) \,\cos \phi(t) \,g_s (t-u)\, \Exp^{-i \xi
t}\,dt \\
&=& \frac 1 2
\int_{-\infty}^{+\infty} a(t) \,\left(
\Exp^{i \phi(t)} + \Exp^{-i \phi(t)}\right)
 \,g_s (t-u)\, \Exp^{-i \xi t}\,dt\\
&=& I(\phi) + I(-\phi) .
\end{eqnarray*}
Commen\c{c}ons par \'{e}tudier
\begin{eqnarray*}
I(\phi)  &=& \frac 1 2
\int_{-\infty}^{+\infty} a(t) \,\Exp^{i\phi(t)} \,g_s (t-u)\, \Exp^{-i \xi
t}\,dt \\
&=& \frac 1 2
\int_{-\infty}^{+\infty} a(t+u)\, \Exp^{i \phi(t+u)}\, g_s (t)\, \Exp^{-i \xi
(t+u)}\, dt .
\end{eqnarray*}
Comme $a(t+u) \approx a(u)$ et
$\phi (t+u) \approx \phi(u) + t \phi'(u)$ pour $|t| \leq s/2$
il vient
\[
I(\phi) \approx
\frac {a(u)\, \Exp^{ i (\phi(u) - \xi u)}  } 2
\int_{-\infty}^{+\infty}  g_s (t)\, \Exp^{-i t (\xi - \phi'(u))}
\, dt = 
\frac {a(u)\, \Exp^{ i (\phi(u) - \xi u)}  } 2 \,
\hat g_s (\xi - \phi'(u)) .
\]
Puisque $\hat g_s (\om) = \sqrt s \, \hat g(s \om)$ 
\begin{equation}
\label{wt-est2}
I(\phi) \approx
\frac {\sqrt s} 2  \,a(u)\,\Exp^{ i (\phi(u) - \xi u)}
\hat g\Bigl(s[\xi - \phi'(u)]\Bigr) .
\end{equation}
De m\^eme
\[
I(-\phi) \approx
\frac {\sqrt s} 2  \,a(u)\,\Exp^{ i (-\phi(u) - \xi u)}
\hat g\Bigl(s[\xi + \phi'(u)]\Bigr) .
\]
Comme $\xi \geq 0$, $s[\xi + \phi'(u)] \geq \Delta \om$ donc
$\hat g\Bigl(s[\xi + \phi'(u)]\Bigr) \ll 1$. On peut donc
n\'egliger $I(-\phi)$ et l'on d\'eduit (\ref{wt-est}) de
(\ref{wt-est2}).
$\Box$
\\
\\
{\bf Points de cr\^{e}te\ } 
Supposons que $a(t)$ et $\phi'(t)$ ont des variations 
negligeables sur 
les intervalles de taille $s$, et que $\phi'(t) \geq \Delta \om/s$.
Comme $|\hat	g(\om)|$ est maximal en $\om	= 0$,
(\ref{wt-est}) montre que, pour chaque $u$, le spectrogramme
$|Sf(u,\xi)|^2 =	|\lb f , g_\suxi \rb|^2$ est maximal en
$\xi(u) = \phi'(u)$. Les points correspondants $(u,\xi(u))$ du plan 
temps-fr\'{e}quence sont appel\'{e}s des {\it cr\^{e}tes}.
Aux points de cr\^{e}te, (\ref{wt-est})	devient
\begin{equation}
\label{ridg-wind-fou-ap}
Sf(u, \xi ) = \frac {\sqrt s  } 2 \,a(u)\, \Exp^{ i
(\phi(u) - \xi u)} \, \hat g(0) 
\end{equation}
La fr\'{e}quence de cr\^{e}te donne la fr\'{e}quence instantan\'{e}e
$\xi(u) = \phi'(u)$, et on calcule $\xi(u) = \phi'(u)$ avec
\begin{equation}
\label{compute-freq}
a(u) = \frac {2\, \Bigl|Sf\Bigl(u,\xi (u)\Bigr)\Bigr|}
{\sqrt s\, |\hat g(0)|} .
\end{equation}
Soit $\Phi_S (u,\xi)$ la phase complexe de  $Sf(u,\xi)$.
L'expression 
(\ref{ridg-wind-fou-ap}) montre que les points de cr\^{e}te sont 
\'{e}galement des points o\`{u} la phase est stationnaire:
\[
\frac {\partial \Phi_S  (u,\xi )} {\partial u} = \phi' (u) -
\xi  = 0 .
\]
La v\'{e}rification de la stationnarit\'{e} de la phase permet de 
mieux situer les cr\^{e}tes.
\\
\\
\noindent{\bf Fr\'{e}quences multiples\ }
Lorsque le signal contient plusieurs lignes spectrales de 
fr\'{e}quences suffisamment distinctes, la transform\'{e}e de Fourier 
fen\^{e}tr\'{e}e en s\'{e}pare les diverses composantes, et les 
cr\^{e}tes permettent de d\'{e}tecter l'\'{e}volution temporelle de chacune d'entre 
elles. Consid\'{e}rons
\[
f(t) = a_1 (t) \cos \phi_1 (t) + a_2 (t) \cos \phi_2 (t) ,
\]
o\`{u} $a_k (t)$ et $\phi^\prime_k (t)$ sont \`{a} variations petites 
sur les intervalles de largeur $s$, et o\`{u} on suppose
$s \phi^\prime_k	(t)	\geq \Delta	\om$.
Comme la 
transform\'{e}e de Fourier fen\^{e}tr\'{e}e est lin\'{e}aire, on 
applique (\ref{wt-est}) \`{a} chaque composante spectrale:
\begin{eqnarray}
\label{twospectrems}
Sf(u, \xi ) & = & 
\frac {\sqrt s } 2  \,a_1 (u)\,
\hat g\Bigl(s[\xi - \phi_1 '(u)]\Bigr) \,\Exp^{ i (\phi_1 (u) - \xi u)} \nonumber \\
&  & +
\frac {\sqrt s}  2  \,a_2 (u)\,
\hat g\Bigl(s[\xi - \phi_2 '(u)]\Bigr) \,\Exp^{ i (\phi_2 (u) - \xi u)} .
\end{eqnarray}
On arrive \`{a} s\'{e}parer les composantes spectrales si, pour tout $u$
\begin{equation}
\label{separ8}
\hat g\Bigl(s |\phi_1 '(u) - \phi_2 '(u)|\Bigr) \ll 1 ,
\end{equation}
ce qui signifie que la diff\'{e}rence entre les fr\'{e}quences est 
plus grande que la largeur de bande de 	$\hat g(s \om)$:
\begin{equation}
\label{separ}
|\phi_1 '(u) - \phi_2 '(u) | \geq \frac{\Delta \om} s .
\end{equation}
Dans ce cas, on peut n\'{e}gliger, pour $\xi = \phi_1 '(u)$, le second 
terme de (\ref{twospectrems}), et le premier terme engendre un point 
de cr\^{e}te permettant de reconstituer $\phi'_1	(u)$ et $a_1	(u)$ en 
utilisant (\ref{compute-freq}).
De m\^{e}me, on peut n\'{e}gliger le premier terme lorsque $\xi =	
\phi_2 '(u)$, et on obtient un second point de cr\^{e}te 
caract\'{e}risant $\phi_2 '	(u)$ et $a_2 (u)$. Les points de cr\^{e}te
sont distribu\'{e}s sur les deux lignes temps-fr\'{e}quence $\xi(u)	= \phi_1 ' (u)$
et $\xi(u) = \phi_2	' (u)$. Ce r\'{e}sultat demeure valable pour un 
nombre quelconque de composantes spectrales instationnaires, tant que 
la distance entre deux fr\'{e}quences instantan\'{e}es v\'{e}rifie (\ref{separ}). 
Lorsque les lignes spectrales sont trop proches, elles cr\'{e}ent des 
interf\'{e}rences qui d\'{e}truisent la structure de cr\^{e}tes.

En g\'{e}n\'{e}ral, on ne connait pas le nombre de fr\'{e}quences 
instantan\'{e}es. On d\'{e}tecte alors tous les maxima locaux de 
$|Sf(u,\xi)|^2$ dont la phase est stationnaire 
$\frac {\partial	\Phi_S	(u,\xi )} {\partial	u} = \phi' (u) - \xi = 0$.
Ces points d\'{e}finissent des courbes dans le plan $(u,\xi)$ qui 
sont les cr\^{e}tes de la transform\'{e}e de Fourier fen\^{e}tr\'{e}e. 
On supprime souvent les cr\^{e}tes de faible amplitude $a(u)$ parce 
qu'elles peuvent \^{e}tre des artifacts provenant de variations 
bruit\'{e}es, ou des ``ombres'' d'autres fr\'{e}quences 
instantan\'{e}es, d\^{u}es aux lobes lat\'{e}raux de $\hat g(\om)$.

Dans la figure \ref{WFTRidgeChirps}, on peut voir les cr\^{e}tes 
obtenues \`{a} partir du module et de la phase de la transform\'{e}e 
de Fourier fen\^{e}tr\'{e}e de la figure \ref{WFTChirps}. Pour $t \in 
[400,500]$, les fr\'{e}quences intantan\'{e}es sont 
trop proches et la r\'{e}solution en fr\'{e}quence de la fen\^{e}tre 
ne permet pas de les s\'{e}parer. En cons\'{e}quence, les cr\^{e}tes y 
d\'{e}tectent une fr\'{e}quence instantan\'{e}e moyenne.

\begin{figure}[bhtp]
\centerline{
        \epsfxsize=8cm
        \leavevmode\epsfbox{figures/chap4/WFTRidgeChirps.eps}}
\caption{
Cr\^{e}tes de plus grande amplitude calcul\'{e}es \`{a} partir du 
spectrogramme de la figure \protect\ref{WFTChirps}. Ces cr\^{e}tes 
donnent les fr\'{e}quences instantan\'{e}es des chirps lin\'{e}aires et 
quadratiques, et des transitoires \`{a} basse et haute fr\'{e}quence 
en  $t=512$ et $t=896$.
}
\label{WFTRidgeChirps}
\end{figure}
\\
\\
\noindent
{\bf Choix de	la fen\^{e}tre}
La mesure des fr\'{e}quences instantan\'{e}es 
aux points de cr\^{e}te 
a \'{e}t\'{e} valid\'{e}e seulement lorsque la taille $s$ de la 
fen\^{e}tre $g_s$ est suffisamment petite pour que
$a(t)$ et $\phi' (t)$ soient approximativement
constant sur $[u-s/2,u+s/2]$. 
D'un autre c\^{o}t\'{e}, il faut que la largeur de 
bande ${\Delta \om}/ s$ soit 
suffisamment petite pour s\'{e}parer les composantes 
spectrales succesives en (\ref{separ}). Le choix de l'\'{e}chelle $s$ de la 
fen\^{e}tre  doit donc r\'{e}aliser un compromis entre ces deux 
contraintes.

\vspace{0cm}\setlength{\tabcolsep}{0cm} % Separation entre les lignes d'images
\setlength{\fboxsep}{0cm} % Separation entre la boite et l'image
\avecboite = 0
\begin{figtab}
\begin{figrow}{4}
{{\label{WFTLinChirps} 
Somme de deux ``chirps'' lin\'{e}aires prall\`{e}les.
(a): Spectrogramme $P_S f(u,\xi)=|Sf(u,\xi)|^2$.
(b): Cr\^{e}tes calcul\'{e}es \`{a} partir du spectrogramme.  
}}\\
\figentry{8cm}{0cm}{figures/chap4/WFTLinChirps.eps}\\

\centry{(a)}\\

\figentry{8cm}{0cm}{figures/chap4/WFTRidgeLinChirps.eps}\\

\centry{(b)}
\end{figrow} 
\end{figtab}

\vspace{0cm}\setlength{\tabcolsep}{0cm} % Separation entre les lignes d'images
\setlength{\fboxsep}{0cm} % Separation entre la boite et l'image
\avecboite = 0
\begin{figtab}
\begin{figrow}{4}
{{\label{WFTHypChirps}
Somme de deux chirps hyperboliques.
(a): Spectrogramme $P_S f(u,\xi)$.
(b): Cr\^{e}tes calcul\'{e}es \`{a} partir du spectrogramme.  
}}\\
\figentry{8cm}{0cm}{figures/chap4/WFTHypChirps.eps}\\

\centry{(a)}\\

\figentry{8cm}{0cm}{figures/chap4/WFTRidgeHypChirps.eps}\\

\centry{(b)}
\end{figrow} 
\end{figtab}


\begin{Examples}
\item 
La somme de deux chirps lin\'{e}aires parall\`{e}les
\begin{equation}
\label{ParallLineCh}
f(t) = a_1 \cos (b t^2+ c t) + a_2 \cos (b t^2) 
\end{equation}
a deux fr\'{e}quences instantan\'{e}es 
$\phi_1'	(t)	= 2	bt +c$ et $\phi_2'	(t)	= 2	bt$.
On voit un exemple num\'{e}rique en figure \ref{WFTLinChirps}. La 
fen\^{e}tre $g_s$ a une r\'{e}solution fr\'{e}quentielle suffisamment 
fine pour s\'{e}parer les deux chirps si
\begin{equation}
\label{ccondit}
|\phi_1'(t) - \phi_2'(t) | = |c| \geq \frac {\Delta \om} s .
\end{equation}
Son support temporel est assez fin compar\'{e} \`{a} la variation 
temporelle des chirps si la fr\'equence instantan\'ee
$\phi'(t)$ est quasiment constante sur $[u-s/2,u+s/2]$,
ce que l'on obtient si
\begin{equation}
\label{bcondit}
s^2 \,|\phi_1''(u)| = s^2 \,|\phi_2''(u)| = 
2 \,b\, s^2 \ll 1 .
\end{equation}
Les conditions (\ref{ccondit}) et (\ref{bcondit}) montrent qu'on peut 
trouver une fen\^{e}tre $g$ ad\'{e}quate si et seulement si 
\begin{equation}
\label{chirp-paral-co}
\frac {c} {\sqrt b} \gg \Delta \om .
\end{equation}
Comme $g$ est une fen\^{e}tre lisse de support $[-\frac	1 2	, \frac	1 2	]$,
sa largeur de bande fr\'{e}quentielle $\Delta \om$ est de l'ordre de 
1. Les chirps lin\'{e}aires de la figure \ref{WFTLinChirps} 
v\'{e}rifient (\ref{chirp-paral-co}). On a calcul\'{e} leurs 
cr\^{e}tes en utilisant une fen\^{e}tre gaussienne tronqu\'{e}e 
avec $s = 50$.

\item 
Le chirp hyperbolique
\[
f(t) = \cos \left(\frac \alpha {\beta -t}\right)
\]
pour $0 \leq t < \beta$ a une fr\'{e}quence instantan\'{e}e
\[
\phi'(t) = \frac \alpha {(\beta - t)^2},
\]
\`{a} variation rapide quand $t$ est proche de $\beta$. Les 
fr\'{e}quences instantan\'{e}es des chirps hyperboliques vont de $0$ 
\`{a} $+\infty$ en un temps fini. Cette propri\'{e}t\'{e} est 
particuli\`{e}rement utile aux radars. 
Ces chirps sont \'{e}galement 
produits par les sonars de navigation des chauves-souris 
\cite{torresani}.

On ne peut pas estimer les fr\'{e}quences instantan\'{e}es des chirps 
hyperboliques par une transform\'{e}e de Fourier fen\^{e}tr\'{e}e 
parce que, pour une taille de fen\^{e}tre donn\'{e}e, la fr\'{e}quence 
instantan\'{e}e varie trop rapidement dans les hautes fr\'{e}quences. 
Lorsque $u$ est assez proche de $\beta$, on ne peut consid\'erer
que la fr\'equence instantan\'ee $\phi'(t)$ est constante sur
le support $[u-s/2,u+s/2]$ de $g_\suxi$ car
\[
s^2 |\phi '' (u)| = 
\frac {s^2 \alpha} {(\beta - u)^3} > 1 .
\]
En figure \ref{WFTHypChirps}, on voit un signal compos\'{e} de la 
somme de deux chirps hyperboliques:
\begin{equation}
\label{ParallHypCh}
f(t) = a_1 \cos \left(\frac {\alpha_1} {\beta_1 - t}\right) +
a_2 \cos \left(\frac {\alpha_2} {\beta_2 - t}\right),
\end{equation}
avec $\beta_1 = 700$	and	$\beta_2 = 740$.
Au d\'{e}but du signal, les deux chirps ont des fr\'{e}quences 
instantan\'{e}es voisines qui sont s\'{e}par\'{e}es par la 
transform\'{e}e de Fourier fen\^{e}tr\'{e}e, qu'on a calcul\'{e}e 
avec une fen\^{e}tre large. Quand on se rapproche de $\beta_1$ ou de 
$\beta_2$, la fr\'{e}quence instantan\'{e}e varie trop rapidement en 
regard de la taille de la fen\^{e}tre. Les cr\^{e}tes correspondantes 
ne permettent plus de suivre ces fr\'{e}quences instantan\'{e}es. 
\index{Chirp!hyperbolique}
\end{Examples}


%\section{Reconnaissance de la parole}
%
%La d\'ecouverte dans les ann\'ees 50 
%des propri\'et\'es acoustiques de la parole ainsi que de la structure
%des formants a ouvert la possibilit\'e d'automatiser la reconnaissance
%de la parole. Plus de 40 ans plus tard, le probl\`eme se r\'ev\`ele
%bien plus difficile qu'on ne s'y attendait. 
%On distingue plusieurs types de probl\`emes. La reconnaissance
%de mots isol\'es s\'epar\'es par une pause, la
%d\'etection de mots appartenant \`a un vocabulaire limit\'e dans
%un flot continu de parole, et la
%reconnaissance de parole sans pose.
%Il existe actuellement
%des produits commerciaux pour la reconnaissance de mots isol\'es mais
%les performances ne sont pas suffisantes pour la parole continue.
%Les difficult\'es de la reconnaissance de parole ont diverses 
%origines.
%\\
%$\bullet$ Variations mono-locuteur.
%La prononciation est souvent d\'eform\'ee. Par exemple, le ``et'' peut
%\^etre r\'eduit \`a un simple grognement. 
%La prononciation d'un phon\`eme est aussi affect\'ee par le son avant
%et apr\`es. Enfin, le d\'ebit de parole peut varier 
%de fa\c{c}on consid\'erable.\\
%$\bullet$ Variations multi-locuteurs. Par exemple, pour
%les sons vois\'es, la position des 2 
%premiers formants varient d'un locuteur \`a l'autre.\\
%$\bullet$ Ambigu\"{\i}t\'e des sons. 
%Les variables acoustiques ne sp\'ecifient pas
%toujours de fa\c{c}on unique les variables phon\'etiques. Il est souvent
%n\'ecessaire d'utiliser des informations compl\'ementaires provenant
%de la structure du langage.\\
%$\bullet$ Bruits et interf\'erences. Un son de parole est souvent
%superpos\'e avec d'autres sons provenant \'eventuellement d'une
%autre conversation, qu'il est n\'ecessaire d'\'eliminer lors de
%la reconnaissance.\\
%\\
%{\bf Reconnaissance de mots isol\'es}
%Les structures temps-fr\'equences sont parmi les plus importantes
%pour reconna\^{\i}tre  les phon\`emes et donc les mots isol\'es qu'ils
%composent. 
%La position des formants peux se calculer \`a partir de la
%position des p\^oles du filtre AR associ\'e, comme on l'a expliqu\'e
%au paragraphe \ref{conduit-sec}. 
%Il est aussi possible de d\'etecter les zones
%de haute concentration d'\'energie dans le spectrogramme, comme le
%montre la figure \ref{spectrogram-parole}. 
%
%L'utilisation de structures locales telles que des formants n'est 
%souvent pas suffisante pour obtenir un bon taux de reconnaissance.
%On optimise souvent la reconnaissance en mesurant les 
%probabilit\'es de transition d'un son \`a un autre. 
%A partir d'un mod\`ele bas\'e sur des cha\^{\i}nes
%de Markov, on cherche alors 
%le mot qui a une probabilit\'e conditionnelle
%maximum, \'etant donn\'e le signal observ\'e.
%
%Le d\'ebit de parole \'etant un param\`etre non contr\^ol\'e,
%il est aussi 
%n\'ecessaire de renormaliser le temps pour faire correspondre
%le son avec les structures de r\'ef\'erence utilis\'ees pour la
%reconnaissance. Cela peut se faire
%avec des dilatations et compressions arbitraires en essayant 
%d'optimiser un crit\`ere de correspondance. D'autres algorithmes
%effectuent
%des dilatations temporelles locales, guid\'ees par les structures du
%son.
%
%


La complexit\'e d'une suite de symboles
peut se mesurer par la taille minimum d'un code
permettant de reconstruire cette suite.
La th\'eorie de l'information de Shannon montre que le nombre
de bit moyen pour coder chaque symbole d\'epend de l'entropie
du processus al\'eatoire sous-jacent.
Pour coder des suites de nombre r\'eels, il est necessaire
de les approximer avec une quantification avant d'effectuer
un codage entropique.
L'optimisation de cette quantification est \'etudi\'ee.
Ces r\'esultats donnent les bases math\'ematiques
et algorithmiques permettant de
comprimer des signaux audios ou des images.

\section{Complexit\'e et Entropie}

La th\'eorie de l'information d\'efinie la complexit\'e d'une
s\'erie num\'erique en \'evaluant la taille des codes permettant
de reproduire cette s\'erie. Les fondations de cette th\'eorie
sont mises en place par Shannon en 1948, qui mod\'elise
des s\'eries num\'eriques comme des r\'ealisations d'un processus
al\'eatoire. Il d\'emontre alors
l'existence d'une complexit\'e intrins\`eque associ\'ee \`a tout
processus al\'eatoire, qu'il appelle {\it entropie}.
En 1965, Kolmogorov introduit une d\'efinition plus
g\'en\'erale de la complexit\'e d'une s\'erie num\'erique,
comme \'etant la longueur minimum du programme binaire
permettant de reproduire cette s\'erie avec un ordinateur.
Le mod\`ele d'ordinateur est une machine de Turing ayant un
nombre fini d'\'etats.
Cette d\'efinition n'a pas recours \`a un mod\`ele probabiliste
mais est plus d\'elicate \`a manipuler math\'ematiquement.
Nous suivront donc ici l'approche de Shannon qui donne des
r\'esultats suffisament pr\'ecis pour la plupart des
probl\`emes de traitement du signal.

\subsection{Suites typiques}

Consid\'erons des suites de symboles de taille $n$ prenant leurs
valeurs dans un alphabet $A = \{a_k \}_{1 \leq k \leq K}$
de taille $K$.
L'approche probabiliste de Shannon mod\'elise
ces s\'equences de symboles comme \'etant les valeurs prises
par des variables al\'eatoires $X_1\, X_2\,...\, X_n$.
Pour simplifier l'analyse, nous nous placerons dans le cas le
plus simple o\`u les $X_i$ sont des variables al\'eatoires
ind\'ependantes et de m\^eme loi. On note
\[
p(a_k) = \PP \{ X_i = a_k \} .
\]
Comme les variables $X_i$ sont ind\'ependantes, la
probabilit\'e d'une suite de valeurs est:
\[
p(x_1,\, ...\,, x_n) = \PP \{X_1 = x_1 ,\, ...\, , X_n = x_n \} =
\prod_{k=1}^n \PP \{X_k = x_k \} = \prod_{k=1}^n p(x_k) .
\]
On peut d\'efinir $p(X_1 ,\, ..\, X_n) = \prod_{k=1}^n p(X_k)$
qui est une variable al\'eatoire donnant la probabilit\'e
d'une suite de valeurs tir\'ee au hasard.
Le th\'eor\`eme suivant montre que pour $n$ fix\'e
et suffisament grand, alors pour la plupart des
tirages, cette probabilit\'e est presque constante et
\'egale \`a l'entropie
\[
H = - \sum_{k=1}^K p(a_k)\, \log_2 p(a_k) = - E\{ \log_2 p(X_i)\} .
\]
L'entropie $H$ peut s'interpr\'eter comme
l'incertitude moyenne sur les valeurs
que prennent les variables al\'eatoire $X_i$.
On peut v\'erifier que
\[
0 \leq H \leq \log_2 K .
\]
L'entropie est maximum, $H = \log_2 K$, si
$p(a_k) = \frac 1 K$ pour $1 \leq k \leq K$.
Il y a en effet une incertitude maximum sur les valeurs prises
par $X_i$.
L'entropie est minimum, $H = 0$, si l'un des symboles $a_k$
a une probabilit\'e $1$.
On conna\^{\i}t alors \`a l'avance la valeur de $X_i$.

\begin{theorem}
\label{EquiTh}
Si les $X_i$ sont des variables al\'eatoires ind\'ependantes
et de m\^eme probabilit\'e $p(x)$ alors
\[
-\frac 1 n \log_2 p(X_1, ...,X_n)~~\mbox{tend vers $H$ avec
une probabilit\'e $1$}
\]
lorsque $n$ tend vers $+\infty$.
\end{theorem}
\begin{proof}
On calcule
\[
- \frac 1 n \log_2 p(X_1,\, \dots ,, X_n) = - \frac 1 n \sum_{i=1}^n
\log_2 p(X_i)~.
\]
Comme les $X_i$ sont ind\'ependants, les $\log_2 p(X_i)$ sont aussi
des variables al\'eatoires ind\'ependentes.
En appliquant la loi forte des grands nombres \cite{neveu}
on d\'emontre que
$- \frac 1 n \sum_{i=1}^n \log_2 p(X_i)$ tend
vers $- E\{ \log p(X_i)\} = H$ lorsque $n$ tend vers $+\infty$,
avec probabilit\'e 1. 
\end{proof}

Bien qu'a priori $X_1,\,\dots\,,X_n$ puisse prendre des
valeurs quelconques dans l'ensemble $A^n$ des vecteurs de
symboles de taille $n$, ce th\'eor\`eme permet de montrer
qu'il y a une probabilit\'e presque $1$ pour que ce vecteur soit
une suite typique appartenant \`a un ensemble beaucoup plus
petit. On appelle {\it ensemble typique} $T^n_\epsilon$
relativement \`a $p(x)$ l'ensemble des suites
$(x_1,...,x_n) \in A^n$ telles que
\begin{equation}
\label{Typique-def}
2^{-n (H+\epsilon)} \leq p(x_1,...,x_n) \leq
2^{-n (H-\epsilon)} .
\end{equation}
On note $|T^n_\epsilon|$ le cardinal de $T^n_\epsilon$.
Le th\'eor\`eme suivant montre que
$|T^n_\epsilon|$
est de l'ordre de $2^{nH}$, et que toutes les
suites typiques ont une probabilit\'e presque \'egale
\`a $2^{-nH}$.

\begin{proposition} [Ensembles typiques]
\begin{enumerate}[label=(\roman*)]
\item  Si $(x_1, \dots , x_n) \in T^n_\epsilon$ alors
\begin{equation}
\label{Typique-th1}
H - \epsilon \leq - \frac 1 n \,\log_2 p(x_1, ... , x_n)  \leq H + \epsilon .
\end{equation}
\item Lorsque $n$ est suffisamment grand
\begin{equation}
\label{Typique-th2}
\PP \{(X_1, ...,X_n) \in T^n_\epsilon \} > 1 - \epsilon .
\end{equation}
\item  Lorsque $n$ est suffisamment grand
\begin{equation}
\label{Typique-th3}
2^{n(H - \epsilon)} \leq |T^n_\epsilon| \leq
2^{n(H + \epsilon)} .
\end{equation}
\end{enumerate}
\end{proposition}
\begin{proof}
La propri\'et\'e (\ref{Typique-th1}) est une
cons\'equence directe de la d\'efinition (\ref{Typique-def})
de $T^n_\epsilon$.

L'in\'egalit\'e (\ref{Typique-th2})
se d\'eduit du th\'eor\`eme \ref{EquiTh} qui montre
que pour tout $\epsilon>0$ et $\delta > 0$ il existe $n_0$
tel que pour tout $n \geq n_0$
\[
\PP \left\{\left|  -\frac 1 n \log_2 p(X_1, ... , X_n) - H \right|
< \epsilon \right\} >
1 - \delta .
\]
En prenant $\delta = \epsilon$ on obtient (\ref{Typique-th2}).

On note $\vec x = (x_1 , ... , x_n)$,
\begin{eqnarray*}
1 &=& \sum_{\vec x \in A^n} p(\vec x) \geq \sum_{\vec x \in T^n_\epsilon} p(\vec x) \\
& \geq & \sum_{\vec x \in T^n_\epsilon} 2^{-n(H +\epsilon)} =
|T^n_\epsilon|\, 2^{-n(H +\epsilon)} ,
\end{eqnarray*}
ce qui d\'emontre l'in\'egalit\'e (\ref{Typique-th3})
\`a droite.

Lorsque $n$ est suffisament grand, on a montr\'e en
(\ref{Typique-th2}) que
\begin{eqnarray*}
1 - \epsilon & < &\PP \{(X_1, ...,X_n) \in T^n_\epsilon \} \\
& \leq & \sum_{x \in T^n_\epsilon} 2^{-n(H-\epsilon)} =
|T^n_\epsilon| \, 2^{-n(H-\epsilon)} ,
\end{eqnarray*}
ce qui d\'emontre l'in\'egalit\'e (\ref{Typique-th3})
\`a gauche.
\end{proof}
\\
\\
\subsection{Codage} 
On peut effectuer un codage ``$\epsilon$-typique'' des
valeurs de $X_1\,...,X_n$ qui utilise des mots binaires plus courts
pour coder les s\'equences typiques qui sont les plus probables.
Comme il y a moins de $2^{n(H+\epsilon)}$ \'el\'ements
dans $T^n_\epsilon$, ces \'el\'ements peuvent \^etre ind\'ex\'es
par des mots binaires de $\lfloor n(H+\epsilon)\rfloor + 1$
bits, o\`u  $\lfloor x \rfloor$ est le plus grand entier inf\'erieur
\`a $x$.
Comme il y a $K^n$ \'elements dans $A$, les \'el\'ements
qui n'appartiennent pas \`a $T^n_\epsilon$ peuvent \^etre
ind\'ex\'es par des mots binaires de
$\lfloor \log_2 K \rfloor + 1$ bits. Afin de savoir si
$x \in T^n_\epsilon$ on rajoute
un $0$ au d\'ebut de son code binaire, qui est donc de longueur
$\lfloor n(H+\epsilon)\rfloor + 2$. Si $x \nnin T^n_\epsilon$
on rajoute un $1$ au d\'ebut de son code binaire, dont la
taille est donc $\lfloor \log_2 K \rfloor + 2$.
On note $R$ le nombre moyen de bits pour coder chaque
symbole d'une sequence $X_1, ... , X_n$.


\begin{theorem}
Il existe $C > 0$ tel que pour tout
$\epsilon > 0$, et $n$ suffisament grand,
le nombre moyen $R$ de bits par symbole d'un
codage $\epsilon$-typique satisfait
\[
R \leq H + C\, \epsilon .
\]
\end{theorem}
\begin{proof}
On note $\vec X = (X_1, \dots , X_n )$,
$\vec x = (x_1 , ... , x_n )$. Soit $l(x_i)$ la longueur du
mot binaire utilis\'e par un code typique pour coder $x_i$.
Le nombre total de bits pour coder $\vec x$ est
\[
l(\vec x) = \sum_{i=1}^n l(x_i)~.
\]
Le nombre moyen de bits par symbole est donc
\begin{eqnarray*}
R &=& E \{ l(\vec X)\} =
\sum_{\vec x \in A^n} l(\vec x)\, p(\vec x) =
\sum_{\vec x \in T^n_\epsilon} l(\vec x)\, p(\vec x) +
\sum_{\vec x \nin T^n_\epsilon} l(\vec x)\, p(\vec x) \\
&\leq& \PP \{\vec X \in T^n_\epsilon \}\,
\Bigl(\lfloor n(H+\epsilon)
\rfloor + 2\Bigr)
+ \PP \{\vec X \nnin T^n_\epsilon \}\,
\Bigl( \lfloor n \log_2 K \rfloor +
2\Bigr) \\
&\leq& n(H+\epsilon) + 2 + \epsilon ( n \log_2 K + 2) \leq
 H + C\, \epsilon
\end{eqnarray*}
avec $C = 5 + \log_2 K $ pour $n \geq 1/\epsilon$.
\end{proof}
Ce th\'eor\`eme d\'emontre que l'on peut construire un code
dont le nombre de bit moyen par symbole est arbitrairement
pr\`es de l'entropie $H$. Par ailleurs, on peut montrer
que tout code n\'ecessite un nombre moyen de bit par symbole $R \geq H$. Le paragraphe
suivant d\'emontre ce r\'esultat pour les codes par blocs.


\subsection{Codage entropique}
\label{entropy-code-sec}

Nous considerons dans un premier temps
les codes instantan\'es, qui d\'efinissent un code binaire
$w_k$ pour chaque symbole $a_k$ de l'alphabet $A$. Cela
permet de d\'ecoder symbole par symbole toute
s\'equence $x_1, ..., x_n$.
Si $\log_2 K$ est un entier, chaque symbole
$a_k$ peut \^etre cod\'e par un mot binaire de
$\lfloor \log_2 K \rfloor + 1$ bits.
Ce code peut cependant \^etre am\'elior\'e
en utilisant des mots
binaires plus courts pour des symboles qui apparaissent plus
souvent.

Soit $l_k$ la longeur du code binaire
$w_k$ associ\'ee \`a $a_k$.
Le nombre moyen de bits n\'ecessaires pour coder les symboles
d'une suite de variables al\'eatoires $X_1 \, ...\, X_n$ de
m\^eme probabilit\'e $p(x)$ est
\begin{equation}
\label{bit-rate}
R  = \sum_{k=1}^{K} l_k\, p(a_k) .
\end{equation}
Le but est de trouver un code instantan\'e qui soit
d\'ecodable et qui minimise $R$.\\

\subsection{Condition de pr\'efixe}
Un code instan\'e n'est pas toujours uniquement d\'ecodable.
Par exemple, le code qui associe \`a
$\{a_k\}_{1 \leq k \leq 4}$
les mots binaires
\[
\{w_1 = 0 ~,~w_2 = 10 ~,~w_3 = 110 ~,~w_4 = 101 \}
\]
n'est pas d\'ecodable de fa\c{c}on unique.
La suite $1010$ peut soit correspondre
\`a $w_2~w_2$ ou \`a $w_4~ w_1$.
La condition de pr\'efixe impose qu'aucun mot binaire n'est le
d\'ebut d'un autre mot binaire.
Cette condition est clairement n\'ecessaire et
suffisante pour garantir que
toute suite de mots binaires se d\'ecode de fa\c con unique.
Dans l'exemple pr\'ec\'edent,
$w_2$ est le pr\'efixe de $w_4$.
Le code suivant
\[
\{w_1 = 0 ~,~w_2 = 10 ~,~w_3 = 110 ~,~w_4 = 111 \}
\]
satisfait la condition de pr\'efixe.


\begin{figure}[bhtp]
\centerline{
        \epsfxsize=5cm
	\leavevmode\epsfbox{figures/chap11/fig2.eps}}
\caption{Arbre binaire d'un code de $6$ symboles,
qui satisfait la condition de pr\'efixe. Le mot binaire
$w_k$ de chaque feuille est indiqu\'e au-dessous.}
\label{prefixe6}
\end{figure}

Un code qui satisfait la condition de pr\'efixe peut \^etre
associ\'e \`a un arbre binaire, dont les $K$ feuilles correspondent
aux symboles $\{a_k \}_\UkK$.
Cette repr\'esentation est utile pour construire le code
qui minimise le nombre de bits moyen $R$.
Les branches de gauche et de droite de l'arbre binaire sont
respectivement cod\'ees par 0 et 1.
La figure \ref{prefixe6} montre un exemple pour un code
de $6$ symboles.
Le mot binaire $w_k$ associ\'e au symbole $a_k$
est la succession de $0$ et de $1$ correspondant aux branches
de gauche et de droite, le long du chemin de la racine de
l'arbre \`a la feuille correspondant \`a $a_k$.
Le code binaire g\'en\'er\'e par un tel arbre satisfait toujours
la condition de pr\'efixe. En effet, $w_m$ est
un pr\'efixe de $w_k$ si et seulement si
$a_m$ est un anc\^etre de $a_k$ dans l'arbre binaire.
Ceci n'est pas possible puisque les deux symboles correspondent
\`a des feuilles de l'arbre.
Inversement, tout code pr\'efixe peut \^etre repr\'esent\'e par un tel
arbre binaire.
La longueur $l_k$ du mot binaire
$w_k$ est la profondeur de la feuille $a_k$ dans l'arbre binaire.
L'optimisation d'un code de pr\'efixe est donc \'equivalente \`a
la construction d'un arbre binaire optimal qui distribue
les profondeur des feuilles de fa\c con \`a minimiser
(\ref{bit-rate}).
\\
\\
\subsectuib{Entropie de Shannon}
Le th\'eor\`eme de Shannon prouve que le nombre moyen de bit $R$
par symbole est plus grand que l'entropie.

\begin{theorem} [Shannon]
\label{shan-th}
On suppose que les symboles
$\{a_k\}_{1 \leq k \leq K}$ apparaissent avec la distribution
de probabilit\'e $\{p(a_k)\}_\UkK$.
Le nombre moyen $R$ de bit d'un code ayant la propri\'et\'e du
pr\'efixe satisfait
\begin{equation}
\label{shan1}
R \geq H = - \sum_{k=1}^K p(a_k) \log_2 p(a_k) .
\end{equation}
Il existe un code ayant la propri\'et\'e du pr\'efixe tel que
\begin{equation}
\label{shan2}
R \leq H + 1.
\end{equation}
\end{theorem}
\begin{proof}
Le th\'eor\`eme de Shannon se d\'emontre \`a partir de
l'in\'egalit\'e de Kraft.

\begin{lemma} [In\'egalit\'e de Kraft]
Tout code ayant la propri\'et\'e du pr\'efixe satisfait
\begin{equation}
\label{kraft}
\sum_{k=1}^K 2^{-l_k} \leq 1 .
\end{equation}
Inversement, si
$\{ l_k \}_\UkK$ sont des entiers positifs tels que
l'in\'egalit\'e (\ref{kraft}) est satisfaite alors il existe
un code de mots binaires $\{w_k \}_\UkK$ de longueurs
$\{ l_k \}_\UkK$ et qui satisfait la condition de pr\'efixe.
\end{lemma}

Pour d\'emontrer (\ref{kraft}) on associe un arbre binaire
au code consid\'er\'e. Chaque $l_k$ correspond \`a un noeud
de l'arbre \`a une profondeur $l_k$ qui d\'epend du mot
binaire $w_k$. Soit
\begin{equation}
\label{mmax}
m = \max \{l_1, l_2, ... , l_K\} .
\end{equation}
On consid\'ere l'arbre binaire complet dont toutes les
feuilles sont \`a la profondeur $m$.
On note $T_k$ le sous-arbre issu du noeud correspondant au
mot binaire $w_k$. Ce sous arbre a une profondeur ${m - l_k}$
et contient donc $2^{m - l_k}$ noeud au niveau $m$, comme
l'illustre la figure \ref{kraft-fig}.
Comme il y a $2^m$ noeud \`a la profondeur $m$ de l'arbre binaire
complet et que
la propri\'et\'e du pr\'efixe implique
que tous les sous arbres $T_1 , ... , T_K$ sont distincts, on
d\'eduit que
\[
\sum_{k=1}^K 2^{m-l_k} \leq 2^m ,
\]
d'o\`u (\ref{kraft}).

\begin{figure}[bhtp]
\centerline{
	\epsfxsize=6cm
	\leavevmode\epsfbox{/home/mallat/X/TREX/figures/SigFig/FIG7.1.EPS.txt}}
\caption{Disposition des sous arbres $T_i$ dont la racine est \`a la
profondeur $l_i$ dans l'arbre d'un code de pr\'efixe.}
\label{kraft-fig}
\end{figure}


Inversement, on consid\`ere $\{ l_k \}_\UkK$ satisfaisant
(\ref{kraft}) avec $l_1 \leq l_2 \leq ... \leq l_K$ et
$m = \max \{l_1, l_2, ... , l_K\}$.
On d\'efinit les ensembles $N_1$ des $2^{m-l_1}$ premiers
noeud au niveau $m$ sur la gauche de l'arbre, puis $N_2$
l'ensemble des $2^{m-l_2}$ noeuds suivants et ainsi de suite
comme l'indique la figure \ref{kraft-fig}.
Les noeuds des ensembles $N_k$ sont les noeuds terminaux
de sous-arbres $T_k$ qui sont disjoints. On associe
\`a la racine de l'arbre $T_k$ qui est \`a la profondeur
$l_k$ le mot binaire $w_k$. Cela d\'efinit un code
qui satisfait la condition du pr\'efixe o\`u chaque
mot a la longueur $l_k$ voulue.
Cela termine la d\'emonstration du lemme.

Pour d\'emontrer les deux in\'egalit\'es (\ref{shan1}) et
(\ref{shan2}) du
th\'eor\`eme, on consid\`ere la minimisation de
\[
R = \sum_{k=1}^K p(a_k) \,l_k
\]
sous la contrainte de Kraft
\[
\sum_{k=1}^K 2^{-l_k} \leq 1 .
\]
Dans un premier temps, nous
supposons que $l_k$ peut \^etre un r\'eel quelconque.
Le minimum se calcule en utilisant un multiplicateur de
Lagrange $\lambda$ et en minimisant
\[
J = \sum_{k=1}^K p(a_k) l_k + \lambda \sum_{k=1}^K 2^{-l_k} .
\]
L'annulation de la d\'erivee par rapport \`a $l_k$ donne
\[
\frac {\partial J} {\partial l_k} = p(a_k) - \lambda \,2^{-l_k}\,
\log_\Exp 2  = 0 .
\]
Le minimum est obtenu pour $\sum_{k=1}^K 2^{-l_k} = 1$
et comme $\sum_{k=1}^K p(a_k) = 1$ on obtient
$\lambda = 1/\log_\Exp 2$. La longueur optimale minimisant
$R$ est donc
\[
l_k = -\log_2 p(a_k) ,
\]
et
\[
R =
\sum_{k=1}^K p(a_k)\, l_k = - \sum_{k=1}^K p(a_k) \log_2 p(a_k) = H .
\]

Pour garantir que $l_k$ est entier, on choisit
\[
l_k = \lceil - \log_2 p(a_k) \rceil
\]
o\`u $\lceil x \rceil$ est la plus petite valeur enti\`ere
sup\'erieure \`a $x$.
Cela correspond au code de Shannon.
Comme $l_k \geq - \log_2 p(a_k)$, l'in\'egalit\'e
de Kraft est satisfaite puisque
\[
\sum_{k=1}^K 2^{-l_k} \leq \sum_{k=1}^K 2^{\log_2 p(a_k)} = 1 .
\]
Il existe donc un code pr\'efixe dont les mots de code
ont une longueur $l_k$. Pour ce code
\[
\sum_{k=1}^K p(a_k) l_k \leq
\sum_{k=1}^K p(a_k) (-\log_2 p(a_k) + 1) = H + 1 .
\]
\end{proof}

\subsection{Codage par blocs}
L'ajout de 1 bit dans l'in\'egalit\'e (\ref{shan2})
vient du fait que $-\log_2 p_i$ n'est pas n\'ecessairement
un entier alors que la longueur d'un mot binaire doit
\^etre un entier. On peut construire des codes tels que
$R$ est plus proche de $H$ en r\'epartissant ce bit
suppl\'ementaire sur un bloc de $n$ \'el\'ements.
Au lieu de faire un codage instantan\'e, symbole par symbole,
on code d'un coup le bloc de symboles
$\vec X = X_1 , \,...\,,X_n$, qui peut \^etre consid\'er\'e
comme une variable al\'eatoire \`a valeurs dans
l'alphabet $A^n$ de taille $K^n$.
A tout bloc de symboles $\vec a \in A^n$ on associe un
mot binaire de longuer $l(\vec a)$. Le nombre de bits
$R$ par symbole pour un tel code par bloc est
\[
R = \frac 1 n \sum_{\vec a \in A^n} p(\vec a) \, l(\vec a) .
\]


\begin{proposition}
\label{shan-prop}
Le nombre moyen $R$ de bit d'un code
par bloc de taille $n$ ayant la propri\'et\'e du
pr\'efixe satisfait
\begin{equation}
\label{shan12}
R \geq H = - \sum_{k=1}^K p(a_k) \log_2 p(a_k) .
\end{equation}
Il existe un code par blocs de taille
$n$ ayant la propri\'et\'e du pr\'efixe tel que
\begin{equation}
\label{shan22}
R \leq H + \frac 1 n.
\end{equation}
\end{proposition}
\begin{proof}
L'entropie associ\'ee \`a $\vec X$ est
\[
\vec H = \sum_{\vec x \in A^n} p(\vec x) \, \log_2 p(\vec x) .
\]
Comme les variables al\'eatoires $X_i$ sont ind\'ependantes
\[
p(\vec x) = p(x_1, \dots , x_n) = \prod_{i=1}^n p(x_i)~.
\]
On d\'emontre par r\'ecurrence sur $n$ que
$\vec H = n H$. Soit $\vec R$ le nombre de bits moyen
pour coder les $n$ symboles $\vec X$. Le th\'eor\`eme
de Shannon \ref{shan-th} montre que $\vec R \geq \vec H$
et qu'il existe un code par bloc tel que
$\vec R \leq \vec H + 1$. On d\'eduit donc
(\ref{shan12},\ref{shan22}) pour $R = \frac{\vec R} n$, qui
est le nombre de bits moyen par symbole.
\end{proof}

Ce th\'eor\`eme d\'emontre que des codes par blocs
utilisent un nombre moyen de bits par symbole qui tendent
vers l'entropie lorsque la taille du bloc augmente.

\subsection{Code de Huffman}
L'algorithme de Huffman est un algorithme de programmation dynamique
qui construit de bas en haut un arbre correspondant \`a un
code pr\'efixe et qui
minimise
\begin{equation}
\label{Rmoyen}
R = \sum_{k=1}^K p(a_k) \, l_k .
\end{equation}
Nous ordonnons
$\{a_k \}_\UkK$ pour que $p(a_k) \leq p(a_{k+1})$.
Pour minimiser (\ref{Rmoyen})
les symboles de plus petites probabilit\'es doivent \^etre
associ\'es aux mot binaires $w_k$ de longueur maximale, ce qui
correspond \`a un noeud au bas de l'arbre.
Nous commen\c cons donc par repr\'esenter les deux symboles de plus
petite probabilit\'e
$a_1$ et $a_2$ comme les enfants d'un noeud commun.
Ce noeud peut \^etre interpr\'et\'e comme un symbole
$a_{1,2}$ correspondant \`a ``$a_1$ ou $a_2$'' et
dont la probabilit\'e est
$p(a_1) + p(a_2)$. La proposition suivante prouve que
l'on peut it\'erer ce regroupement \'el\'ementaire et construire un
code optimal.

\begin{proposition}
On consid\`ere $K$ symboles avec leurs probabilit\'es
ordonn\'ees en ordre croissant: $p(a_k) \leq p(a_{k+1})$.
On regroupe les deux symboles $a_1$ et $a_2$ de probabilit\'e
minimum en un seul symbole $a_{1,2}$ de probabilit\'e
\[
p(a_{1,2}) = p(a_1) + p(a_2) .
\]
Un arbre correspondant \`a un code pr\'efixe optimal pour
les $K$ symboles se construit \`a partir
d'un arbre de code pr\'efixe optimal pour les $K-1$ symboles
$\{a_{1,2}\} \cup \{ a_k \}_{3 \leq k \leq K}$,
en divisant la feuille de
$a_{1,2}$ en deux noeuds correspondant \`a $a_1$ et $a_2$.
\end{proposition}

La d\'emonstration de cette proposition se trouve dans
\cite{bremaud-proba}.

Cette proposition r\'eduit la construction d'un code optimal de
$K$ symboles \`a la construction
d'un code optimal pour les $K-1$ symboles.
Le code de Huffman it\`ere $K-1$ fois
ce regroupement et fait progressivement pousser l'arbre
d'un code de pr\'efixe optimal depuis le bas jusqu'en haut.
Le Th\'eor\`eme \ref{shan-th} de Shannon prouve que
\begin{equation}
\label{entropy-bound}
H \leq R  \leq H + 1 .
\end{equation}
\\
\\
\subsection*{Exemple}
Les probabilit\'es des $\{a_k\}_{1 \leq k \leq 6}$ sont
\begin{equation}
\label{proba-code}
\{p(a_k)\}_{1 \leq k \leq 6} = \{0.05~,~0.1~,~0.1~,~0.15~,~0.2~,~0.4\}.
\end{equation}

La figure \ref{arbre-binaire}
donne l'arbre binaire construit avec l'algorithme
de Huffman.
Les symboles $a_1$ et $a_2$ sont regroup\'es
en un symbole $a_{1,2}$
de probabilit\'e $p(a_{1,2}) = p(a_1)+p(a_2)= 0.15$. A l'it\'eration suivante,
les symboles de plus basse probabilit\'e sont
$p(a_3) = 0.1$ et $p(a_{1,2}) = 0.15$. On regroupe donc
$a_{1,2}$ et $a_3$ en un symbole $a_{1,2,3}$ dont la probabilit\'e
est $0.25$. Les deux symboles de probabilit\'es les plus faibles sont
alors $a_4$ et
$a_5$ qui sont regroup\'es en $a_{4,5}$
de probabilit\'e $0.35$. On regroupe ensuite $a_{4,5}$ et
$a_{1,2,3}$ pour obtenir un symbole $a_{1,2,3,4,5}$ de probabilit\'e
$0.6$ qui est finalement regroup\'e avec $a_6$, ce qui finit
le code, comme l'illustre
l'arbre de la figure \ref{arbre-binaire}.
Le nombre moyen de bits obtenu par ce code est
$R= 2.35$ alors que l'entropie est $H = 2.28$.

\begin{figure}[bhtp]
\centerline{
        \epsfxsize=6cm
	\leavevmode\epsfbox{NewFig/fig3.eps}}
\caption {Arbre correspondant au code de Huffman pour une
source dont les probabilit\'es sont donn\'ees par
(\protect \ref{proba-code}) \protect \cite{vetterli}.}
\label{arbre-binaire}
\end{figure}
\\
\\
\noindent
\subsection{Sensibilit\'e au bruit}
Un code de Huffman est plus compact qu'un code de taille fixe
$\log_2 K$ mais est aussi plus sensible au bruit.
Pour un code de taille constante, une erreur
de transmission d'un bit modifie
seulement la valeur d'un symbole.
Au contraire, une erreur d'un bit
dans un code de taille variable peut modifier toute la suite
des symboles.
Lors de transmissions bruit\'ees, de telles erreurs peuvent se
produire. Il est alors n\'ecessaire d'utiliser un code correcteur
qui introduit une l\'eg\`ere redondance de facon \`a identifier les
erreurs.

\section{Quantification scalaire}
\label{scalar-quant-sec}

Si une variable al\'eatoire $X$
prend des valeurs r\'eelles quelconques, on ne peut pas
obtenir un code exact de taille finie.
Il est alors n\'ecessaire d'approximer $X$ par
$\tilde X$ qui prend ses valeurs dans un alphabet fini, et
l'erreur r\'esultante est
\[
D = E \{|X - \tilde X|^2\} .
\]
Un quantificateur scalaire d\'ecompose l'axe r\'eel en
$K$ intervalles $\{[y_{k-1} , y_k ]\}_{1 \leq k \leq K}$
de tailles variables, avec $y_0 = -\infty$ et $y_K = +\infty$.
Le quantificateur associe \`a tout
$x \in [y_{k-1} , y_k]$ une valeur
$Q(x) =a_k$.
Si les $K$ niveaux de quantification $\{a_k\}_{\UkK}$
sont fix\'es a priori,
pour minimiser $|x - Q(x)| = |x - a_k|$,
il faut que la quantification
associe \`a $x$ son niveau de quantification $a_k$ le plus
proche. On doit alors choisir des intervalles de quantification
qui satisfont
\begin{equation}
\label{opt-interv}
y_k = \frac {a_k + a_{k+1}} 2
\end{equation}
\\
\\
\subsection{Quantification haute r\'esolution}
Soit $p(x)$ la densit\'e de probabilit\'e de $X$.
On note $\tilde X = Q (X)$ la variable quantifi\'ee.
L'erreur quadratique moyenne est
\begin{equation}
\label{quantize-error}
D = E\{(X- \tilde X)^2\} =
\int_{-\infty}^{+\infty} |x - Q(x)|^2 p(x) dx .
\end{equation}

On dit que le quantificateur a une haute
r\'esolution si
$p(x)$ peut \^etre approxim\'e par une constante sur tout intervalle
de quantification $[{y_{k-1}},{y_{k}}]$.
La taille de ces intervalles est $\Delta_k = y_k - y_{k-1}$.
L'hypoth\`ese de haute r\'esolution implique que
\begin{equation}
\label{high-resol-quant}
p(x) = \frac {p_k} {\Delta_k} ~~\mbox{pour $x \in [y_{k-1},{y_{k}}]$},
\end{equation}
avec
\[
p_k = \PP \{X \in [{y_{k-1}},y_{k}] \} = \PP \{ \tilde X = a_k \} .
\]
La proposition suivant calcule l'erreur $D$ sous cette hypoth\`ese.

\begin{proposition}
Pour un quantificateur de haute r\'esolution sur des intervalles
$[{y_{k-1}},{y_{k}}]$, l'erreur $D$
minimum obtenue en optimisant la position des niveaux
$\{ a_k \}_{0 \leq k \leq K}$ est
\begin{equation}
\label{quadr-error}
D = \frac 1 {12} \sum_{k=1}^{K} {p_k} \, {\Delta_k ^2} .
\end{equation}
\end{proposition}
\begin{proof} 
Comme $Q(x) = a_k$ si
$x \in [y_{k-1},y_k)$, on peut re\'ecrire (\ref{quantize-error})
\[
D = \sum_{k=1}^{K} \int_{y_{k-1}}^{y_{k}} (x - a_k)^2 p(x) dx .
\]
En rempla\c{c}ant $p(x)$ par son expression
(\ref{high-resol-quant}) on a
\begin{equation}
D =
\sum_{k=1}^{K} \frac {p_k} {\Delta_k}
\int_{y_{k-1}}^{y_{k}} (x - a_k)^2  dx.
\end{equation}
Cette erreur est minimum
pour $a_k = \half (y_{k}+{y_{k-1}})$, et l'int\'egration donne
(\ref{quadr-error}).
\end{proof}

\sibsection{Quantification uniforme}
Le quantificateur uniforme est un cas particulier important o\`u
tous les intervalles de quantification sont de m\^eme taille
\[
y_{k} - y_{k-1} = \Delta ~~~\mbox{pour $1 \leq k \leq K$} .
\]
L'erreur quadratique moyenne (\ref{quadr-error})
devient
\begin{equation}
\label{unifoquantiz}
D = \frac {\Delta^2} {12} \sum_{k=1}^{K} {p_k}
= \frac {\Delta^2} {12} .
\end{equation}
Elle est ind\'ependante de la distribution de probabilit\'e
$p(x)$ de la source.
\\
\\
\subsection{Quantification optimale}
On veut optimiser
le quantificateur pour minimiser le nombre de bits
n\'ecessaires pour coder les valeurs quantifi\'ees
$\tilde X$, \'etant donn\'ee une distortion $D$ admissible.
Le th\'eor\`eme de
Shannon \ref{shan-th} prouve que la valeur moyenne minimum de bits
n\'ecessaire pour coder $\tilde X$ est sup\'erieure \`a l'entropie $H$
de la variable al\'eatoire $\tilde X$.
Comme le
code de Huffman donne un r\'esultat proche de cette entropie,
il nous faut minimiser l'entropie $H$ pour $D$ fixe.

La source quantifi\'ee $\tilde X$ prend $K$ valeurs diff\'erentes
$\{a_k\}_{1 \leq k \leq K}$ avec probabilit\'es
$\{p_k\}_{1 \leq k \leq K}$.
L'entropie du signal quantifi\'e est donc
\[
H = - \sum_{k=1}^K p_k\, \log_2 p_k .
\]
On d\'efinit l'entropie diff\'erentielle de la variable al\'eatoire
$X$ \`a valeurs r\'eelles
\begin{equation}
\label{entropie-dffD}
H_d = - \int_{-\infty}^{+\infty} p(x) \log_2 p(x) dx .
\end{equation}
Le th\'eor\`eme suivant montre que pour un quantificateur de haute
r\'esolution produisant une erreur $D$,
l'entropie est minimum lorsque le quantificateur est
uniforme.

\begin{theorem}
\label{quanti-theo}
L'entropie de
tout quantificateur de haute r\'esolution satisfait
\begin{equation}
\label{lower-quantX}
H \geq H_d - \frac 1 2 \log_2 (12 D) .
\end{equation}
Le minimum est atteint si et seulement si $Q$ est un
quantificateur uniforme.
\end{theorem}
\begin{proof}
Pour un quantificateur de haute r\'esolution
$p(x)$ est approximativement constant sur $[y_{k-1},y_k]$
et donc
\[
p_k = \int_{y_{k-1}}^{y_k} p(x)\,dx = p_k \Delta_k
\]
avec $\Delta_k = y_k - y_{k-1}$.
Donc
\begin{eqnarray*}
H & = & - \sum_{k=1}^K p_k \log_2 (p(a_k)\, \Delta_k )\\
& = &- \sum_{k=1}^K \int_{y_{k-1}}^{y_k} p(x) \log_2 p(a_k)\, dx
- \sum_{k=1}^K p_k \log_2 \Delta_k \\
& = & H_d - \frac 1 2 \sum_{k=1}^K p_k \log_2 \Delta_k^2 ,
\end{eqnarray*}
car $p(x) = p(a_k)$ pour $x \in [y_{k-1},y_k]$.
Pour toute fonction concave $\phi (x)$, l'in\'egalit\'e de Jensen
montre que pour tout $\sum_{k=1}^K p_k = 1$
et $\{a_k \}_\UkK$ alors
\begin{equation}
\label{jensen}
\sum_{k=1}^N p_k \phi ( a_k) \leq \phi(\sum_{k=1}^N  p_k a_k) .
\end{equation}
Si $\phi (x)$ est strictement concave, l'in\'egalit\'e devient une \'egalit\'e
si et seulement si tous les $a_k$ sont \'egaux lorsque
$p_k \neq 0$. Comme $\log_2(x)$ est strictement concave,
(\ref{quadr-error}) montre que
\[
 \frac 1 2 \sum_{k=1}^N p_k \log_2 \Delta_k^2 \leq
 \frac 1 2 \log_2 \sum_{k=1}^N p_k \Delta_k^2 =
 \frac 1 2 \log_2 (12 D ) .
\]
On en d\'eduit donc que
\[
H \geq H_d - \frac 1 2 \log_2 (12 D ).
\]
Cette in\'egalit\'e devient une \'egalit\'e si et seulement si tous les
$\Delta_k$ sont \'egaux, ce qui correspond \`a un quantificateur
uniforme.
\end{proof}
Ce th\'eor\`eme montre que pour un quantificateur haute r\'esolution
le nombre minimum de bits
$R = H$ est obtenu pour un quantificateur uniforme et
\begin{equation}
\label{bit-rate-uniform}
R = H_d -  \frac 1 2 \log_2 (12 D ).
\end{equation}
La distortion en fonction du nombre de bits est donc
\[
D(R) = \frac 1 {12}\, 2^{2 H_d} \,2^{-2R} .
\]

\chapter{Compression de Signaux}
\label{comp-code-chap}

La compression de signaux s'apparente \`a la deshydratation d'un
litre de jus d'orange en quelques grammes de poudre concentr\'ee.
Le go\^ut de la boisson orange restitu\'ee est similaire au jus d'orange
mais a souvent perdu de sa subtilit\'e.
En traitement du signal, nous sommes plus interess\'es
par des sons ou des images, mais l'on rencontre le m\^eme
conflit entre qualit\'e et compression.
Minimiser la d\'egradation pour un taux de compression donn\'e est
le but des algorithmes de codage.
Les applications principales concernent le stockage des donn\'ees
et la transmission \`a travers des canaux \`a d\'ebit limit\'e.

Nous \'etudions les algorithmes de codage 
qui d\'ecomposent le
signal sur une base orthogonale et approximent efficacement les
coefficients de d\'ecomposition.
Ce type de codage est actuellement le plus performant pour
restituer des signaux audios ou des images de bonne qualit\'e.

\section{Codage compact}

La performance ultime d'un algorithme de codage est mesur\'ee par
un ``score d'opinion moyen''.
Pour un taux de compression donn\'e,
la qualit\'e des signaux cod\'es est \'evalu\'ee par plusieurs personnes
et calibr\'es selon une proc\'edure pr\'ecise.
De telles \'evaluations sont longues \`a faire et les r\'esultats 
difficiles 
\`a interpr\'eter math\'ematiquement. La qualit\'e d'un algorithme de codage
est donc le plus souvent optimis\'ee avec une distance qui
a une forme analytique simple et qui tient compte partiellement de
notre sensibilit\'e visuelle ou auditive.
La distance euclidienne, bien que relativement
grossi\`ere d'un point de vue perceptuel,
a l'avantage d'\^etre facile \`a manipuler analytiquement.


\subsection{Etat de l'art}
\label{stateofart}
\noindent
{\bf Parole}
Le codage de la parole est particuli\`erement important pour
la t\'el\'ephonie o\`u il peut \^etre de qualit\'e m\'ediocre tout en
maintenant une bonne intelligibilit\'e.
Un signal de parole par t\'el\'ephone est limit\'e \`a la bande de
fr\'equence 200-3400 Hz et est \'echantillonn\'e \`a 8kHz.
Un ``Pulse Code Modulation (PCM)'' qui quantifie chaque
\'echantillon sur
8 bits produit un code de 64kb/s ($64~10^3$ bits par seconde).
Ceci peut \^etre consid\'erablement r\'eduit en supprimant certaines
composantes redondantes de la parole.

Nous avons vu dans le paragraphe \ref{modelisation-paro} que
la production d'un signal de parole est bien comprise. 
Des filtres autor\'egressifs 
excit\'es par des trains d'impulsions ou un bruit
blanc Gaussien 
permettent de restaurer un signal
intelligible \`a partir de peu de param\`etres.
Ces codes d'analyse-synth\`ese, tel que le standard LPC-10 d\'ecrit
dans le paragraphe \ref{compre-LPC},
produisent un signal de parole intelligible
\`a 2,4kb/s. \\

{\bf Audio}
Les signaux audios peuvent inclure de la parole mais aussi de la
musique et n'importe quel type de son.
Ils sont donc beaucoup plus difficiles \`a mod\'eliser que la parole.
Sur un disque compact, le signal audio est limit\'e \`a un maximum 
de 20kHz. Il est \'echantillonn\'e \`a 44.1kHz et chaque \'echantillon est
cod\'e sur 16bits. Le d\'ebit du code PCM r\'esultant est donc de
706kb/s. Le signal audio d'un disque compact ou d'une cassette
digitale doit \^etre cod\'e sans distortion auditive.

Les codeurs par transform\'ee orthogonale qui d\'ecomposent les
signaux sur des bases locales en temps et en fr\'equence sont
parmi les plus performants pour les signaux audios, car ils ne
n\'ecessitent pas la mise en place d'un mod\`ele.
Une qualit\'e de disque compact est obtenue par des codeurs
n\'ecessitant 128kb/s.
Avec 64kb/s les d\'egradations sont \`a peine audibles.
Ces algorithmes sont particuli\`erement importants pour les 
CD-ROM en multim\'edia.\\

{\bf Images}
Une image couleur est compos\'ee de trois canaux d'intensit\'e 
dans le rouge, le vert et le bleu.
Chacune de ces images a typiquement
500 par 500 pixels, qui sont cod\'es sur 8 bits (256 niveaux de gris).
Un canal t\'el\'ephonique digital ISDN  a un d\'ebit de 64kb/s.
Il faut donc environ 2 minutes pour transmettre une telle image. 

Les images tout comme les signaux audios incluent le plus souvent
des structures de type diff\'erent qu'il est difficile de mod\'eliser.
Courament, les algorithmes de compression les plus efficaces sont
bas\'es sur des transform\'ees orthogonales utilisant des bases de
cosinus locaux ou des bases d'ondelettes.
Avec moins de 1 bit/pixel, ces codes reproduisent une image
de qualit\'e visuelle presque parfaite.
A 0.25 bit/pixel, l'image reste de bonne qualit\'e visuelle et
peut \^etre transmise en 4 secondes sur une ligne t\'el\'ephonique
digitale.

\subsection{Codage dans une base orthogonale}
\label{transf-sec}

Un code orthogonal d\'ecompose le signal dans une base
orthogonale bien choisie de fa\c con \`a optimiser la compression
des coefficients de d\'ecomposition.
La classe des signaux cod\'es est represent\'ee 
par un vecteur al\'eatoire $Y[n]$ de taille $N$.
On d\'ecompose $Y[n]$ sur une 
base orthogonale $\{ g_m [n] \}_\ZnN$
\[
Y[n] = \sum_{m=0}^{N-1} A[m] \,g_m [n] .
\]
Les coefficients de d\'ecomposition
$A[m]$ sont des variables al\'eatoires
\[
A[m] = <Y[n] , g_m [n] > = 
\sum_{n=0}^{N-1} Y[n] \, g^*_m [n] .
\]
Si $A[m]$ n'est pas de moyenne nulle, 
on code $A[m] - E\{A[m]\}$ et on m\'emorise la valeur moyenne
$E\{A[m]\}$. Nous supposerons par la suite que 
$E\{A[m]\}$ = 0.
\\
\\
{\bf Quantification}
Les valeurs r\'eelles 
$\{A[m]\}_\ZmN$ doivent \^etre approxim\'ees avec une pr\'ecision finie
pour construire un code de taille finie.
Une quantification scalaire approxime chaque 
$A[m]$ individuellement.
Si les coefficients $A[m]$ ont une forte d\'ependance,
le taux de compression peut \^etre am\'elior\'e 
avec une quantification vectorielle qui regroupe les coefficients
en blocs.
Cette approche n\'ecessite cependant plus de calculs. 
Si la base $\{ g_m\}_\ZmN$ est choisie de fa\c con \`a ce que 
les coefficients $A[m]$ soient presque ind\'ependants, 
l'am\'elioration d'une quantification vectorielle est
marginale.

Chaque $A[m]$ est une variable al\'eatoire dont la valeur
quantifi\'ee est $\tilde A[m] = Q_m (A[m])$ 
Le signal quantifi\'e reconstruit est
\[
\tilde Y[n] = \sum_{m=0}^{N-1} \tilde A[m] \, g_m [n] .
\]
Comme la base est orthogonale
\[
\| Y - \tilde Y \|^2 = \sum_{m=0}^{N-1} 
|A[m] - \tilde A[m]|^2
\]
et donc la valeur moyenne de l'erreur est
\[
E\{\| Y - \tilde Y \|^2\} = \sum_{m=0}^{N-1} 
E\{|A[m] - \tilde A[m]|^2\}.
\]
Si l'on note 
\[
D_m = E\{|A[m] - \tilde A[m]|^2\} ,
\]
l'erreur totale devient
\[
D =  \sum_{m=0}^{N-1}  D_m .
\]
Soit $R_m$ le nombre moyen de bits pour coder $\tilde A [m]$.
Pour une quantification \`a haute r\'esolution,
le th\'eor\`eme \ref{quanti-theo} montre que si
$D_m$ est fix\'e alors on minimise 
$R_m$ avec une quantification scalaire uniforme. On note $\Delta_m$
la taille des intervales de quantification. On a montr\'e en 
(\ref{unifoquantiz}) que
\[
D_m = \frac {\Delta_m^2} {12} .
\]
et (\ref{lower-quantX}) prouve que
\[
R_m = H_d (X) -  \frac 1 2 \log_2 (12 D_m) =
H_d (X) -  \log_2 \Delta_m ~,
\]
o\`u $H_d (X)$ est l'entropie diff\'erentielle de $X$ d\'efinie
par (\ref{entropie-dffD}).
\\
\\
{\bf Allocation de bits}
Si l'on fixe l'erreur totale $D$, il nous faut 
optimiser le choix de $\{\Delta_m\}_\ZmN$ 
afin de minimiser le nombre total de bits 
\[
R  = \sum_{m=0}^{N-1} R_m .
\]
Soit $\bar R = \frac R N$ le nombre moyen de bits par 
coefficient.
Le th\'eor\`eme suivant montre que le codage est optimis\'e
lorsque tous les $\Delta_m$ sont \'egaux.

\begin{theorem}
\label{bit-alloc-th} 
Pour une quantification haute r\'esolution et une erreur 
totale $D$, on minimize $\bar R$ avec
\begin{equation}
\label{allDeltamequal}
\Delta_m^2 = \frac {12\, D} N~~~\mbox{pour $0 \leq m < N$},
\end{equation}
auquel cas
\begin{equation}
\label{dist-rat-gen}
D(\bar R) = \frac N {12} \, 2^{2 \overline H_d} \, 2^{-2 \bar R} ~,
\end{equation}
o\`u $\overline H_d$ est l'entropie diff\'erentielle moyenne
\[
\overline H_d = \frac 1 N \sum_{m=0}^{N-1} H_d (A[m]) .
\]
\end{theorem}

{\bf D\'emonstration} Pour une quantification haute r\'esolution
uniforme, (\ref{bit-rate-uniform}) montre que
\[
R_m = H_d (A[m]) -  \frac 1 2 \log_2 (12 \, D_m ).
\]
Donc
\begin{equation}
\label{valueofR}
\bar R = \frac 1 N \sum_{m=0}^{N-1} R_m = 
\frac 1 N \sum_{m=0}^{N-1}  H_d (A[m]) -  
\frac 1 N \sum_{m=0}^{N-1}  \frac 1 2 \log_2 (12 \, D_m ).
\end{equation}
Minimiser $\bar R$ revient \`a minimiser
$\sum_{m=0}^{N-1}  \log_2 (12 D_m )$.
En appliquant l'in\'egalit\'e de Jensen (\ref{jensen}) \`a
la fonction concave $\phi(x) = \log_2 (x)$ pour $p_k = \frac 1 N$
on obtient
\[
\frac 1 N \sum_{m=0}^{N-1} \log_2 (12 \, D_m ) \leq 
\log_2 \left(\frac {12} N \sum_{m=0}^{N-1} D_m \right) = 
\log_2 \left(\frac {12 \, D} N \right)  .
\]
Cette in\'egalit\'e est une \'egalit\'e si et seulement si tous
les $D_m$ sont \'egaux.
Donc $\frac {\Delta_m^2} {12} = D_m = \frac D N$, ce qui
prouve (\ref{allDeltamequal}). On d\'eduit aussi de
(\ref{valueofR}) que
\[
\bar R = \frac 1 N \sum_{m=0}^{N-1}  H_d (A[m]) -  
\frac 1 2 \log_2 \left(\frac {12\, D} N \right)
\]
ce qui implique (\ref{dist-rat-gen}). $\Box$

Ce th\'eor\`eme montre que le codage est optimis\'e en
introduisant la m\^eme erreur moyenne
$D_m = \frac {\Delta_m^2} {12} = \frac D N$ dans la direction 
de chaque vecteur $g_m$ de la base. Le nombre moyen de bits
$R_m$ pour coder $A[m]$ d\'epend alors seulement 
de l'entropie diff\'erentielle: 
\begin{equation}
\label{RmOptiBiallO}
R_m = H_d (A[m]) -  \frac 1 2 \log_2 \left(\frac {12 D} N \right) .
\end{equation}
On note $\sigma_m^2$ la variance de $A[m]$, et 
$\hat A [m] = \frac 1 {\sigma_m} \, A[m]$ la variable al\'eatoire
normalis\'ee de variance $1$. Un changement de variable dans
l'int\'egrale de l'entropie diff\'erentielle montre que
\[
H_d (A[m]) = H_d (\hat A[m]) + \log_2 \sigma_m~.
\]
L'allocation de bit optimal $R_m$ donn\'e par 
(\ref{RmOptiBiallO}) peut donc devenir n\'egative si
la variance $\sigma_m$ est trop petite, ce qui n'est clairement
pas une solution admissible. 
En pratique $R_m$ doit \^etre un entier positif mais imposer cette
contrainte suppl\'ementaire ne permet pas de faire un
calcul analytique simple. La formule (\ref{RmOptiBiallO}) n'est donc
utilisable que si les valeurs $\{R_m\}_{0 \leq m < N}$ 
sont positives et on les
approxime alors par les entiers les plus proches.
\\
\\
{\bf Normes quadratiques pond\'er\'ees\ } \index{Norm!weighted}
Nous avons mentionn\'e qu'une erreur quadratique souvent ne
mesure pas bien l'erreur per\c{c}ue pour des images ou des
signaux audios. Lorsque les vecteurs
$g_m$ sont bien localis\'es en temps/espace et en fr\'equence,
on peut am\'eliorer cette norme 
en pond\'erant les erreurs par des poids qui d\'ependent
de la fr\'equence, afin de se rapprocher de notre sensibilit\'e
auditive/visuelle, qui varie avec la fr\'equence du signal.
Une norme pond\'er\'ee est d\'efinie par
\begin{equation}
\label{weight-normed}
D =  \sum_{m=0}^{N-1} \frac{D_m} {w_m^2} , 
\end{equation}
o\`u $\{ w_m^2\}_\ZmN$ sont des constantes.

Le th\'eor\`eme \ref{bit-alloc-th} s'applique
\`a une norme pond\'er\'ee en observant que
\[
D =  \sum_{m=0}^{N-1} D^w_m, 
\]
o\`u $D^w_m =  \frac {D_m} {w_m^2}$
est l'erreur de quantification de $A^w [m] = \frac {A[m]} {w_m}$. 
Le th\'eor\`eme \ref{bit-alloc-th} 
prouve que l'allocation de bits optimale
est obtenue en quantifiant uniform\'ement
tous les $A^w [m]$ avec des intervales de m\^eme tailles $\Delta$.
Cela implique que les
coefficients $A[m]$ sont uniform\'ement quantifi\'es
avec des intervales de taille
$\Delta_m =  {\Delta}\, {w_m}$, et donc
$D_m = \frac {w_m^2 D}  {N}$.
Comme on s'y attendait, en augmentant les poids $w_m$ on
augment l'erreur dans la direction de $g_m$.
La quantification uniform
$Q_{\Delta_m}$ \`a intervales 
$\Delta_m$ est souvent calcul\'ee avec un quantificateur
$Q$ qui associe a tout nombre r\'eel l'entier le plus proche:
\begin{equation}
\label{variable-unifo-quan}
Q_{\Delta_m}\Bigl(A[m]\Bigr) = \Delta_m\, Q\left (\frac {A[m]} {\Delta_m}\right) =
{\Delta}\, {w_m}\, Q \left(\frac {A[m]} {\Delta\,w_m}\right) .
\end{equation}

\section{Bases de Cosinus Locaux}

{\bf Choix de la base}
Le codage d'une classe de signaux $Y[n]$ dans une base
orthogonale $\{g_m [n]\}_\UmN$ est d'autant meilleur que la base
supprime les corr\'elations entre les coefficients $Y[n]$.
La base $\{g_m [n]\}_\UmN$ est donc choisie de fa\c con \`a obtenir des
coefficients $A[m] = <Y,g_m>$ aussi d\'ecorrel\'es que possible.
De m\^eme il est souvent d\'esirable d'obtenir des coefficients
$A[m]$ qui ont une forte probabilit\'e d'\^etre proches de z\'ero.
De tels coefficients sont en effet annul\'es par la quantification
et efficacement cod\'es par en s\'equences.
Les bases de cosinus d\'ecrites dans ce paragraphe sont bien
adapt\'ees pour le codage des signaux audios et des images.
L'existence d'algorithmes de calcul rapide bas\'es sur la transform\'ee
de Fourier rapide permet d'utiliser ces bases pour du codage
en temps r\'eel.
\\
\\
{\bf Bases de Cosinus}
Le codage de signaux r\'eels
$f[n]$ defini pour $0 \leq n < N$
se fait plut\^ot sur des bases de
cosinus que sur des bases de Fourier discr\`etes
$\{\frac 1 {\sqrt N} e ^{\frac {i2 \pi k n} N } \}_\ZkN$.
En effet nous avons vu dans le paragraphe \ref{transf-four-discr}
que la d\'ecompositon d'un signal de taille $N$ dans une base
de Fourier discr\`ete effectue une p\'eriodisation de $f[n]$ sur
$N$ \'echantillons. Si $f[0] \neq f[N-1]$, ce signal p\'eriodique
a des transitions brutales en $n=0$ et $n=N-1$ ce qui produit des
coefficients de Fourier de large amplitude. La quantification
de ces coefficients de large amplitude cr\'ee des erreurs aux
bords que l'on veut \'eviter. 
On montre que la base
de cosinus sp\'ecifi\'ee par le Th\'eor\`eme 
\ref{th-discr-cos} poss\`ede essentiellement
les m\^emes propri\'et\'es 
qu'une base de Fourier discr\`ete, mais ne 
produit 
pas des coefficients de large amplitude par effet de bord.
Cette base de cosine est bas\'ee sur une extension $\tilde f [n]$
p\'eriodique de $f[n]$,
qui \'evite l'introduction de transition brutale aux bords.

Le signal $f[n]$ d\'efini sur $0 \leq n  < N$ 
est \'etendu par sym\'etrie par rapport \`a $- \half$ en un
signal $\tilde f [n]$ de taille $2N$ 
\begin{equation} 
\tilde f[n] = 
   \left\{ \begin{array}{ll} 
        f[n]& \mbox {for $0 \leq n \leq N$}\\
        f[-n-1] &\mbox {for $-N \leq n \leq -1$}
            \end{array}
   \right.  
\end{equation}
La sym\'etrie \'evite d'introduire une transition brutale
lors de la periodisation sur $2N$ coefficients
car $\tilde f[0] = \tilde f[2N-1]$.
La transform\'ee de Fourier discr\`ete
de taille $2N$ d\'ecompose $\tilde f[n]$ comme une somme de 
sinus et de cosinus 
\[
\tilde f [n] = 
\sum_{k=0}^{N-1} a_k \cos \Bigl[\frac {2 k\pi} {2N} (n+\half)\Bigr] +
\sum_{k=0}^{N-1} b_k \sin \Bigl[\frac {2k\pi} {2N} (n+\half)\Bigr] .
\]
Comme $\tilde f [n]$ est sym\'etrique par rapport
\`a $-\half$, on d\'eduit que
$b_k = 0$ pour tout $0 \leq k < N$.
De plus $f[n] = \tilde f[n]$ pour $0 \leq n <N$, ce qui prouve 
que tout signal $f \in \R^N$ peut s'\'ecrire comme une somme
de ces cosinus. On v\'erifie aussi facilement que ces
cosinus sont
orthogonaux dans $\R^N$. On obtient donc le th\'eor\`eme suivant.


\begin{theorem}
\label{th-discr-cos}
La famille de cosinus discrets
\[
\left\{ c_k [n] = {\lambda_k} \, \sqrt {\frac 2 N}\,
\cos\Bigl[\frac {k \pi} N (n+\half)\Bigr] \right\}_\ZkN,
\]
avec
\begin{equation} 
\lambda_k = 
   \left \{ \begin{array}{ll} 
            \frac 1 {\sqrt 2 } & \mbox{si $k = 0$} \\
            1 &  \mbox{sinon} 
            \end{array}
   \right.  
\end{equation}
est une base orthonormale de $\R^N$.
\end{theorem}

Les produits scalaires
\begin{equation}
\label{DCT_I}
{\hat f_c [k]}  = <f[n], c_k[n] > = 
\lambda_k\, \sqrt {\frac 2 N}\,
\sum_{n=0} ^{N-1} f[n] \cos\Bigl[\frac {k \pi} N (n+\half)\Bigr]  
\end{equation}
d\'efinissent la transform\'ee en cosinus de 
$f[n]$. Comme ces cosinus forment une base orthonormale
\begin{equation}
\label{cosin-recons}
{f[n]}  = 
\sum_{k=0} ^{N-1} <f, c_k >\, c_k [n] =
\sum_{k=0} ^{N-1} \hat f_c [k] \,c_k [n] .
\end{equation}
En modifiant l'algorithme de transform\'ee de Fourier rapide,
on peut calculer les coefficients $\{\hat f_c [k]\}_\ZkN$
avec $O(N \log N)$ op\'erations \cite{vetterli}. 
De m\^eme la reconstruction
(\ref{cosin-recons}) s'obtient avec
$O(N \log N)$ op\'erations.\\ 
\\
{\bf Localisation temporelle}
Lorsque le signal
inclut des structures vari\'ees \`a diff\'erents instants,
il est pr\'ef\'erable de s\'eparer ces composantes avec des fen\^etres
temporelles et d'effectuer une transform\'ee en cosinus \`a l'int\'erieur
de ces fen\^etres. Cette approche est similaire \`a 
la transform\'ee de Fourier \`a fen\^etre que nous avons \'etudi\'ee
dans le chapitre \ref{chap-ana-temp-fre}.


Un signal de taille $N$ peut etre s\'epar\'e en $\frac N M$ composantes
de tailles $M$ en utilisant 
des fen\^etres rectangulaires de taille $M$
\[
g[n] = \left\{
\begin{array}{ll}
1 & \mbox{si $0 \leq n < M $}\\
0 & \mbox{sinon}
\end{array}
\right.
\]
Clairement
\[
f[n] = \sum_{p=1}^{\frac N M} g[n-pM] f[n] .
\]
Le produit $g[n - pM] f[n]$ est la
restriction de $f[n]$ pour $p M \leq n < (p+1) M$.
On peut d\'ecomposer cette restriction sur une base de
de cosinus de taille $M$ obtenue en translatant la famille
\[
\left\{ c_k [n] = {\lambda_k} \sqrt {\frac 2 M} 
\cos\Bigl[\frac {k \pi} M (n+\half)\Bigr] \right\}_\ZkM .
\]
Chaque partie 
$g[n - pM] f[n]$ se d\'ecompose sur la
famille $\{ c_k [n-pM] g[n-pM] \}_\ZkM$ restreinte \`a 
$p M \leq n < (p+1) M$.
Cela revient \`a d\'ecomposer $f[n]$ sur une base orthogonale
de taille $N$ compos\'ee de $\frac N M$ fen\^etres de taille $M$,
modul\'ees par des cosinus 
\begin{equation}
\label{base-loc-cs}
\Bigl\{ g_{p,k} [n] =
g[n-pM] c_k [n-pM] \Bigr\}_{\ZkM, 0 \leq p < \frac N M} .
\end{equation}
Cette famille est une base orthonormale
de l'espace des signaux de taille $N$.\\

{\bf Bases bidimensionnelles}
La proposition suivante montre que
des bases orthogonales d'images $f[n,m]$ de taille $N^2$ peuvent
se construire par un produit s\'eparable de bases orthogonales
de signaux mono-dimensionnels $f[n]$ de taille $N$.

\begin{proposition}
Si $\{e_k [n]\}_\ZkN$ est une
base orthonormale de l'espace des signaux de taille $N$ alors 
\[
\left\{e_{k,j} [n,m] = e_k [n] e_j [m] \right\}_{\ZkN, 0 \leq j < N}
\]
est une base orthogonale de l'espace
des images $f[n,m]$ de taille $N^2$.
\end{proposition}

{\bf D\'emonstration}
Il suffit pour cela de montrer que ces $N^2$ vecteurs sont
orthogonaux. En effet
\begin{eqnarray*}
<e_{k,j} , e_{k',j'}> &=& 
\sum_{n=0}^{N-1}\sum_{m=0}^{N-1}
e_{k,j}[n,m] e_{k',j'}^* [n,m] \\
&=&
\sum_{n=0}^{N-1}
e_{k}[n] e_{k'}^* [n] \sum_{m=0}^{N-1}
e_{j}[m] e_{j'}^* [m] .
\end{eqnarray*}
Ces produits scalaires sont donc tous nuls si $k \neq k'$
et $j \neq j'$ car $\{e_k [n]\}_\UkN$ est une base orthogonale.
$\Box$

Cette proposition nous permet de construire des 
bases
de cosinus locaux d'images par un produit s\'eparable de 
bases de cosinus locaux (\ref{base-loc-cs})
de signaux monodimensionnels.
La famille de $N^2$ fen\^etres modul\'ees
\begin{equation}
\left\{ g_{k,j} [n-pM,m-qM] = 
g[n-pM] g[m-qM] c_k [n-pM] c_j [m-qM]
\right\}_{0 \leq k,l < M, 0 \leq p,q \leq \frac N M}
\end{equation}
est une base orthogonale de l'espace des images $f[n,m]$ de
taille $N^2$.
Cette base d\'ecompose l'image en $\frac {N^2} {M^2}$ carr\'es
de taille $M$. Elle d\'ecompose ensuite
la restriction de l'image
sur chaque carr\'e de $M^2$ pixels dans une base de cosinus.

\section{Codage perceptuel}
\label{percept-code}

Les codes par transform\'ee orthogonale sont particuli\`erement
efficaces pour de larges classes de signaux pour lesquelles on ne
peut d\'efinir des mod\`eles de production.
C'est le cas pour les signaux audios ou les images.
Le choix de la base et la quantification des coefficients 
doivent cependant \^etre adapt\'es \`a la perception humaine de
fa\c con \`a produire des erreurs qui introduisent peu de d\'egradations
perceptuelles.

\subsection{Codage audio}
\label{transparent-audio}

Un disque compact m\'emorise un son audio de haute qualit\'e
avec un \'echantillonnage a
44.1 kHz. Les \'echantillons sont quantifi\'es uniform\'ement sur
16 bits ce qui produit un
``Pulse Code Modulation'' de 706kb/s.
Un code ``transparent'' peut introduire des erreurs num\'eriques
mais ces erreurs doivent rester inaudibles pour un
auditeur ``moyen''.
De forts taux de compression sont obtenus en adaptant la 
quantification aux propri\'et\'es de masquage auditif.\\
\\
{\bf Masquage auditif}
Une petite erreur de quantification n'est pas entendue si
elle est additionn\'ee \`a un signal qui a une forte \'energie
dans la m\^eme bande de fr\'equence.
Cet effet de masquage se fait \`a l'int\'erieur de bandes
de fr\'equence ``critiques'' de la forme
$[\om_c - \frac {\Delta \om} 2,\om_c + \frac {\Delta \om} 2]$,
qui ont \'et\'e mesur\'ees 
par des exp\'eriences de psycho-physiologie
auditive.
Un fort signal dont la transform\'ee de Fourier a un support
contenu dans la bande de fr\'equence
$[\om_c - \frac {\Delta \om} 2,\om_c + \frac {\Delta \om} 2]$
d\'ecro\^{\i}t la sensibilit\'e d'un auditeur
pour d'autres composantes qui sont \`a l'int\'erieur de cette
bande de fr\'equence.
Dans l'intervalle de fr\'equences
$[0,20$kHz$]$, il y a approximativement
25 bandes critiques dont la largeur
$\Delta \om$ et la fr\'equence centrale $\om_c$ satisfont
\begin{equation}
\label{criticalband}
{\Delta \om} \approx 
\left\{
\begin{array}{ll}
100 ~~\mbox{for $\om_c \leq 700$}\\
0.15 \om_c ~~\mbox{for $700 \leq \om_c \leq 20~000$}
\end{array}
\right.
\end{equation}
{\bf Quantification adaptative}
Le signal $f[n]$ est d\'ecompos\'e dans une base de cosinus locaux
\[
\Bigl\{ g_{p,k} [n] = g[n-pM] c_k [n-pM]  \Bigr\}_
{\ZkM, 0 \leq p < \frac N M} 
\]
construits avec une fen\^etre $g[n]$ couvrant g\'en\'eralement 
$M = 1024$ \'echantillons.
Par une transform\'ee de Fourier rapide, pour chaque
composante du signal $f[n] g[n-pM]$ de $M$ \'echantillons, 
on calcule l'\'energie en fr\'equence dans les
bandes critiques 
$[\om_c - \frac{\Delta \om} 2, \om_c + \frac{\Delta \om} 2 ]$.
Cette \'energie permet de calculer le niveau de masquage et 
donc l'amplitude maximum $\Delta$
des erreurs de quantification qui ne sont pas audibles dans
cette bande de fr\'equence. On quantifie alors 
uniformement avec un pas $\Delta$ chaque produit scalaire
$<f , g_{p,k}>$ pour des cosinus locaux
dont la fr\'equence se trouve \`a l'int\'erieur de la bande critique
$[\om_c - \frac{\Delta \om} 2, \om_c + \frac{\Delta \om} 2 ]$.
Cet algorithme introduit dans chaque bande critique des erreurs
de quantification qui sont au-dessous du niveau d'audition.\\
\\
{\bf MUSICAM}
L'algorithme MUSICAM
(Masking-pattern Universal Subband Integrated Coding and
Multiplexing) utilis\'e par le standard MPEG-I est 
le plus simple des
codeurs perceptuels. Il d\'ecompose le signal en 32 bandes de fr\'equences
de tailles \'egales dont la largeur  est de 750Hz.
Cette d\'ecomposition est tr\`es semblable \`a une d\'ecomposition dans
une base de cosinus locaux.
Chaque $8~ 10^{-3}$ seconde
la quantification est adapt\'ee dans chaque bande de 
fr\'equence pour tenir compte des propri\'et\'es de masquage du signal.
Ce syst\`eme comprime des signaux audios jusqu'\`a 128 kb/s sans 
introduire d'erreur audible.

\subsection{Codage d'Images par JPEG}
\label{still-image-comp}

Le standard JPEG est le plus 
couramment utilis\'e pour la compression
d'images.
L'image est d\'ecompos\'ee dans une base s\'eparable de
cosinus locaux, en utilisant des fen\^etres de $M = 8$ par 8 pixels
\[
\left\{ 
g_{k,j} [n - pM ,m - qM] = g[n-pM] g[m-qM] 
c_k [n-pM] c_j [n-qM]
\right\}_{0 \leq k,j < M, 0 \leq p,q \leq \frac N M}.
\]
Cela signifie que l'image est d\'ecompos\'ee en carr\'es de
$8$ par 8 pixels et que chacun de ces carr\'es est d\'ecompos\'e
dans une base s\'eparable de cosinus. Dans chaque fen\^etre,
les $64$ coefficients de d\'ecomposition
sont quantifi\'es uniform\'ement.
Dans les zones
o\`u l'image est r\'eguli\`ere, 
les cosinus de hautes fr\'equences g\'en\`erent des
petits coefficients, qui sont annul\'es par la quantification.
Pour coder efficacement la position de ces coefficients nuls,
on utilise un code par s\'equences.
\\
\\
{\bf Codage des z\'eros}
Enregistrer la position des coefficients nuls correspond au codage
d'une source binaire \'egale \`a 1 lorsque le coefficient est non-nul
et $0$ lorsque qu'il est nul.
On peut utiliser la redondance de cette source de $0$ et de $1$
en codant les valeurs sous forme de s\'equences.
Un code ``run-length'' enregistre la taille
$Z$ des s\'equences successives de $0$ et la taille $I$ des sequences
successives de $1$. Les variables al\'eatoires $Z$ et $I$ 
sont ensuite
enregistr\'ees avec un code entropique.
Un code ``run-length'' est un algorithme de codage vectoriel 
dont on peut d\'emontrer qu'il est optimal lorsque la s\'equence
de $0$ et de $1$ est produite par une cha\^{\i}ne de Markov d'ordre 1.
Ce type de codage binaire est utilis\'e pour la transmission de
fax.

Chaque bloc de 64 coefficients en cosinus est parcouru en
zig-zag, comme l'illustre la figure \ref{JPEG-zigzag}.
Sur ce parcours, on enregistre la taille des s\'equences
de $0$ et de $1$ correspondant aux coefficients quantifi\'es
\`a zero ou pas. 

\begin{figure}[bhtp]
\centerline{
        \epsfxsize=5cm
	\leavevmode\epsfbox{figures/chap11/fig7.eps}}
\caption{Sur une fen\^etre de 64 coefficients en cosinus, la
fr\'equence zero (DC) est en haut \`a gauche. Le codage des
z\'ero effectue un parcours en
zig-zag depuis les basses vers les hautes fr\'equences.}
\label{JPEG-zigzag}
\end{figure}

Sur chaque bloc,
le vecteur de fr\'equence
z\'ero $g_{0,0} [n-pM,m-qM]$ a une valeur constante.
Le coefficient correspondant est donc proportionel \`a 
l'intensit\'e moyenne de l'image sur le bloc.
Ces coefficients sont cod\'ees s\'epar\'ements car 
ils sont fortement
correl\'ees d'un bloc \`a l'autre. Plut\^ot que de quantifier
individuellement les
fr\'equences nulles ($k=j=0$), on code 
la diff\'erence des coefficients de fr\'equence nulles, 
entre un bloc de l'image
et le suivant qui se trouve \`a sa droite.
\\
\\
\noindent{\bf Perception visuelle}
La sensibilit\'e visuelle d\'epend de la fr\'equence 
et de l'orientation
du stimulus.
Les propri\'et\'es de masquage sont aussi importantes en vision
que pour l'audition mais sont beaucoup plus compliqu\'ees \`a mod\'eliser.
Des exp\'eriences psycho-physiologiques montrent qu'un stimulus
dont la transform\'ee de Fourier est localis\'ee dans une bande
de fr\'equence \'etroite produit un effet de masquage sur une bande
de fr\'equence de l'ordre de 
1 \`a 1.5 octave. Ce masquage d\'epend de la fr\'equence et de 
l'orientation du signal mais aussi d'autres param\`etres
d'intensit\'e, de couleur et de texture.
Cette complexit\'e rend beaucoup plus difficile l'utilisation 
des propri\'et\'es de masquage en vision.
Les codeurs adaptent donc plut\^ot l'erreur de quantification 
suivant 
une sensibilit\'e "moyenne" dans chaque bande de fr\'equence,
sans utiliser les propri\'et\'es de masquage
Nous sommes typiquement moins sensibles aux oscillations de
hautes fr\'equences qu'aux variations de basses fr\'equences.

Pour minimiser la d\'egration visuelle des images cod\'ees,
JPEG effectue une quantification avec des intervales
de quantification dont les valeurs sont proportionelles
\`a des poids qui sont calcul\'es grace des exp\'eriences
psychophysiques.
Ceci revient \`a optimiser l'erreur pond\'er\'ee
(\ref{weight-normed}).
La table \ref{TableOfWeights} est un exemple de matrice
de 8 par 8 poids qui peuvent \^etre utilis\'es par 
JPEG. Les poids des fr\'equences les
plus basses, qui apparaissent en haut \`a gauche de la 
table \ref{TableOfWeights}, sont  10 fois plus petit qu'aux
plus hautes fr\'equences, qui apparaissent en bas \`a droite.

\begin{table}
\footnotesize
\begin{center}
\begin{tabular}{| c | c | c | c | c | c | c | c |} \hline
\ 16\ \  &\ 11\ \ &\ 10\ \ &\ 16\ \ &\ 24\ \ &\ 40\ \ &\ 51\ \ &\ 61\
\ \\ \hline
 12 &  12 & 14 &  19 & 26 & 58&  60&  55\\ \hline
 14 & 13&  16&  24 & 40&  57&  69&  56  \\ \hline
 14 & 17 & 22&  29 &  51 & 87&  80 & 62\\ \hline
 18 & 22 & 37 & 56 & 68 & 108 & 103&  77\\ \hline
 24&  35 & 55&  64 & 81 & 194 & 113 & 92\\ \hline
 49 & 64 & 78&  87 & 103&  121&  120 & 101\\ \hline
 72&  92&  95 & 98 & 121&  100 & 103 & 99\\ \hline
\end{tabular}
\end{center}
\normalsize
\caption{Matrice de poids $w_\kj$ utilis\'es pour la
quantification des coefficients 
correspondant \`a chaque bloc de cosinus
$g_\kj$. Les fr\'equences augmentent de gauche \`a droite et
de haut en bas.}
\label{TableOfWeights}
\end{table}
\\
\\
{\bf Qualit\'e de compression}
Lorsque $\bar R \in [0.2,0.5]$bit/pixel, la figure \ref{JPEG} montre que
les images restor\'ees \`a partir d'un code JPEG sont de qualit\'e
mod\'er\'ee.
A fort taux de compression, on voit appara\^{\i}tre les blocs
de 8 par 8 pixels sur lesquels la transform\'ee en cosinus est
calcul\'ee.
A 0.75-1 bit/pixel, les images cod\'ees avec JPEG sont
visuellement quasiement parfaites. L'algorithme JPEG est
souvent utilis\'e avec $\bar R \in [0.5,1]$.



%\begin{figure}
\vspace{5cm}\setlength{\tabcolsep}{0cm} % Separation entre les lignes d'images
\setlength{\fboxsep}{0cm} % Separation entre la boite et l'image
\avecboite = 0

\begin{figtab}
\begin{figrow}{2} 
{{Images comprim\'ees avec l'algorithme JPEG.}
\label{JPEG}}
\figentry{6cm}{0cm}{/home/mallat/X/TREX/figures/falzon/Compression/PS/lenna.ps}&\figentry{6cm}{0cm}{/home/mallat/X/TREX/figures/falzon/Compression/PS/lenna_1.28.ps}\\
\centry{image originale}& \centry{ 1.28 bits/pixel}\\ 
\figentry{6cm}{0cm}{/home/mallat/X/TREX/figures/falzon/Compression/PS/lenna_1.09.ps} & 
\figentry{6cm}{0cm}{/home/mallat/X/TREX/figures/falzon/Compression/PS/lenna_0.4.ps} \\
\centry{ 1.09 bits/pixel} & \centry{ 0.4 bits/pixel} \\
\figentry{6cm}{0cm}{/home/mallat/X/TREX/figures/falzon/Compression/PS/lenna_0.27.ps} &
\figentry{6cm}{0cm}{/home/mallat/X/TREX/figures/falzon/Compression/PS/lenna_0.18.ps}\\
\centry{ 0.27 bits/pixel} & \centry{ 0.18 bits/pixel}\\
\end{figrow} 
\end{figtab} 

%\end{figure}




\begin{thebibliography}{99}

\bibitem{bremaud} P. Br\'emaud,
``Signaux al\'eatoires pour le traitement du signal et les communications'',
{\em Collection: Cours de l'Ecole Polytechnique, Ellipses}, 1993.

\bibitem{bremaud-proba} P. Br\'emaud,
``Introduction aux probabilit\'es'', Springer Verlag, Berlin.

\bibitem{bony} J.M. Bony,
``Cours d'Analyse'',
{\em Ecole Polytechnique}, 1994.

\bibitem{bony2} J.M. Bony,
``M\'ethodes math\'ematiques pour les sciences physiques,''
{\em Ecole Polytechnique}, 1995.

\bibitem{genat} J.F. Genat,
``Synth\`ese et traitement des sons en temps r\'eel'',
{\em Ecole Polytechnique}, 1994.

\bibitem{karar} J.F Genat et A. Karar,
``Introduction \`a l'analyse et synth\`ese de la parole'',
{\em Ecole Polytechnique}, 1994.

\bibitem{haykin} S. Haykin,
``Adaptive filter theory'',
{\em Prentice Hall}, 1991.


\bibitem{oppenheim} A. Oppenheim et R. Schafer,
``Discrete-time signal processing'',
{\em Prentice Hall}, 1989.

\bibitem{parson} T. Parsons,
``Voice and speech processing'',
{\em Mc-Graw Hill}, 1987.

\bibitem{pap} A. Papoulis,
``Signal analysis'',
{\em Mc Graw Hill}, 1977.

\bibitem{priest} M.B. Priestley,
``Spectral analysis and time series'',
{\em Academic Press}, 1981.

\bibitem{thomas} Y. Thomas,
``Signaux et syst\`emes lin\'eaires'',
{\em Masson}, 1994.

\bibitem{torresani} B. Torr\'esani,
``Analyse continue par ondelettes'',
{\em CNRS editions}, {1995}.

\bibitem{vetterli} M. Vetterli and J. Kovacevic,
``Wavelets and subband coding'',
{\em Prentice Hall}, 1995.

\end{thebibliography} 
\end{document}
