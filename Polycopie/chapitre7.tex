\chapter{Analyse Temps-Fr\'equence}
\label{chap-ana-temp-fre}

En \'{e}coutant de la musique, 
nous percevons clairement les variations temporelles
des ``fr\'{e}quences'' sonores. 
On met en \'{e}vidence les propri\'{e}t\'{e}s temporelles
et fr\'equentielles
des sons gr\^{a}ce \`{a} 
la transform\'{e}e de Fourier \`a fen\^etre,
qui d\'{e}compose les 
signaux en fonctions \'{e}l\'{e}mentaires bien concentr\'{e}es en 
temps et en fr\'{e}quence. La 
mesure des variations temporelles des
``fr\'{e}quences instantan\'{e}es'' est une application
importante, qui illustre les limitations impos\'ees par
le principe d'incertitude de Heisenberg.

\section{Transform\'ee de Fourier \`a Fen\^etre}
\label{temp-fre-se}

On peut classifier les sons suivant leurs propri\'et\'es 
fr\'equentielles. 
Par exemple, les fr\'equences de r\'esonance du conduit
vocale produisent des ``formants'' qui caract\'erisent les voyelles.
La transform\'ee de Fourier ne peut pas \^etre utilis\'ee car
\[
\hat f (\om) = \int_{-\infty}^{+\infty} f(t) \Exp^{-i\om t} dt
\]
d\'epend des valeurs de $f(t)$ \`a tout instant. Pour diff\'erencier
des sons produits successivement, on d\'efinit une  
transform\'ee de Fourier \`a fen\^etre qui 
s\'epare les diff\'erentes
composantes du signal gr\^ace \`a une fen\^etre translat\'ee.

La fen\^etre $g(t)$
est une fonction paire dont le support est concentr\'e au
voisinage de $0$. 
La transform\'ee de Fourier \`a fen\^etre
au voisinage de $u$, \`a la fr\'equence $\xi$ est d\'efinie par
\[
Sf(u,\xi) = \int_{-\infty}^{+\infty} f(t) \,g(t-u) \,\Exp^{-i\om t} dt .
\]
\\
\\
{\bf Localisation temps-fr\'equence}
On d\'efinit
\[
g_\uxi (t) = g(t-u) \,\Exp^{i\xi t}
\] 
qui peut \^etre interpr\'et\'e comme une ``note de musique''
localis\'ee au voisinage de $t =u$, et autour de 
la fr\'equence $\xi$.
La transform\'ee de Fourier \`a fen\^etre mesure la corr\'elation
entre le signal $f(t)$ et cette note \'el\'ementaire
\begin{equation}
\label{temp-fre-at}
Sf(u,\xi) = \int_{-\infty}^{+\infty} f(t) g_\uxi^*(t) dt .
\end{equation}
Comme $g$ est paire, $g_{u,\xi} (t ) = \Exp^{i \xi t}	g(t-u)$ en 
centr\'{e} sur $u$. 
Le produit $f\, g_\uxi$ isole donc les composantes de
$f$ au voisinage de $u$, et $Sf(u,\xi)$ ne d\'epend que
des propri\'et\'es de $f$ dans ce voisinage.
L'\'{e}talement en temps de $g_\uxi$ est 
ind\'{e}pendant de $u$ et de $\xi$:
\begin{equation}
\label{sigma-t}
\sigma^2_t = \int_{-\infty}^{+\infty} (t-u) ^2 \,
|g_\uxi (t)|^2\, dt = 
\int_{-\infty}^{+\infty} t ^2 \,|g(t)|^2\, dt .
\end{equation}

Si l'on applique la formule de Parseval (\ref{parseval})
\`a (\ref{temp-fre-at}), on obtient une int\'egrale fr\'equentielle
\[
Sf(u,\xi) = \frac 1 {2 \pi} 
\int_{-\infty}^{+\infty} \hat f(\om) \,\hat g_\uxi^* (\om)\, d\om .
\]
La valeur $Sf(u,\xi)$
ne d\'epend donc que du comportement de $\hat f (\om)$ dans
le domaine fr\'equentiel o\`u $\hat g_\uxi^* (\om)$ n'est pas
n\'egligeable.
La transform\'{e}e de Fourier $\hat g$ de $g$ est r\'{e}elle et 
sym\'{e}trique car $g$ est r\'{e}elle et sym\'{e}trique. 
En utilisant les propri\'et\'es (\ref{trans}) et (\ref{modul}),
on montre que
la transform\'ee de Fourier de $g_\uxi (t) = g(t-u) e^{i\xi t}$
peut s'\'ecrire
\[
\hat g_\uxi (\om) = \Exp^{-iu(\om-\xi)} \hat g(\om - \xi) .
\] 
C'est donc une fonction centr\'ee \`a la fr\'equence $\om = \xi$.
Son \'{e}talement fr\'{e}quentiel autour de $\xi$ vaut
\begin{equation}
\label{sigma-om}
\sigma^2_\om = \frac 1 {2 \pi}
\int_{-\infty}^{+\infty} (\om-\xi) ^2\, |\hat g_\uxi (\om)|\, d\om = 
\frac 1 {2 \pi} \int_{-\infty}^{+\infty} \om ^2\, |\hat g(\om)|\, d\om .
\end{equation}
Il est ind\'{e}pendant de $u$ et de $\xi$. 
Dans un plan temps-fr\'equence $(t,\om)$, 
on repr\'esente $g_{\uxi}$ par une 
{\it bo\^{\i}te de Heisenberg} 
de taille $\sigma_t \times \sigma_\om$, 
centr\'{e}e en $(u,\xi)$, comme on peut le voir sur la figure 
\ref{4.2}. 
La taille de cette bo\^{\i}te ne d\'{e}pend pas de $(u,\xi)$, 
ce qui veut dire que la r\'{e}solution de la transform\'{e}e de 
Fourier fen\^{e}tr\'{e}e reste constante sur tout le plan 
temps-fr\'{e}quence.

\begin{figure}[bhtp]
\centerline{
        \epsfxsize=6cm
        \leavevmode\epsfbox{figures/chap1/wf-tfloc.eps}}
\caption{
Les bo\^{\i}tes de Heisenberg de deux atomes de Fourier fen\^{e}tr\'{e}s
$g_\uxi$ and $g_{\nu,\gamma}$.}
\label{4.2}
\end{figure}
\\
\\
\noindent
{\bf Incertitude de Heisenberg}
Pour mesurer les composantes de $f$ et de $\hat f$ dans
des petits voisinages de $u$ et $\xi$ il faut construire
une fen\^etre $g(t)$ qui est 
bien localis\'{e}e dans le temps, et dont l'\'{e}nergie de la 
transform\'{e}e de Fourier est concentr\'{e}e dans un petit domaine 
fr\'{e}quentiel. 
Le Dirac $g(t) = \delta(t)$ a un support ponctuel $t=0$, 
mais sa transform\'{e}e de Fourier  $\hat \delta (\om) = 1$ a une 
\'{e}nergie qui est distribu\'{e}e uniform\'{e}ment sur toutes les 
fr\'{e}quences. On sait que $|\hat g(\om)|$ d\'{e}croit rapidement 
dans les hautes fr\'{e}quences seulement si $g(t)$ est une
fonction qui varie r\'eguli\`erement.
L'\'{e}nergie de $g$ est donc n\'ecessairement 
r\'{e}partie sur un domaine temporel relativement large.

Pour r\'{e}duire l'\'{e}talement temporel de $g$, 
on peut op\'{e}rer un 
changement d'\'{e}chelle de temps d'un facteur $s<1$ sans changer son 
\'{e}nergie totale, soit
\[
g_s (t) = \frac 1 {\sqrt s} g( \frac t s ) ~~~\mbox{avec}~~~
\| g_s \|^2 = \|g \|^2.
\]
La transform\'{e}e de Fourier $\hat g_s ( \om ) = \sqrt s\, \hat	g( s \om )$
est dilat\'{e}e d'un facteur $1/s$, et on perd donc en fr\'{e}quentiel 
ce qu'on a gagn\'{e} en temporel. On voit appara\^{\i}tre un compromis 
entre la localisation en temps et celle en fr\'{e}quence.

Les concentrations en temps en en fr\'{e}quences sont limit\'{e}es par 
le principe d'incertitude d'Heisenberg\index{Principe 
d'incertitude}\index{Heisenberg!principe d'incertitude}. Ce principe 
d'incertitude \`{a} une interpr\'{e}tation, particuli\`{e}rement 
importante en m\'{e}canique quantique, comme une incertitude sur la 
position et l'impulsion d'une particule libre.
Plus $\sigma_t$ et $\sigma_\om$ sont grandes, plus on a 
d'incertitude 
sur la position et l'impulsion de la particule libre.
Le th\'eor\`eme suivant montre que le produit
$\sigma_t \times \sigma_\om$ ne peut \^etre arbitrairement
petit.

\begin{theorem}
[Incertitude de Heisenberg]
\label{uncert}
On suppose que $g \in \LD$ est une fonction centr\'ee en
$0$ et dont la transform\'ee de Fourier est aussi centr\'ee en
$0$:
\[
\int_{-\infty}^{+\infty} t \,|g(t)|^2\, dt = 
\int_{-\infty}^{+\infty} \om \,|\hat g(\om)|^2\, d\om = 0 ~.
\]
Alors les variances definies en (\ref{sigma-t},\ref{sigma-om})
satisfont
\begin{equation}
\label{uncertainty}
\sigma^2_t\, \sigma^2_\om \geq \frac 1 4 .
\end{equation}

Cette in\'{e}galit\'{e} est une \'{e}galit\'{e} si et seulement si il 
existe $(a,b) \in \C^2$  tel que
\begin{equation}
\label{Gabor}
g(t) =  a \, \Exp^{-b t^2} .
\end{equation}
\end{theorem}

{\bf D\'emonstration}
La preuve suivante, due \`{a} Weyl, suppose que
$\lim _{|t| \rightarrow + \infty} \sqrt t g(t) = 0$,
mais le th\'{e}or\`{e}me est vrai pour tout $g \in \LD$. 
Remarquons que
\begin{equation}
\sigma^2_t\, \sigma^2_\om = \frac 1 {2 \pi \|g\|^4}
{\int_{- \infty}^{+ \infty} |t\, g(t)|^2\, dt} \,
{\int_{- \infty}^{+ \infty} |\om\, \hat g( \om )|^2 \,d\om} .
\end{equation}
Comme $i	\om	\hat g(\om)$ est la transform\'{e}e de Fourier de $g'(t)$,
l'identit\'{e} de Plancherel (\ref{plancherel}) appliqu\'{e}e \`{a} 
$i	\om	\hat g(	\om	)$ donne
\begin{equation}
\label{tobeplisch}
\sigma^2_t \sigma^2_\om = \frac 1 {\|g\|^4}
{\int_{- \infty}^{+ \infty} |t \,g(t)|^2 \,dt} \,
{\int_{- \infty}^{+ \infty} |g'( t )|^2\, dt} .
\end{equation}
L'in\'{e}galit\'{e} de Schwarz implique
\begin{eqnarray*}
\sigma^2_t \sigma^2_\om & \geq & \frac 1 {\|g\|^4}
\left[ {\int_{- \infty}^{+ \infty} |t\, g'(t) \,g^* (t)|\, dt} \right]^2\\
& \geq & \frac 1 {\|g\|^4} \left[
{\int_{- \infty}^{+ \infty} \frac t 2\, [ g'(t)\, g^* (t) 
+ {g'}^*(t)\, g(t)]\, dt}
\right]^2\\  
&\geq &\frac 1 { 4 \|g\|^4}
\left[ {\int_{- \infty}^{+ \infty} t \,(|g(t)|^2)'\, dt} \right]^2 .
\end{eqnarray*}
Comme
$\lim _{|t| \rightarrow + \infty} \sqrt t \,g(t) = 0$,
on obtient, apr\`{e}s int\'{e}gration par parties
\begin{equation}
\sigma^2_t \sigma^2_\om \geq \frac 1 { 4 \|g\|^4}
\left[ {\int_{- \infty}^{+ \infty} |g(t)|^2 \,dt} \right]^2 
= \frac 1 4 .
\end{equation}
Pour atteindre l'\'{e}galit\'{e}, il faut que l'in\'{e}galit\'{e} de 
Schwarz appliqu\'{e}e \`{a} (\ref{tobeplisch}) soit elle-m\^{e}me une 
\'{e}galit\'{e}. Cela implique qu'il existe $b \in \C$ tel que
\begin{equation}
g'(t) = -2\,b\, t\, g(t) .
\end{equation}
Il existe donc $a \in \C$ tel que
$g(t) = a \,\Exp^{-b t^2}$.
Les in\'{e}galit\'{e}s suivantes dans la preuve sont alors des 
\'{e}galit\'{e}s, ce qui fait qu'on atteint effectivement le minorant.
$\Box$


En m\'{e}canique quantique, ce th\'{e}or\`{e}me montre qu'on ne peut 
arbitrairement r\'{e}duire l'incertitude \`{a} la fois sur la position 
et sur l'impulsion d'une particule libre. Les gaussiennes 
(\ref{Gabor}) ont une localisation minimale \`{a} la fois en 
temps et en fr\'{e}quences.
\\
\\
{\bf Spectrogramme}
On peut associer \`a la transfrom\'ee de Fourier \`a fen\^etre
une densit\'{e} 
d'\'{e}nergie qu'on appelle {\it spectrogramme}, et qu'on note $P_S$:
\begin{equation}
\label{SpectroDef}
P_S f (u,\xi) = |Sf(u,\xi)|^2 
= \left|\int_{- \infty}^{+ \infty} f(t)\, g(t-u)\, \Exp^{- i \xi t} \,dt 
\right|^2 .
\end{equation}
Il mesure l'\'{e}nergie de $f$ et de $\hat f$ dans le voisinage 
temps-fr\'{e}quence ou l'\'energie de $g_\uxi$ est concentr\'ee.

\begin{Examples} 
\item 
Une sinuso\"{\i}de $f(t)=	\Exp^{i	\xi_0 t}$, dont la transform\'{e}e 
de Fourier est le Dirac $\hat f(\om) = 2 \pi \delta	(\om - \xi_0)$ a 
pour transform\'{e}e de Fourier \`a fen\^{e}tr{e}
\[
Sf(u,\xi) = \Exp^{-iu(\xi - \xi_0)} \, \hat g(\xi - \xi_0). 
\]
Son \'{e}nergie est r\'{e}partie sur l'intervalle fr\'{e}quentiel
$[\xi_0 - \frac {\sigma_\om} 2 , \xi_0 + \frac {\sigma_\om} 2 ]$.

\item 
La transform\'{e}e de Fourier fen\^{e}tr\'{e}e d'un Dirac
$f(t) = \delta	(t - u_0)$ vaut
\[
Sf(u,\xi) = \Exp^{-i \xi u_0} \,g(u_0 - u) .
\]
Son \'{e}nergie est localis\'{e}e dans l'intervalle temporel
$[u_0 - \frac {\sigma_t} 2 , u_0 + \frac {\sigma_t} 2 ]$.

\item 
Dans la figure \ref{WFTChirps}, on voit le spectrogramme 
d'un signal qui a une composantes dont la ``fr\'equence
instantan\'ee''
augmente lin\'eairement dans le temps (chirp lin\'eaire)
et une seconde composante
dont la fr\'equence d\'ecroit de fa\c{c}on quadratique 
dans le temps (chirp quadratique). S'ajoute \`a cela 
deux gaussiennes modul\'{e}es. 
On a calcul\'{e} le spectrogramme avec une 
fen\^{e}tre gaussienne dilat\'{e}e d'un facteur $s = 50$. 
Le chirp lin\'{e}aire a des coefficients de grande amplitude 
le long de la trajectoire de sa fr\'{e}quence instantan\'ee.
Le chirp quadratique donne des grands coefficients le long 
d'une parabole. Les deux gaussiennes modul\'{e}es donnent deux taches 
fr\'{e}quentielles \`{a} haute et basse fr\'{e}quence, en
 $u =	512$ et $u	= 896$.

\item La figure \ref{spectrogram-parole} 
montre le spectrogramme du son ``greasy'' dont le graphique
est donn\'e au-dessus. L'amplitude de $|Sf(u,\xi)|^2$ est
d'autant plus grande que l'image du spectrogramme est sombre.
Le ``ea'' tout comme le ``y'' sont
des sons vois\'es dont les formants apparaissent clairement sur
le spectrogramme. Le ``s'' est un son non-vois\'e dont l'\'energie
est diffus\'e en hautes fr\'equences.
\end{Examples}

\vspace{0cm}\setlength{\tabcolsep}{0cm} % Separation entre les lignes d'images
\setlength{\fboxsep}{0cm} % Separation entre la boite et l'image
\avecboite = 0
\begin{figtab}
\begin{figrow}{4}
{{\label{WFTChirps}
Le signal comprend un chirp lin\'{e}aire de fr\'{e}quence croissante, 
un chirp quadratique de fr\'{e}quence d\'{e}croissante, et deux 
gaussiennes modul\'{e}es situ\'{e}es en $t=512$ et $t=896$.
(a) Spectrogramme $P_S	f(u,\xi)$. Les axes horizontaux et
verticaux correspondent respectivement au temps $u$ et \`a la
fr\'equence $\xi$. Les points sombres correspondent
\`{a} des coefficients de grande amplitude.
(b) Phase complexe de $Sf(u,\xi)$ dans les r\'{e}gions o\`{u} le 
module de $P_S	f (u,\xi)$ est non nul.}}\\
\figentry{8cm}{0cm}{figures/chap4/WFTChirps.eps}\\

\centry{(a)}\\

\figentry{8cm}{0cm}{figures/chap4/WFTPhaseChirps.eps}\\

\centry{(b)}
\end{figrow} 
\end{figtab}

\setlength{\tabcolsep}{0cm} % Separation entre les lignes d'images
\setlength{\fboxsep}{0cm} % Separation entre la boite et l'image
\avecboite = 0
\begin{figtab}
\begin{figrow}{2} {{Le graphique du dessus correspond au son ``greasy''
enregistr\'e \`a 8kHz. Son spectrogramme $|Sf(u,\xi)|^2$
est montr\'e au-dessous, dans le plan temps-fr\'equence.}
\label{spectrogram-parole}}\\

\figentry{10cm}{0cm}{/home/mallat/X/TREX/figures/gribonval/greasy.ps} \\
\avecboite = 1
\figentry{8cm}{10cm}{/home/mallat/X/TREX/figures/gribonval/sonog1.ps} 

\end{figrow} 
\end{figtab} 
\\
\\
\noindent
{\bf Compl\'etude et stabilit\'e}
Lorsque les coordonn\'{e}es temps-fr\'{e}quence $(u,\xi)$ parcourent 
$\R^2$, les bo\^{\i}tes de Heisenberg des atomes $g_\uxi$ recouvrent 
tout le plan temps-fr\'{e}quence. On peut donc s'attendre \`{a} 
pouvoir reconstituer $f$ \`{a} partir de sa transform\'{e}e de 
Fourier \`a fen\^{e}tre $Sf(u,\xi)$. Le th\'{e}or\`{e}me suivant nous fournit 
une formule de reconstruction et montre qu'on a conservation de 
l'\'{e}nergie.

\begin{theorem}
\label{window-four-form}
\label{formual-WF}
Si $f \in \LD$ 
alors
\begin{equation}
\label{inverse-WF}
f(t) = \frac 1 {2 \pi} \int_{-\infty}^{+\infty}  \int_{-\infty}^{+\infty} 
Sf(u,\xi) \,g(t-u) \,\Exp^{i \xi t} \, d\xi \, du 
\end{equation}
et 
\begin{equation}
\label{energy-WF}
\int_{-\infty}^{+\infty} |f(t)|^2 \,dt = 
\frac 1 {2 \pi} \int_{-\infty}^{+\infty}  \int_{-\infty}^{+\infty} 
|S f(u,\xi)|^2 \, d\xi \, du .
\end{equation}
\end{theorem}

{\bf D\'emonstration\ }
On commence par la preuve de la formule de reconstruction 
(\ref{inverse-WF}). Appliquons la formule de Fourier Parseval 
(\ref{parseval}) \`{a} l'int\'{e}grale (\ref{inverse-WF}) en la 
variable $u$. On calcule la transform\'{e}e de Fourier de 
$f_\xi (u) = Sf(u,\xi)$ en $u$ en remarquant que
\begin{equation}
\label{wind-filt}
Sf(u,\xi) = \Exp^{-iu\xi} 
\int_{- \infty}^{+ \infty} f(t) \,g(t-u)\, \Exp^{i \xi (u-t)}\, dt
= \Exp^{-iu\xi} \,f \star g_\xi (u) ,
\end{equation}
avec $g_\xi	(t)	= g(t) \Exp^{i \xi t}$, car $g(t) =	g(-t)$. Sa 
transform\'{e}e de Fourier vaut donc
\[
\hat f_\xi (\om) = \hat f(\om + \xi) \,\hat g_\xi (\om + \xi) = 
\hat f(\om + \xi)\, \hat g( \om ) .
\]
La transform\'{e}e de Fourier de $g(t-u)$ en $u$ vaut 
$\hat	g(\om) \Exp^{-it\om}$. On a donc
\begin{eqnarray*}
\frac 1 {2 \pi} \left( \int_{-\infty}^{+\infty}  
\int_{-\infty}^{+\infty} Sf(u,\xi) \,g(t-u) \,
\Exp^{i \xi t}\,  du \right) d\xi & = & \\
\frac 1 {2 \pi} \int_{-\infty}^{+\infty}  \left(
\frac 1 {2 \pi} \int_{-\infty}^{+\infty}  
\hat f(\om+\xi)\, |\hat g(\om)|^2\, \Exp^{it(\om+\xi)}\, d \om \right)  d\xi &  . &
\end{eqnarray*}
Si  on peut appliquer le th\'{e}or\`{e}me de Fubini pour 
changer l'ordre d'int\'{e}gration. Le th\'{e}or\`{e}me de 
transform\'{e}e de Fourier inverse montre que
\[
\frac 1 {2 \pi} \int_{-\infty}^{+\infty}  
\hat f(\om + \xi) \,\Exp^{it(\om+\xi)}\,  d\xi = f(t) .
\]
Comme
$\frac 1 {2 \pi} \int_{-\infty}^{+\infty}  
|\hat g(\om)|^2\,  d\om  = 1,$
on en d\'{e}duit (\ref{inverse-WF}). Si $\hat	f \nnin	\,\LU$,
on d\'{e}montre la formule \`{a} partir de l\`{a} gr\^{a}ce \`{a}
un argument de densit\'{e}.

Occupons nous maintenant de la conservation de l'\'{e}nergie 
(\ref{energy-WF}). Comme la transform\'{e}e de Fourier en $u$ de 
$Sf(u,\xi)$ est $\hat	f(\om +	\xi) \,\hat	g(\om )$,
on obtient, en appliquant la formule de Plancherel au membre de 
droite de (\ref{energy-WF}):
\[
\frac 1 {2 \pi} \int_{-\infty}^{+\infty}  \int_{-\infty}^{+\infty} 
|Sf(u,\xi)|^2 \,du\, d\xi = 
\frac 1 {2 \pi} \int_{-\infty}^{+\infty}  
\frac 1 {2 \pi} \int_{-\infty}^{+\infty}  
|\hat f(\om + \xi) \,\hat g(\om ) |^2 \,d\om\, d\xi .
\]
On peut appliquer le th\'{e}or\`{e}me de Fubini, et la formule de 
Plancherel montre que
\[
\frac 1 {2 \pi} \int_{-\infty}^{+\infty}  
|\hat f(\om + \xi) |^2\,  d\xi = \|f \|^2,
\]
ce qui implique (\ref{energy-WF}). $\Box$


On peut r\'{e}\'{e}crire la formule de reconstruction 
(\ref{inverse-WF}) sous la forme
\begin{equation}
f(t) = \frac 1 {2 \pi} \int_{-\infty}^{+\infty}  \int_{-\infty}^{+\infty} 
\lb f,g_\uxi\rb \, g_\uxi (t)\,  d\xi\, du .
\end{equation}
Cette formule ressemble \`{a} celle d'une d\'{e}composition sur une base 
orthogonale, mais ce n'est pas le cas, car la famille $\{g_\uxi \}_\uxiR$ 
est largement redondante dans $\LD$. La seconde identit\'{e} 
(\ref{energy-WF})  justifie qu'on interpr\`{e}te le spectrogramme 
$P_S f(u,\xi)=|Sf(u,\xi)|^2$ comme une densit\'{e} d'\'{e}nergie, car 
son int\'{e}grale en temps-fr\'{e}quence est \'{e}gale \`{a} 
l'\'{e}nergie du signal.
\\
\\
{\bf Discr\'{e}tisation}
La discr\'{e}tisation et le calcul rapide de la transform\'{e}e de 
Fourier \`a fen\^{e}tre rel\`{e}ve des m\^{e}mes id\'{e}es que la 
discr\'{e}tisation de la transform\'{e}e de Fourier classique, 
d\'{e}crite pr\'{e}c\'edemment dans le paragraphe \ref{finite-sig}. On
consid\`{e}re des signaux discrets de p\'{e}riode $N$. On prend comme 
fen\^{e}tre $g[n]$ un signal discret sym\'{e}trique et de p\'{e}riode 
$N$ et de norme unit\'{e} $\|g \| =	1.$ On d\'{e}finit les atomes de 
Fourier fen\^{e}tr\'{e}s discrets
\index{Transform\'{e}e de Fourier!fen\^{e}tr\'{e}e!discr\`{e}te}
\index{Fourier!transform\'{e}e de!fen\^{e}tr\'{e}e discr\`{e}te}
par
\[
g_\ml [n] = g[n-m] \, \Exp^{\frac {i 2 \pi  ln} N}. 
\]
La transform\'{e}e de Fourier discr\`{e}te $g_\ml$ a pour valeurs
\[
\hat g_\ml [k] = \hat g[k-l] \, \Exp^{\frac {-i 2 \pi  m (k-l)} N}. 
\]
La transform\'{e}e de Fourier fen\^{e}tr\'{e}e discr\`{e}te d'un 
signal de p\'{e}riode $N$ est
\begin{equation}
\label{DisWdinFTa}
Sf[m,l] = \lb f,g_\ml\rb  = \sum_{n=0}^{N-1} 
f[n]\, g[n-m]\, \Exp^{\frac {-i2\pi ln} N} ,
\end{equation}
Pour chaque $0 \leq	m <	N$, on calcule $Sf[m,l]$ pour $0 \leq l < N$ 
par transform\'{e}e de Fourier discr\`{e}te sur $f[n] g[n-m]$. On 
r\'{e}alise ce calcul au moyen de $N$ FFT de taille $N$, ce qui donne 
un total de $O(N^2 \log_2 N)$ op\'{e}rations. C'est cet algorithme qui 
a servi pour le calcul des figures \ref{WFTChirps} et 
\ref{spectrogram-parole}.


\section{Fr\'{e}quence instantan\'{e}e}
\label{inst-fre-sec}

Dans un morceau de musique, on distingue plusieurs fr\'{e}quences 
variant dans le temps. Il reste \`{a} d\'{e}finire la notion de 
fr\'{e}quence instantan\'{e}e.  Afin d'estimer plusieurs 
fr\'{e}quences instantan\'{e}es, on s\'{e}pare les composantes 
fr\'{e}quentielles \`{a} l'aide d'une transform\'{e}e de 
r\'{e}solution suffisante en fr\'{e}quence, mais \'{e}galement en 
temps, de mani\`{e}re \`{a} pouvoir r\'{e}aliser des mesures variables 
dans le temps. On \'{e}tudie la mesure des fr\'{e}quences 
instantan\'{e}es par des transform\'{e}es de Fourier 
fen\^{e}tr\'{e}es.
\\
\\
{\bf 
Fr\'{e}quence instantan\'{e}e analytique\ } 
Un cosinus modul\'{e}
\[
f(t) = a \cos (w_0 t + \phi_0 ) = a \cos \phi(t)
\]
a une fr\'{e}quence $\om_0$ \'{e}gale \`{a} la d\'{e}riv\'{e}e de la 
phase $\phi(t) =	w_0	t +	\phi_0$. Afin de g\'{e}n\'{e}raliser cette 
notion, on \'{e}crit les signaux r\'{e}els $f$ comme ayant une 
amplitude $a$ et un phase $\phi$ variant dans le temps:
\begin{equation}
\label{phase-fre}
f(t) = a(t)\, \cos \phi(t) ~~\mbox{
avec}~~ a(t) \geq 0~.
\end{equation}
On d\'{e}finit {\it la fr\'{e}quence instantan\'{e}e} comme la 
d\'{e}riv\'{e}e de la phase:\index{Analytique!signal}
\[
\om(t) = \phi' (t) \geq 0 ~.
\]
On peut se ramener \`{a} une d\'{e}riv\'{e}e positive en jouant sur le 
signe de $\phi (t)$. Il convient n\'{e}anmoins d'\^{e}tre prudent car
il existe de nombreuses valeurs pour $a(t)$ et de  $\phi (t)$,
et $\om (t)$ n'est donc pas d\'{e}fini de mani\`{e}re unique pour un 
$f$ donn\'{e}.

On obtient une 
d\'{e}compostion particuli\`{e}re de type (\ref{phase-fre}) 
en calculant la partie analytique $f_a $ de $f$, 
dont la transform\'{e}e de Fourier est d\'efinie par
\begin{equation}
\label{analyt-par-f}
\hat f_a (\om) = \left\{
\begin{array}{ll}
2\, \hat f(\om) & \mbox{si $\om \geq 0$}\\
0 & \mbox{si $\om < 0$}
\end{array} 
\right. .
\end{equation}
On dit que le signal complexe $f_a (t)$ est analytique car
on peut d\'emontrer qu'il a une extension analytique sur
le demi plan complexe sup\'erieur.
Par ailleurs on peut verifier que
$f = \Real [f_a ]$ car
$\hat f (\om) = \frac 1 2 (\hat f_a (\om) + \hat f_a^* (-\om))$.

On peut repr\'{e}senter $f_a$ en s\'{e}parant le module de la phase 
complexe
\[
f_a (t) = a (t)\, \Exp^{i \phi (t)}~.
\]
On en d\'{e}duit
\[
f(t) = a(t) \,\cos \phi(t) .
\]
On dira que $a(t)$ est l'amplitude {\it analytique} de $f(t)$, et 
$\phi'(t) $ sa fr\'{e}quence analytique
instantan\'{e}e; elles sont d\'{e}finies 
de mani\`{e}re unique.

\begin{Example}
\item Si $f(t) = a(t)\, \cos (\om_0 t + \phi_0)$, alors
\[
\hat f (\om) =  \frac 1 2
\left(
\Exp^{i \phi_0} \, \hat a(\om - \om_0) +
\Exp^{-i \phi_0} \, \hat a(\om +\om_0)
\right).
\]
Si les variations de $a(t)$ sont lentes en comparaison de la 
p\'{e}riode $\frac {2 \pi} {\om_0}$, ce qu'on peut obtenir en 
for\c{c}ant le support de
$\hat a (\om)$ \`{a} \^{e}tre dans $[-\om_0,\om_0]$, alors
\[
\hat f_a (\om) = \Exp^{i \phi_0} \, \hat a(\om - \om_0)
\]
d'o\`{u} $f_a (t) = a(t)\, \Exp^{i (\om_0 t + \phi_0)}$.
\end{Example}

Si $f$ est un  signal constitu\'{e} de la somme de deux 
sinuso\"{\i}des:
\[
f (t) =  a \cos({ \om_1 t}) + a \cos({ \om_2 t}) ,
\]
alors
\[
f_a (t) = a \,\Exp^{i \om_1 t} + a \,\Exp^{i \om_2 t} =
 a \, \cos \left( \frac {\om_1 - \om_2} 2\, t \right)
\,\Exp^{i \frac {\om_1 + \om_2} 2 t}.  
\]
La fr\'{e}quence instantan\'{e}e vaut
$\phi'(t) = \frac {\om_1 + \om_2} 2$
et l'amplitude instan\'ee est
\[
a(t) = a \, \left| \cos \left(\frac {\om_1 - \om_2} 2 \,t\right)\right| .
\]
Ce r\'{e}sultat n'est pas satisfaisant parce qu'il ne montre pas que 
le signal est compos\'{e} de deux sinuso\"{\i}des de m\^{e}me 
amplitude. On a obtenu une fr\'{e}quence moyenne. Nous allons 
expliquer dans la section qui suit comment mesurer les 
fr\'{e}quences instantan\'{e}es de plusieurs composantes spectrales en 
les s\'{e}parant \`{a} l'aide de transform\'{e}es de Fourier 
\`a fen\^{e}tre.
\\
\\
{\bf Modulation de fr\'{e}quence\ }
En communications, on peut transmettre l'information \`{a} travers 
son amplitude $a(t)$ (modulation d'amplitude) ou sa fr\'{e}quence 
instantan\'{e}e $\phi'(t)$ (modulation de fr\'{e}quence).
La modulation de fr\'{e}quence est plus robuste en pr\'{e}sence de 
bruits blancs gaussiens additifs. De plus, elle r\'{e}siste 
mieux aux 
interf\'{e}rences entre chemins 
multiples, qui d\'{e}truisent l'information d'amplitude. Une 
modulation de fr\'{e}quence envoie un message $m(t)$ \`{a} 
travers un signal
\[
f(t) = a \cos \phi (t)~~~\mbox{with}~~~ 
\phi'(t) = \om_0 + k \,m(t) .
\]
La largeur de bande de $f$ est proportionnelle \`{a} $k$. On 
ajuste cette constante en fonction des bruits de transmission et de la 
bande passante disponible. A la r\'{e}ception, on r\'{e}cup\`{e}re le 
message $m(t)$ gr\^{a}ce \`{a} une d\'{e}modulation de fr\'{e}quence 
qui calcule la fr\'{e}quence instantan\'{e}e $\phi'(t)$.
\\
\\
{\bf Mod\`{e}les de sons additifs\ }
On peux mod\'{e}liser les sons musicaux et les phon\`{e}mes comme des somme de
{\it partielles} sinuso\"{\i}dales: \index{Partielle}
\begin{equation}
\label{NewSoundMOd}
f(t) = \sum_{k=1}^K f_k (t) =  \sum_{k=1}^K a_k (t) \cos \phi_k (t)~ ,
\end{equation}
o\`{u} $a_k$ et $\phi^\prime_k$	sont lentement variables.
De telles d\'{e}compositions sont 
utilis\'{e}es pour reconna\^{i}tre des formes et modifier des sons.
Le paragraphe \ref{sec-wind-four-ridges} explique
comment calculer les $a_k$ et les fr\'{e}quences instantan\'{e}es 
$\phi_k'$ de chaque partielle, 
dont on d\'{e}duit la phase $\phi_k$ par 
int\'{e}gration.

Pour comprimer de son $f$ d'un facteur $\alpha$ dans le temps, et sans 
modifier les valeurs des $\phi_k'$ et des $a_k$, on synth\'{e}tise
\begin{equation}
\label{Accelerate-sound}
g (t) =  \sum_{k=1}^K a_k (\alpha\, t)\,
\cos \Bigl( \frac 1 \alpha \,\phi_k (\alpha\, t) \Bigr) .
\end{equation}
Les partielles de $g$ en $t = \alpha \, t_0$ et les partielles de $f$ 
en $t = t_0$ ont les m\^{e}mes amplitudes et les m\^{e}mes 
fr\'{e}quences instantan\'{e}es. Pour $\alpha >	1$, le son $g$ est 
plus court que $f$ tout en \'{e}tant per\c{c}u comme ayant le 
m\^{e}me ``contenu fr\'{e}quentiel'' que $f$.

On op\`{e}re un d\'{e}placement fr\'{e}quentiel en multipliant chaque 
phase par une constante $\alpha$:
\begin{equation}
\label{Transpose-sound}
g (t) =  \sum_{k=1}^K b_k (t)\,
\cos \Bigl( \alpha\, \phi_k (t ) \Bigr) .
\end{equation}
La fr\'{e}quence instantan\'{e}e de chaque partielle vaut maintenant 
$\alpha \, \phi_k'(t)$. Pour calculer les nouvelles amplitudes $b_k (t)$, 
on utilise un mod\`{e}le de r\'{e}sonnance, qui suppose que ces 
amplitudes sont les \'{e}chantillons d'une enveloppe fr\'{e}quentielle 
lisse $F(t,	\om)$:
\[
a_k (t) = F\Bigl(t, \phi_k'(t) \Bigr) ~~\mbox{et}~~
b_k (t) = F\Bigl(t, \alpha\, \phi_k'(t) \Bigr) ~.
\]
En traitement de la parole, cette enveloppe est compos\'e de plusieurs
{formants\/}. 
Il est fonction du type de phon\`{e}me qui a \'{e}t\'{e} 
prononc\'{e}. Comme $F(t,\om)$ est une fonction r\'{e}guli\`{e}re\index{Formant}
de $\om$, on calcule son amplitude en $\om	= \alpha\, \phi_k'(t)$ par 
interpolation des valeurs $a_k (t)$	correspondant \`{a} $\om = \phi_k'(t)$.

\subsection{Cr\^{e}tes de transform\'ee de Fourier \`a fen\^{e}tre}
\label{sec-wind-four-ridges}

Le spectrogramme $P_S f(u,\xi)= |Sf(u,\xi)|^2$
mesure l'\'{e}nergie de $f$ dans un voisinage temps-fr\'{e}quence de $(u,\xi)$.
L'algorithme de cr\^{e}tes calcule les fr\'{e}quences 
instantan\'{e}es \`{a} partir des maxima locaux de $P_S	f(u,\xi)$
\cite{torresani}.

On calcule la transform\'{e}e de Fourier \`a fen\^{e}tre \`{a} 
l'aide d'une fen\^{e}tre sym\'{e}trique $g(t) =	g(-t)$ de support
$[-\frac	1 2	, \frac	1 2]$. La transform\'{e}e de Fourier $\hat g$ de 
$g$ est une fonction sym\'{e}trique r\'{e}elle avec
$|\hat g	(\om)| \leq	\hat g (0)$	pour tout	$\om \in \R$.
Son maximum $\hat g(0) =	\int_{1/2}^{-1/2} g(t)\,dt$ est de l'ordre de 
1. 
On normalise la fen\^{e}tre $g$ de mani\`{e}re 
\`{a} avoir $\|g\| = 1$. 
On d\'efinit la largeur de bande $\Delta \om$ de
$\hat g$ par
\begin{equation}
\label{Band-widt-g}
|\hat g(\om)| \ll 1 ~~~
\mbox{
pour}~~~ |\om| \geq \Delta \om.
\end{equation}

A une \'{e}chelle donn\'{e}e $s$,
$g_s	(t)	= \frac	1 {\sqrt s}	g(\frac	t s)$ a un support de taille $s$ 
et est de norme unit\'{e}. Les atomes de Fourier correspondants sont
\[
g_{s,u,\xi} (t) = g_s (t-u) \, \Exp^{i \xi t} .
\]
La transform\'{e}e de Fourier \`a fen\^{e}tre est alors
\begin{equation}
\label{scale-wind-Four}
Sf(u,\xi) \,=\, \lb f,g_\suxi \rb \, =\,
\int_{-\infty}^{+\infty} f(t)\, g_s (t-u)\, \Exp^{-i\xi t} \,dt.
\end{equation}
Le th\'{e}or\`{e}me suivant exprime $Sf(u,\xi)$ en fonction de la 
fr\'{e}quence instantan\'{e}e de $f$.

\begin{theorem}
\label{approx-window}
Soit $f(t) = a(t)\, \cos \phi (t)$.
Si les variations de $a(t)$ et $\phi'(t)$ sont
n\'egligeables sur le support $[u-s/2,u+s/2]$
de $g_s(t-u)$ et que $\phi'(u) \geq s^{-1} \, \Delta \om$
alors pour tout $\xi \geq 0$ on a
\begin{equation}
\label{wt-est}
Sf(u,\xi) 
\approx \frac {\sqrt s} 2  \,a(u)\,\Exp^{ i (\phi(u) - \xi u)}
\hat g\Bigl(s[\xi - \phi'(u)]\Bigr) .
\end{equation}
\end{theorem}

{\bf D\'emonstration\ }
Remarquons que
\begin{eqnarray*}
\lb f,g_\suxi \rb  &=&
\int_{-\infty}^{+\infty} a(t) \,\cos \phi(t) \,g_s (t-u)\, \Exp^{-i \xi
t}\,dt \\
&=& \frac 1 2
\int_{-\infty}^{+\infty} a(t) \,\left(
\Exp^{i \phi(t)} + \Exp^{-i \phi(t)}\right)
 \,g_s (t-u)\, \Exp^{-i \xi t}\,dt\\
&=& I(\phi) + I(-\phi) .
\end{eqnarray*}
Commen\c{c}ons par \'{e}tudier
\begin{eqnarray*}
I(\phi)  &=& \frac 1 2
\int_{-\infty}^{+\infty} a(t) \,\Exp^{i\phi(t)} \,g_s (t-u)\, \Exp^{-i \xi
t}\,dt \\
&=& \frac 1 2
\int_{-\infty}^{+\infty} a(t+u)\, \Exp^{i \phi(t+u)}\, g_s (t)\, \Exp^{-i \xi
(t+u)}\, dt .
\end{eqnarray*}
Comme $a(t+u) \approx a(u)$ et
$\phi (t+u) \approx \phi(u) + t \phi'(u)$ pour $|t| \leq s/2$
il vient
\[
I(\phi) \approx
\frac {a(u)\, \Exp^{ i (\phi(u) - \xi u)}  } 2
\int_{-\infty}^{+\infty}  g_s (t)\, \Exp^{-i t (\xi - \phi'(u))}
\, dt = 
\frac {a(u)\, \Exp^{ i (\phi(u) - \xi u)}  } 2 \,
\hat g_s (\xi - \phi'(u)) .
\]
Puisque $\hat g_s (\om) = \sqrt s \, \hat g(s \om)$ 
\begin{equation}
\label{wt-est2}
I(\phi) \approx
\frac {\sqrt s} 2  \,a(u)\,\Exp^{ i (\phi(u) - \xi u)}
\hat g\Bigl(s[\xi - \phi'(u)]\Bigr) .
\end{equation}
De m\^eme
\[
I(-\phi) \approx
\frac {\sqrt s} 2  \,a(u)\,\Exp^{ i (-\phi(u) - \xi u)}
\hat g\Bigl(s[\xi + \phi'(u)]\Bigr) .
\]
Comme $\xi \geq 0$, $s[\xi + \phi'(u)] \geq \Delta \om$ donc
$\hat g\Bigl(s[\xi + \phi'(u)]\Bigr) \ll 1$. On peut donc
n\'egliger $I(-\phi)$ et l'on d\'eduit (\ref{wt-est}) de
(\ref{wt-est2}).
$\Box$
\\
\\
{\bf Points de cr\^{e}te\ } 
Supposons que $a(t)$ et $\phi'(t)$ ont des variations 
negligeables sur 
les intervalles de taille $s$, et que $\phi'(t) \geq \Delta \om/s$.
Comme $|\hat	g(\om)|$ est maximal en $\om	= 0$,
(\ref{wt-est}) montre que, pour chaque $u$, le spectrogramme
$|Sf(u,\xi)|^2 =	|\lb f , g_\suxi \rb|^2$ est maximal en
$\xi(u) = \phi'(u)$. Les points correspondants $(u,\xi(u))$ du plan 
temps-fr\'{e}quence sont appel\'{e}s des {\it cr\^{e}tes}.
Aux points de cr\^{e}te, (\ref{wt-est})	devient
\begin{equation}
\label{ridg-wind-fou-ap}
Sf(u, \xi ) = \frac {\sqrt s  } 2 \,a(u)\, \Exp^{ i
(\phi(u) - \xi u)} \, \hat g(0) 
\end{equation}
La fr\'{e}quence de cr\^{e}te donne la fr\'{e}quence instantan\'{e}e
$\xi(u) = \phi'(u)$, et on calcule $\xi(u) = \phi'(u)$ avec
\begin{equation}
\label{compute-freq}
a(u) = \frac {2\, \Bigl|Sf\Bigl(u,\xi (u)\Bigr)\Bigr|}
{\sqrt s\, |\hat g(0)|} .
\end{equation}
Soit $\Phi_S (u,\xi)$ la phase complexe de  $Sf(u,\xi)$.
L'expression 
(\ref{ridg-wind-fou-ap}) montre que les points de cr\^{e}te sont 
\'{e}galement des points o\`{u} la phase est stationnaire:
\[
\frac {\partial \Phi_S  (u,\xi )} {\partial u} = \phi' (u) -
\xi  = 0 .
\]
La v\'{e}rification de la stationnarit\'{e} de la phase permet de 
mieux situer les cr\^{e}tes.
\\
\\
\noindent{\bf Fr\'{e}quences multiples\ }
Lorsque le signal contient plusieurs lignes spectrales de 
fr\'{e}quences suffisamment distinctes, la transform\'{e}e de Fourier 
fen\^{e}tr\'{e}e en s\'{e}pare les diverses composantes, et les 
cr\^{e}tes permettent de d\'{e}tecter l'\'{e}volution temporelle de chacune d'entre 
elles. Consid\'{e}rons
\[
f(t) = a_1 (t) \cos \phi_1 (t) + a_2 (t) \cos \phi_2 (t) ,
\]
o\`{u} $a_k (t)$ et $\phi^\prime_k (t)$ sont \`{a} variations petites 
sur les intervalles de largeur $s$, et o\`{u} on suppose
$s \phi^\prime_k	(t)	\geq \Delta	\om$.
Comme la 
transform\'{e}e de Fourier fen\^{e}tr\'{e}e est lin\'{e}aire, on 
applique (\ref{wt-est}) \`{a} chaque composante spectrale:
\begin{eqnarray}
\label{twospectrems}
Sf(u, \xi ) & = & 
\frac {\sqrt s } 2  \,a_1 (u)\,
\hat g\Bigl(s[\xi - \phi_1 '(u)]\Bigr) \,\Exp^{ i (\phi_1 (u) - \xi u)} \nonumber \\
&  & +
\frac {\sqrt s}  2  \,a_2 (u)\,
\hat g\Bigl(s[\xi - \phi_2 '(u)]\Bigr) \,\Exp^{ i (\phi_2 (u) - \xi u)} .
\end{eqnarray}
On arrive \`{a} s\'{e}parer les composantes spectrales si, pour tout $u$
\begin{equation}
\label{separ8}
\hat g\Bigl(s |\phi_1 '(u) - \phi_2 '(u)|\Bigr) \ll 1 ,
\end{equation}
ce qui signifie que la diff\'{e}rence entre les fr\'{e}quences est 
plus grande que la largeur de bande de 	$\hat g(s \om)$:
\begin{equation}
\label{separ}
|\phi_1 '(u) - \phi_2 '(u) | \geq \frac{\Delta \om} s .
\end{equation}
Dans ce cas, on peut n\'{e}gliger, pour $\xi = \phi_1 '(u)$, le second 
terme de (\ref{twospectrems}), et le premier terme engendre un point 
de cr\^{e}te permettant de reconstituer $\phi'_1	(u)$ et $a_1	(u)$ en 
utilisant (\ref{compute-freq}).
De m\^{e}me, on peut n\'{e}gliger le premier terme lorsque $\xi =	
\phi_2 '(u)$, et on obtient un second point de cr\^{e}te 
caract\'{e}risant $\phi_2 '	(u)$ et $a_2 (u)$. Les points de cr\^{e}te
sont distribu\'{e}s sur les deux lignes temps-fr\'{e}quence $\xi(u)	= \phi_1 ' (u)$
et $\xi(u) = \phi_2	' (u)$. Ce r\'{e}sultat demeure valable pour un 
nombre quelconque de composantes spectrales instationnaires, tant que 
la distance entre deux fr\'{e}quences instantan\'{e}es v\'{e}rifie (\ref{separ}). 
Lorsque les lignes spectrales sont trop proches, elles cr\'{e}ent des 
interf\'{e}rences qui d\'{e}truisent la structure de cr\^{e}tes.

En g\'{e}n\'{e}ral, on ne connait pas le nombre de fr\'{e}quences 
instantan\'{e}es. On d\'{e}tecte alors tous les maxima locaux de 
$|Sf(u,\xi)|^2$ dont la phase est stationnaire 
$\frac {\partial	\Phi_S	(u,\xi )} {\partial	u} = \phi' (u) - \xi = 0$.
Ces points d\'{e}finissent des courbes dans le plan $(u,\xi)$ qui 
sont les cr\^{e}tes de la transform\'{e}e de Fourier fen\^{e}tr\'{e}e. 
On supprime souvent les cr\^{e}tes de faible amplitude $a(u)$ parce 
qu'elles peuvent \^{e}tre des artifacts provenant de variations 
bruit\'{e}es, ou des ``ombres'' d'autres fr\'{e}quences 
instantan\'{e}es, d\^{u}es aux lobes lat\'{e}raux de $\hat g(\om)$.

Dans la figure \ref{WFTRidgeChirps}, on peut voir les cr\^{e}tes 
obtenues \`{a} partir du module et de la phase de la transform\'{e}e 
de Fourier fen\^{e}tr\'{e}e de la figure \ref{WFTChirps}. Pour $t \in 
[400,500]$, les fr\'{e}quences intantan\'{e}es sont 
trop proches et la r\'{e}solution en fr\'{e}quence de la fen\^{e}tre 
ne permet pas de les s\'{e}parer. En cons\'{e}quence, les cr\^{e}tes y 
d\'{e}tectent une fr\'{e}quence instantan\'{e}e moyenne.

\begin{figure}[bhtp]
\centerline{
        \epsfxsize=8cm
        \leavevmode\epsfbox{figures/chap4/WFTRidgeChirps.eps}}
\caption{
Cr\^{e}tes de plus grande amplitude calcul\'{e}es \`{a} partir du 
spectrogramme de la figure \protect\ref{WFTChirps}. Ces cr\^{e}tes 
donnent les fr\'{e}quences instantan\'{e}es des chirps lin\'{e}aires et 
quadratiques, et des transitoires \`{a} basse et haute fr\'{e}quence 
en  $t=512$ et $t=896$.
}
\label{WFTRidgeChirps}
\end{figure}
\\
\\
\noindent
{\bf Choix de	la fen\^{e}tre}
La mesure des fr\'{e}quences instantan\'{e}es 
aux points de cr\^{e}te 
a \'{e}t\'{e} valid\'{e}e seulement lorsque la taille $s$ de la 
fen\^{e}tre $g_s$ est suffisamment petite pour que
$a(t)$ et $\phi' (t)$ soient approximativement
constant sur $[u-s/2,u+s/2]$. 
D'un autre c\^{o}t\'{e}, il faut que la largeur de 
bande ${\Delta \om}/ s$ soit 
suffisamment petite pour s\'{e}parer les composantes 
spectrales succesives en (\ref{separ}). Le choix de l'\'{e}chelle $s$ de la 
fen\^{e}tre  doit donc r\'{e}aliser un compromis entre ces deux 
contraintes.

\vspace{0cm}\setlength{\tabcolsep}{0cm} % Separation entre les lignes d'images
\setlength{\fboxsep}{0cm} % Separation entre la boite et l'image
\avecboite = 0
\begin{figtab}
\begin{figrow}{4}
{{\label{WFTLinChirps} 
Somme de deux ``chirps'' lin\'{e}aires prall\`{e}les.
(a): Spectrogramme $P_S f(u,\xi)=|Sf(u,\xi)|^2$.
(b): Cr\^{e}tes calcul\'{e}es \`{a} partir du spectrogramme.  
}}\\
\figentry{8cm}{0cm}{figures/chap4/WFTLinChirps.eps}\\

\centry{(a)}\\

\figentry{8cm}{0cm}{figures/chap4/WFTRidgeLinChirps.eps}\\

\centry{(b)}
\end{figrow} 
\end{figtab}

\vspace{0cm}\setlength{\tabcolsep}{0cm} % Separation entre les lignes d'images
\setlength{\fboxsep}{0cm} % Separation entre la boite et l'image
\avecboite = 0
\begin{figtab}
\begin{figrow}{4}
{{\label{WFTHypChirps}
Somme de deux chirps hyperboliques.
(a): Spectrogramme $P_S f(u,\xi)$.
(b): Cr\^{e}tes calcul\'{e}es \`{a} partir du spectrogramme.  
}}\\
\figentry{8cm}{0cm}{figures/chap4/WFTHypChirps.eps}\\

\centry{(a)}\\

\figentry{8cm}{0cm}{figures/chap4/WFTRidgeHypChirps.eps}\\

\centry{(b)}
\end{figrow} 
\end{figtab}


\begin{Examples}
\item 
La somme de deux chirps lin\'{e}aires parall\`{e}les
\begin{equation}
\label{ParallLineCh}
f(t) = a_1 \cos (b t^2+ c t) + a_2 \cos (b t^2) 
\end{equation}
a deux fr\'{e}quences instantan\'{e}es 
$\phi_1'	(t)	= 2	bt +c$ et $\phi_2'	(t)	= 2	bt$.
On voit un exemple num\'{e}rique en figure \ref{WFTLinChirps}. La 
fen\^{e}tre $g_s$ a une r\'{e}solution fr\'{e}quentielle suffisamment 
fine pour s\'{e}parer les deux chirps si
\begin{equation}
\label{ccondit}
|\phi_1'(t) - \phi_2'(t) | = |c| \geq \frac {\Delta \om} s .
\end{equation}
Son support temporel est assez fin compar\'{e} \`{a} la variation 
temporelle des chirps si la fr\'equence instantan\'ee
$\phi'(t)$ est quasiment constante sur $[u-s/2,u+s/2]$,
ce que l'on obtient si
\begin{equation}
\label{bcondit}
s^2 \,|\phi_1''(u)| = s^2 \,|\phi_2''(u)| = 
2 \,b\, s^2 \ll 1 .
\end{equation}
Les conditions (\ref{ccondit}) et (\ref{bcondit}) montrent qu'on peut 
trouver une fen\^{e}tre $g$ ad\'{e}quate si et seulement si 
\begin{equation}
\label{chirp-paral-co}
\frac {c} {\sqrt b} \gg \Delta \om .
\end{equation}
Comme $g$ est une fen\^{e}tre lisse de support $[-\frac	1 2	, \frac	1 2	]$,
sa largeur de bande fr\'{e}quentielle $\Delta \om$ est de l'ordre de 
1. Les chirps lin\'{e}aires de la figure \ref{WFTLinChirps} 
v\'{e}rifient (\ref{chirp-paral-co}). On a calcul\'{e} leurs 
cr\^{e}tes en utilisant une fen\^{e}tre gaussienne tronqu\'{e}e 
avec $s = 50$.

\item 
Le chirp hyperbolique
\[
f(t) = \cos \left(\frac \alpha {\beta -t}\right)
\]
pour $0 \leq t < \beta$ a une fr\'{e}quence instantan\'{e}e
\[
\phi'(t) = \frac \alpha {(\beta - t)^2},
\]
\`{a} variation rapide quand $t$ est proche de $\beta$. Les 
fr\'{e}quences instantan\'{e}es des chirps hyperboliques vont de $0$ 
\`{a} $+\infty$ en un temps fini. Cette propri\'{e}t\'{e} est 
particuli\`{e}rement utile aux radars. 
Ces chirps sont \'{e}galement 
produits par les sonars de navigation des chauves-souris 
\cite{torresani}.

On ne peut pas estimer les fr\'{e}quences instantan\'{e}es des chirps 
hyperboliques par une transform\'{e}e de Fourier fen\^{e}tr\'{e}e 
parce que, pour une taille de fen\^{e}tre donn\'{e}e, la fr\'{e}quence 
instantan\'{e}e varie trop rapidement dans les hautes fr\'{e}quences. 
Lorsque $u$ est assez proche de $\beta$, on ne peut consid\'erer
que la fr\'equence instantan\'ee $\phi'(t)$ est constante sur
le support $[u-s/2,u+s/2]$ de $g_\suxi$ car
\[
s^2 |\phi '' (u)| = 
\frac {s^2 \alpha} {(\beta - u)^3} > 1 .
\]
En figure \ref{WFTHypChirps}, on voit un signal compos\'{e} de la 
somme de deux chirps hyperboliques:
\begin{equation}
\label{ParallHypCh}
f(t) = a_1 \cos \left(\frac {\alpha_1} {\beta_1 - t}\right) +
a_2 \cos \left(\frac {\alpha_2} {\beta_2 - t}\right),
\end{equation}
avec $\beta_1 = 700$	and	$\beta_2 = 740$.
Au d\'{e}but du signal, les deux chirps ont des fr\'{e}quences 
instantan\'{e}es voisines qui sont s\'{e}par\'{e}es par la 
transform\'{e}e de Fourier fen\^{e}tr\'{e}e, qu'on a calcul\'{e}e 
avec une fen\^{e}tre large. Quand on se rapproche de $\beta_1$ ou de 
$\beta_2$, la fr\'{e}quence instantan\'{e}e varie trop rapidement en 
regard de la taille de la fen\^{e}tre. Les cr\^{e}tes correspondantes 
ne permettent plus de suivre ces fr\'{e}quences instantan\'{e}es. 
\index{Chirp!hyperbolique}
\end{Examples}


%\section{Reconnaissance de la parole}
%
%La d\'ecouverte dans les ann\'ees 50 
%des propri\'et\'es acoustiques de la parole ainsi que de la structure
%des formants a ouvert la possibilit\'e d'automatiser la reconnaissance
%de la parole. Plus de 40 ans plus tard, le probl\`eme se r\'ev\`ele
%bien plus difficile qu'on ne s'y attendait. 
%On distingue plusieurs types de probl\`emes. La reconnaissance
%de mots isol\'es s\'epar\'es par une pause, la
%d\'etection de mots appartenant \`a un vocabulaire limit\'e dans
%un flot continu de parole, et la
%reconnaissance de parole sans pose.
%Il existe actuellement
%des produits commerciaux pour la reconnaissance de mots isol\'es mais
%les performances ne sont pas suffisantes pour la parole continue.
%Les difficult\'es de la reconnaissance de parole ont diverses 
%origines.
%\\
%$\bullet$ Variations mono-locuteur.
%La prononciation est souvent d\'eform\'ee. Par exemple, le ``et'' peut
%\^etre r\'eduit \`a un simple grognement. 
%La prononciation d'un phon\`eme est aussi affect\'ee par le son avant
%et apr\`es. Enfin, le d\'ebit de parole peut varier 
%de fa\c{c}on consid\'erable.\\
%$\bullet$ Variations multi-locuteurs. Par exemple, pour
%les sons vois\'es, la position des 2 
%premiers formants varient d'un locuteur \`a l'autre.\\
%$\bullet$ Ambigu\"{\i}t\'e des sons. 
%Les variables acoustiques ne sp\'ecifient pas
%toujours de fa\c{c}on unique les variables phon\'etiques. Il est souvent
%n\'ecessaire d'utiliser des informations compl\'ementaires provenant
%de la structure du langage.\\
%$\bullet$ Bruits et interf\'erences. Un son de parole est souvent
%superpos\'e avec d'autres sons provenant \'eventuellement d'une
%autre conversation, qu'il est n\'ecessaire d'\'eliminer lors de
%la reconnaissance.\\
%\\
%{\bf Reconnaissance de mots isol\'es}
%Les structures temps-fr\'equences sont parmi les plus importantes
%pour reconna\^{\i}tre  les phon\`emes et donc les mots isol\'es qu'ils
%composent. 
%La position des formants peux se calculer \`a partir de la
%position des p\^oles du filtre AR associ\'e, comme on l'a expliqu\'e
%au paragraphe \ref{conduit-sec}. 
%Il est aussi possible de d\'etecter les zones
%de haute concentration d'\'energie dans le spectrogramme, comme le
%montre la figure \ref{spectrogram-parole}. 
%
%L'utilisation de structures locales telles que des formants n'est 
%souvent pas suffisante pour obtenir un bon taux de reconnaissance.
%On optimise souvent la reconnaissance en mesurant les 
%probabilit\'es de transition d'un son \`a un autre. 
%A partir d'un mod\`ele bas\'e sur des cha\^{\i}nes
%de Markov, on cherche alors 
%le mot qui a une probabilit\'e conditionnelle
%maximum, \'etant donn\'e le signal observ\'e.
%
%Le d\'ebit de parole \'etant un param\`etre non contr\^ol\'e,
%il est aussi 
%n\'ecessaire de renormaliser le temps pour faire correspondre
%le son avec les structures de r\'ef\'erence utilis\'ees pour la
%reconnaissance. Cela peut se faire
%avec des dilatations et compressions arbitraires en essayant 
%d'optimiser un crit\`ere de correspondance. D'autres algorithmes
%effectuent
%des dilatations temporelles locales, guid\'ees par les structures du
%son.
%
%

