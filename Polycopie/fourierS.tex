\section{L'espace de Schwarz}

Nous avons vu comment les propri{\'e}t{\'e}s de d{\'e}croissances se traduisent par des
propri{\'e}t{\'e}s de d{\'e}rivabilit{\'e} sur la transform{\'e}e de Fourier, et \textit{vice
  versa}.
Afin de construire un espace fonctionnel stable par transformation de Fourier,
il est donc naturel d'introduire un espace contenant des fonctions comportant
{\`a} la fois ces deux propri{\'e}t{\'e}s.

\index{Espace de Schwarz}
\begin{definition} On d{\'e}signe par $\mcs(\rset)$ ou tout simplement $\mcs$
l'espace vectoriel des fonctions de $\rset$ dans $\cset$ qui v{\'e}rifient les deux propri{\'e}t{\'e}s suivantes:
\begin{enumerate}
\item $f$ est ind{\'e}finiment d{\'e}rivable sur $\rset$;
\item $f$ est {\`a} d{\'e}croissance rapide, ainsi que toutes ses d{\'e}riv{\'e}es.
\end{enumerate}
On appelle $\mcs(\rset)$ l'\emph{espace de Schwarz}.
\end{definition}

Donnons quelques exemples importants de fonctions appartenant {\`a} $\mcs(\rset)$.
\begin{example}
  Il est facile de v{\'e}rifier que la fonction $f$ d{\'e}finie par $f(x)=\rme^{-x^2}$
  appartient {\`a} $\mcs$. Par suite, il en est de m{\^e}me de la fonction $g_\sigma$
  d{\'e}finie par~(\ref{eq:gauss_func}) quelque soit $\sigma>0$.
\end{example}
Il faut travailler un peu plus pour trouver un exemple de fonction dans $\mcs$
{\`a} \emph{support compact} c'est-{\`a}-dire nulle en dehors d'un ensemble
born{\'e}.
\begin{example}\label{exple:cinfty-supp-compact}
  On consid{\`e}re la fonction $g$ d{\'e}finie par
$$
g(x)=
\begin{cases}
\rme^{-1/x} &\text{ si $x>0$}\\
0 &\text{ sinon.}
\end{cases}
$$
On montre facilement que $f$ est
$\mathcal{C}^\infty$. En revanche elle n'est pas {\`a} support compact puisque
$g(x)\to1$ quand $x\to\infty$. Cependant la fonction $f$ d{\'e}finie par
$$
f(x)=g(x) \times g(1-x) ,\quad x\in\rset\;,
$$
est nulle en dehors de $[0,1]$ et $\mathcal{C}^\infty$. C'est donc une fonction
de $\mcs$ {\`a} support compact.
\end{example}

\begin{proposition}
L'espace $\mcs$ a les propri{\'e}t{\'e}s suivantes :
\begin{enumerate}
\item $\mcs$ est stable pour la multiplication par un polyn{\^o}me.
\item $\mcs$ est stable par d{\'e}rivation ($f \in \mcs \Rightarrow f' \in  \mcs$).
\item $\mcs \subset  L_1(\rset)$.
\end{enumerate}
\end{proposition}
La d{\'e}monstration de cette proposition est laiss{\'e}e en exercice.  Le r{\'e}sultat
essentiel, qui d{\'e}coule essentiellement de propri{\'e}t{\'e}s d{\'e}j{\`a} d{\'e}montr{\'e}es, est
contenu dans le th{\'e}or{\`e}me suivant.
\begin{theorem}
\label{theo:stabiliteS}
L'espace $\mcs$ est stable par transformation de Fourier: si $f \in \mcs$ alors $\hat{f} \in \mcs$.
\end{theorem}
\begin{proof}
Soit $f \in \mcs$. Comme
$f$ est dans $L_1(\rset)$ et {\`a} d{\'e}croissance rapide, $\hat{f} \in C_\infty (\rset)$ d'apr{\`e}s la Proposition \ref{prop:1913}.
Pour tout $k \in \nset$, $f^{(k)}$ {\'e}tant {\`a} d{\'e}croissance rapide est int{\'e}grable d'apr{\`e}s le lemme \ref{lem:decroissancerapide}.
On en d{\'e}duit de la proposition \ref{prop:1914} que $\hat{f}$ est {\`a} d{\'e}croissance rapide.
Il reste {\`a} examiner la d{\'e}rivabilit{\'e} de $f$.
Comme toutes les d{\'e}riv{\'e}es de $f$ sont {\`a} d{\'e}croissance rapide, les fonctions $x\mapsto(x^q f (x))^{(p)}$ sont dans
$L_1(\rset)$ pour tout entiers $p$ et $q$; d'o{\`u}, d'apr{\`e}s la Proposition \ref{prop:FourierDerivation}, pour tout
$\xi\in\rset$,
\begin{equation}
  \label{eq:stabilite-S-en-action}
\xi^p \hat{f}^{(q)}(\xi) = \xi^p  \TFA{(-2 \rmi \pi)^q m_q\times f}(\xi) =
\frac{1}{(\rmi 2 \pi)^p} \TFA{[(- \rmi 2 \pi)^q m_q \times f]^{(p)}}(\xi) \;,
\end{equation}
o{\`u} $m_q$ est le mon{\^o}me de degr{\'e} $q$, $m_q(x)=x^q$.  Le th{\'e}or{\`e}me
de Rieman-Lebesgue montre que $\lim_{|\xi| \to \infty} |\xi^q
\hat{f}^{(q)}(\xi)|= 0$.
\end{proof}

Nous allons voir cette stabilit{\'e} se traduit en fait par une propri{\'e}t{\'e} encore
plus remarquable: $\TF$ est une bijection de $\mcs$ dans $\mcs$. Pour cela, il
faut montrer que $\TF$ admet une r{\'e}ciproque sur $\mcs$, ce sera le r{\^o}le de sa
soeur jumelle $\TFC$. Le proc{\'e}d{\'e} de r{\'e}gularisation sera fondamental pour
arriver {\`a} cette fin.


\section{R{\'e}gularisation par convolution}

On rappelle qu'{\'e}tant donn{\'e}es $f$ et $g$ deux fonctions de $\rset$ dans $\cset$,
la convolution de $f$ et $g$ est la fonction $f \star g$
d{\'e}finie par
$$
f \star g(x) = \int f(x - t)g(t) \rmd t = \int f(u)g(x - u) \rmd u\;,
$$
en tout point $x$ o{\`u} cette int{\'e}grale est correctement d{\'e}finie.\

Examinons d'abord quelques exemples permettant de visualiser les propri{\'e}t{\'e}s de
la convolution.

\begin{example}
Prenons $f = g = \1_{[0,1]}$. On obtient
$$
\int f (x - t)g(t) \rmd t = \int_0^1 \1_{[0,1]}(x - t) \rmd t = \mathrm{Leb} ([0,1] \cap [x - 1,x]).
$$
La convolution de ces deux fonctions discontinues est donc continue.

Prenons maintenant $f \in L_1(\rset)$ et $g = \frac{1}{2h} \1_{[-h,h]}$ o{\`u} $h > 0$.
Calculons $f \star g$.
\begin{equation}
f * g(x) = \frac{1}{2h} \int_{x-h}^{x+h}  f (u) \rmd u
\end{equation}
ce qui repr{\'e}sente la moyenne de $f$ sur $[x - h, x + h]$.
On voit directement sur l'int{\'e}grale que $f \star g$ est une fonction continue.
\end{example}
Ces deux exemples illustrent la propri{\'e}t{\'e} essentielle de la convolution qui est de
\textit{r{\'e}gulariser} (qui est li{\'e} au fait de moyenner).

Le lemme suivant montre que la convolution peut {\^e}tre utilis{\'e}e dans le but de
fournir une approximation, ce qui constitue le premier principe de
r{\'e}gularisation.

\begin{lemma}\label{lem:regularisation}
  Soit $f\in\mathcal L^1$.
  Soit $K$ une fonction positive telle que $\int K(x)\,\rmd x=1$ et, pour une
  constante $C>0$ et un exposant $\alpha>2$, $K(x)\leq C (1+|x|)^{-\alpha}$
  pour tout $x\in\rset$.  D{\'e}finissons, pour tout $\sigma>0$, la fonction
  $K_\sigma$ par
\begin{equation}
  \label{eq:Kband}
  K_\sigma(x)=\frac1\sigma K(x/\sigma)\;.
\end{equation}
Alors, on a les propri{\'e}t{\'e}s suivantes.
\begin{enumerate}
\item\label{item:regularisation1} En tout point $t$ o{\`u} $f$ est continue, on a
$$
\lim_{\sigma\downarrow0} (f \star K_\sigma)(t) = f(t)\;.
$$
\item\label{item:regularisation2} Si $f$ est continue {\`a} support compact, alors
  $f \star K_\sigma$ converge uniform{\'e}ment vers $f$ quand $\sigma\downarrow0$.
\end{enumerate}
\end{lemma}
\begin{proof}
Comme $\int K_\sigma(x)\, \rmd x= 1$, par un changement de variable {\'e}l{\'e}mentaire,
\begin{equation}
\label{eq:inversionL1-2}
\int f(t-u) K_\sigma(u) \rmd u - f(t) = \int \left\{ f(t - u) - f(t) \right \} K_\sigma(u) \rmd u \eqsp.
\end{equation}
Montrons le point~\ref{item:regularisation1}.
Comme $f$ est continue au point $t$, pour tout $\epsilon > 0$, il existe $\eta$ tel que $|u-t| \leq \eta$
implique que $|f(u) - f(t) | \leq \epsilon$. On a pour tout $\sigma > 0$,
\begin{equation}
\label{eq:conv-local}
\int_{|u| \leq \eta} \left|  f(t-u) - f(t) \right | K_\sigma(u) \rmd u \leq \epsilon \int K_\sigma(u) \rmd u = \epsilon \eqsp.
\end{equation}
De plus,
$$
\int_{|u| \geq \eta} \left| f(t-u) - f(t) \right | K_\sigma(u) \rmd u  \leq \|f\|_1\,\sup_{|u|\geq\eta} {K}_\sigma(u)
+|f(t)|\,\int_{|u|\geq \eta} {K}_\sigma(u) \rmd u  \eqsp.
$$
D'autre part, lorsque $\sigma \downarrow 0$,
$$
\sup_{|u|\geq\eta} {K}_\sigma(u) = \frac1\sigma \sup_{|v|\geq\eta/\sigma} {K}_\sigma(v)\leq  \frac{C}\sigma (1+\eta/\sigma)^{-\alpha}
$$
et,
$$
\int_{u\geq \eta} K_\sigma(u) \rmd u \leq \frac{C}\sigma\int_{|v|\geq\eta/\sigma} (1+|v|)^{-\alpha}\,dv =O(\sigma^{\alpha-2})\eqsp.
$$
Par cons{\'e}quent, en combinant ces majorations avec $\alpha>2$,
\begin{equation}
\label{eq:conv-hors-local}
\lim_{\sigma \downarrow 0} \int_{|u| \geq \eta} \left|  f(t-u) - f(t) \right | K_\sigma(u) \rmd u =0 \eqsp.
\end{equation}
Le r{\'e}sultat suit donc avec~(\ref{eq:inversionL1-2}).

Le point~\ref{item:regularisation2} se montre de fa�on semblable.
Comme $f$ est continue {\`a} support compact, elle est uniform{\'e}ment continue.
On peut donc choisir $\eta>0$ tel que~(\ref{eq:conv-local}) soit valide pour
tout $t$. De m{\^e}me la convergence~(\ref{eq:conv-hors-local}) est uniforme en $t$
en utilisant que $f$ est born{\'e}. D'o{\`u} le r{\'e}sultat.
\end{proof}

Le second principe de r{\'e}gularisation est que la version convolu{\'e}e $f\star K$ de
$f$ adopte la r{\'e}gularit{\'e} du \emph{noyau} $K$. Ce principe est donn{\'e} par le
lemme suivant.

\begin{lemma}\label{lem:regulairsation-est-reguliere}
  Soient $f,g\in L^1$. Alors on a les propri{\'e}t{\'e}s suivantes
  \begin{enumerate}
  \item\label{item:approx-cont-S-1} Si $g$ est $\mathcal{C}^k$, $f\star g$ est
  $\mathcal{C}^k$ et $(f\star g)^{(k)}=f\star g^{(k)}$.
\item\label{item:approx-cont-S-2} On a pour tout entier $p\geq0$,
  \begin{equation}
    \label{eq:decroissance-concolution}
    \|(f\star g)\times m_p\|_\infty\leq \|f\times (1+|m_p|)\|_\infty \;
    \|g\times (1+|m_p|)\|_1  \;,
  \end{equation}
o{\`u} $m_p$ d{\'e}signe le mon{\^o}me de degr{\'e} $p$, $m_p(x)=x^p$.
\item \label{item:approx-cont-S-2prim} Si $f$ et $g$ sont {\`a} d{\'e}croissance rapide,
  il en est de m{\^e}me de $f\star g$.
\item\label{item:approx-cont-S-3}  Si $f$ est {\`a} d{\'e}croissance rapide et
  $g\in\mcs$, $f\star g\in\mcs$.
  \end{enumerate}
\end{lemma}
\begin{proof}
  La propri{\'e}t{\'e}~\ref{item:approx-cont-S-1} est une simple application du lemme
  de d{\'e}rivation sous le signe somme.

Montrons la propri{\'e}t{\'e}~\ref{item:approx-cont-S-2}.
Pour tous $t,x\in\rset$, on a $|x|^p\leq(|x-t|+|t|)^p\leq|x-t|^p+|t|^p$.
D'o{\`u}
$$
|f\star g(x)||x|^p \leq \int |f(x-t)|\,|x-t|^p\,|g(t)|\rmd t+
\int |f(x-t)|\,|t|^p\,|g(t)|\rmd t \;.
$$
On obtient donc~(\ref{eq:decroissance-concolution}).

La propri{\'e}t{\'e}~\ref{item:approx-cont-S-2prim} est obtenu en appliquant
de~\ref{item:approx-cont-S-2} pour tout $p\geq1$.

La propri{\'e}t{\'e}~\ref{item:approx-cont-S-3} d{\'e}coule de \ref{item:approx-cont-S-1}
et~\ref{item:approx-cont-S-2prim}.
\end{proof}


On remarque facilement que l'on peut prendre dans les lemmes pr{\'e}c{\'e}dents
$K=g_1$, c'est-{\`a}-dire $K_\sigma=g_\sigma$, o{\`u} $g_\sigma$ est d{\'e}finie
en~(\ref{eq:gauss_func}). On peut parler dans ce cas de \emph{r{\'e}gularisation
  gaussienne}.

\section{Formules d'inversion}

On est maintenant en mesure de montrer des r{\'e}sultats d'inversion de la
transform{\'e}e de Fourier dans $L_1(\rset)$. La preuve se base sur l'observation
suivante. D'apr{\`e}s~(\ref{eq:gaussienneTF}), $\TF(g_\sigma)$ est une fonction
r{\'e}elle paire de $\mathcal{L}_1(\rset)$ et
$\TF\TF(g_\sigma)=\TFC\TF(g_\sigma)=g_\sigma$. Le r{\'e}sultat suivant est une
premi{\`e}re g{\'e}n{\'e}ralisation de la formule d'inversion ``$\TFC\TF(f)=f$'' qui sera
par la suite {\'e}tendue {\`a} des cadres bien plus g{\'e}n{\'e}raux.

\begin{proposition}\label{prop:inversionFourierL1}
Soit $f\in\mathcal{L}_1(\rset)$ et supposons que $\hat{f}$ appartiennent aussi
{\`a} $\mathcal{L}_1(\rset)$. Alors, en tout point $x$ o{\`u} $f$ est continue, on a
\begin{equation}
\label{eq:inversionFT}
[\bar{\TF} \hat{f}](x) = f(x).
\end{equation}
\end{proposition}
\begin{proof}
On a vu ci-dessus que la fonction $\hat{g}_\sigma(x) = \rme^{- 2 \pi^2 \sigma^2 x^2}$
a ${g}_\sigma$ pour transform{\'e}e de Fourier. Donc
$[\TF(x\mapsto \hat{g}_\sigma(x)\rme^{\rme 2 \pi t x})](\xi)={g}_\sigma(\xi-t)={g}_\sigma(t-\xi)$ par la
proposition~\ref{prop:FourierTranslation} puis par parit{\'e} de $g_\sigma$. La proposition \ref{prop:echangeTF} appliqu{\'e}e avec
$x \mapsto f(x)$ et $x \mapsto \rme^{\rmi 2 \pi t x} \hat{g}_\sigma(x)$ donne donc, pour tout $t\in \rset$,
\begin{equation}
\label{eq:inversionL1-1}
\int \hat{f}(x) \hat{g}_\sigma(x) \rme^{\rmi 2 \pi t x} \rmd x = \int f(u) {g}_\sigma(t-u) \rmd u = f\star g_\sigma(t) \eqsp.
\end{equation}
Lorsque $\sigma \to 0$, on peut passer {\`a} la limite dans l'int{\'e}grale de gauche, puisque l'on a, pour tout
$x$, $\lim_{\sigma \to 0} \hat{g}_\sigma(x)= 1$ pour tout $x$ et $|\hat{f}(x) \hat{g}_\sigma(x) \rme^{\rmi 2 \pi t x}| \leq |\hat{f}(x)|$.
Comme $\hat{f} \in L_1(\rset)$, on applique le th{\'e}or{\`e}me de convergence domin{\'e}e, qui montre
$$
\lim_{\sigma \to 0} \int \hat{f}(x) \hat{g}_\sigma(x) \rme^{ \rmi 2 \pi t x} \rmd x = \int \hat{f}(x) \rme^{\rmi 2 \pi t x} \rmd x \eqsp.
$$
Le passage {\`a} la limite dans le membre de droite de~(\ref{eq:inversionL1-1}) est lui une cons{\'e}quence du
lemme~\ref{lem:regularisation}. On obtient bien le r{\'e}sultat annonc{\'e} si $f$ est continue en $t$.
\end{proof}


Le r{\'e}sultat pr{\'e}c{\'e}dent conjugu{\'e} avec le th{\'e}or{\`e}me~\ref{theo:stabiliteS} permet de
d{\'e}finir une transform{\'e}e inverse comme application r{\'e}ciproque de $\TF$ d{\'e}finie
comme application de $\mcs$ dans $\mcs$.

En effet, si $f$ est un {\'e}l{\'e}ment de $\mcs$, $\hat{f}$ est dans $\mcs$ et donc
int{\'e}grable. $f$ {\'e}tant partout continue, la formule
d'inversion~(\ref{eq:inversionFT}) est valable pour tout $x \in \rset$. Donc,
pour tout $f \in \mcs$, $f = \TFAC{\TF f}$. De la m{\^e}me fa�on, on a $f =
\TFA{\TFC f}$. $\TF$ est donc une bijection sur $\mcs$ et son inverse est
$\TFC$.

\begin{theorem}\label{thm:Schwarz}
La transformation de Fourier $\TF$ est une application lin{\'e}aire bijective de $\mcs$ sur $\mcs$.
L'application inverse est $\TF^{-1} = \TFC$.
\end{theorem}

Ayant montr{\'e} le th{\'e}or{\`e}me~\ref{thm:Schwarz}, de nombreuses formules d'inversion peuvent \^etre d{\'e}duite par dualit{\'e}.
Nous verrons dans la section suivante comment appliquer ce principe dans un cadre hilbertien.
Ici nous l'appliquons dans le cadre $L_1(\rset)$ grace au r{\'e}sultat suivant, qui nous permettra de compl{\'e}ter la
proposition~\ref{prop:inversionFourierL1} par un th{\'e}or{\`e}me d'inversion.
\begin{proposition}\label{prop:DualiteSchwartz}
Soient deux fonctions $f$ et $g$ dans $L_1(\rset)$. Si, pour toute fonction \textit{test} $\phi$ dans $\mcs$, on a
$$
\int f(x)\,\phi(x)\,\rmd x= \int g(x)\,\phi(x)\,\rmd x,
$$
alors $f=g$ (au sens $L_1(\rset)$).
\end{proposition}
\begin{proof}
En prenant la diff{\'e}rence entre les deux membre de l'{\'e}galit{\'e} de l'hypoth{\`e}se, on voit qu'il suffit de montrer ce r{\'e}sultat pour
$g=0$. De plus, comme les fonctions continues sont denses dans l'ensemble des fonctions int�grables, on peut se contenter de prendre $f$ continue, le cas g{\'e}n{\'e}ral {\'e}tant
obtenu par passage {\`a} la limite. Or, pour $f$ continue et $g=0$, le r{\'e}sultat est imm{\'e}diat par application du principe de
r{\'e}gularisation en choisissant une fonction $K\in\mcs$ positive int{\'e}grant {\`a} 1 (par exemple $g_1$) puis en appliquant le
lemme~\ref{lem:regularisation} en tout point de la droite r{\'e}elle.
\end{proof}

On en d{\'e}duit le r{\'e}sultat annonc{\'e} qui compl{\`e}te la proposition~\ref{prop:inversionFourierL1}.
\begin{theorem}
Soit $f\in L_1(\rset)$ et supposons que $\hat{f}$ appartiennent aussi {\`a} $L_1(\rset)$. Alors
la fonction (continue) $\bar{\TF} \hat{f}$ est l'unique
repr{\'e}sentant continu de $f$.
\end{theorem}
\begin{proof}
La conitnuit{\'e} de $\bar{\TF} \hat{f}$ d{\'e}coule du th{\'e}or{\`e}me~\ref{thm:rieman-lebesgue}.
Pour toute fonction test $\phi$ de $\mcs$, on a, d'apr{\`e}s la proposition~\ref{prop:echangeTF} et le
th{\'e}or{\`e}me~\ref{thm:Schwarz},
$$
\int f(x) \phi(x)\,\rmd x=\int \hat{f}(\xi) \TFC(\phi)(\xi)\,d\xi.
$$
Mais comme $\hat{f}$, on peut r{\'e}appliquer l'{\'e}quivalent de la proposition~\ref{prop:echangeTF} mais pour la transform{\'e}e
inverse, ce qui donne alors directement
$$
\int \hat{f}(\xi) \TFC(\phi)(\xi)\,d\xi=\int \TFC(\hat{f})(x) \phi(x) \,\rmd x.
$$
D'o{\`u} le r{\'e}sultat en appliquant la proposition~\ref{prop:DualiteSchwartz}.
\end{proof}

\section{Convolution et transform�e de Fourier}

On conclut se chapitre en explorant l'effet de le transform{\'e}e de Fourier sur la
convolution. Nous explorons ici uniquement le cas de la convolution d{\'e}finie sur
$L^1\times L^1$.

\begin{theorem}%\label{thm:inversionFourierL1}
Etant donn{\'e}es deux fonctions $f$ et $g$ de $L_1(\rset)$ on a:
\begin{enumerate}
\item $f \star g$ est d{\'e}finie presque partout et $f \star g$ appartient {\`a} $L_1(\rset)$.
\item La convolution est un op{\'e}rateur bilin{\'e}aire continu de $L_1(\rset) \times L_1(\rset) \to L_1(\rset)$
tel que
$$
\| f \star g \|_1 \leq \| f \|_1 \| g \|_1
$$
\item la transform{\'e}e de Fourier du produit de convolution $\widehat{f \star g}$ est {\'e}gal au produit des transform{\'e}es
de Fourier des fonctions $\hat{f}$ et $\hat{g}$: $\widehat{f \star g} = \hat{f} \hat{g}$.
\end{enumerate}
\end{theorem}
\begin{proof}
Comme $f,g \in L_1(\rset)$, la fonction $(y,z) \mapsto f(y) g(z) \in L_1(\rset^2)$ d'apr{\`e}s le th{\'e}or{\`e}me de Fubini.
En faisant le changement de variable $y = x-t$ et $z=t$, on obtient:
$$
\iint f(y) g(z) d y d z = \iint f(x-t) g(t) \rmd x \rmd t \eqsp
$$
et les int{\'e}grales des modules sont finies.
La fonction $x \mapsto \int f(x-t) g(t) \rmd t$ est donc d{\'e}finie presque partout et appartient {\`a} $L_1(\rset)$, toujours d'apr{\`e}s le th{\'e}or{\`e}me de Fubini.
La seconde in{\'e}galit{\'e} d{\'e}coule de:
$$
\int | f \star g(x)| \rmd x = \int |g(t)| \left( \int  |f(x-t) \rmd x \right) \rmd t = \|f\|_1 \| g \|_1 \eqsp.
$$
La troisi{\`e}me assertion s'obtient de fa�on similaire par une application du th{\'e}or{\`e}me de Fubini.
\end{proof}

%%% Local Variables:
%%% mode: latex
%%% ispell-local-dictionary: "francais"
%%% TeX-master: "Polycopie-Fourier-L1L2"
%%% End:
