
\section{L'espace de Schwartz}

Nous avons vu comment les propri{\'e}t{\'e}s de d{\'e}croissances se traduisent par des
propri{\'e}t{\'e}s de d{\'e}rivabilit{\'e} sur la transform{\'e}e de Fourier, et \textit{vice
  versa}.
Afin de construire un espace fonctionnel stable par transformation de Fourier,
il est donc naturel d'introduire un espace contenant des fonctions comportant
{\`a} la fois ces deux propri{\'e}t{\'e}s.

\index{Espace de Schwartz}
\begin{definition} On d{\'e}signe par $\mcs(\rset)$ ou tout simplement $\mcs$
l'espace vectoriel des fonctions de $\rset$ dans $\cset$ qui v{\'e}rifient les deux propri{\'e}t{\'e}s suivantes:
\begin{enumerate}
\item $f$ est ind{\'e}finiment d{\'e}rivable sur $\rset$;
\item $f$ est {\`a} d{\'e}croissance rapide, ainsi que toutes ses d{\'e}riv{\'e}es.
\end{enumerate}
On appelle $\mcs(\rset)$ l'\emph{espace de Schwartz}.
\end{definition}

Donnons quelques exemples importants de fonctions appartenant {\`a} $\mcs(\rset)$.
\begin{example}
  Il est facile de v{\'e}rifier que la fonction $f$ d{\'e}finie par $f(x)=\rme^{-x^2}$
  appartient {\`a} $\mcs$. Par suite, il en est de m{\^e}me de la fonction $g_\sigma$
  d{\'e}finie par~(\ref{eq:gauss_func}) quelque soit $\sigma>0$.
\end{example}
Il faut travailler un peu plus pour trouver un exemple de fonction dans $\mcs$
{\`a} \emph{support compact} c'est-{\`a}-dire nulle en dehors d'un ensemble
born{\'e}.
\begin{example}\label{exple:cinfty-supp-compact}
  On consid{\`e}re la fonction $g$ d{\'e}finie par
$$
g(x)=
\begin{cases}
\rme^{-1/x} &\text{ si $x>0$}\\
0 &\text{ sinon.}
\end{cases}
$$
On montre facilement que $f$ est
$\mathcal{C}^\infty$. En revanche elle n'est pas {\`a} support compact puisque
$g(x)\to1$ quand $x\to\infty$. Cependant la fonction $f$ d{\'e}finie par
$$
f(x)=g(x) \times g(1-x) ,\quad x\in\rset\;,
$$
est nulle en dehors de $[0,1]$ et $\mathcal{C}^\infty$. C'est donc une fonction
de $\mcs$ {\`a} support compact.
\end{example}

\begin{proposition}
L'espace $\mcs$ a les propri{\'e}t{\'e}s suivantes :
\begin{enumerate}
\item $\mcs$ est stable pour la multiplication par un polyn{\^o}me.
\item $\mcs$ est stable par d{\'e}rivation ($f \in \mcs \Rightarrow f' \in  \mcs$).
\item $\mcs \subset  \lone(\rset)$.
\end{enumerate}
\end{proposition}
La d{\'e}monstration de cette proposition est laiss{\'e}e en exercice.  Le r{\'e}sultat
essentiel, qui d{\'e}coule essentiellement de propri{\'e}t{\'e}s d{\'e}j{\`a} d{\'e}montr{\'e}es, est
contenu dans le th{\'e}or{\`e}me suivant.
\begin{theorem}
\label{theo:stabiliteS}
L'espace $\mcs$ est stable par transformation de Fourier: si $f \in \mcs$ alors $\hat{f} \in \mcs$.
\end{theorem}
\begin{proof}
Soit $f \in \mcs$. Comme
$f$ est dans $\lone(\rset)$ et {\`a} d{\'e}croissance rapide, $\hat{f} \in C_\infty (\rset)$ d'apr{\`e}s la Proposition \ref{prop:1913}.
Pour tout $k \in \nset$, $f^{(k)}$ {\'e}tant {\`a} d{\'e}croissance rapide est int{\'e}grable d'apr{\`e}s le lemme \ref{lem:decroissancerapide}.
On en d{\'e}duit de la proposition \ref{prop:1914} que $\hat{f}$ est {\`a} d{\'e}croissance rapide.
Il reste {\`a} examiner la d{\'e}rivabilit{\'e} de $f$.
Comme toutes les d{\'e}riv{\'e}es de $f$ sont {\`a} d{\'e}croissance rapide, les fonctions $x\mapsto(x^q f (x))^{(p)}$ sont dans
$\lone(\rset)$ pour tout entiers $p$ et $q$; d'o{\`u}, d'apr{\`e}s la Proposition \ref{prop:FourierDerivation}, pour tout
$\xi\in\rset$,
\begin{equation}
  \label{eq:stabilite-S-en-action}
\xi^p \hat{f}^{(q)}(\xi) = \xi^p  \TFA{(-2 \rmi \pi)^q m_q\times f}(\xi) =
\frac{1}{(\rmi 2 \pi)^p} \TFA{[(- \rmi 2 \pi)^q m_q \times f]^{(p)}}(\xi) \;,
\end{equation}
o{\`u} $m_q$ est le mon{\^o}me de degr{\'e} $q$, $m_q(x)=x^q$.  Le th{\'e}or{\`e}me
de Rieman-Lebesgue montre que $\lim_{|\xi| \to \infty} |\xi^q
\hat{f}^{(q)}(\xi)|= 0$.
\end{proof}

Nous allons voir cette stabilit{\'e} se traduit en fait par une propri{\'e}t{\'e} encore
plus remarquable: $\TF$ est une bijection de $\mcs$ dans $\mcs$. Pour cela, il
faut montrer que $\TF$ admet une r{\'e}ciproque sur $\mcs$, ce sera le r{\^o}le de sa
soeur jumelle $\TFC$. Le proc{\'e}d{\'e} de r{\'e}gularisation sera fondamental pour
arriver {\`a} cette fin.




\section{Formules d'inversion}

On est maintenant en mesure de montrer des r{\'e}sultats d'inversion de la
transform{\'e}e de Fourier dans $\lone(\rset)$. La preuve se base sur l'observation
suivante. D'apr{\`e}s~(\ref{eq:gaussienneTF}), $\TF(g_\sigma)$ est une fonction
r{\'e}elle paire de $\mathcal{L}_1(\rset)$ et
$\TF\TF(g_\sigma)=\TFC\TF(g_\sigma)=g_\sigma$. Le r{\'e}sultat suivant est une
premi{\`e}re g{\'e}n{\'e}ralisation de la formule d'inversion ``$\TFC\TF(f)=f$'' qui sera
par la suite {\'e}tendue {\`a} des cadres bien plus g{\'e}n{\'e}raux.

\begin{proposition}\label{prop:inversionFourierL1}
Soit $f\in\mathcal{L}_1(\rset)$ et supposons que $\hat{f}$ appartiennent aussi
{\`a} $\mathcal{L}_1(\rset)$. Alors, en tout point $x$ o{\`u} $f$ est continue, on a
\begin{equation}
\label{eq:inversionFT}
[\bar{\TF} \hat{f}](x) = f(x).
\end{equation}
\end{proposition}
\begin{proof}
On a vu ci-dessus que la fonction $\hat{g}_\sigma(x) = \rme^{- 2 \pi^2 \sigma^2 x^2}$
a ${g}_\sigma$ pour transform{\'e}e de Fourier. Donc
$[\TF(x\mapsto \hat{g}_\sigma(x)\rme^{\rmi 2 \pi t x})](\xi)={g}_\sigma(\xi-t)={g}_\sigma(t-\xi)$ par la
proposition~\ref{prop:FourierTranslation} puis par parit{\'e} de $g_\sigma$. La proposition \ref{prop:echangeTF} appliqu{\'e}e avec
$x \mapsto f(x)$ et $x \mapsto \rme^{\rmi 2 \pi t x} \hat{g}_\sigma(x)$ donne donc, pour tout $t\in \rset$,
\begin{equation}
\label{eq:inversionL1-1}
\int \hat{f}(x) \hat{g}_\sigma(x) \rme^{\rmi 2 \pi t x} \rmd x = \int f(u) {g}_\sigma(t-u) \rmd u  \eqsp.
\end{equation}
Lorsque $\sigma \to 0$, on peut passer {\`a} la limite dans l'int{\'e}grale de gauche, puisque l'on a, pour tout
$x$, $\lim_{\sigma \to 0} \hat{g}_\sigma(x)= 1$ pour tout $x$ et $|\hat{f}(x) \hat{g}_\sigma(x) \rme^{\rmi 2 \pi t x}| \leq |\hat{f}(x)|$.
Comme $\hat{f} \in \lone(\rset)$, on applique le th{\'e}or{\`e}me de convergence domin{\'e}e, qui montre
$$
\lim_{\sigma \to 0} \int \hat{f}(x) \hat{g}_\sigma(x) \rme^{ \rmi 2 \pi t x} \rmd x = \int \hat{f}(x) \rme^{\rmi 2 \pi t x} \rmd x \eqsp.
$$
Le passage {\`a} la limite dans le membre de droite de~(\ref{eq:inversionL1-1}) est lui une cons{\'e}quence du
lemme~\ref{lem:regularisation}. On obtient bien le r{\'e}sultat annonc{\'e} si $f$ est continue en $t$.
\end{proof}

\begin{lemma}\label{lem:regularisation}
  Soit $f\in \lone(\rset)$.
  Soit $K$ une fonction positive telle que $\int K(x)\,\rmd x=1$ et, pour une
  constante $C>0$ et un exposant $\alpha>2$, $K(x)\leq C (1+|x|)^{-\alpha}$
  pour tout $x\in\rset$.  D{\'e}finissons, pour tout $\sigma>0$, la fonction
  $K_\sigma$ par
\begin{equation}
  \label{eq:Kband}
  K_\sigma(x)=\frac1\sigma K(x/\sigma)\;.
\end{equation}
Alors, on a les propri{\'e}t{\'e}s suivantes.
\begin{enumerate}[label=(\roman*)]
\item\label{item:regularisation1} En tout point $t$ o{\`u} $f$ est continue, on a
$$
\lim_{\sigma\downarrow0}  \int f(u) K_\sigma(t-u) = f(t)\;.
$$
\item\label{item:regularisation2} Si $f$ est continue {\`a} support compact, alors
  $t \mapsto \int f(u) K_\sigma(t-u)$ converge uniform{\'e}ment vers $f$ quand $\sigma\downarrow0$.
\end{enumerate}
\end{lemma}
\begin{proof}
Comme $\int K_\sigma(x)\, \rmd x= 1$, par un changement de variable {\'e}l{\'e}mentaire,
\begin{equation}
\label{eq:inversionL1-2}
\int f(t-u) K_\sigma(u) \rmd u - f(t) = \int \left\{ f(t - u) - f(t)  \right\} K_\sigma(u) \rmd u \eqsp.
\end{equation}
Montrons le point~\ref{item:regularisation1}.
Comme $f$ est continue au point $t$, pour tout $\epsilon > 0$, il existe $\eta$ tel que $|u-t| \leq \eta$
implique que $|f(u) - f(t) | \leq \epsilon$. On a pour tout $\sigma > 0$,
\begin{equation}
\label{eq:conv-local}
\int_{|u| \leq \eta} \left|  f(t-u) - f(t) \right| K_\sigma(u) \rmd u \leq \epsilon \int K_\sigma(u) \rmd u = \epsilon \eqsp.
\end{equation}
De plus,
$$
\int_{|u| \geq \eta} \left| f(t-u) - f(t) \right| K_\sigma(u) \rmd u  \leq \|f\|_1\,\sup_{|u|\geq\eta} {K}_\sigma(u)
+|f(t)|\,\int_{|u|\geq \eta} {K}_\sigma(u) \rmd u  \eqsp.
$$
D'autre part, lorsque $\sigma \downarrow 0$,
$$
\sup_{|u|\geq\eta} {K}_\sigma(u) = \frac1\sigma \sup_{|v|\geq\eta/\sigma} {K}_\sigma(v)\leq  \frac{C}\sigma (1+\eta/\sigma)^{-\alpha}
$$
et,
$$
\int_{u\geq \eta} K_\sigma(u) \rmd u \leq \frac{C}\sigma\int_{|v|\geq\eta/\sigma} (1+|v|)^{-\alpha}\,dv =O(\sigma^{\alpha-2})\eqsp.
$$
Par cons{\'e}quent, en combinant ces majorations avec $\alpha>2$,
\begin{equation}
\label{eq:conv-hors-local}
\lim_{\sigma \downarrow 0} \int_{|u| \geq \eta} \left|  f(t-u) - f(t) \right| K_\sigma(u) \rmd u =0 \eqsp.
\end{equation}
Le r{\'e}sultat suit donc avec~(\ref{eq:inversionL1-2}).

Le point~\ref{item:regularisation2} se montre de fa\c{c}on semblable.
Comme $f$ est continue {\`a} support compact, elle est uniform{\'e}ment continue.
On peut donc choisir $\eta>0$ tel que~(\ref{eq:conv-local}) soit valide pour
tout $t$. De m{\^e}me la convergence~(\ref{eq:conv-hors-local}) est uniforme en $t$
en utilisant que $f$ est born{\'e}. D'o{\`u} le r{\'e}sultat.
\end{proof}



Le r{\'e}sultat pr{\'e}c{\'e}dent conjugu{\'e} avec le th{\'e}or{\`e}me~\ref{theo:stabiliteS} permet de
d{\'e}finir une transform{\'e}e inverse comme application r{\'e}ciproque de $\TF$ d{\'e}finie
comme application de $\mcs$ dans $\mcs$.

En effet, si $f$ est un {\'e}l{\'e}ment de $\mcs$, $\hat{f}$ est dans $\mcs$ et donc
int{\'e}grable. $f$ {\'e}tant partout continue, la formule
d'inversion~(\ref{eq:inversionFT}) est valable pour tout $x \in \rset$. Donc,
pour tout $f \in \mcs$, $f = \TFAC{\TF f}$. De la m{\^e}me fa\c{c}on, on a $f =
\TFA{\TFC f}$. $\TF$ est donc une bijection sur $\mcs$ et son inverse est
$\TFC$.

\begin{theorem}\label{thm:Schwartz}
La transformation de Fourier $\TF$ est une application lin{\'e}aire bijective de $\mcs$ sur $\mcs$.
L'application inverse est $\TF^{-1} = \TFC$.
\end{theorem}

Ayant montr{\'e} le th{\'e}or{\`e}me~\ref{thm:Schwartz}, de nombreuses formules d'inversion peuvent \^etre d{\'e}duites par dualit{\'e}.
Nous verrons dans la section suivante comment appliquer ce principe dans un cadre hilbertien.



%%% Local Variables:
%%% mode: latex
%%% ispell-local-dictionary: "francais"
%%% TeX-master: "Polycopie-Fourier-L1L2"
%%% End:
