
\section{Transform{\'e}e de Fourier sur $L_1(\rset)$}

Nous abordons dans cette partie la d{\'e}finition et les propri{\'e}t{\'e}s de la transform{\'e}e de Fourier des fonctions.
\begin{definition}[Transform{\'e}e de Fourier]
Soit $f \in \lone(\rset)$. On pose, pour tout $\xi\in\rset$,
\begin{align}
\label{eq:TFL1}
&[\TFA{f}](\xi)= \hat{f}(\xi) = \int_{\rset} \rme^{- \rmi 2 \pi \xi x} f(x) \rmd x \\
&[\TFAC{f}](\xi) = \int_{\rset} \rme^{ \rmi 2 \pi \xi x} f(x) \rmd x
\end{align}
On appelle $\TFA{f}$ (not{\'e} aussi $\TF f$) la \emph{Transform{\'e}e de Fourier} de $f$ et $\TFAC{f}$  (not{\'e} aussi $\TFC f$) la
transform{\'e}e de Fourier conjugu{\'e}e de $f$.
\end{definition}
Cette int{\'e}grale a un sens pour $f \in \lone(\rset)$, parce que $x\mapsto \rme^{- \rmi 2 \pi \xi x} f(x)$ est alors aussi dans
$\lone(\rset)$ pour tout $\xi\in\rset$.
\begin{example}
Soit $f = \1_{[a,b]}(x)$, la fonction indicatrice de l'intervalle $[a,b]$. Un calcul imm{\'e}diat montre que
$$
\hat{f}(\xi) =
\begin{cases}
b-a & \quad \xi = 0, \\
\frac{\sin \pi (b-a) \xi}{\pi \xi} \rme^{- \rmi \pi (a+b) \xi} & \quad \xi\neq0\eqsp.
\end{cases}
$$
On remarque que $\hat{f} \not \in \lone(\rset)$. En revanche c'est une fonction continue born{\'e}e
telle que $\lim_{|\xi|\to \infty} \hat{f}(\xi)= 0$.
\end{example}
En fait les propri{\'e}t{\'e}s d{\'e}crites pour $\hat{f}$ dans cet exemple sont des propri{\'e}t{\'e}s g{\'e}n{\'e}rales des
fonctions de $\TF(L^1(\rset))$ comme le montre le r{\'e}sultat suivant.

\begin{theorem}[Rieman-Lebesgue]
\label{thm:rieman-lebesgue}
Etant donn{\'e} $f \in \lone(\rset)$ on a
\begin{enumerate}
\item $\TF f$ est une fonction continue et born{\'e}e sur $\rset$,
\item $\TF$ est un op{\'e}rateur lin{\'e}aire et continu de $(\lone(\rset),\|\cdot\|_1)$ dans $(C_\infty,\|\cdot\|)$ (l'espace des
  fonctions continues munie de la norme sup) et $\| \hat{f} \| \leq \| f \|_1$,
\item $\lim_{|\xi| \to \pm \infty} |\hat{f}(\xi) |= 0 $.
\end{enumerate}
\end{theorem}
\begin{proof}
\begin{enumerate}
\item La fonction $\xi \mapsto \rme^{- \rmi 2 \pi \xi x} f(x)$ est continue sur $\rset$
  et major{\'e}e en module par $|f(x)|$ (qui ne d{\'e}pend pas de $\xi$), qui est dans $\lone(\rset)$. On conclut en appliquant le th\'eor\`eme de convergence domin\'ee.
\item La lin{\'e}arit{\'e} de $\TF$ d{\'e}coule directement de la lin\'earit\'e de l'int{\'e}grale. Pour tout $\xi \in \rset$, on a $|\hat{f}(\xi)| \leq
  \int |f(x)| \rmd x = \| f \|_1$. On en d{\'e}duit que $\hat{f}$ est born{\'e}e par $\| f\|_1$ et que $\TF$ est continue.
\item Supposons tout d'abord que $f$ est continue {\`a} support compact (il existe $M>0$ tel que $f(x)=0$ si $|x|>M$). Par
  changement de variable $x=t-1/(2\xi)$ dans~\eqref{eq:TFL1}, on a, pour tout $\xi\neq0$,
$$
\hat{f}(\xi) = \int_{\rset} \rme^{- \rmi 2 \pi \xi t - \rmi \pi} f(t+1/(2\xi)) \rmd t =
- \int_{\rset} \rme^{- \rmi 2 \pi \xi t} f(t+1/(2\xi)) \rmd t \eqsp.
$$
D'o{\`u} l'expression, en sommant cette {\'e}quation avec~(\ref{eq:TFL1}),  pour tout $\xi\neq0$,
$$
2 \hat{f}(\xi) = \int_{\rset} \rme^{- \rmi 2 \pi \xi x} (f(x)-f(x+1/(2\xi))) \,\rmd x
$$
Il s'en suit, par convergence domin{\'e}e (en observant que $|f(x)-f(x+1/(2\xi))|$ est major{\'e} ind{\'e}pendamment de $x$ et $\xi$ et
est nul pour $x\notin[-M-1,M+1]$ pour tout $|\xi|\geq1$),
$$
\lim_{|\xi| \to \pm \infty} |\hat{f}(\xi)| \leq \frac12 \lim_{|\xi| \to \pm \infty}
\int_{\rset} |f(x)-f(x+1/(2\xi))| \,\rmd x = 0.
$$
Soit maintenant $f \in \lone(\rset)$. Il existe une suite $\{g_n\}$ de fonctions continues dans $\lone(\rset)$ telles que $\|f - g_n \|_1 \to 0$. Comme,
$\|\hat{f} - g_n\|_\infty \leq \| f - g_n \|_1$ et $\lim_{\xi \to \pm \infty} g_n(\xi) = 0$,
on en d{\'e}duit ais{\'e}ment que $\lim_{\xi  \to \pm \infty} \hat{f}(\xi)= 0$.
\end{enumerate}
\end{proof}

La propri{\'e}t{\'e} {\`a} la fois la plus imm{\'e}diate et la plus  fondamentale de la transform{\'e}e de Fourier est son effet sur les
translations.
\begin{proposition}[Transform{\'e}e de Fourier et translation]
\label{prop:FourierTranslation}
Soit $f\in \lone(\rset)$. Alors, pour tout $t\in\rset$, les fonctions $x\mapsto f(x-x_0)$ et
$x\mapsto \rme^{\rmi 2 \pi \xi_0 x}f(x)$ sont dans $\lone(\rset)$ et v{\'e}rifient
\begin{enumerate}
\item $[\TFA{x\mapsto f(x-x_0)}](\xi)=\rme^{- \rmi 2 \pi x_0}\hat{f}(\xi)$ pour tout $\xi\in\rset$;
\item $[\TFA{x\mapsto  \rme^{\rmi 2 \pi \xi_0 x}f(x)}](\xi)=\hat{f}(\xi-\xi_0)$ pour tout $\xi\in\rset$.
\end{enumerate}
\end{proposition}
\begin{proof}
La preuve \'el\'ementaire est laiss\'ee aux lecteurs.
\end{proof}

Une des propri{\'e}t{\'e}s remarquables de la transform{\'e}e de Fourier est d'{\'e}changer la \emph{d{\'e}rivation} et la multiplication par un mon{\^o}me
\begin{proposition}[Transform{\'e}e de Fourier et D{\'e}rivation]
\label{prop:FourierDerivation}
Soit $n$ un entier naturel.
\begin{enumerate}
\item Si $x\mapsto x^k f(x)$ est dans $\lone(\rset)$  pour tout $k=0,1, \dots, n$, alors $\hat{f}$ est $n$ fois contin\^ument
  d{\'e}rivable et on a
$$
\hat{f}^{(n)} = \TFA{x\mapsto(-2 \rmi \pi x)^n f(x)}
$$
\item Si $f$ est $n$ fois contin\^ument d{\'e}rivables avec $f^{(k)} \in \lone(\rset)$ pour tout $k=0,1, \dots, n$, alors
$$
[\TF(f^{(n)})](\xi) = (2 \rmi \pi \xi)^n \hat{f}(\xi) \quad\text{pour tout $\xi\in\rset$}\eqsp.
$$
\end{enumerate}
\end{proposition}
\begin{proof}
Dans les deux cas, il suffit de d{\'e}montrer le r{\'e}sultat pour $n=1$ puis d'appliquer une r{\'e}currence {\'e}vidente.
\begin{enumerate}
\item La fonction $h:\xi \mapsto \rme^{- \rmi 2 \pi \xi x} f(x)$ est contin\^ument d{\'e}rivable
et $h'(\xi)= -2 \rmi \pi x \rme^{- \rmi 2 \pi \xi x} f(x)$. De plus $|h'(\xi)| \leq 2 \pi |xf(x)|$.
Le r{\'e}sultat d{\'e}coule du th{\'e}or{\`e}me de d{\'e}rivation sous le signe somme.
\item Comme $f' \in \lone(\rset)$, on peut calculer $\TFA{f'}$ par la formule
$$
[\TFA{f'}](\xi) = \lim_{a \to \infty} \int_{-a}^a \rme^{- \rmi 2 \pi \xi x} f'(x) \rmd x, \quad\xi\in\rset\eqsp.
$$
De plus, par int{\'e}gration par parties, pour tout $\xi\in\rset$ et tout  $a>0$,
$$
\int_{-a}^{+a} \rme^{- \rmi 2 \pi \xi x} f'(x) \rmd x = [ \rme^{- \rmi 2 \pi \xi x} f(x) ]_{-a}^a + \int_{-a}^a (2 \rmi \pi \xi) \rme^{- \rmi 2 \pi \xi x} f(x) \rmd x \eqsp.
$$
Comme $f' \in \lone(\rset)$ et $f(a) = f(0) + \int_0^a f'(t) \rmd x$, $\lim_{a \to \infty} \int_0^a f'(t) \rmd x$ existe et donc
$\lim_{a \to \infty} f(a) $ existe. Cette limite est n{\'e}cessairement nulle car $f \in \lone(\rset)$. De la m{\^e}me fa\c{c}on,
$\lim_{a \to \infty} f(-a) = 0$. D'o{\`u} le r{\'e}sultat.
\end{enumerate}
\end{proof}

La proposition suivante sera tr{\`e}s utile pour {\'e}tablir des formules d'inversion de la transform{\'e}e de Fourier.
\begin{proposition}
\label{prop:echangeTF}
Soit $f$ et $g$ deux fonctions de $\lone(\rset)$. Alors $f \hat{g}$ et $\hat{f}g$ sont dans $\lone(\rset)$ et on a
\begin{equation}
\label{eq:echange}
\int f(x) \hat{g}(x) \rmd x = \int \hat{f}(x) g(x) \rmd x \eqsp.
\end{equation}
\end{proposition}
\begin{proof}
Comme $\hat{g} \in L_\infty(\rset)$, les fonctions $f \hat{g}$ et
$\hat{f} g$ appartiennent {\`a} $\lone(\rset)$. Comme la fonction $(t,x)
\mapsto \rme^{- \rmi 2 \pi t x} f(t) g(x) \in \lone(\rset^2)$, il
r{\'e}sulte du th{\'e}or{\`e}me de Fubini que
\begin{multline*}
\int f(t) \hat{g}(t) dt = \int f(t) \left( \int \rme^{- \rmi 2 \pi t x} g(x) \rmd x \right) dt =
\\ \int g(x) \left( \int \rme^{- \rmi 2 \pi t x} f(t) dt \right) \rmd x = \int g(x) \hat{f}(x) \rmd x\eqsp.
\end{multline*}
\end{proof}


\section{D{\'e}croissance et d{\'e}rivation}
%Nous avons observ{\'e} dans la partie pr{\'e}c{\'e}dente la n{\'e}cessit{\'e} de restreindre l'espace $\lone(\rset)$
%pour pour d{\'e}finir la transform{\'e}e de Fourier inverse.
Nous avons observ{\'e} dans la partie pr{\'e}c{\'e}dente d{\`e}s le premier exemple de calcul de transform{\'e}e de Fourier
que $\lone(\rset)$  n'est pas stable sous l'effet de $\TF$.
Nous allons introduire un sous-espace de $\lone(\rset)$ stable par transformation de Fourier,
d{\'e}rivation et multiplication par un polyn{\^o}me. Cet espace introduit par Laurent Schwartz et que l'on notera $\mcs$ joue un r\^ole essentiel en analyse de Fourier.

\begin{definition}[Fonction {\`a} d{\'e}croissance rapide]\index{Fonction|{\`a}
    d{\'e}croissance rapide}
Une fonction $f : \rset \to \cset$ est dite {\`a} \emph{d{\'e}croissance rapide} si, pour tout $p \in \nset$,
on a
$$
\lim_{|x| \to \infty} |x|^p |f (x)| = 0 \eqsp.
$$
\end{definition}
C'est le cas par exemple de $f (x) = \rme^{-|x|}$.
Mais on notera que contrairement {\`a} son nom, cette d{\'e}finition n'implique aucune monotonie pour
$f$ m{\^e}me dans un voisinage de l'infini (prendre par exemple $f (x) = \rme^{-|x|} \sin x$).
Une propri{\'e}t{\'e} utile sur l'int{\'e}grabilit{\'e} des fonctions {\`a} d{\'e}croissance rapide est la suivante.
\begin{lemma}
\label{lem:decroissancerapide}
 Si $f$ est une fonction de $L_{1\mathrm{loc}}(\rset)$ {\`a} d{\'e}croissance rapide alors pour tout
 $p \in \nset$, $x \mapsto x^p f (x)$ appartient {\`a} $\lone(\rset)$.
\end{lemma}
\begin{proof}
L'indice ``loc'' signifie que la restriction de $f$ {\`a} tout compact est dans $\lone(\rset)$.
 $f$ {\'e}tant {\`a} d{\'e}croissance rapide, il existe $M > 0$ tel que pour tout $|x| \geq  M$,
on ait $|x|^{p+2} |f(x)| \leq  1$. D'o{\`u}
\begin{align*}
\int |x^p f(x)| \rmd x &\leq \int_{|x| \leq M}  |x|^p |f(x)| \rmd x+ \int_{|x| > M} |x|^{-2} |x^{p+2}  f(x)| \rmd x \\
&\leq  M^p \int_{|x| \leq M} |f(x)|\, \rmd x +  \int_{|x| > M}   x^{-2} \rmd x < \infty\eqsp.
\end{align*}
\end{proof}
On en d{\'e}duit une propri{\'e}t{\'e} remarquable de la transform{\'e}e de Fourier des fonctions {\`a} d{\'e}croissance rapide.
\begin{proposition}
\label{prop:1913}
Soit $f$ une fonction de $\lone(\rset)$ {\`a} d{\'e}croissance rapide. Alors $\hat{f}$ est ind{\'e}finiment d{\'e}rivable.
\end{proposition}
\begin{proof}
D'apr{\`e}s la proposition \ref{prop:FourierDerivation}, $\hat{f}$ est dans $C_\infty(\rset)$ d{\'e}s que, pour tout $p \in  \nset$,
$x^p f (x)$ est dans $\lone(\rset)$; ce qui est assur{\'e} par  le lemme \ref{lem:decroissancerapide}.
\end{proof}
Inversement si $f$ est dans $C_\infty(\rset)$ quelles propri{\'e}t{\'e}s poss{\`e}de $\hat{f}$ ? Le r{\'e}sultat suivant am{\`e}ne un {\'e}l{\'e}ment de r{\'e}ponse.

\begin{proposition}
\label{prop:1914}
Soit $f$ une fonction de $C_\infty(\rset)$. Si pour tout $k \in \nset$, $f^{(k)}$ est dans $\lone(\rset)$
alors $\hat{f}$ est {\`a} d{\'e}croissance rapide.
\end{proposition}
\begin{proof}
D'apr{\`e}s la proposition \ref{prop:FourierDerivation} on a, pour tout $k \in \nset$,
$\widehat{f^{(k)}}(\xi) = (2 \rmi \pi \xi)^k \hat{f}(\xi)$.
En appliquant le th{\'e}or{\`e}me de Riemann-Lebesgue,  il vient $\lim_{|\xi| \to \infty} |\xi|^k |\hat{f}(\xi)| = 0$.
\end{proof}
Autrement dit nous venons de voir que
\begin{enumerate}
\item plus $f$ d{\'e}cro{\^i}t rapidement {\`a} l'infini, plus $\hat{f}$ est r{\'e}guli{\`e}re;
\item plus $f$ est r{\'e}guli{\`e}re, plus $\hat{f}$ d{\'e}cro{\^i}t rapidement {\`a} l'infini.
En particulier si $f \in C_\infty(\rset)$ et est {\`a} d{\'e}croissance rapide, il en est de m{\^e}me pour
$\hat{f}$.
\end{enumerate}

\section{Un exemple remarquable}

Nous allons consid{\'e}rer une famille de fonctions  qui reste stable par transformation de Fourier.

On introduit pour tout $\sigma > 0$ la fonction de \emph{densit{\'e} gaussienne}
\index{Densit{\'e} gaussienne}
\begin{equation}
  \label{eq:gauss_func}
  g_\sigma(x)=\frac{1}{\sigma \sqrt{2 \pi} }\rme^{- \xi^2/2 \sigma^2}.
\end{equation}


\begin{lemma}\label{lem:gaussenneTF}
La fonction  $g_1(x)= 1 / \sqrt{2\pi} \exp( -x^2/2)$ est la densit{\'e} d'une probabilit{\'e} sur $\rset$, et sa transform{\'e}e de Fourier
est $\hat{g_1}(\xi)= \rme^{- 2 \pi^2 \xi^2}$.
\end{lemma}
\begin{proof}
La fonction  est positive et v{\'e}rifie  $\int_\rset g_1(x) \rmd x= 1$ (on peut le montrer en exercice en {\'e}crivant le carr{\'e} de
l'int{\'e}grale comme une double int{\'e}grale).

La fonction $x\mapsto xg_1(x)$ {\'e}tant aussi dans $\lone(\rset)$, on peut appliquer la
proposition~\ref{prop:FourierDerivation}(i), et on obtient, pour tout $\xi\in\rset$,
$$
\hat{g_1}'(\xi) = - 2\rmi \pi \int x \,g_1(x) \rme^{- \rmi 2 \pi \xi x} \rmd x \eqsp.
$$
Un calcul imm{\'e}diat donne $g_1'(x)= -xg_1(x)$; une int{\'e}gration par partie donne donc
$$
\hat{g_1}'(\xi) = - 2\rmi \pi  \int g_1'(x) \,(- \rmi 2 \pi \xi)\,\rme^{- \rmi 2 \pi \xi x} \rmd x \eqsp.
$$
D'o{\`u} l'on tire finalement que $\hat{g}_1'(\xi) = - 4 \pi^2 \xi \hat{g}_1(\xi)$.
La solution g{\'e}n{\'e}rale de l'{\'e}quation diff{\'e}rentielle {\`a} variables s{\'e}parables  $f'(u)= - 4 \pi^2 u f(u)$
{\'e}tant $f(u) = C \rme^{- 2 \pi^2 u^2}$, et comme on a  $\hat{g_1}(0) = \int g_1(x) \rmd x = 1$,
on voit que n{\'e}cessairement $\hat{g_1}(\xi)= \rme^{-2 \pi^2 \xi^2}$.
\end{proof}

Par un changement de variable {\'e}vident, ce r{\'e}sultat se g{\'e}n{\'e}ralise ais{\'e}ment {\`a} tout $\sigma>0$. En particulier,
\begin{equation}\label{eq:gaussienneTF}
\hat{g_\sigma}(\xi) =  \rme^{-2 \pi^2 (\xi\sigma)^2} = \frac1{\sigma\sqrt{2\pi}}\,g_{1/2\pi\sigma}(\xi),\quad\xi\in\rset.
\end{equation}

Comme annonc\'e plus haut, la famille $(g_\sigma)_{\sigma>0}$ est donc stable par
transform{\'e} de Fourier.
\section{Quelques exemples classiques}
Rappelons que  $u$ est la fonction de Heaviside, d\'efinie par $u(x)=1$ pour $x>0$ et $u(x)=0$ pour $x\leq 0$.
\begin{enumerate}[label=(\roman*)]
\item $f_{1}(x)=\rme^{-ax}u(x)$ , ${\rm Re}(a)>0.$
$$
\hat{f_{1}}(\xi)=\int_{0}^{\infty}\rme^{-2i\pi x\xi}\rme^{-ax}dx=\lim_{b\rightarrow+\infty}[\frac{-\rme^{-x(a+2i\pi\xi)}}{a+2i\pi\xi}]_{0}^{b}=\frac{1}{a+2i\pi\xi}.
$$
\item $f_{2}(x)=\rme^{ax}u(-x)$ , ${\rm Re}(a)>0.$
$$
\hat{f_{2}}(\xi)=\int_{-\infty}^{0}\rme^{-2i\pi x\xi}\rme^{ax}dx=\frac{-1}{-a+2i\pi\xi}.
$$
\item $f_{3}(x)=\displaystyle \frac{x^{k}}{k!}\rme^{-ax}u(x)$ , ${\rm Re}(a)>0.$

$f_{3}(x)=\displaystyle \frac{1}{(-2i\pi)^{k}}\frac{1}{k!}(-2i\pi x)^{k}f_{1}(x)$ , and $\displaystyle \hat{f_{3}}(\xi)=\frac{1}{k!}\frac{1}{(-2i\pi)^{k}}\hat{f_{1}}^{(k)}(\xi)$. Comme $\hat{f_{1}}^{(k)}(\xi)=k!(a+2i\pi\xi)^{-(k+1)}(-2i\pi)^{k},$
$$
\hat{f_{3}}(\xi)=\frac{1}{(a+2i\pi\xi)^{k+1}}.
$$
\item $f_{4}(x)=\displaystyle \frac{x^{k}}{k!}\rme^{ax}u(-x)$ , ${\rm Re}(a)>0.$
Nous avons
$$
\hat{f_{4}}(\xi)=\frac{-1}{(-a+2i\pi\xi)^{k+1}}.
$$
\item $f_{5}(x)=\rme^{-a|x|}, {\rm Re}(a)>0$. Nous d\'eduisons des calculs pr\'ec\'edents
$$
\hat{f_{5}}(\xi)=\frac{2a}{a^{2}+4\pi^{2}\xi^{2}}.
$$
\item $f_{6}(x)=\mathrm{sign}(x)\rme^{-a|x|}, {\rm Re}(a)>0$. Nous avons
$$
\hat{f_{6}}(\xi)=\frac{-4i\pi\xi}{a^{2}+4\pi^{2}\xi^{2}}.
$$
\end{enumerate}
\section{Formulaire}
\label{sec:formulaire}
\begin{enumerate}[label=(\roman*)]
\item
\begin{align*}
\hat{f}^{(k)}(\xi)&=\widehat{(-2\rmi\pi x)^{k}f}(\xi)\\
\widehat{f^{(k)}}(\xi)&=(2\rmi \pi\xi)^{k}\hat{f}(\xi)
\end{align*}
\item
\begin{align*}
f(x-a) &\TFyield \rme^{-2\rmi \pi a\xi}\hat{f}(\xi) \\
\rme^{2 \rmi \pi ax}f(x)&\TFyield \hat{f}(\xi-a) \\
\end{align*}
\item $a\neq 0$.
$$ f(ax)\TFyield \frac{1}{|\xi|}\hat{f}(\frac{\xi}{a})$$
\item  $a\in \cset$, ${\rm Re}(a)>0, k=0,1,2\ldots.$
\begin{align*}
\frac{x^{k}}{k!}\rme^{-ax}u(x) &\TFyield \frac{1}{(a+2i\pi\xi)^{k+1}} \\
\frac{x^{k}}{k!}\rme^{ax}u(-x) &\TFyield \frac{-1}{(-a+2i\pi\xi)^{k+1}} \\
\rme^{-a|x|} &\TFyield \frac{2a}{a^{2}+4\pi^{2}\xi^{2}} \\
\mathrm{sign}(x)\rme^{-a|x|} &\TFyield \frac{-4i\pi\xi}{a^{2}+4\pi^{2}\xi^{2}} \\
\end{align*}
\item  $a\in \rset$, $a>0$.
\begin{align*}
\rme^{-ax^{2}}  &\TFyield \sqrt{\frac{\pi}{a}}\rme^{-\frac{\pi^{2}}{a}\xi^{2}} \\
\1_{\ccint{-a,+a}}(x) &\TFyield \frac{\sin 2a\pi\xi}{\pi\xi}
\end{align*}
\end{enumerate}
%%% Local Variables:
%%% mode: latex
%%% ispell-local-dictionary: "francais"
%%% TeX-master: "Polycopie-Fourier-L1L2"
%%% End:
