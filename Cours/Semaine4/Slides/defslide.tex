\documentclass{beamer}
\usepackage{graphicx}
\usepackage{amsmath,amssymb,amstext,amsthm}
\usepackage{amsfonts}
\usepackage{bbm}
\usepackage{beamerthemesplit}

\usepackage[utf8]{inputenc}
\usepackage[french]{babel}
\usepackage{bbm}

\usetheme{Antibes}
\mode<presentation>
\useoutertheme{tree}
\usecolortheme{beaver}
\useinnertheme{rectangles}

\setbeamerfont{block title}{size={}}
%\usecolortheme[rgb={0.55,0.1,0.05}]{structure}
%\usecolortheme[rgb={0.75,0.1,0.05}]{structure}
\usepackage{color}

\newenvironment{disarray}{\everymath{\displaystyle\everymath{}}\array} {\endarray}
\newtheorem{theo}{Théorème}
\newtheorem{prop}[theo]{Proposition}
\newtheorem{conj}[theo]{Conjecture}
\newtheorem{cor}{Corollary}[theo]
\newtheorem{lem}{Lemme}
\newtheorem{nota}{Notation}
\newtheorem{rk}{Remark}
\newtheorem{exa}{Example}
\newtheorem{df}{Definition}
\newtheorem{terminologie}{Terminologie}
\def\rme{\mathrm{e}}
\def\rmi{\mathrm{i}}
\def\rset{\mathbb{R}}
\def\nset{\mathbb{N}}
\def\dlim{\stackrel{d}{\rightarrow}}
\def\plim{\stackrel{\PP}{\longrightarrow}}
\def\iid{i.i.d.}
\def\1{\mathbbm{1}}
\newenvironment{dem}{\textbf{Proof}}{\flushright$\blacksquare$\\}
%\def\blankframe{
%\mode<presentation>{
%  { \setbeamertemplate{background canvas}[default]
%    \setbeamercolor{background canvas}{bg=black}
%    \begin{frame}[plain]{}
%    \end{frame}
%  }
%}
%\mode<presentation>{
%\setbeamertemplate{background canvas}[default]
%\setbeamercolor{background canvas}{bg=white}}
%\mode*
%}
\def\eqsp{\,}
\DeclareMathOperator{\E}{{\mathbb E}}
\def\PE{\E}
\DeclareMathOperator{\F}{{\mathbb F}}
\DeclareMathOperator{\G}{{\mathbb G}}
\DeclareMathOperator{\D}{{\mathbb D}}
\DeclareMathOperator{\R}{{\mathbb R}}
\DeclareMathOperator{\C}{{\mathbb C}}
\DeclareMathOperator{\Z}{{\mathbb Z}}
\DeclareMathOperator{\N}{{\mathbb N}}
\DeclareMathOperator{\K}{{\mathbb K}}
\DeclareMathOperator{\T}{{\mathbb T}}
\DeclareMathOperator{\PP}{{\mathbb P}}
\DeclareMathOperator{\QQ}{{\mathbb Q}}
\DeclareMathOperator{\Q}{{\mathbb Q}}
\DeclareMathOperator{\IF}{{\mathbb I}}


%%%%%%%%%%%%%%%%%%%%%%%%%%%%%%% Pour le modle lin\'eaire

\DeclareMathOperator{\bX}{\boldsymbol{X}}
\DeclareMathOperator{\bY}{\boldsymbol{Y}}
\DeclareMathOperator{\bx}{\boldsymbol{x}}
\DeclareMathOperator{\vp}{\boldsymbol{p}}
\DeclareMathOperator{\vq}{\boldsymbol{q}}
\DeclareMathOperator{\estMC}{\widehat \theta_n^{\,\,{\tt mc}}}
\DeclareMathOperator{\estMCNL}{\widehat \theta_n^{\,\,{\tt mcnl}}}
\DeclareMathOperator{\estMV}{\widehat \theta_n^{\,\,{\tt mv}}}
\DeclareMathOperator{\design}{\mathbb{M}}
\DeclareMathOperator{\est}{\widehat \theta_{\mathnormal{n}}}
\DeclareMathOperator{\var}{\mathrm{Var}}
\def\Var{\var}
\DeclareMathOperator{\estMVc}{\widehat \theta_{n,0}^{\,{\tt mv}}}
\DeclareMathOperator{\Xbar}{\overline{\mathnormal{X}}_\mathnormal{n}}

\newcommand{\indi}[1]{\mathbbm{1}_{\{#1\}}}
\newcommand{\coint}[1]{\left[#1\right)}
\newcommand{\ocint}[1]{\left(#1\right]}
\newcommand{\ooint}[1]{\left(#1\right)}
\newcommand{\ccint}[1]{\left[#1\right]}
\newcommand{\Id}{\mathrm{I}}

\definecolor{LightYell}{rgb}{0.95,0.83,0.70}
\definecolor{orange}{rgb}{1.0,0.50,0.01}
\definecolor{StroYell}{rgb}{0.95,0.88,0.72}
\definecolor{lightred}{rgb}{0.75,0.033,0}
\definecolor{shadecolor1}{rgb}{0.90,0.83,0.70}
\definecolor{myem}{rgb}{0.797,0.598,0.598}
\definecolor{BrickRed}{cmyk}{0,0.89,0.94,0.28}
\definecolor{RoyalPurple}{cmyk}{0.75,0.9,0,0}

\newcommand{\tco}[1]{\textcolor{orange}{#1}}
\newcommand{\tcr}[1]{\textcolor{lightred}{#1}}

\def\gauss{\mathcal{N}}
\def\truetheta{\theta}
\def\curtheta{\vartheta}