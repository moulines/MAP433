\documentclass[a4paper,11pt,fleqn]{article}

\usepackage[francais]{babel}
%\usepackage[latin1]{inputenc}
%\usepackage[applemac]{inputenc}
\usepackage{a4wide,amsmath,amssymb,bbm,fancyhdr,hyperref, graphicx}

\RequirePackage[OT1]{fontenc}

\usepackage[latin1]{inputenc}
% THE variable
\newcommand{\thisyear}{Ann\'ee 2015-2016}

% Definitions (pas trop!)
\newcommand{\R}{\ensuremath{\mathbb{R}}}
\newcommand{\N}{\mathbb{N}}
\newcommand{\rset}{\ensuremath{\mathbb{R}}}
\renewcommand{\P}{\ensuremath{\operatorname{P}}}
\newcommand{\E}{\ensuremath{\mathbb{E}}}
\newcommand{\V}{\ensuremath{\mathbb{V}}}
\newcommand{\gaus}{\ensuremath{\mathcal{N}}}
\newcommand{\1}{\ensuremath{\mathbbm{1}}}
\newcommand{\dlim}{\ensuremath{\stackrel{\mathcal{L}}{\longrightarrow}}}
\newcommand{\plim}{\ensuremath{\stackrel{\mathrm{P}}{\longrightarrow}}}
\newcommand{\PP}{\ensuremath{\mathbb{P}}}
\newcommand{\p}{\ensuremath{\mathbb{P}}}
\newcommand{\eps}{\varepsilon}
\newcommand{\pa}[1]{\left(#1\right)}
\newcommand{\ac}[1]{\left\{#1\right\}}


% Style
\pagestyle{fancyplain}
\renewcommand{\sectionmark}[1]{\markright{#1}}
\renewcommand{\subsectionmark}[1]{}
\lhead[\fancyplain{}{\thepage}]{\fancyplain{}{\footnotesize {\sf
MAP433 Statistique, \thisyear / PC5-6}}}
\rhead[\fancyplain{}{\footnotesize {\sf MAP433 Statistique,
\thisyear / \rightmark}}]{\fancyplain{}{\thepage}}
\cfoot[\fancyplain{}{}]{\fancyplain{}{}}
\renewcommand{\thefootnote}{\fnsymbol{footnote}}
\parindent=0mm

% Titre
%\title{{\bf MAP433 Statistique}}
\title{\includegraphics[width=3.5cm]{logo_x.jpg}\hfill {\bf MAP433 Statistique}\hfill \quad\quad\quad\quad\quad\quad \ }
\author{{\bf PC 5 et 6: R\'egression}}
\date{25 septembre et 2 octobre 2015}

\begin{document}

\maketitle




\section{Mod\`ele de r\'egression multiple}
On consid\`ere le mod\`ele de regression multiple
$$y=\theta_{0}e+X\theta+\xi,\quad\textrm{o\`u}\ \ \E[\xi]=0,~\E[\xi\xi^T]=\sigma^2I_n,\ e=(1,1,\ldots,1)^T$$
avec $X$ une matrice $n\times k$ de rang $k$ et  $y$, $\xi$ des vecteurs de $\mathbb{R}^n$. Les param\`etres
$\theta_{0}\in\R$ et $\theta\in\R^k$ sont inconnus. On note $\hat \theta_{0}$ et $\hat \theta$ les estimateurs des moindres carr\'es de $\theta_{0}$ et $\theta$.
\begin{enumerate}
\item On note $\hat y = \hat \theta_{0} e+X\hat \theta$. 
Montrer que $\bar{\hat{y}}=\bar{y}$, o\`u $\bar y$ (resp. $\bar{\hat{y}}$) est la moyenne des $y_{i}$ (resp. des $\hat y_{i}$). En d\'eduire que  $\bar y = \hat\theta_{0}+\bar X\hat \theta$ o\`u
$\bar X={1\over n}e^TX=\left[\ldots,\bar X_{:\,j},\ldots\right].$
\item Montrer l'\'equation d'analyse de la variance:
$$\|y-\bar ye\|^2=\|y-\hat y\|^2+\|\hat y-\bar ye\|^2.$$
En d\'eduire que 
le {\it coefficient de d\'etermination}
$$R^2=\frac{\sum_{i=1}^n(\hat{y}_i-\bar{y})^2}{\sum_{i=1}^n(y_i-\bar{y})^2}$$
est toujours inf\'erieur \`a 1.
\item Supposons que $Z=[e,X]$ est de rang $k+1$.
Calculez en fonction de $Z$ la matrice de covariance de $(\hat \theta_{0},\hat\theta)$. Comment acc\`ede-t-on \`a Var$(\hat\theta_{j})$, pour $j=0,\ldots,p$?
\item Proposer un estimateur sans biais de $\sigma^2$ puis de  la matrice de covariance de $(\hat \theta_{0},\hat \theta)$.
\item On suppose dor\'enavant que $\theta_{0}=0$ et donc
$$y=X\theta+\xi,\quad\E[\xi]=0,~\E[\xi\xi^T]=\sigma^2I_n.$$
L'estimateur des moindres carr\'es $\tilde \theta$ dans ce mod\`ele est-il \'egal \`a $\hat \theta$?
\item A-t-on la relation $\bar{\hat{y}}=\bar{y}$? Que dire du $R^2$ dans ce mod\`ele?
\end{enumerate}

\section{Le mod\`ele ANOVA}


On dispose d'observations de variables al\'eatoires
$$
Y_{ij}= m_i + \xi_{ij}, \quad i=1,\dots, k, \quad j=1,\dots, l,
$$
o\`u $(m_1,\dots,m_k)\in \R^k$ et les $\xi_{ij}$ sont des variables
al\'eatoires i.i.d. de loi ${\cal N}(0,\sigma^2)$. 



\begin{enumerate}
\item Montrer qu'il s'agit d'un mod\`ele de r\'egression lin\'eaire
avec la matrice ${\bf X}$ que l'on pr\'ecisera. Que vaut $B={\bf X}^T{\bf X}$?
\item Montrer que la condition $m_1=m_2=\dots=m_k$  s'\'ecrit sous la forme $Gm=0$ avec une
matrice $G$ que l'on pr\'ecisera.
\item On estime $m$ par l'estimateur des moindre carr\'es $\hat m$. Quelle est la covariance de $\hat m$? 
\item Proposer un estimateur de $Gm$. Quel est son biais? sa covariance?
\item Proposer un estimateur $\hat \sigma^2$ de $\sigma^2$. Quelle est sa distribution?
\end{enumerate}





\section{Th\'eor\`eme de Gauss-Markov}
On consid\`ere le mod\`ele de r\'egression
$$\underset{(n,1)}{Y}=\underset{(n,k)}{X}\underset{(k,1)}{\theta}
+\underset{(n,1)}{\xi}.$$ On suppose que $X$ est une matrice
d\'eterministe, $\E[\xi]=0$, $\E[\xi\xi^T]=\sigma^2I_n$,
Rang$(X)=k$. On note $\hat{\theta}$ l'estimateur des MC de $\theta$.
\begin{enumerate}
\item Montrer que $\hat{\theta}$ est sans biais et expliciter
sa matrice de covariance.
\item Soit $\tilde\theta$ un estimateur de $\theta$ lin�aire en $Y$,
i.e., $\tilde\theta = LY$ pour une matrice $L\in\R^{k\times n}$ d\'eterministe. Donner une condition n�cessaire et suffisante sur $L$ pour
que $\tilde\theta$ soit sans biais. On supposera maintenant cette
hypoth�se v�rifi�e.
\item Calculer la matrice de covariance de $\tilde\theta$.
En posant $\Delta=L-(X^TX)^{-1}X^T$ montrer que $\Delta X=0$ et cov$(\tilde \theta)=\ $cov$(\hat\theta)+\sigma^2\Delta\Delta^T$.
En d\'eduire que
$$
\E[(\tilde\theta - \theta)(\tilde\theta - \theta)^T]\ge \E[(\hat{\theta} -
\theta)(\hat{\theta} - \theta)^T] \quad\textrm{(in\'egalit\'e au sens matriciel).}
$$
\item En passant au risques quadratiques
$\E\big[\|\tilde\theta - \theta\|^2\big]$ et $\E\big[\|\hat{\theta} -
\theta\|^2\big]$, en d\'eduire que l'estimateur des MC est optimal
dans la classe de tous les estimateurs lin\'eaires sans biais.

\end{enumerate}


\section{R\'egression Ridge}
On consid\`ere le mod\`ele de r\'egression
$$\underset{(n,1)}{Y}=\underset{(n,k)}{X}\underset{(k,1)}{\theta}+
\underset{(n,1)}{\xi}.$$ On suppose que $X$ est une matrice
d\'eterministe, $\E[\xi]=0$, $\E[\xi\xi^T]=\sigma^2I_n$.
\begin{enumerate}
\item On suppose que $k> n$. Que dire de l'estimation par moindres carr\'es?
\item On appelle estimateur \texttt{Ridge regression} de
param\`etre de r\'egularisation $\lambda>0$ l'estimateur
$$
\hat \theta_{\lambda}= \arg \min_{\theta\in\R^k}\left\{
\|Y-X\theta\|^2+\lambda \|\theta\|^2\right\}.
$$
Exprimez $\hat \theta_{\lambda}$ en fonction de $X$, $Y$ et
$\lambda$. Cet estimateur est-il d\'efini pour $k>n$?
\item Calculez la moyenne et la matrice de covariance de
l'estimateur Ridge. Est-il sans biais?
\item On suppose maintenant que $k=1$, ce qui correspond
au mod\`ele de r\'egression simple.
 Montrer qu'il existe une valeur de $\lambda$ telle que, pour certaines valeurs de $\theta$,  le risque $\E\big[(\hat\theta_\lambda - \theta)^2\big]$ de l'estimateur Ridge de param\`etre $\lambda$ est inf\'erieur au risque  $\E\big[(\hat\theta_0 - \theta)^2\big]$ de l'estimateur des MC.
\end{enumerate}





\section{Analyse de donn\'ees atmosph\'eriques}
Nous allons analyser des relev\'es atmosph\'eriques effectu\'es par l'association``Air Breizh". 
Ces relev\'es se pr\'esentent sous la forme d'un tableau dont chaque ligne donne les mesures de l'ozone du jour (O3), de la temp\'erature \`a 12h (T12) et 15h (T15), d'un indice de n\'ebulosit\'e \`a 12h (Ne12), des relev\'es de vents \`a 12h (N12, S12, E12, W12),
d'un indice du vent moyen (Vx) et de la concentration en ozone de la veille (03v). 
Notre objectif sera de trouver parmi  les facteurs pr\'ec\'edents ceux qui sont influents sur la quantit\'e d'ozone (O3) pr\'esente dans la basse atmosph\`ere.  \medskip

Les analyses seront r\'ealis\'ees avec  {\tt R} : c'est un logiciel gratuit et tr\`es largement utilis\'e par les statisticiens car la plupart des m\'ethodes statistiques (anciennes et nouvelles) ont \'et\'e impl\'ement\'ees dans ce langage. Il est 
t\'el\'echargeable sur: \url{http://cran.r-project.org/}.
 Pour apprendre \`a s'en servir:  \url{http://cran.r-project.org/doc/manuals/R-intro.pdf}. Vous pouvez aussi consulter l'ouvrage 
 \emph{R\'egression avec R} de Cornillon \& Matzner-Lober. La syntaxe est proche de scilab et matlab. \medskip
 
 Pour commencer, t\'el\'echargez les donn\'ees sur la page \url{http://www.cmap.polytechnique.fr/~giraud/MAP433/ozone.Rdata}. Lancez {\tt R}, puis chargez les donn\'ees dans {\tt R} avec la commande {\tt load("ozone.Rdata")}. Nous effectuerons une r\'egression lin\'eaire \`a l'aide de la fonction {\tt lm}. Par exemple
 \begin{verbatim}
 reg = lm(O3~T12+Vx, data=ozone)
 \end{verbatim}
 r\'ealise la r\'egression de O3 par rapport aux variables T12 et Vx. 
 Tapez {\tt ?lm} pour avoir une description de cette fonction.
 Si {\tt reg} est le r\'esultat d'une r\'egression de $Y\in\R^n$ contre $X\in\R^{n\times k}$, l'instruction {\tt summary(reg)} retourne un tableau de valeurs dont la premi\`ere  colonne donne l'estimateur 
 $$\widehat \theta\in\mathop{\textrm{argmin}}_{\theta\in\R^k}\|Y-X \theta\|^2$$  
 et l'avant derni\`ere colonne donne les $t$-values
$$\hat t_{j}={\widehat\theta_{j}\over \sqrt{\hat \sigma^2[(X^TX)^{-1}]_{jj}}}\quad\textrm{o\`u}\quad \widehat \sigma^2={1\over n-k}\|Y-X\widehat\theta\|^2.$$
 La derni\`ere colonne donne les $p$-values $\hat p_{j}= \mathcal{T}_{n-k}(\hat t_{j})$ o\`u $\mathcal{T}_{n-k}(t)=\p(|T_{n-k}|>|t|)$  avec $T_{n-k}$ une variable de Student \`a $n-k$ degr\'es de libert\'e.



\subsection*{A) Inspection des r\'esidus}
\begin{enumerate}
\item Calculer avec la fonction {\tt lm} la r\'egression de O3 par rapport aux autres variables. Identifier $Y$ et $X$ dans ce cas.  Que vaut $n$? Que vaut $k$?
\item On note $\hat\xi=Y-\hat Y$ o\`u $\hat Y=X\widehat \theta$. Tracer l'histogramme des $\{\hat \xi_{i}:i=1,\ldots,n\}$ \`a l'aide de la fonction ${\tt hist}$. 
\item L'histogramme sugg\`ere que les r\'esidus pourraient suivre une loi Gaussienne. On va inspecter cette hypoth\`ese en regardant les quantiles de la loi empirique. On note $x_{q}(Q)=\min\{x : {Q}(]-\infty,x])\geq q\}$ le quantile d'ordre $q$ d'une loi $Q$. Tracer le QQplot 
\begin{verbatim}
qqnorm(lm(O3~.,data=ozone)$residuals)
\end{verbatim}
 Que repr\'esente ce graphique?
 \item Pour savoir si la variance d\'epend du signal tracer les points $\big\{(\hat Y_{i},|\hat\xi_{i}|):i=1,\ldots,n\big\}$ \`a l'aide de la fonction {\tt plot}. Effectuer la r\'egression des $|\hat\xi_{i}|$ en fonction des $\hat Y_{i}$.
\end{enumerate}


\subsection*{B) Choix des variables}
\begin{enumerate}
\item Quelles variables $j$ ont une $p$-value $\hat p_{j}$ inf\'erieure \`a 5\%?
\item Calculer la r\'egression de O3 par rapport \`a Ne12+O3v et inspecter les r\'esidus comme pr\'ec\'edemment.
\item Calculer la r\'egression de O3 par rapport \`a Ne12+O3v+T15+Vx. Que constatez-vous au niveau des  $p$-values $\hat p_{j}$?
\end{enumerate}



\subsection*{C) R\'egressions partielles}
Dor\'enavant on ne travaille qu'avec les variables Ne12, O3v, T15 et Vx. On veut inspecter les questions suivantes:
\begin{itemize}
\item le mod\`ele lin\'eaire par rapport \`a la variable $j$ est-il raisonnable?
\item quelle est l'influence de la variable $j$?
\end{itemize}
\begin{enumerate}
\item Montrer que si le mod\`ele $Y=\sum_{k}\theta_{k}X_{k}+\xi$ est vrai, alors:

lm($Y\sim -X_{j}$)\$residuals = $\theta_{j}\times$ lm($X_{j} \sim -X_{j}$)\$residuals + lm($\xi \sim -X_{j}$)\$residuals 

o\`u lm($Z\sim -X_{j}$) repr\'esente la r\'egression de $Z$ par rapport \`a toutes les variables $X_{1},\ldots,X_{k}$ sauf $X_{j}$.
\item Si les $\xi_{1},\ldots,\xi_{n}$ sont i.i.d.\ de loi $\mathcal{N}(0,\sigma^2)$,
quelle est la loi du vecteur lm($\xi \sim -X_{j}$)\$residuals?
\item Calculer la r\'egression de lm($Y\sim -X_{j}$)\$residuals par lm($X_{j} \sim -X_{j}$)\$residuals pour $j=$Ne12. Le mod\`ele lin\'eaire semble-t-il raisonnable pour cette variable?
\item M\^eme question avec la variable $j=$T15.
 \end{enumerate}









\end{document}
