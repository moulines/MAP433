\documentclass[a4paper,11pt,fleqn]{article}

\usepackage[francais]{babel}
%\usepackage[latin1]{inputenc}
%\usepackage[applemac]{inputenc}
\usepackage{a4wide,amsmath,amssymb,bbm,fancyhdr,hyperref, graphicx}
\usepackage{mdwlist}

\RequirePackage[OT1]{fontenc}

\usepackage[latin1]{inputenc}
% THE variable
\newcommand{\thisyear}{Ann\'ee 2015-2016}

% Definitions (pas trop!)
\newcommand{\R}{\ensuremath{\mathbb{R}}}
\newcommand{\N}{\mathbb{N}}
\newcommand{\rset}{\ensuremath{\mathbb{R}}}
\renewcommand{\P}{\ensuremath{\operatorname{P}}}
\newcommand{\E}{\ensuremath{\mathbb{E}}}
\newcommand{\V}{\ensuremath{\mathbb{V}}}
\newcommand{\gaus}{\ensuremath{\mathcal{N}}}
\newcommand{\1}{\ensuremath{\mathbbm{1}}}
\newcommand{\dlim}{\ensuremath{\stackrel{\mathcal{L}}{\longrightarrow}}}
\newcommand{\plim}{\ensuremath{\stackrel{\mathrm{P}}{\longrightarrow}}}
\newcommand{\PP}{\ensuremath{\mathbb{P}}}
\newcommand{\p}{\ensuremath{\mathbb{P}}}
\newcommand{\eps}{\varepsilon}
\newcommand{\pa}[1]{\left(#1\right)}
\newcommand{\ac}[1]{\left\{#1\right\}}
\newcommand{\HH}{\mathcal{H}}


% Style
\pagestyle{fancyplain}
\renewcommand{\sectionmark}[1]{\markright{#1}}
\renewcommand{\subsectionmark}[1]{}
\lhead[\fancyplain{}{\thepage}]{\fancyplain{}{\footnotesize {\sf
MAP433 Statistique, \thisyear / PC5-6}}}
\rhead[\fancyplain{}{\footnotesize {\sf MAP433 Statistique,
\thisyear / \rightmark}}]{\fancyplain{}{\thepage}}
\cfoot[\fancyplain{}{}]{\fancyplain{}{}}
\renewcommand{\thefootnote}{\fnsymbol{footnote}}
\parindent=0mm

% Titre
%\title{{\bf MAP433 Statistique}}
\title{\includegraphics[width=3.5cm]{../images/logo_x.jpg}\hfill {\bf MAP433 Statistique}\hfill \quad\quad\quad\quad\quad\quad \ }
\author{{\bf PC8 - Tests et regression}}
\date{}

\begin{document}

\maketitle

\section{Test dans un mod\`ele ANOVA}

On observe une quantit\'e $Y$ parmi $k$ populations.  On veut tester si la moyenne des $Y$ est la m\^eme parmi ces $k$ populations. Pour chaque population, on a un \'echantillon de taille $\ell$. Plus pr\'ecis\'ement,
on observe
$$
Y_{ij}= m_i + \xi_{ij}, \quad i=1,\dots, k, \quad j=1,\dots, \ell,
$$
o\`u $(m_1,\dots,m_k)\in \R^k$ et les $\xi_{ij}$ sont des variables
al\'eatoires i.i.d. de loi ${\cal N}(0,\sigma^2)$. On consid\`ere le
probl\`eme de test d'\'egalit\'e des moyennes
$$ {\cal H}_0: m_1=m_2=\dots=m_k \quad\text{contre}\quad
{\cal H}_1: \text{ $\exists i\not=i'$ tels que $m_i\not=m_{i'}$}.$$
\begin{enumerate}
\item Montrer qu'il s'agit d'un mod\`ele de r\'egression lin\'eaire
avec une matrice ${\bf X}$ que l'on pr\'ecisera. Que vaut $B={\bf X}^T{\bf X}$?
\item Montrer que l'hypoth\`ese nulle s'\'ecrit
$${\cal H}_0:\, m\in \Theta_0= \{m: Gm =0\}$$ pour une
matrice $G$ que l'on pr\'ecisera.
\item Proposez un test de ${\cal H}_{0}$ contre ${\cal H}_{1}$ dans ce contexte.
\end{enumerate}








\section{Analyse de donn\'ees atmosph\'eriques}
Nous allons analyser des relev\'es atmosph\'eriques effectu\'es par l'association``Air Breizh".
Ces relev\'es se pr\'esentent sous la forme d'un tableau dont chaque ligne donne les mesures de l'ozone du jour (O3), de la temp\'erature \`a 12h (T12) et 15h (T15), d'un indice de n\'ebulosit\'e \`a 12h (Ne12), des relev\'es de vents \`a 12h (N12, S12, E12, W12),
d'un indice du vent moyen (Vx) et de la concentration en ozone de la veille (03v).
Notre objectif sera de trouver parmi  les facteurs pr\'ec\'edents ceux qui sont influents sur la quantit\'e d'ozone (O3) pr\'esente dans la basse atmosph\`ere.  \medskip

Les analyses seront r\'ealis\'ees avec  {\tt R} : c'est un logiciel gratuit et tr\`es largement utilis\'e par les statisticiens car la plupart des m\'ethodes statistiques (anciennes et nouvelles) ont \'et\'e impl\'ement\'ees dans ce langage. Il est
t\'el\'echargeable sur: \url{http://cran.r-project.org/}.
 Pour apprendre \`a s'en servir:  \url{http://cran.r-project.org/doc/manuals/R-intro.pdf}. Vous pouvez aussi consulter l'ouvrage
 \emph{R\'egression avec R} de Cornillon \& Matzner-Lober. La syntaxe est proche de scilab et matlab. \medskip

 Pour commencer, t\'el\'echargez les donn\'ees sur la page \url{http://www.cmap.polytechnique.fr/~giraud/MAP433/ozone.Rdata}. Lancez {\tt R}, puis chargez les donn\'ees dans {\tt R} avec la commande {\tt load("ozone.Rdata")}. Nous effectuerons une r\'egression lin\'eaire \`a l'aide de la fonction {\tt lm}. Par exemple
 \begin{verbatim}
 reg = lm(O3~T12+Vx, data=ozone)
 \end{verbatim}
 r\'ealise la r\'egression de O3 par rapport aux variables T12 et Vx.
 Tapez {\tt ?lm} pour avoir une description de cette fonction.
 Si {\tt reg} est le r\'esultat d'une r\'egression de $Y\in\R^n$ contre $X\in\R^{n\times k}$, l'instruction {\tt summary(reg)} retourne un tableau de valeurs dont la premi\`ere  colonne donne l'estimateur
 $$\widehat \theta\in\mathop{\textrm{argmin}}_{\theta\in\R^k}\|Y-X \theta\|^2$$
 et l'avant derni\`ere colonne donne les $t$-values
$$\hat t_{j}={\widehat\theta_{j}\over \sqrt{\hat \sigma^2[(X^TX)^{-1}]_{jj}}}\quad\textrm{o\`u}\quad \widehat \sigma^2={1\over n-k}\|Y-X\widehat\theta\|^2.$$
 La derni\`ere colonne donne les $p$-values $\hat p_{j}= \mathcal{T}_{n-k}(\hat t_{j})$ o\`u $\mathcal{T}_{n-k}(t)=\p(|T_{n-k}|>|t|)$  avec $T_{n-k}$ une variable de Student \`a $n-k$ degr\'es de libert\'e.



\subsection*{A) Inspection des r\'esidus}
\begin{enumerate}
\item Calculer avec la fonction {\tt lm} la r\'egression de O3 par rapport aux autres variables. Identifier $Y$ et $X$ dans ce cas.  Que vaut $n$? Que vaut $k$?
\item On note $\hat\xi=Y-\hat Y$ o\`u $\hat Y=X\widehat \theta$. Tracer l'histogramme des $\{\hat \xi_{i}:i=1,\ldots,n\}$ \`a l'aide de la fonction ${\tt hist}$.
\item L'histogramme sugg\`ere que les r\'esidus pourraient suivre une loi Gaussienne. On va inspecter cette hypoth\`ese en regardant les quantiles de la loi empirique. On note $x_{q}(Q)=\min\{x : {Q}(]-\infty,x])\geq q\}$ le quantile d'ordre $q$ d'une loi $Q$. Tracer le QQplot
\begin{verbatim}
qqnorm(lm(O3~.,data=ozone)$residuals)
\end{verbatim}
 Que repr\'esente ce graphique?
 \item Pour savoir si la variance d\'epend du signal tracer les points $\big\{(\hat Y_{i},|\hat\xi_{i}|):i=1,\ldots,n\big\}$ \`a l'aide de la fonction {\tt plot}. Effectuer la r\'egression des $|\hat\xi_{i}|$ en fonction des $\hat Y_{i}$.
\end{enumerate}


\subsection*{B) Choix des variables}
\begin{enumerate}
\item Quelles variables $j$ ont une $p$-value $\hat p_{j}$ inf\'erieure \`a 5\%?
\item Calculer la r\'egression de O3 par rapport \`a Ne12+O3v et inspecter les r\'esidus comme pr\'ec\'edemment.
\item Calculer la r\'egression de O3 par rapport \`a Ne12+O3v+T15+Vx. Que constatez-vous au niveau des  $p$-values $\hat p_{j}$?
\end{enumerate}



\subsection*{C) R\'egressions partielles}
Dor\'enavant on ne travaille qu'avec les variables Ne12, O3v, T15 et Vx. On veut inspecter les questions suivantes:
\begin{itemize}
\item le mod\`ele lin\'eaire par rapport \`a la variable $j$ est-il raisonnable?
\item quelle est l'influence de la variable $j$?
\end{itemize}
\begin{enumerate}
\item Montrer que si le mod\`ele $Y=\sum_{k}\theta_{k}X_{k}+\xi$ est vrai, alors:

lm($Y\sim -X_{j}$)\$residuals = $\theta_{j}\times$ lm($X_{j} \sim -X_{j}$)\$residuals + lm($\xi \sim -X_{j}$)\$residuals

o\`u lm($Z\sim -X_{j}$) repr\'esente la r\'egression de $Z$ par rapport \`a toutes les variables $X_{1},\ldots,X_{k}$ sauf $X_{j}$.
\item Si les $\xi_{1},\ldots,\xi_{n}$ sont i.i.d.\ de loi $\mathcal{N}(0,\sigma^2)$,
quelle est la loi du vecteur lm($\xi \sim -X_{j}$)\$residuals?
\item Calculer la r\'egression de lm($Y\sim -X_{j}$)\$residuals par lm($X_{j} \sim -X_{j}$)\$residuals pour $j=$Ne12. Le mod\`ele lin\'eaire semble-t-il raisonnable pour cette variable?
\item M\^eme question avec la variable $j=$T15.
 \end{enumerate}





\end{document}

