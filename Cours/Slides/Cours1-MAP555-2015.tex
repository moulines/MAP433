\documentclass{beamer}
\usepackage{graphicx}
\usepackage{amsmath,amssymb,amstext,amsthm,xargs}
\usepackage{amsfonts}
\usepackage{bbm}
\usepackage{beamerthemesplit}

\usepackage[utf8]{inputenc}
\usepackage[french]{babel}
\usepackage{bbm}

\usetheme{Antibes}
\mode<presentation>
\useoutertheme{tree}
\usecolortheme{beaver}
\useinnertheme{rectangles}

\setbeamerfont{block title}{size={}}
%\usecolortheme[rgb={0.55,0.1,0.05}]{structure}
%\usecolortheme[rgb={0.75,0.1,0.05}]{structure}
\usepackage{color}

\newenvironment{disarray}{\everymath{\displaystyle\everymath{}}\array} {\endarray}
\newtheorem{theo}{Théorème}
\newtheorem{prop}[theo]{Proposition}
\newtheorem{conj}[theo]{Conjecture}
\newtheorem{cor}{Corollary}[theo]

\newtheorem{lem}{Lemme}
\newtheorem{nota}{Notation}
\newtheorem{rk}{Remark}
\newtheorem{exa}{Example}
\newtheorem{df}{Definition}
\newtheorem{terminologie}{Terminologie}
\def\rme{\mathrm{e}}
\def\rmi{\mathrm{i}}
\def\rset{\mathbb{R}}
\def\nset{\mathbb{N}}
\def\dlim{\stackrel{d}{\rightarrow}}
\newcommandx{\plim}[1][1=]{\stackrel{\PP_{#1}}{\longrightarrow}}
\def\iid{i.i.d.}
\def\1{\mathbbm{1}}
\newenvironment{dem}{\textbf{Proof}}{\flushright$\blacksquare$\\}
%\def\blankframe{
%\mode<presentation>{
%  { \setbeamertemplate{background canvas}[default]
%    \setbeamercolor{background canvas}{bg=black}
%    \begin{frame}[plain]{}
%    \end{frame}
%  }
%}
%\mode<presentation>{
%\setbeamertemplate{background canvas}[default]
%\setbeamercolor{background canvas}{bg=white}}
%\mode*
%}
\def\eqsp{\,}
\DeclareMathOperator{\E}{{\mathbb E}}
\def\PE{\E}
\def\PCov{\mathrm{Cov}}
\DeclareMathOperator{\F}{{\mathbb F}}
\DeclareMathOperator{\G}{{\mathbb G}}
\DeclareMathOperator{\D}{{\mathbb D}}
\DeclareMathOperator{\R}{{\mathbb R}}
\DeclareMathOperator{\C}{{\mathbb C}}
\DeclareMathOperator{\Z}{{\mathbb Z}}
\DeclareMathOperator{\N}{{\mathbb N}}
\DeclareMathOperator{\K}{{\mathbb K}}
\DeclareMathOperator{\T}{{\mathbb T}}
\DeclareMathOperator{\PP}{{\mathbb P}}
\DeclareMathOperator{\QQ}{{\mathbb Q}}
\DeclareMathOperator{\Q}{{\mathbb Q}}
\DeclareMathOperator{\IF}{{\mathbb I}}


%%%%%%%%%%%%%%%%%%%%%%%%%%%%%%% Pour le modèle lin\'eaire

\DeclareMathOperator{\bX}{\boldsymbol{X}}
\DeclareMathOperator{\bY}{\boldsymbol{Y}}
\DeclareMathOperator{\bx}{\boldsymbol{x}}
\DeclareMathOperator{\vp}{\boldsymbol{p}}
\DeclareMathOperator{\vq}{\boldsymbol{q}}
\DeclareMathOperator{\estMCNL}{\widehat \theta_n^{\,\,{\tt mcnl}}}
\DeclareMathOperator{\estMV}{\widehat \theta_n^{\,\,{\tt mv}}}
\DeclareMathOperator{\est}{\widehat \theta_{\mathnormal{n}}}
\DeclareMathOperator{\var}{\mathrm{Var}}
\def\Var{\var}
\DeclareMathOperator{\estMVc}{\widehat \theta_{n,0}^{\,{\tt mv}}}
\DeclareMathOperator{\Xbar}{\overline{\mathnormal{X}}_\mathnormal{n}}

\newcommand{\indi}[1]{\mathbbm{1}_{\{#1\}}}
\newcommand{\coint}[1]{\left[#1\right)}
\newcommand{\ocint}[1]{\left(#1\right]}
\newcommand{\ooint}[1]{\left(#1\right)}
\newcommand{\ccint}[1]{\left[#1\right]}

\definecolor{LightYell}{rgb}{0.95,0.83,0.70}
\definecolor{orange}{rgb}{1.0,0.50,0.01}
\definecolor{StroYell}{rgb}{0.95,0.88,0.72}
\definecolor{lightred}{rgb}{0.75,0.033,0}
\definecolor{shadecolor1}{rgb}{0.90,0.83,0.70}
\definecolor{myem}{rgb}{0.797,0.598,0.598}
\definecolor{BrickRed}{cmyk}{0,0.89,0.94,0.28}
\definecolor{RoyalPurple}{cmyk}{0.75,0.9,0,0}

\newcommand{\tco}[1]{\textcolor{orange}{#1}}
\newcommand{\tcr}[1]{\textcolor{lightred}{#1}}

\def\gauss{\mathcal{N}}
\def\truetheta{\theta}
\def\truebeta{\boldsymbol{\beta}}
\def\projX{A}
\def\curtheta{\alpha}
\def\argmin{\mathrm{argmin}}
\def\ie{\emph{i.e.}}
\def\regressmat{\mathbb{X}}
\def\errpred{\boldsymbol{\hat{\xi}}}
\def\bnoise{\boldsymbol{\xi}}
\def\predY{\hat{\mathbf{Y}}}
\DeclareMathOperator{\estregress}{\widehat{\truebeta}_n}
\DeclareMathOperator{\estMC}{\widehat \theta_n^{\,\,{\tt mc}}}
\def\curbeta{b}
\def\bcurbeta{\mathbf{b}}
\newcommand{\indep}{\rotatebox[origin=c]{90}{$\models$}} 
\newcommand{\Id}[1]{\mathrm{Id}_{#1}}
\def\1{\mathbbm{1}}
\def\mcb{\ensuremath{\mathcal{B}}}
\def\mcc{\ensuremath{\mathcal{C}}}
\def\mce{\ensuremath{\mathcal{E}}}
\def\mcf{\ensuremath{\mathcal{F}}}
\def\nset{\ensuremath{\mathbb{N}}}
\def\qset{\ensuremath{\mathbb{Q}}}
\def\rset{\ensuremath{\mathbb{R}}}
\def\zset{\ensuremath{\mathbb{R}}}
\def\cset{\ensuremath{\mathbb{C}}}
\def\rsetc{\ensuremath{\overline{\rset}}}
\def\Xset{\ensuremath{\mathsf{X}}}
\def\Tset{\ensuremath{\mathsf{T}}}
\def\Yset{\ensuremath{\mathsf{Y}}}
\def\rmd{\mathrm{d}}
\def\Qint{\ensuremath{\mathrm{QInt}}}
\def\Int{\ensuremath{\mathrm{Int}}}
\def\eqdef{\ensuremath{\stackrel{\mathrm{def}}{=}}}
\def\eqsp{\;}
\def\lleb{\lambda^{\mathrm{Leb}}}
\newcommand{\rmi}{\mathrm{i}}
\newcommand{\rme}{\mathrm{e}}
\def\supp{\mathrm{supp}}



%notation fourier
\def\1{\mathbbm{1}}
\def\mcb{\ensuremath{\mathcal{B}}}
\def\mcc{\ensuremath{\mathcal{C}}}
\def\mce{\ensuremath{\mathcal{E}}}
\def\mcf{\ensuremath{\mathcal{F}}}
\def\nset{\ensuremath{\mathbb{N}}}
\def\qset{\ensuremath{\mathbb{Q}}}
\def\rset{\ensuremath{\mathbb{R}}}
\def\zset{\ensuremath{\mathbb{R}}}
\def\cset{\ensuremath{\mathbb{C}}}
\def\rsetc{\ensuremath{\overline{\rset}}}
\def\Xset{\ensuremath{\mathsf{X}}}
\def\Tset{\ensuremath{\mathsf{T}}}
\def\Yset{\ensuremath{\mathsf{Y}}}
\def\rmd{\mathrm{d}}
\def\Qint{\ensuremath{\mathrm{QInt}}}
\def\Int{\ensuremath{\mathrm{Int}}}
\def\eqdef{\ensuremath{\stackrel{\mathrm{def}}{=}}}
\def\eqsp{\;}
\def\lleb{\lambda^{\mathrm{Leb}}}
\newcommand{\coint}[1]{\left[#1\right[}
\newcommand{\ocint}[1]{\left]#1\right]}
\newcommand{\ooint}[1]{\left]#1\right[}
\newcommand{\ccint}[1]{\left[#1\right]}


\newcommand{\TF}{\mathcal{F}}
\newcommand{\TFC}{\overline{\mathcal{F}}}
\newcommand{\TFA}[1]{\mathcal{F}\left( #1 \right)}
\newcommand{\TFAC}[1]{\overline{\mathcal{F}}\left( #1 \right)}

\def\TFyield{\stackrel{\mathcal{F}}{\mapsto}}

\def\tore{\mathbb{T}}
\def\btore{\mathcal{B}(\tore)}
\def\espaceproba{(\Omega,\mathcal{A},\PP)}
\def\limn{\lim_{n \rightarrow \infty}}
\newcommand{\ps}{\ensuremath{\text{p.s.}}}
\newcommand{\pp}{\ensuremath{\text{p.p.}}}
\def\cA{\mathcal{A}}
\def\cC{\mathcal{C}}
\def\cL{\mathcal{L}}
\def\cM{\mathcal{M}}
\def\cN{\mathcal{N}}
\def\cO{\mathcal{O}}
\def\cP{\mathcal{P}}
\def\cS{\mathcal{S}}
\newcommand{\filtop}[1]{\operatorname{F}_{#1}}
\def\bfphi{{\boldsymbol{\phi}}}
\def\bfpsi{{\boldsymbol{\psi}}}
\def\bfgamma{{\boldsymbol{\gamma}}}
\def\bfpi{{\boldsymbol{\pi}}}
\def\bfsigma{{\boldsymbol{\sigma}}}
\def\bftheta{{\boldsymbol{\theta}}}
\def\bfhphi{{\hat{\boldsymbol{\phi}}}}
\def\bfhrho{{\hat{\boldsymbol{\rho}}}}
\def\bfhgamma{{\hat{\boldsymbol{\gamma}}}}

\def\ltwo{L_2}
\newcommand{\lone}{\ensuremath{L_1}}

\newcommand{\pltwo}{\ensuremath{\ell^2}(\zset)}
\newcommand{\plone}{\ensuremath{\ell^1}(\zset)}
\def\calG{\mathcal{G}}
\def\calM{\mathcal{M}}
\def\calI{\mathcal{I}}
\def\calH{\mathcal{H}}


\newcommand\BL[1]{\mathrm{BL}(#1)}%bande limit{\'e}e
%Espace de Schwarz
\def\mcs{\ensuremath{\mathcal{S}}}
%produit scalaire
\newcommand{\pscal}[2]{\left\langle #1, #2 \right\rangle}
\newcommand{\proj}[3][]{
\ifthenelse{\equal{#1}{}}{\ensuremath{\operatorname{proj}\left( \left. #2\right|#3\right)}}
{\ensuremath{\operatorname{proj}_{#1}\left( \left. #2 \right|#3\right)}}
}
%espaces engendr{\'e}s
\newcommand{\lspan}{\mathrm{Vect}}
\newcommand{\cspan}{\overline{\mathrm{Vect}}}
\def\oplusperp{\stackrel{\perp}{\oplus}}
\def\ominusperp{\ominus}%\def\ominusperp{\stackrel{\perp}{\ominus}}


%Operation sur les fonctions/distributions

\newcommand{\translation}{\mathcal{T}}
\newcommand{\multiplication}{\mathcal{M}}


%
\def\Rset{\mathbb{R}}
\def\Cset{\mathbb{C}}
\def\Zset{\mathbb{Z}}
\def\Nset{\mathbb{N}}
\def\Tset{\mathrm{T}}
% et d'autres
\newcommand{\vvec}[1]{\mathbf{#1}}
\newcommand{\signe}{\mathrm{sgn}}
\newcommand{\rect}{\mathrm{rect}}
\newcommand{\sinc}{\mathrm{sinc}}
\newcommand{\cov}{\mathrm{cov}}
\newcommand{\corr}{\mathrm{corr}}
\newcommand{\vp}{\mathrm{vp}}
\newcommand{\erf}{\mathrm{erf}}
\def\mod{{\ \rm mod\ }}
\def\cF{\mathcal{F}}
\def\cE{\mathcal{E}}
\def\cB{\mathcal{B}}
\def\cH{\mathcal{H}}
\def\cG{\mathcal{G}}
\def\cI{\mathcal{I}}
\def\PP{\mathbb{P}}
\newcommand\PE[1]{{\mathbb E}\left[ #1 \right]}
\newcommand{\Var}[1]{\mathrm{Var}\left( #1 \right)}
\def\BB{\mathrm{B.B.}}
\def\BBF{\mathrm{B.B.F.}}
\newcommandx{\norm}[2][2=]{\Vert #1 \Vert_{#2}}
\def\L1loc{L_{1,\mathrm{loc}}}
\def\Leb{\mathrm{Leb}}


\newcommandx\sequence[3][2=,3=]
{\ifthenelse{\equal{#3}{}}{\ensuremath{\{ #1_{#2}\}}}{\ensuremath{\{ #1_{#2}, \eqsp #2 \in #3 \}}}}
\newcommandx\sequencePar[3][2=,3=]
{\ifthenelse{\equal{#3}{}}{\ensuremath{\{ #1({#2})\}}}{\ensuremath{\{ #1({#2}), \eqsp #2 \in #3 \}}}}
\def\pp{\ensuremath{\mathrm{p.p.}}}
\def\ie{i.e.} 

\newcommand{\ensemble}[2]{\left\{#1\,:\eqsp #2\right\}}
\newcommand{\set}[2]{\ensemble{#1}{#2}}

\title{MAP 455 : Fourier transform}
\begin{document}
\date{11 Septembre 2015}
\maketitle



\begin{frame}
\frametitle{Today}
\tableofcontents
\end{frame}
\section{Main definitions and properties}
\begin{frame}
\frametitle{Definition}
\begin{definition}
Given $f\in \lone(\rset)$ we write
\begin{align*}
&\TF f(\xi)=\hat{f}(\xi)=\int_{\rset} \rme^{-2\rmi\pi\xi x}f(x)\ \rmd x, \\
&\TFC f(\xi)=\int_{\mathbb{R}}\rme^{2\rmi\pi\xi x}f(x) \rmd x. \\
\end{align*}
By definition, the function $\TF f$ is the Fourier transform of $f$, and $\TFC f$ is the conjugate Fourier transform of $f$.
\end{definition}
\end{frame} 

\begin{frame}
\frametitle{Rieman-Lebesgue theorem}
\begin{theorem}[Riemann-Lebesgue]  If $f\in \lone(\rset)$ , then $\hat{f}$ satisfies the following conditions:
\begin{enumerate}[label=(\roman*)]
\item $\TF f$ is continuous  and bounded on $\rset$.
\item $\TF$  is a  continuous linear operator from $\lone(\rset)$  to $L^{\infty}(\rset)$, and
$$
\Vert\hat{f}\Vert_{\infty}\leq\Vert f\Vert_{1}.
$$
\item $\lim_{|\xi|\rightarrow+\infty} |\hat{f}(\xi)|=0$. 
\end{enumerate}
\end{theorem}
\end{frame}

\begin{frame}
\frametitle{The exchange formula}
\begin{theorem}
 Let $f$ and $g$  be two functions in $\lone(\rset)$ . Then $f\hat{g}$
and $\hat{f}g$ both belong to $\lone(\rset)$ and
$$
\int f(t)\hat{g}(t)\rmd t=\int\hat{f}(x)g(x)\ \rmd x.
$$
\end{theorem}
\end{frame}

\begin{frame}
\frametitle{Derivation}

\begin{theorem}
\begin{enumerate}[label=(\roman*)]
\item If $x^{k}f(x)$ is in $\lone(\rset)$ , $k=0,1,2,\ \ldots,\ n$,  then $\hat{f}$ is n   times differentiable, and
$$
\hat{f}^{(k)}(\xi)=\widehat{(-2\rmi\pi x)^{k}f}(\xi) \quad  \text{for} \ k=1,2,\ \ldots,\ n,
$$
where $\widehat{(-2 \rmi \pi x)^{k}f}(\xi)$  denotes $\TFA{x \to (-2\rmi\pi x)^{k}f(x)}(\xi)$ .
\item If $f\in C^{n}(\rset)\cap \lone(\rset)$ and if all the derivatives $f^{(k)},\ k=1,2,\ \ldots,\ n$, are  in $\lone(\rset)$, then
$$
\hat{f^{(k)}}(\xi)=(2\rmi \pi\xi)^{k}\hat{f}(\xi)\ \quad \text{for}\ k=1,2,\ \ldots,\ n.
$$
\item If $f\in \lone(\rset)$  has bounded support, then $\hat{f}\in C^{\infty}(\rset)$.
\end{enumerate}
\end{theorem}
\end{frame}

\begin{frame}
\frametitle{Translations, symmetries}
\begin{enumerate}[label=(\roman*)]
\item If $f$ has values in $\cset$, then $\overline{f}(x)=\overline{f(x)}$, the complex conjugate of $f(x)$ . 
\item $f_{\sigma}$ denotes the reflection of $f$ defined by $f_{\sigma}(x)=f(-x)$ .
\item The translate $\tau_{a}f$ of $f$ is defined by $\tau_{a}f(x)=f(x-a)$ .
\end{enumerate}
\begin{theorem}[conjugation and parity]
For $f\in \lone(\rset)$ we have the following relations:
\begin{enumerate}[label=(\roman*)]
\item $\overline{\TFA{f}}=\overline{\TF}(\overline{f})$ .
\item $(\TFA f)_{\sigma}=\overline{\TF}(f)=\TF (f_{\sigma})$ .
\item $f$ even (odd) $\Rightarrow$  $\hat{f}$ even (odd).
\item $f$ real and even (real and odd) $\Rightarrow$ $\hat{f}$ real  and even (imaginary and odd).
\end{enumerate}
\end{theorem}
\end{frame}

\begin{frame}
\frametitle{Fourier transform of the Gaussian kernel}
\begin{lemma}\label{lem:gaussenneTF}
La fonction  $g_1(x)= 1 / \sqrt{2\pi} \exp( -x^2/2)$ est la densit{\'e} d'une probabilit{\'e} sur $\rset$, et sa transform{\'e}e de Fourier
est $\hat{g_1}(\xi)= \rme^{- 2 \pi^2 \xi^2}$.
\end{lemma}
\end{frame}

\begin{frame}
\frametitle{Summary of useful formulas}
\section{Formulaire}
\label{sec:formulaire}
\begin{enumerate}[label=(\roman*)]
\item $\hat{f}^{(k)}(\xi)= \widehat{(-2\rmi\pi x)^{k}f}(\xi)$
$\widehat{f^{(k)}}(\xi)=(2\rmi \pi\xi)^{k}\hat{f}(\xi)$
\item $f(x-a) \TFyield \rme^{-2\rmi \pi a\xi}\hat{f}(\xi)$,
$\rme^{2 \rmi \pi ax}f(x)\TFyield \hat{f}(\xi-a)$
\item $a\neq 0$, $ f(ax)\TFyield \frac{1}{|\xi|}\hat{f}(\frac{\xi}{a})$
\item  $a\in \cset$, ${\rm Re}(a)>0, k=0,1,2\ldots.$
\begin{align*}
\frac{x^{k}}{k!}\rme^{-ax}u(x) &\TFyield \frac{1}{(a+2i\pi\xi)^{k+1}} \\
\frac{x^{k}}{k!}\rme^{ax}u(-x) &\TFyield \frac{-1}{(-a+2i\pi\xi)^{k+1}} \\
\rme^{-a|x|} &\TFyield \frac{2a}{a^{2}+4\pi^{2}\xi^{2}} \\
\mathrm{sign}(x)\rme^{-a|x|} &\TFyield \frac{-4i\pi\xi}{a^{2}+4\pi^{2}\xi^{2}} \\
\end{align*}
\item  $a\in \rset$, $a>0$.
\begin{align*}
\rme^{-ax^{2}}  &\TFyield \sqrt{\frac{\pi}{a}}\rme^{-\frac{\pi^{2}}{a}\xi^{2}} \\
\1_{\ccint{-a,+a}}(x) &\TFyield \frac{\sin 2a\pi\xi}{\pi\xi}
\end{align*}
\end{enumerate}

\end{frame}

\section{The space $\mcs(\rset)$}
\begin{frame}
\begin{definition}[Rapidly decreasing function]
A function  $f : \rset \to \cset$ is \alert{rapidly decreasing} if, for all $p \in \nset$,
$$
\lim_{|x| \to \infty} |x|^p |f (x)| = 0 \eqsp.
$$
\end{definition}
\begin{theorem}
\label{prop:1913}
\begin{enumerate}[label=(\roman*)]
\item Let  $f \in \lone(\rset)$ be a rapidly decreasing function. Then $\hat{f}$ is indefinitely differentiable.
\item Let $f \in C_\infty(\rset)$. If for all $k \in \nset$, $f^{(k)} \in \lone(\rset)$ then $\hat{f}$ is rapidly decreasing.
\end{enumerate}
\end{theorem}
\end{frame}

\begin{frame}
\index{The Schwartz space $\mcs(\rset)$}
\begin{definition}[Schwartz space] $\mcs(\rset)$ or simply $\mcs$ denotes the vector space of functions that have the following two properties
\begin{enumerate}[label=(\roman*)]
\item $f$ is infinitely differentiable on $\rset$;
\item $f$ and all of its derivatives decay rapidly.
\end{enumerate}
\end{definition}
We call $\mcs(\rset)$ the \alert{Schwartz space} (named after the french mathematician Laurent Schwartz).
\end{frame}

\begin{frame}
\frametitle{Properties of the Schwartz space}
\begin{lemma}
The space $\mcs$ has the following properties :
\begin{enumerate}[label=(\roman*)]
\item $\mcs$ is invariant under multiplication by a polynomial..
\item $\mcs$  is invariant under derivation; that is, ($f \in \mcs \Rightarrow f' \in  \mcs$).
\item $\mcs \subset  \lone(\rset)$.
\end{enumerate}
\end{lemma}
\end{frame}



\begin{frame}
\frametitle{Invariance}
\begin{theorem}
The space $\mcs$ is invariant by Fourier transform; that is, if $f \in \mcs$, then $\hat{f} \in \mcs$.
\end{theorem}
\end{frame}

\section{The inverse Fourier transform}
\begin{frame}
\frametitle{A key result}
\begin{theorem}\label{prop:inversionFourierL1}
Let $f\in \lone(\rset)$. Assume that $\hat{f} \in \lone(\rset)$. Then, for any $x \in \rset$ where $f$ is continuous, we get
\begin{equation*}
[\bar{\TF} \hat{f}](x) = f(x).
\end{equation*}
\end{theorem}
\end{frame}

\begin{frame}
\frametitle{The inverse Fourier transform on $\mcs$}
\begin{theorem}\label{thm:Schwartz}
The Fourier transform $\TF$ is a linear $1$-to-$1$ mapping from $\mcs$ to $\mcs$.
The inverse transform is $\TF^{-1} = \TFC$. In other words, the relations
\begin{align*}
\hat{f}(\xi) &= \int_{\rset} \rme^{-2 \rmi \pi \xi x} f(x) \rmd x \eqsp, \\
f(x) &= \int_{\rset} \rme^{+ 2 \rmi \pi x \xi} \hat{f}(\xi) \rmd \xi \eqsp. 
\end{align*}
\end{theorem}
The function $g(x) = \rme^{- \pi x^2}$ is a fixed point of this mapping.
\end{frame}

\end{document}