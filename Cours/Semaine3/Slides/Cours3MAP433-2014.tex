\documentclass{beamer}
\usetheme[right,hideothersubsections]{Berkeley}
\usepackage{graphicx}
\usepackage{amsmath,amssymb,amstext}
\usepackage{amsfonts}

\usepackage[utf8]{inputenc}
\usepackage[french]{babel}


\newenvironment{disarray}{\everymath{\displaystyle\everymath{}}\array} {\endarray}
\newtheorem{theo}{Théorème}
\newtheorem{prop}[theo]{Proposition}
\newtheorem{conj}[theo]{Conjecture}
\newtheorem{cor}{Corollary}[theo]
\newtheorem{lm}{Lemma}
\newtheorem{nota}{Notation}
\newtheorem{rk}{Remark}
\newtheorem{exa}{Example}
\newtheorem{df}{Définition}
\newtheorem{terminologie}{Terminologie}
\newenvironment{dem}{\textbf{Proof}}{\flushright$\blacksquare$\\}

\DeclareMathOperator{\E}{{\mathbb E}}
\DeclareMathOperator{\F}{{\mathbb F}}
\DeclareMathOperator{\G}{{\mathbb G}}
\DeclareMathOperator{\D}{{\mathbb D}}
\DeclareMathOperator{\R}{{\mathbb R}}
\DeclareMathOperator{\C}{{\mathbb C}}
\DeclareMathOperator{\Z}{{\mathbb Z}}
\DeclareMathOperator{\N}{{\mathbb N}}
\DeclareMathOperator{\K}{{\mathbb K}}
\DeclareMathOperator{\T}{{\mathbb T}}
\DeclareMathOperator{\PP}{{\mathbb P}}
\DeclareMathOperator{\QQ}{{\mathbb Q}}
\DeclareMathOperator{\Q}{{\mathbb Q}}
\DeclareMathOperator{\IF}{{\mathbb I}}


%%%%%%%%%%%%%%%%%%%%%%%%%%%%%%% Pour le modèle linéaire

\DeclareMathOperator{\bX}{\boldsymbol{X}}
\DeclareMathOperator{\bY}{\boldsymbol{Y}}
\DeclareMathOperator{\bx}{\boldsymbol{x}}
\DeclareMathOperator{\vp}{\boldsymbol{p}}
\DeclareMathOperator{\vq}{\boldsymbol{q}}
\DeclareMathOperator{\estMC}{\widehat \vartheta_n^{\,\,{\tt mc}}}
\DeclareMathOperator{\estMCNL}{\widehat \vartheta_n^{\,\,{\tt mcnl}}}
\DeclareMathOperator{\estMV}{\widehat \vartheta_n^{\,\,{\tt mv}}}
\DeclareMathOperator{\design}{\mathbb{M}}
\DeclareMathOperator{\est}{\widehat \vartheta_{\mathnormal{n}}}
\DeclareMathOperator{\var}{\mathrm{Var}}
\DeclareMathOperator{\estMVc}{\widehat \vartheta_{n,0}^{\,{\tt mv}}}
\DeclareMathOperator{\Xbar}{\overline{\mathnormal{X}}_\mathnormal{n}}




\title{MAP 433 : Introduction aux méthodes statistiques. Cours 3}
%\author{M. Hoffmann}
%\institute{Université Paris-Est and ETG}
\begin{document}
\date{14 février 2014}
\maketitle



\begin{frame}
\frametitle{Aujourd'hui}
\tableofcontents
\end{frame}
\section{Rappel succint du Cours 2}

\begin{frame}
\frametitle{Estimation de fonctionnelles régulières}
\begin{itemize}
\item $X_1,\ldots, X_n \sim_{\text{i.i.d.}} F$.
%\item \underline{Principe} : si $F \leadsto T(F)$ est \og régulière \fg{}, alors $T(\widehat F_n)$ est un \og bon \fg{} estimateur de $T(F)$ (estimateur par plug-in).
\item Estimation de $T(F) = h\big(\int_{\R}g(x)dF(x)\big)$.
%\item $\int_{\R} \varphi(x)dF(x) = \int_{\R}\varphi(x) \PP^X(dx)$, où $X \sim F$.
%\item \underline{Formule de calcul} :
%$$\boxed{\int_{\R}g(x)d\widehat F_n(x) =
%\frac{1}{n}\sum_{i = 1}^n g(X_i).}$$
%Traduction : {\color{red}une
%variable aléatoire de loi $\widehat F_n$ prend les valeurs $X_i$
%avec probabilité $1/n$.}
\item Estimateur par {\color{red} substitution} ou {\it plug-in} de $T(F)$ :
$$\boxed{T(\widehat F_n) = h\Big(\tfrac{1}{n}\sum_{i = 1}^ng(X_i)\Big)}$$
\end{itemize}
\end{frame}


%\begin{frame}
%\frametitle{Performance de l'estimateur par substitution}
%\begin{itemize}
%\item {\color{red}Convergence} si $g,h:\R\rightarrow \R$, $h$
%continue et $\E|g(X)|<\infty$, alors $T(\widehat
%F_n)\stackrel{\mathrm{p.s.}}{\rightarrow} T(F)$ (loi forte des
%grands nombres).
%\item {\color{red} Vitesse de convergence, Etape 1.}
%TCL :
%$$\sqrt{n}\big(\tfrac{1}{n}\sum_{i=1}^ng(X_i)-\int_{\R}g(x)dF(x)\big) \stackrel{d}{\rightarrow} {\mathcal N}\big(0,\mathrm{Var}\big[g(X)\big]\big),$$
%où $X$ est une v.a. de loi $F$ et
%\begin{align*}
%\text{Var}\big[g(X)\big] &= \E\big[g(X)^2\big]-\big(\E[g(X)]\big)^2 \\
%&= \int_{\R}g(x)^2dF(x)-\big(\int_{\R}g(x)dF(x)\big)^2.
%\end{align*}
%\end{itemize}
%\end{frame}
%
%\begin{frame}
%\frametitle{Vitesse de convergence (suite)}
%\begin{itemize}
%\item{\color{red} Etape 2.} On a $\sqrt{n}(Z_n -c_1)
%\stackrel{d}{\rightarrow} {\mathcal N}(0,c_2)$. Comment transférer
%ce résultat à $\sqrt{n}(h(Z_n) - h(c_1))\stackrel{d}{\rightarrow}$ ?
%\item {\color{red} Méthode \og delta\fg{}} : si $h$ continément différentiable
%$$\sqrt{n}\big(h(Z_n)-h(c_1)\big) = \sqrt{n}(Z_n-c_1)h'(\eta_n),\;\;\eta_n \in \big[Z_n,c_1\big].$$
%On a  $\sqrt{n}(Z_n-c_1)\stackrel{d}{\rightarrow} {\mathcal N}(0,c_2)$ et $h'(\eta_n)\stackrel{\PP}{\rightarrow}h'(c_1)$.
%%car $Z_n\stackrel{\PP}{\rightarrow} c_1$.
%
%{\color{red}Lemme de Slutsky} :
%$$ \sqrt{n}(Z_n-c_1)h'(\eta_n) \stackrel{d}{\rightarrow} {\mathcal N}(0,c_2) h'(c_1).$$
%% \stackrel{d}{=} {\mathcal N}\big(0,h'(c_1)^2c_2\big).$$
%Finalement
%$$\boxed{\sqrt{n}\big(h(Z_n)-h(c_1)\big) \stackrel{d}{\rightarrow}
%{\mathcal N}\big(0,c_2[h'(c_1)]^2\big)}$$
%\end{itemize}
%\end{frame}
%
%\begin{frame}
%\frametitle{Conclusion}
%\begin{prop}
%Si $\E[g(X)^2]<+\infty$ et $h$ continément différentiable, alors
%$$\sqrt{n}\big(T(\widehat F_n)-T(F)\big)\stackrel{d}{\rightarrow} {\mathcal N}\big(0,v({\color{red}F})\big),
%$$
%où $v({\color{red}F}) = h'\big(\E\big[g(X)\big]\big)^2\mathrm{Var}\big[g(X)\big]$.
%\end{prop}
%Pour construire un {\color{red} intervalle de confiance}, il faut encore remplacer $v({\color{red}F})$ par $v(\widehat F_n)$.
%{\color{red}On montre que} $v(\widehat F_n)\stackrel{\PP}{\rightarrow} v(F)$ et, via le lemme de Slutsky,
%$$
%\sqrt{n}\frac{T(\widehat F_n)-T(F)}{v(\widehat F_n)^{1/2}}\stackrel{d}{\rightarrow} {\mathcal N}\big(0,1\big).
%$$
%On {\color{red}en déduit} un intervalle de confiance asymptotique comme précédemment.
%\end{frame}
%\begin{frame}
%\frametitle{Le cas de la dimension $d>1$}
%\begin{itemize}
%\item Il s'agit de fonctionnelles de la forme
%$$T(F) = h\left(\int_{\R}g_1(x)dF(x),\ldots, \int_{\R}g_k(x)dF(x)\right)$$
%où $h:\R^k\rightarrow \R$ continément différentiable.
%\item {\color{red} Exemple} : le coefficient d'asymétrie
%$$T(F) = \frac{\int_{\R}\big(x-m(F)\big)^3dF(x)}{\sigma^{3/2}(F)},$$
%$m(F)=$ moyenne de $F$, $\sigma^2(F) =$ variance de $F$.
%\item {\color{red} Outil :} Version multidimensionnelle du TCL et de la \og méthode delta \fg{}.
%\end{itemize}
%\end{frame}
%
%\begin{frame}
%\frametitle{Méthode \og delta\fg{} multidimensionnelle}
%\begin{itemize}
%\item {\color{red}TCL multidimensionnel :}
%$(\bX_n)_{n\ge 1}$ vecteurs aléatoires dans $\R^k$, i.i.d., de
%moyenne ${\boldsymbol \mu} = \E[\bX_1]$ et de matrice de
%variance-covariance $\Sigma =
%\E\big[(\bX_1-\boldsymbol{\mu})(\bX_1-\boldsymbol{\mu})^T\big]$ bien
%définie. Alors $\bar \bX_n=\tfrac{1}{n}\sum_{i = 1}^n\bX_i$
%v\'erifie:
%$$\sqrt{n}\big(\overline{\bX}_n-\boldsymbol{\mu}\big) \stackrel{d}{\rightarrow} {\mathcal N}\big(0,\Sigma\big).$$
%%\end{itemize}
%%\end{frame}
%%\begin{frame}
%%\begin{itemize}
%\item {\color{red} Méthode \og delta\fg{} multidimensionnelle :} Si, de plus, $h : \R^k \rightarrow \R$ continément différentiable, alors
%$$\sqrt{n}\big(h(\overline{\bX}_n)-h(\boldsymbol{\mu})\big) \stackrel{d}{\rightarrow} {\mathcal N}\Big(0, \nabla h(\boldsymbol{\mu}) \Sigma \nabla h(\boldsymbol{\mu})^T\Big).$$
%\end{itemize}
%\end{frame}
%\begin{frame}
%\frametitle{Application : coefficient d'asymétrie}
%\begin{itemize}
%\item {\color{red} Coefficient d'asymétrie :} on a
%$$T(F) = h\Big(\int_{\R} xdF(x),\int_{\R} x^2 dF(x), \int_{\R} x^3 dF(x)\Big)$$ avec
%$$h(\alpha,\beta,\gamma) = \frac{\gamma-3\alpha \beta+2\alpha^3}{(\beta-\alpha^2)^{3/2}}.$$
%$$T(\widehat F_n) = h\Big(\tfrac{1}{n}\sum_{i = 1}^n X_i, \tfrac{1}{n}\sum_{i = 1}^n X_i^2,\tfrac{1}{n}\sum_{i = 1}^n X_i^3\Big).$$
%\item On applique le TCL multidimensionnel avec $\bX_i = (X_i,X_i^2,X_i^3)^T$ et $\boldsymbol{\mu} = \big(\int_{\R} xdF(x),\int_{\R} x^2 dF(x),\int_{\R} x^3 dF(x)\big)^T$, puis la méthode \og delta\fg{} avec $h$.
%\end{itemize}
%\end{frame}
\begin{frame}
\frametitle{Limites de l'approche empirique} L'estimation de $T(F)$
par $T(\widehat F_n)$ n'est pas toujours {\color{red}possible}:
%
%
\begin{itemize}
\item La fonctionnelle $F \leadsto T(F)$ n'est pas \og régulière\fg{},
\item La paramétrisation $F \leadsto T(F)$ ne donne {\color{red}pas} lieu à une {\color{red}forme analytique simple}.
%et ne s'étudie pas à force d'arguments standard basés sur la loi des grands nombres et le TCL
$\rightarrow$ autres approches.
\end{itemize}
%
\underline{Exemple}. {\color{red} Hypothèse} : $F$ admet une densité
$f$ par rapport à le mesure de Lebesgue, {\color{red}continue} ($=$
pp à une fonction continue $f$).
$$T(F) = f(x_0),\;\;x_0\in \R \,\;\text{(donné)}.$$
On ne {\color{red}peut pas prendre} comme estimateur $\widehat
F_n'(x_0)$ car $\widehat F_n$ n'est pas diff\'erentiable (constante
par morceaux...)

\end{frame}

\begin{frame}
\frametitle{Limites de l'approche empirique} L'estimation de $T(F)$
par $T(\widehat F_n)$ n'est pas toujours {\color{red}souhaitable} :
\begin{itemize}
\item
Souvent on dispose d'information {\color{red} a priori}
supplémentaire : $F$ appartient à une sous-classe {\color{red}
particulière} de distributions, et il y a des choix plus judicieux
que l'estimateur par plug-in.
\end{itemize}
\end{frame}

%\begin{frame}
%\frametitle{Exemple 2 : information supplémentaire}
%\end{frame}
%\begin{frame}
%\frametitle{Exemple 3 : paramétrisation non-standard}
%\end{frame}

\begin{frame}
\frametitle{Conclusion}
\begin{itemize}
\item {\color{red}L'approche empirique}, basée sur $\widehat F_n$ permet d'estimer une distribution inconnue $F$ ou une fonctionnelle $T(F)\in \R$ à partir d'un $n$-échantillon, mais
\begin{itemize}
\item reste très générale, pas toujours adaptée.
%(cas de fonctionnelles irrégulières ou difficiles à paramétriser en $F$)
\item restreinte à la situation d'un $n$-échantillon.
%\item reste très générale...
%(en particulier, n'incorpore pas d'information supplémentaire de modélisation)
\end{itemize}
\item Formalisation de la notion {\color{red}d'expérience statistique }
\begin{itemize}
\item incorporation d'information de modélisation {\color{red}supplémentaire}.
\item construction de méthodes d'estimation -- de décision -- {\color{red}systématiques}.
\item comparaison et {\color{red}optimalité} des méthodes.
\end{itemize}
\end{itemize}
\end{frame}

\section{Modélisation statistique}
\subsection{Expérience statistique}
\begin{frame}
\frametitle{Expérience statistique} Consiste à identifier:
\begin{itemize}
\item {\color{red}Des observations}
$${\tt x}_1,{\tt x}_2,\ldots, {\tt x}_n$$
{\color{red}considérées} comme des {\color{red} réalisations} de variables aléatoires $Z = (X_1,\ldots, X_n)$ de loi $\PP^Z$.
\item {\color{red}Une famille de lois}
$$\left\{\PP_\vartheta,\,\vartheta \in \Theta\right\}.$$
\item {\color{red}Une problématique} : retrouver le paramètre
$\vartheta$ tel que $\PP^Z=\PP_\vartheta$ (estimation) ou bien prendre une décision sur une propriété relative à $\vartheta$ (test).
\end{itemize}
\end{frame}

\begin{frame}
\frametitle{Expérience statistique}
\begin{itemize}
\item Approche g\'en\'erale empirique:
\begin{itemize}
\item $\vartheta=F$, $\Theta$ est l'ensemble de
toutes les lois (s'il s'agit de l'estimation de $F$);
\item $\vartheta=F$, $\Theta$ est l'ensemble de
toutes les lois v\'erifiant une hypoth\`ese tr\`es g\'en\'erale, par
exemple, la bornitude d'un moment (s'il s'agit de l'estimation de
$T(F)$).
\end{itemize}
\item Approche param\'etrique: {\color{red} on suppose} que $F$ appartient \`a une
{\color{red} famille de lois connue} index\'ee par un param\`etre
$\vartheta$ de dimension finie: $\vartheta\in \Theta \subset \R^d$.
\begin{itemize}
\item \underline{Exemple}: $\Theta = \R$,
$$ X_i= \vartheta +\xi_i, \quad i=1,\dots,n,$$
$\xi_i$ v.a. i.i.d. de densit\'e {\color{red} connue} $f$ sur $\R$
et $\E(X_i)=\vartheta$.

\underline{Question}: en utilisant cette information
suppl\'ementaire, peut-on construire un estimateur plus performant
que l'estimateur $\bar X_n$ bas\'e sur l'approche empirique?
\end{itemize}
%
\end{itemize}
\end{frame}


\begin{frame}
\frametitle{Expérience statistique}
\begin{itemize}
\item En \'ecrivant
$$ X_i= \vartheta +\xi_i, \quad i=1,\dots,n,$$
$\xi_i$ v.a. i.i.d. de densit\'e {\color{red} connue} $f$, nous
pr\'ecisons la forme de la loi $\PP_{\vartheta}$ de
$(X_1,\dots,X_n)$:
$$\PP_\vartheta\big[A\big] = \int_A
\left(\prod_{i=1}^n f(x_i-\vartheta)\right) dx_1\ldots dx_n,
$$
pour tout $A\in {\mathcal B}(\R^n)$.
\end{itemize}
\end{frame}


\begin{frame}
\frametitle{Expérience statistique}
\begin{df}
Une expérience (un modèle) statistique ${\mathcal E}$ est le triplet
$${\mathcal E} = \left(\mathfrak{Z}, {\mathcal Z}, \,\big\{\PP_\vartheta, \vartheta \in \Theta\big\}\right),$$
avec
\begin{itemize}
\item $\big(\mathfrak{Z}, {\mathcal Z}\big)$ espace mesurable (souvent
$(\R^n,{\mathcal B}(\R^n))$),
\item $\{\PP_\vartheta,\,\vartheta \in \Theta\}$ famille de probabilités définies {\color{red}simultanément} sur le même espace  $\big(\mathfrak{Z}, {\mathcal Z}\big)$,
\item $\vartheta$ est le {\color{red}paramètre inconnu}, et $\Theta$ est {\color{red} l'ensemble des paramètres connu}.
\end{itemize}
\end{df}
\end{frame}


%\begin{frame}
%\frametitle{Description (mathématique) d'une expérience statistique}
%\begin{itemize}
%\item Deux points de vue {\color{red} équivalents} et (parfois ?) source de confusion :
%\begin{itemize}
%\item Expérience \underline{engendrée par une observation}.
%\item Expérience \underline{canonique}
%\end{itemize}
%\item {\color{red}Traitement sur un exemple} : on \og observe \fg{} $X_1,\ldots, X_n$ i.i.d. de loi exponentielle de paramètre $\vartheta >0$.
%\end{itemize}
%\end{frame}


\begin{frame}
\frametitle{Experience engendrée par ($X_1,\ldots, X_n$)}
\begin{itemize}
\item {\color{red}Traitement sur un exemple} : on observe
$$Z = (X_1,\ldots, X_n), \quad\quad X_i= \vartheta + \xi_i,$$
$\xi_i$ v.a. i.i.d. de densit\'e {\color{red} connue} $f$.
\item La famille de lois $\big\{\PP_\vartheta^n,\vartheta \in \Theta  =
\R\big\}$ est définie sur $\mathfrak{Z}=\R^n$ par
$$\PP_\vartheta^n\big[A\big] = \int_A
\left(\prod_{i=1}^n f(x_i-\vartheta)\right) dx_1\ldots dx_n,
$$
pour $A\in{\mathcal Z}= {\mathcal B}(\R^n)$ (et $\PP^Z$ est l'une
des $\PP_\vartheta^n$).
\item Expérience {\color{red}engendrée par l'observation $Z$} :
$${\mathcal E}^n = \big(\R^n,{\mathcal B}(\R^n),\big\{\PP_\vartheta^n,\,\vartheta \in \Theta\big\}\big).$$
\end{itemize}
\end{frame}

%\begin{frame}
%\frametitle{Expérience -- observation canonique}
%\begin{itemize}
%\item Si on part de ${\mathcal E}^n = \big(\R^n,{\mathcal B}(\R^n),\big\{\PP_\vartheta^n,\,\vartheta \in \Theta\big\}\big)$, {\color{red}il n'y a plus d'observation !} On a perdu la structure (trop lourde) de tout l'espace $\big(\Omega, {\mathcal F}, \PP\big)$ et des observations $X_1,\ldots, X_n$.
%\item On peut {\color{red}toujours \og fabriquer \fg{} une observation} $Z$ dont l'expérience engendrée soit ${\mathcal E}^n$.
%Il suffit de poser
%$$\big(\Omega, {\mathcal F}\big):=\big(\mathfrak{Z}, {\mathcal Z}\big) \stackrel{\text{ici}}{=}\big(\R^n,{\mathcal B}(\R^n)\big),$$
%et
%$$Z(\omega) = \omega\;\;\;\text{observation canonique}.$$
%\end{itemize}
%\end{frame}

\begin{frame}
\frametitle{Expérience (mod\`ele) paramétrique, non-paramétrique}
\begin{itemize}
\item Si $\Theta$ peut être \og pris \fg{} comme un sous-ensemble
de $\R^d$ : {\color{red} expérience (=mod\`ele) paramétrique}.
\item Sinon (par exemple si le paramètre $\vartheta$ est un élément d'un espace fonctionnel) : {\color{red} expérience (=mod\`ele) non-paramétrique}.
\end{itemize}
\end{frame}

\subsection{Expériences dominées}
\begin{frame}
\frametitle{Expériences dominées}
\begin{itemize}
\item On fait une hypothèse minimale de \og complexité \fg{} sur le modèle statistique. {\color{red} But} : ramener l'étude de la famille
$$\{\PP_\vartheta,\,\vartheta \in \Theta\}$$
à l'étude d'une famille de fonctions
$$\left\{z \in \mathfrak{Z} \leadsto f(\vartheta,z) \in \R_+,\,\vartheta \in \Theta\right\}.$$
\item Via la notion de {\color{red} domination}. Si $\mu,\nu$ sont deux mesures $\sigma$-finies sur $\mathfrak{Z}$, alors $\mu$ {\color{red} domine} $\nu$ (notation $\nu \ll \mu$) si
$$\mu\big[A\big]=0 \Rightarrow \nu\big[A\big]=0.$$
\end{itemize}
\end{frame}
\begin{frame}
\frametitle{Théorème de Radon-Nikodym}
\begin{theo}
Si $\nu \ll \mu$, il existe une fonction positive
$$z \leadsto  p(z) \stackrel{\text{notation}}{=} \frac{d\nu}{d\mu}(z),$$ définie $\mu-$p.p., $\mu-$ intégrable, telle que
$$\nu\big[A\big] = \int_{A}p(z) \mu(dz) = \int_{A}\tfrac{d\nu}{d\mu}(z)\mu(dz),\;\;A \in {\mathcal Z}.$$
%\underline{Notation} :
%$$p(z)=\frac{d\nu}{d\mu}(z),\;\;\text{densité de}\;\nu\;\text{par rapport à}\;\mu.$$
\end{theo}
\end{frame}
\begin{frame}
\frametitle{Expérience dominée}
\begin{df}
Une expérience statistique ${\mathcal E} = \big(\mathfrak{Z}, {\mathcal Z}, \big\{\PP_\vartheta, \vartheta \in \Theta\big\}\big)$ est {\color{red}dominée} par la mesure $\sigma$-finie $\mu$ définie sur $\mathfrak{Z}$ si
$$\forall \vartheta \in \Theta : \PP_\vartheta \ll \mu.$$
\end{df}
On appelle {\color{red} densités} de la famille $\{\PP_\vartheta,\vartheta \in \Theta\}$ la famille de fonctions (définies $\mu-$ p.p.)
$$z\leadsto \frac{d\PP_\vartheta}{d\mu}(z),\;z\in \mathfrak{Z},\;\vartheta \in \Theta.$$
\end{frame}

\begin{frame}
\frametitle{Densité, régression}
Deux classes d'expériences statistiques {\color{red}dominées} fondamentales :
\begin{itemize}
\item Le modèle de {\color{red} densité}
\item Le modèle de {\color{red}régression }
\end{itemize}
\end{frame}


\subsection{Modèle de densité}

\begin{frame}
\frametitle{Modèle de densité (paramétrique)}
\begin{itemize}
\item On observe un $n$-échantillon de v.a.r. $X_1,\ldots, X_n$.
\item La loi des $X_i$ appartient à
$\{\PP_\vartheta,\,\vartheta \in \Theta\}$, famille de {\color{red}probabilités sur $\R$}, {\color{red} dominée} par une mesure ($\sigma$-finie) $\mu(dx)$ sur $\R$.
\item La loi de $(X_1,\ldots,X_n)$ s'écrit
\begin{align*}
\PP_\vartheta^n(dx_1\cdots dx_n) & = \PP_\vartheta(dx_1)\otimes \cdots \otimes \PP_\vartheta(dx_n) \\
& \ll  \mu(dx_1)\otimes \cdots \otimes \mu(dx_n) \\
& \stackrel{\text{{\color{red}notation}}}{=} \mu^n(dx_1\cdots dx_n)
\end{align*}
%\item {\color{red}L'expérience statistique} engendrée par $(X_1,\ldots, X_n)$ s'écrit :
%$${\mathcal E}^n = \Big(\R^n, {\mathcal B}(\R^n), \big\{\PP_\vartheta^n,\vartheta \in \Theta\big\}\Big),\;\;{\color{red}\Theta \subset \R^d}.$$
\end{itemize}
\end{frame}

\begin{frame}
\frametitle{Modèle de densité (paramétrique)}
\begin{itemize}
\item {\color{red} Densité du modèle} : on part de
$$f(\vartheta,x)=\frac{d\PP_\vartheta}{d\mu}(x),\;\;x\in \R$$
et
$$\frac{d\PP_\vartheta^n}{d\mu^{n}}(x_1,\ldots, x_n) = \prod_{i = 1}^n f(\vartheta,x_i),\;\;x_1,\ldots, X_n \in \R.$$
\item  {\color{red}L'expérience statistique} engendrée par $(X_1,\ldots, X_n)$ s'écrit :
$${\mathcal E}^n = \Big(\R^n, {\mathcal B}(\R^n), \big\{\PP_\vartheta^n,\vartheta \in \Theta\big\}\Big),\;\;{\color{red}\Theta \subset \R^d}.$$
\end{itemize}
\end{frame}

\begin{frame}
\frametitle{Exemple 1 : modèle de densité gaussienne univariée}
\begin{itemize}
\item $X_i\sim {\mathcal N}(m,\sigma^2)$,
avec
$$\vartheta = (m,\sigma^2) \in \Theta = \R\times \R_+\setminus\{0\}.$$
\begin{align*}
\PP_\vartheta(dx) = f(\vartheta,x)dx & =\frac{1}{\sqrt{2\pi \sigma^2}}\exp\Big(-\frac{(x-m)^2}{2\sigma^2}\Big)dx \\
& \ll \mu(dx)=dx.
\end{align*}
\item Puis
\begin{align*}
\frac{d\PP_{{\color{red}\vartheta}}^n}{d\mu^n}(x_1,\ldots, x_n) & = \prod_{i = 1}^n f({\color{red}\vartheta},x_i) \\
& =(2\pi {\color{red}\sigma^2})^{-n/2}\exp\big(-\frac{1}{2{\color{red}\sigma^2}}\sum_{i = 1}^n (x_i-{\color{red}m})^2\big),
\end{align*}
avec $x_1,\ldots,x_n \in \R$.
\end{itemize}
\end{frame}

\begin{frame}
\frametitle{Exemple 2 :  modèle de Bernoulli}
\begin{itemize}
\item $X_i \sim \text{Bernoulli}(\vartheta)$, avec $\vartheta \in \Theta = [0,1]$.
\begin{align*}
\PP_\vartheta(dx)& = (1-\vartheta)\, \delta_{0}(dx) + \vartheta\, \delta_1(dx) \\
& \ll \mu(dx) = \delta_0(dx)+\delta_1(dx)\;\;\text{(mesure de comptage)}.
\end{align*}
\item Puis
$$\boxed{\frac{d\PP_{\color{red}\vartheta}}{d\mu}(x) = (1-{\color{red}\vartheta})\,1_{\{x=0\}}+{\color{red}\vartheta}\,1_{\{x=1\}} = {\color{red}\vartheta}^x(1-{\color{red}\vartheta})^{1-x}}$$
{\color{red}avec $x\in \{0,1\}$} (et $0$ sinon), et
$$\frac{d\PP_\vartheta^n}{d\mu^n}(x_1\cdots x_n) = \prod_{i = 1}^n \vartheta^{x_i}(1-\vartheta)^{1-x_i},$$
{\color{red}avec $x_i \in \{0,1\}$} (et $0$ sinon).
\end{itemize}
\end{frame}

\begin{frame}
\frametitle{Exemple 3 : temps de panne \og arrêtés\fg{}}
\begin{itemize}
\item On observe $X_1,\ldots, X_n$, où $X_i = Y_i \wedge T$, avec $Y_i$ {\color{red}lois exponentielles} de paramètre $\vartheta$ et $T$ {\color{red}temps fixe} (censure).
\item Cas 1 : $T=\infty$ (pas de censure). Alors  $\vartheta \in \Theta = \R_+\setminus \{0\}$ et
$$\PP_\vartheta(dx) = \vartheta \exp(-\vartheta x)1_{\{x \geq 0\}}dx \ll \mu(dx) = dx$$
et
$$\frac{d\PP_{\color{red}\vartheta}^n}{d\mu^n}(x_1,\ldots,x_n) = {\color{red}\vartheta}^n \exp\Big(-{\color{red}\vartheta} \sum_{i = 1}^n x_i\Big),$$
{\color{red}avec $x_i \in \R_+$} (et $0$ sinon).
\item Cas 2 : {\color{red}Comment s'écrit le modèle} dans la cas où $T<\infty$ (présence de censure) ? Comment choisir $\mu$ ?
\end{itemize}
\end{frame}




%\subsection{Modèle de régression}
%
%
%
%
%\begin{frame}
%\frametitle{Expérience statistique}
%%\begin{itemize}
%%\item
%Une {\color{red}expérience (un modèle) statistique} ${\mathcal E}$ est la donnée d'un triplet
%$$\boxed{{\mathcal E} = \left(\mathfrak{Z}, {\mathcal Z}, \,\big\{\PP_\vartheta, \vartheta \in \Theta\big\}\right)}$$
%avec
%\begin{itemize}
%\item $\big(\mathfrak{Z}, {\mathcal Z}\big)$ espace d'état (des observables, en général $\R^n$),
%\item $\{\PP_\vartheta,\,\vartheta \in \Theta\}$ famille de probabilités définies {\color{red}simultanément} sur le même espace  $\big(\mathfrak{Z}, {\mathcal Z}\big)$,
%\item $\vartheta$ {\color{red}paramètre inconnu}, $\Theta$ {\color{red} l'ensemble des paramètres}.
%\end{itemize}
%%\item L'expérience ${\mathcal E}$ est {\color{red} dominée} s'il existe une mesure $\sigma$-finie $\mu$ sur $\mathfrak{Z}$ telle que $\PP_\vartheta \ll \mu$ pour tout $\vartheta \in \Theta$.
%%\end{itemize}
%\end{frame}
%
%
%\begin{frame}
%\frametitle{Experience engendrée par ($X_1,\ldots, X_n$)}
%\begin{itemize}
%\item {\color{red}Traitement sur un exemple} : on observe
%$$Z = (X_1,\ldots, X_n), \quad\quad X_i= \vartheta + \xi_i,$$
%$\xi_i$ v.a. i.i.d. de densit\'e {\color{red} connue} $f$.
%\item La famille de lois $\big\{\PP_\vartheta^n,\vartheta \in \Theta  =
%\R\big\}$ est définie sur $\mathfrak{Z}=\R^n$ par
%$$\PP_\vartheta^n\big[A\big] = \int_A
%\left(\prod_{i=1}^n f(x_i-\vartheta)\right) dx_1\ldots dx_n,
%$$
%pour $A\in{\mathcal Z}= {\mathcal B}(\R^n)$ (et $\PP^Z$ est l'une
%des $\PP_\vartheta^n$).
%\item Expérience {\color{red}engendrée par l'observation $Z$} :
%$${\mathcal E}^n = \big(\R^n,{\mathcal B}(\R^n),\big\{\PP_\vartheta^n,\,\vartheta \in \Theta\big\}\big).$$
%\end{itemize}
%\end{frame}
%
%
%\begin{frame}
%\frametitle{Expérience (mod\`ele) paramétrique, non-paramétrique}
%\begin{itemize}
%\item Si $\Theta$ peut être \og pris \fg{} comme un sous-ensemble
%de $\R^d$ : {\color{red} expérience (=mod\`ele) paramétrique}.
%\item Sinon (par exemple si le paramètre $\vartheta$ est un élément d'un espace fonctionnel) : {\color{red} expérience (=mod\`ele) non-paramétrique}.
%\end{itemize}
%\end{frame}
%
%\subsection{Expériences dominées}
%\begin{frame}
%\frametitle{Expériences dominées}
%\begin{itemize}
%\item On fait une hypothèse minimale de \og complexité \fg{} sur le modèle statistique. {\color{red} But} : ramener l'étude de la famille
%$$\{\PP_\vartheta,\,\vartheta \in \Theta\}$$
%à l'étude d'une famille de fonctions
%$$\left\{z \in \mathfrak{Z} \leadsto f(\vartheta,z) \in \R_+,\,\vartheta \in \Theta\right\}.$$
%\item Via la notion de {\color{red} domination}. Si $\mu,\nu$ sont deux mesures $\sigma$-finies sur $\mathfrak{Z}$, alors $\mu$ {\color{red} domine} $\nu$ (notation $\nu \ll \mu$) si
%$$\mu\big[A\big]=0 \Rightarrow \nu\big[A\big]=0.$$
%\end{itemize}
%\end{frame}
%\begin{frame}
%\frametitle{Théorème de Radon-Nikodym}
%\begin{theo}
%Si $\nu \ll \mu$, il existe une fonction positive
%$$z \leadsto  p(z) \stackrel{\text{notation}}{=} \frac{d\nu}{d\mu}(z),$$ définie $\mu-$p.p., $\mu-$ intégrable, telle que
%$$\nu\big[A\big] = \int_{A}p(z) \mu(dz) = \int_{A}\tfrac{d\nu}{d\mu}(z)\mu(dz),\;\;A \in {\mathcal Z}.$$
%%\underline{Notation} :
%%$$p(z)=\frac{d\nu}{d\mu}(z),\;\;\text{densité de}\;\nu\;\text{par rapport à}\;\mu.$$
%\end{theo}
%\end{frame}
%\begin{frame}
%\frametitle{Expérience dominée}
%\begin{df}
%Une expérience statistique ${\mathcal E} = \big(\mathfrak{Z}, {\mathcal Z}, \big\{\PP_\vartheta, \vartheta \in \Theta\big\}\big)$ est {\color{red}dominée} par la mesure $\sigma$-finie $\mu$ définie sur $\mathfrak{Z}$ si
%$$\forall \vartheta \in \Theta : \PP_\vartheta \ll \mu.$$
%\end{df}
%On appelle {\color{red} densités} de la famille $\{\PP_\vartheta,\vartheta \in \Theta\}$ la famille de fonctions (définies $\mu-$ p.p.)
%$$z\leadsto \frac{d\PP_\vartheta}{d\mu}(z),\;z\in \mathfrak{Z},\;\vartheta \in \Theta.$$
%\end{frame}
%
%\begin{frame}
%\frametitle{Densité, régression}
%Deux classes d'expériences statistiques {\color{red}dominées} fondamentales :
%\begin{itemize}
%\item Le modèle de {\color{red} densité} (Cours 3)
%\item Le modèle de {\color{red}régression } (Cours 4)
%\end{itemize}
%\end{frame}
%
%
%\begin{frame}
%\frametitle{Modèle de densité paramétrique}
%\begin{itemize}
%\item On observe un $n$-échantillon de v.a.r. $X_1,\ldots, X_n$.
%\item La loi des $X_i$ appartient à
%$\{\PP_\vartheta,\,\vartheta \in \Theta\}$, famille de {\color{red}probabilités sur $\R$}, indicée par $\Theta \subset \R^d$, {\color{red} dominée} par une mesure ($\sigma$-finie) $\mu(dx)$ sur $\R$.
%\item Loi de $(X_1,\ldots,X_n)$
%$$
%\boxed{\PP_\vartheta^n(dx_1\cdots dx_n)  =\prod_{i=1}^n f(\vartheta, x_i)\mu^n(dx_1\ldots dx_n)}$$
%avec
%$$ f(\vartheta,x) = \frac{d\PP_\vartheta}{d\mu}(x)$$
%et $\mu^n(dx_1\ldots dx_n) =\mu(dx_1)\otimes \ldots \otimes \mu(dx_n)$.
%\end{itemize}
%\end{frame}
%
%
%\begin{frame}
%\frametitle{Exemple : modèle de Bernoulli}
%\begin{itemize}
%\item $X_i \sim \text{Bernoulli}(\vartheta)$, avec $\vartheta \in \Theta = [0,1]$.
%\begin{align*}
%\PP_\vartheta(dx)& = (1-\vartheta)\, \delta_{0}(dx) + \vartheta\, \delta_1(dx)\\
%& \ll \mu(dx) = \delta_0(dx)+\delta_1(dx).
%\end{align*}
%\item Puis
%$$\boxed{\frac{d\PP_{\color{red}\vartheta}}{d\mu}(x) = (1-{\color{red}\vartheta})\,1_{\{x=0\}}+{\color{red}\vartheta}\,1_{\{x=1\}} = {\color{red}\vartheta}^x(1-{\color{red}\vartheta})^{1-x}}$$
%{\color{red}avec $x\in \{0,1\}$} (et $0$ sinon), et
%\begin{align*}
%\frac{d\PP_\vartheta^n}{d\mu^n}(x_1,\cdots, x_n) & = \prod_{i = 1}^n \vartheta^{x_i}(1-\vartheta)^{1-x_i} \\
%& =\vartheta^{\sum_{i = 1}^n x_i}(1-\vartheta)^{n-\sum_{i = 1}^nx_i}.
%\end{align*}
%%{\color{red}avec $x_i \in \{0,1\}$} (et $0$ sinon).
%\item Remarque : choix des {\color{red} espaces...}
%\end{itemize}
%\end{frame}
%
%\begin{frame}
%\frametitle{Exemple : temps de panne \og arrêtés\fg{}}
%\begin{itemize}
%\item On observe $X_1,\ldots, X_n$, où $X_i = Y_i \wedge T$, avec $Y_i$ {\color{red}lois exponentielles} de paramètre $\vartheta$ et $T$ {\color{red}temps fixe} (censure).
%\item Cas 1 : $T=\infty$ (pas de censure). Alors  $\vartheta \in \Theta = \R_+\setminus \{0\}$ et
%$$\PP_\vartheta(dx) = \vartheta \exp(-\vartheta x)1_{\{x \geq 0\}}dx \ll \mu(dx) = dx$$
%et
%$$\frac{d\PP_{\color{red}\vartheta}^n}{d\mu^n}(x_1,\ldots,x_n) = {\color{red}\vartheta}^n \exp\Big(-{\color{red}\vartheta} \sum_{i = 1}^n x_i\Big),
%%\prod_{i = 1}^n 1_{x_i \geq 0},
%$$
%{\color{red}pour $x_i \geq 0$}, et $0$ sinon.
%
%\end{itemize}
%\end{frame}
%
\begin{frame}
\frametitle{Exemple : temps de panne \og arrêtés\fg{}}
\begin{itemize}
%\item Cas 2 : {\color{red}Comment s'écrit le modèle} dans la cas où $T<\infty$ (présence de censure) ? Comment choisir $\mu$ ?
\item \underline{Loi $\PP_\vartheta(dx)$ de $X = Y \wedge T$} : $Y \sim$ exponentielle de paramètre $\vartheta$ :
$$\boxed{X = Y 1_{\{Y < T\}} + T 1_{\{Y \geq T\}}}$$
d'où
\begin{align*}\PP_\vartheta(dx) &
= \vartheta e^{-\vartheta x} 1_{\{0 \leq x < T\}} dx+ \big(\int_{T}^{+\infty}\vartheta e^{-\vartheta y}dy\big) \delta_T(dx)\\
&= \vartheta e^{-\vartheta x} 1_{\{0 \leq x < T\}}dx + e^{-\vartheta T} \delta_T(dx) \\
&\ll \mu(dx) = dx + \delta_T(dx)\;\;\; \text{(par exemple)}.
\end{align*}
%(définie $\mu$-pp).
\end{itemize}
\end{frame}

\begin{frame}
\frametitle{Exemple : temps de panne \og arrêtés\fg{} (fin)}
\begin{itemize}
\item Alors, pour ce choix de mesure dominante
$$\boxed{\frac{d\PP_{\color{red}\vartheta}}{d\mu}(x) = {\color{red}\vartheta} e^{-{\color{red}\vartheta} x} 1_{\{0 \leq x < T\}} +  e^{-{\color{red}\vartheta} T}1_{\{x=T\}}}$$
\item Finalement,
$$\PP_\vartheta^n(dx_1,\ldots dx_n)
\ll \mu^n(dx_1 \ldots dx_n) = \bigotimes_{i = 1}^n \big[dx_i +
\delta_{T}(dx_i)\big]
$$
et
\begin{align*}
\frac{d\PP_\vartheta^n}{d\mu^n}(x_1,\ldots, x_n) & = \prod_{i = 1}^n \big(\vartheta e^{-\vartheta x_i} 1_{\{0 \leq x_i < T\}} +  e^{-\vartheta T}1_{\{x_i=T\}}\big) \\
& = \vartheta^{N_n(T)} e^{-\vartheta \sum_{i = 1}^{n}x_i 1_{\{x_i<T\}}}e^{-\vartheta T\big(n-N_n(T)\big)},
\end{align*}
{\color{red} avec $0 \leq x_i \leq T$} et $0$ sinon, et  $N_n(T) = \sum_{i = 1}^n1_{\{x_i < T\}}$.
\end{itemize}
\end{frame}




\section{Méthodes d'estimation pour le modèle de densité}


\begin{frame}
\frametitle{Méthodes d'estimation }
\begin{itemize}
%\item Situation
\item Méthode de substitution (ou des moments)
\item $Z$-estimation
\item $M$-estimation
\item Le principe du {\color{red} maximum de vraisemblance}
\end{itemize}
\end{frame}

%\begin{frame}
%\frametitle{Situation}
%\end{frame}

\subsection{Méthode des moments}

\begin{frame}
\frametitle{Méthode des moments : dimension 1}
\begin{itemize}
\item $X_1,\ldots, X_n \sim_{\text{i.i.d.}} \PP_\vartheta$, avec $\vartheta \in \Theta \;{\color{red} \subset \R}$.
\item \underline{Principe} : trouver $g:\R\rightarrow \R$ (en général $g(x)=x^k$) et $h:\R\rightarrow \R$ {\color{red} régulières} de sorte que
$$\vartheta  = h\big(\E_{{\color{red}\vartheta}}\big[g(X)\big]\big) = h\Big(\int_{\R}g(x)dF_{{\color{red}\vartheta}}(x)\Big)=T(F_{{\color{red}\vartheta}})$$
et $T$ fonctionnelle régulière de la distribution inconnnue $F_{{\color{red}\vartheta}}$.
\item \underline{Estimateur} : \og plug-in \fg{}
$$\widehat \vartheta_n = h\big(\tfrac{1}{n}\sum_{i = 1}^n g(X_i)\big).$$
\end{itemize}
\end{frame}

\begin{frame}
\frametitle{Méthode des moments}
\begin{itemize}
\item \underline{Précision d'estimation} via les techniques empiriques :
$$\sqrt{n}\big(\widehat \vartheta_n - \vartheta\big) \stackrel{d}{\rightarrow} {\mathcal N}\big(0, h'(\E_{{\color{red}\vartheta}}[g(X)])^2\mathrm{Var}_{{\color{red} \vartheta}}[g(X)]\big)$$
en {\color{red} loi sous $\PP_\vartheta$} et la variance asymptotique dépend en général de ${\color{red}\vartheta}$ $\rightarrow$ élimination par estimation préliminaire licite via le lemme de Slutsky.
\item \underline{Exemple} : $X_1,\ldots, X_n\sim_{\text{i.i.d.}}$ exponentielle de paramètre $\vartheta$. On a
$$\E_\vartheta\big[X\big] = \frac{1}{\vartheta},$$
l'estimateur par moment associé s'écrit
$$\est = \frac{1}{\overline{X}_n}.$$
\end{itemize}
\end{frame}

\begin{frame}
\frametitle{Exemple en dimension $d>1$}
\begin{itemize}
%\item Le cas de la dimension $d >1$ {\color{red}sur un exemple} :
\item $X_1,\ldots,X_n \sim_{\text{i.i.d.}}$ Béta$(\alpha,\beta)$, de densité
$$x \leadsto \frac{\Gamma(\alpha+\beta)}{\Gamma(\alpha)\Gamma(\beta)}x^{\alpha-1}(1-x)^{\beta-1}1_{\{0 < x < 1\}},$$
\item Le paramètre est ${\color{red}\vartheta} = (\alpha,\beta) \in {\color{red}\Theta}=\R_+\setminus\{0\} \times \R_+\setminus\{0\}$.
\item On a
$$\boxed{\E_\vartheta\big[X\big]=\frac{\alpha}{\alpha+\beta},\;\;\E_\vartheta\big[X^2\big]=\frac{\alpha(\alpha+1)}{(\alpha+\beta+1)(\alpha+\beta)}}$$
\end{itemize}
\end{frame}

\begin{frame}
\frametitle{Exemple en dimension $d>1$}
\begin{itemize}
\item {\color{red}L'estimateur par moment} $\est = (\widehat \vartheta_n^{(1)},\widehat \vartheta_n^{(2)})$ associé est défini par
$$
\left\{\begin{array}{cll}
\overline{X}_n & = &\displaystyle \frac{\widehat \vartheta_n^{(1)}}{\widehat \vartheta_n^{(1)}+\widehat \vartheta_n^{(2)}} \\
\frac{1}{n}\sum_{i = 1}^n X_i^2 & = &\displaystyle  \frac{\widehat \vartheta_n^{(1)}(\widehat \vartheta_n^{(1)}+1)}{(\widehat \vartheta_n^{(1)}+\widehat \vartheta_n^{(2)}+1)(\widehat \vartheta_n^{(1)}+\widehat \vartheta_n^{(2)})}.
\end{array}
\right.
$$
\item {\color{red}Etude asymptotique} via le TCL multidimensionnel et la méthode \og delta\fg{} multidimensionnelle.
\end{itemize}
\end{frame}

\begin{frame}
\frametitle{Limites de la méthode des moments}
\begin{itemize}
%\item \underline{{\color{red}Limites} de l'approche par moments} :
%\begin{itemize}
\item Méthode {\color{red}non systématique}
\item Représentation pas toujours explicite
\item Choix de la fonction $g$, notion d'optimalité parmi une classe d'estimateurs...
%\end{itemize}
\item {\color{red}Généralisation} : $Z$-estimation (ou estimation par méthode
des moments généralisés, GMM= {\it generalized method of moments}).
\end{itemize}
\end{frame}
\subsection{$Z$-estimation}
\begin{frame}
\frametitle{$Z$-estimation}
\begin{itemize}
\item La méthode des moments  (en dimension 1) est basée sur l'inversibilité de la fonction
$$m_g(\vartheta) = \int_{\R} g(x)\PP_\vartheta(dx)$$
i.e. pour tout $\vartheta \in \Theta$
$$\int_{\R}\big(m_g(\vartheta)-g(x)\big)\PP_\vartheta(dx)=0.$$
\item \underline{Principe de construction d'un $Z$-estimateur} :
%(dimension 1)
{\color{red} remplacer} $m_g(\vartheta)-g(x)$ par une fonction $\phi(\vartheta,x):
\Theta \times \R \rightarrow \R$ {\color{red}arbitraire}
telle que
$$\boxed{\forall \vartheta \in \Theta,\int_{\R}\phi(\vartheta, x) \PP_\vartheta(dx)=0.}$$
\end{itemize}
\end{frame}

\begin{frame}
\frametitle{$Z$-estimation}
\begin{itemize}
\item Résoudre l'équation {\color{red} empirique} associée :
$$\boxed{\frac{1}{n}\sum_{i = 1}^n \phi(a, X_i)=0\;\;\text{pour}\;\;a\in \Theta.}$$
\end{itemize}
\begin{df}
On appelle {\color{red}$Z$-estimateur} associé à $\phi$ tout estimateur ${\color{red}\est}$ satisfaisant
$$\sum_{i = 1}^n \phi({\color{red}\est}, X_i)=0$$
\end{df}
\begin{itemize}
\item Il n'y a pas unicité de $\est$ (à ce niveau).
\item \underline{Programme} {\color{red} Etablir des conditions}
sur $\phi$ et sur la famille $\{\PP_\vartheta, \vartheta \in
\Theta\}$ pour obtenir la convergence et les performances
asymptotiques de $\est$.
  \end{itemize}
\end{frame}

\begin{frame}
\frametitle{$Z$-estimation: \`a quoi \c{c}a sert?}
\begin{itemize}
\item \underline{Exemple.} $\Theta = \R$,
$\PP_{\color{red}\vartheta}(dx) = f(x-{\color{red}\vartheta})dx$, et
$f$ sym\'etrique: $f(-x)=f(x)$, $\forall x\in \R$.
\item {\color{red} Il n'y a pas de bornitude des moments!}
\item On pose
$$\phi(a,x)={\rm Arctg}(x-a).$$
\item La fonction
$$a \leadsto \E_{\color{red}\vartheta}\big[\phi(a,X)\big] =
\int_{\R}{\rm Arctg}(x-a)f(x-{\color{red}\vartheta})dx$$ est
strictement d\'ecroissante et s'annule seulement en $a=\vartheta.$
\item {\color{red}$Z$-estimateur associé :} solution $\est$ de
$$\sum_{i = 1}^n{\rm Arctg}(X_i-\est) = 0$$
{\color{red} (unicit\'e)}.
\end{itemize}
\end{frame}


\begin{frame}
\frametitle{Le cas multidimensionnel}
Si ${\color{red} \Theta \subset \R^d}$ avec {\color{red} $d  >1$}, la fonction $\phi$ est remplacée par
$$\Phi = (\phi_1,\ldots,\phi_d):\Theta \times \R \rightarrow \R^d.$$
\begin{definition}
On appelle $Z$-estimateur associé à $\Phi$ tout estimateur ${\color{red}\est}$ satisfaisant
$$\sum_{i = 1}^n \phi_\ell({\color{red}\est}, X_i)=0,\;\;\ell = 1,\ldots, d.$$
\end{definition}
\end{frame}

\begin{frame}
\frametitle{$Z$-estimation $\rightarrow M$-estimation}
\begin{itemize}
\item \underline{En dimension 1} : si
$$\boxed{\phi(\vartheta,x) = \partial_\vartheta\psi(\vartheta, x)}$$
pour une certaine fonction $\psi$, résoudre
$\sum_{i = 1}^n \phi(\vartheta, X_i)=0$
revient à {\color{red}chercher un point critique} de
$$\vartheta \leadsto \sum_{i = 1}^n\psi(\vartheta, X_i).$$
\item \underline{En dimension $d \geq 1$}, il faut $\phi(\vartheta, x) = \nabla_\vartheta \psi(\vartheta, x)$ (moins facile à obtenir).
\item {\color{red} Invite à généraliser} la recherche d'estimateurs via la maximisation d'un critère $\rightarrow M$-estimation.
\end{itemize}
\end{frame}

\subsection{$M$-estimation}

\begin{frame}
\frametitle{$M$-estimation}
\begin{itemize}
\item \underline{Principe} : Se donner une application $\psi : \Theta \times \R \rightarrow \R_+$ telle que, pour tout $\vartheta \in \Theta \subset {\color{red}\R^d}$,
$$a \leadsto \E_\vartheta\big[\psi(a,X)\big] = \int \psi(a,x)\PP_\vartheta(dx)$$
admet {\color{red}un maximum en $a=\vartheta$}.
\end{itemize}
\begin{df}
On appelle $M$-estimateur associé à $\psi$ tout estimateur ${\color{red}\est}$ satisfaisant
$$\sum_{i = 1}^n \psi({\color{red}\est}, X_i) = \max_{a \in \Theta}\sum_{i = 1}^n\psi(a, X_i).$$
\end{df}
\begin{itemize}
\item Il n'y a pas unicité de $\est$ (à ce niveau).
\end{itemize}
\end{frame}

\begin{frame}
\frametitle{Un exemple classique : paramètre de localisation}
\begin{itemize}
\item $\Theta = \R$, $\PP_{\color{red}\vartheta}(dx) = f(x-{\color{red}\vartheta})dx$, et $\int_{\R}xf(x)dx=0$, $\int_{\R}x^2\PP_\vartheta(dx)<+\infty$ pour tout $\vartheta \in \R$. On pose
$$\boxed{\psi(a,x)=-(a-x)^2}$$
\item La fonction
$$a \leadsto \E_\vartheta\big[\psi(a,X)\big] =
-\int_{\R}(a-X)^2f(x-\vartheta)dx$$
admet un {\color{red}maximum} en $a=\E_\vartheta\big[X\big] = \int_{\R}xf(x-\vartheta)dx=\vartheta.$
\item {\color{red}$M$-estimateur associé :}
$$\sum_{i = 1}^n(X_i-\est)^2 = \min_{a \in \R}\sum_{i = 1}^n (X_i-a)^2.$$
\end{itemize}
\end{frame}

\begin{frame}
\frametitle{Paramètre de localisation}
\begin{itemize}
\item C'est {\color{red}aussi} un $Z$-estimateur associé à $\phi(a,x)=2(x-a)$: on résout
$$\sum_{i = 1}^n (a-X_i)=0\;\;\text{d'où}\;\;\est = \overline{X}_n.$$
\item Dans cet {\color{red}exemple très simple},
tous les points de vue co•ncident.
%
\item Si, dans le même contexte,
$\int_{\R}x^2\PP_\vartheta(dx)=+\infty$ et $f(x)=f(-x)$, on peut
utiliser $Z$-estimateur avec $\phi(a,x)={\rm Arctg}(x-a)$. M\'ethode
robuste, mais est-elle optimale? Peut-on faire mieux {\color{red}si
$f$ est connue? A suivre...}
\end{itemize}
\end{frame}

\begin{frame}
\frametitle{Lien entre $Z$- et $M$- estimateurs}
\begin{itemize}
\item {\color{red}Pas d'inclusion} entre ces deux classes d'estimateurs {\color{red}en général} :
\begin{itemize}
\item Si $\psi$ non-régulière, $M$-estimateur $\nRightarrow$ $Z$-estimateur
\item Si une équation d'estimation admet plusieurs solutions distinctes, $Z$-estimateur $\nRightarrow$ $M$-estimateur (cas d'un extremum local).
\end{itemize}
\item Toutefois, si $\psi$ {\color{red}est régulière}, les $M$-estimateurs {\color{red}sont} des $Z$-estimateurs : si $\Theta \subset \R$ ($d=1$), en posant
$$\phi(a,x) = \partial_a\psi(a,x),$$
on a
$$\boxed{\sum_{i = 1}^n \partial_{a} \psi(\vartheta, X_i)\big|_{a = \est}
= \sum_{i = 1}^n \phi(\est, X_i)=0}.$$
%\item On travaillera (presque) toujours dans ce cadre.
\end{itemize}
\end{frame}

%\begin{frame}
%\frametitle{$M$-estimateur : exemple}
%\end{frame}

\subsection{Principe de maximum de vraisemblance}

\begin{frame}
\frametitle{Maximum de vraisemblance}
\begin{itemize}
\item Principe {\color{red} fondamental} et
{\color{red}incontournable} en statistique. Cas particuliers connus
depuis le XVIII\`eme si\`ecle. D\'efinition g\'en\'erale:
Fisher~(1922).
\item Fournit une première {\color{red}méthode systématique} de construction d'un $M$-estimateur
(souvent un $Z$-estimateur, souvent aussi {\it a posteriori} un
estimateur par substitution simple).
\item Procédure {\color{red}optimale} (dans quel sens ?)
sous des hypothèses de {\color{red} régularité} de la famille
$\{\PP_\vartheta, \vartheta \in \Theta\}$ (Cours 6).
\item Parfois difficile à mettre en oeuvre en pratique
$\rightarrow$ {\color{red}méthodes numériques}, statistique
computationnelle.
\end{itemize}
\end{frame}

\begin{frame}
\frametitle{Fonction de vraisemblance}
\begin{itemize}
\item La famille $\{\PP_\vartheta,\vartheta \in \Theta\}$ est dominée par une mesure $\sigma$-finie $\mu$. On se donne, pour $\vartheta \in \Theta$
$$f(\vartheta,x) = \frac{d\PP_\vartheta}{d\mu}(x),\;x \in \R.$$
\end{itemize}
\begin{df}
{\color{red}Fonction de vraisemblance} du $n$-échantillon associée à la famille $\{f(\vartheta,\cdot),\vartheta \in \Theta\}$ :
$$\boxed{\vartheta \leadsto {\mathcal L}_n(\vartheta, X_1,\ldots, X_n) = \prod_{i = 1}^n f(\vartheta, X_i)}$$
\end{df}
%\end{frame}
\begin{itemize}
\item C'est une fonction aléatoire (définie $\mu$-presque partout).
\end{itemize}
\end{frame}

\begin{frame}
\frametitle{Exemples}
\begin{itemize}
\item \underline{Exemple 1}: {\color{red}Modèle de Poisson}. On observe
$$X_1,\ldots, X_n \sim_{\text{i.i.d.}}\text{Poisson}({\color{red}\vartheta}),$$
${\color{red}\vartheta} \in \Theta = \R_+\setminus \{0\}$ et prenons
$\mu(dx) = \sum_{k \in \N}\delta_k(dx)$.
\item La densit\'e de $\PP_\vartheta$ par rapport \`a $\mu$ est
$$f({\color{red}\vartheta}, x) = e^{-{\color{red}\vartheta}}
\frac{{\color{red}\vartheta}^x}{x!}, \quad x=0,1,2,\dots.$$
\item La {\color{red} fonction de vraisemblance} associée s'écrit
\begin{align*}
\vartheta \leadsto {\mathcal L}_n(\vartheta, X_1,\ldots, X_n)
&= \prod_{i = 1}^n e^{-\vartheta}\frac{\vartheta^{X_i}}{X_i!} \\
&= \frac{1}{\prod_{i = 1}^nX_i!} e^{-n\vartheta} \vartheta^{\sum_{i = 1}^n X_i}.
\end{align*}
\end{itemize}
\end{frame}

\begin{frame}
\frametitle{Exemples}
\begin{itemize}
\item \underline{Exemple 2} {\color{red}Mod\`ele de Cauchy}. On observe
$$X_1,\ldots, X_n \sim_{\text{i.i.d.}}\text{Cauchy},$$
${\color{red}\vartheta} \in \Theta = \R$ et $\mu(dx)=dx$ ({\color{red} par exemple}).
\item On a alors
$$\PP_{\color{red}\vartheta}(dx)=f({\color{red}\vartheta},x)dx=\frac{1}{\pi\big(1+(x-{\color{red}\vartheta})^2\big)}dx.$$
\item La {\color{red} fonction de vraisemblance} associée s'écrit
$$\vartheta \leadsto {\mathcal L}_n(\vartheta, X_1,\ldots, X_n) = \frac{1}{\pi^n}\prod_{i = 1}^n \big(1+(X_i-\vartheta)^2\big)^{-1}.$$
\end{itemize}
\end{frame}

\begin{frame}
\frametitle{Principe de maximum de vraisemblance}
\begin{itemize}
\item Cas d'une famille de lois {\color{red} restreinte à deux points}
$$\Theta  = \{\vartheta_1,\vartheta_2\} \subset \R,$$
avec $\PP_{\vartheta_i}$ discrète et $\mu(dx)$ la mesure de comptage.
\item {\color{red}A priori}, pour tout $(x_1,\ldots, x_n)$, et pour $\vartheta \in \{\vartheta_1,\vartheta_2\}$,
\begin{align*}
\PP_\vartheta\big[X_1=x_1,\ldots, X_n=x_n\big] & = \prod_{i=1}^n \PP_\vartheta\big[X_i=x_i\big] \\
&=\prod_{i = 1}^nf(\vartheta, x_i).
\end{align*}
La probabilit\'e d'avoir la r\'ealisation fix\'ee $(x_1,\ldots,
x_n)$.
\end{itemize}
\end{frame}

\begin{frame}
\frametitle{Principe de maximum de vraisemblance}
\begin{itemize}
\item {\color{red}A posteriori, on observe $(X_1,\ldots, X_n)$.} L'événement
$$\Big\{\prod_{i = 1}^n f({\color{red}\vartheta_1},X_i) > \prod_{i = 1}^n f({\color{blue}\vartheta_2},X_i)\Big\}\;\;\;\text{(Cas 1)}$$
{\color{red}ou bien} l'événement
$$\Big\{\prod_{i = 1}^n f({\color{blue}\vartheta_2},X_i) > \prod_{i = 1}^n f({\color{red}\vartheta_1},X_i)\Big\}\;\;\;\text{(Cas 2)}$$
est réalisé. (On ignore le cas d'égalité.)
\item {\color{red} Principe de maximum de vraisemblance:}
$$\estMV = {\color{red}\vartheta_1} 1_{\{\text{Cas 1}\}}+ {\color{blue}\vartheta_2} 1_{\{\text{Cas 2}\}}.$$
\end{itemize}
\end{frame}

\begin{frame}
\frametitle{Estimateur du maximum de vraisemblance}
\begin{itemize}
\item On généralise le principe précédent pour une famille de lois et un ensemble de paramètres {\color{red}quelconques}.
\item \underline{Situation} : $X_1,\ldots, X_n\sim_{\text{i.i.d.}}\PP_\vartheta$, $\{\PP_\vartheta,\vartheta \in \Theta\}$ dominée, $\Theta \subset \R^d$, $\vartheta \leadsto {\mathcal L}_n(\vartheta, X_1,\ldots, X_n)$ vraisemblance associée.
\end{itemize}
\begin{df}
On appelle {\color{red} estimateur du maximum de vraisemblance} tout estimateur $\estMV$ satisfaisant
$${\mathcal L}_n(\estMV,X_1,\ldots, X_n) = \max_{\vartheta \in \Theta} {\mathcal L}_n(\vartheta, X_1,\ldots, X_n).$$
\end{df}
\begin{itemize}
\item {\color{red}Existence, unicité...}
\end{itemize}
\end{frame}

\begin{frame}
\frametitle{Remarques}
\begin{itemize}
\item \underline{Log-vraisemblance}:
\begin{align*}\vartheta \leadsto \ell_n(\vartheta, X_1,\ldots, X_n)& = \log {\mathcal L}_n(\vartheta, X_1,\ldots, X_n)\\
& = \sum_{i = 1}^n \log f(\vartheta, X_i).
\end{align*}
{\color{red}Bien défini} si $f(\vartheta, \cdot) >0$ $\mu$-pp.
$$\text{Max. vraisemblance = max. log-vraisemblance.}$$
\item L'estimateur du maximum de vraisemblance {\color{red} ne dépend pas} du choix de la mesure dominante $\mu$.
\item Notion de {\color{red} racine de l'équation de vraisemblance} : tout estimateur $\widehat \vartheta_n^{\,{\tt rv}}$ vérifiant
$$\nabla_\vartheta \ell_n(\widehat \vartheta_n^{\,{\tt rv}}, X_1,\ldots, X_n) = 0.$$
\end{itemize}
\end{frame}

\begin{frame}
\frametitle{Exemple : modèle normal } L'expérience statistique est
engendrée par un $n$-échantillon de loi ${\mathcal
N}(\mu,\sigma^2)$, le paramètre est $\vartheta = (\mu,\sigma^2)\in
\Theta = \R\times \R_+\setminus\{0\}$.
\begin{itemize}
\item
{\color{red}Vraisemblance} $${\mathcal L}_n((\mu,\sigma^2),
X_1,\ldots, X_n) =
\frac1{(2\pi\sigma^2)^{n/2}}\exp\big(-\tfrac{1}{2\sigma^2}
\sum_{i=1}^n(X_i-\mu)^2\big).$$
\item {\color{red}Log-vraisemblance}
$$\ell_n\big((\mu,\sigma^2),X_1,\ldots, X_n\big) = -\frac{n}{2}
\log(2\pi \sigma^2)-\frac{1}{2\sigma^2}\sum_{i = 1}^n (X_i-\mu)^2.$$
\end{itemize}
\end{frame}

\begin{frame}
\frametitle{Exemple : modèle normal }
%\begin{itemize}
%\item
{\color{red}Equation(s) de vraisemblance}
$$
\left\{
\begin{array}{lll}
\partial_\mu\ell_n \big((\mu,\sigma^2),X_1,\ldots, X_n\big) & = &\displaystyle\frac{1}{\sigma^2}\sum_{i = 1}^n (X_i-\mu) \\ \\
\partial_{\sigma^2}\ell_n \big((\mu,\sigma^2),X_1,\ldots, X_n\big)&
 = &\displaystyle -\frac{n}{2\sigma^2}+\frac{1}{2\sigma^4}
 \sum_{i = 1}^n (X_i-\mu)^2.
\end{array}
\right.
$$
Solution de ces \'equations (pour $n \geq 2$):
$$\boxed{\widehat
\vartheta_n^{\,{\tt rv}} = \big(\overline{X}_n,\frac{1}{n} \sum_{i =
1}^n(X_i-\overline{X}_n)^2\big)}$$ et on vérifie que $\widehat
\vartheta_n^{\,{\tt rv}} =\estMV$.
%\end{itemize}
\end{frame}



\begin{frame}
\frametitle{Exemple : modèle de Poisson}
\begin{itemize}
\item
{\color{red}Vraisemblance}
$${\mathcal L}_n(\vartheta, X_1,\ldots, X_n) =
\frac{1}{\prod_{i = 1}^n X_i!}e^{-n\vartheta}\vartheta^{\sum_{i = 1}^n X_i}.$$
\item {\color{red}Log-vraisemblance}
$$\ell_n(\vartheta, X_1,\ldots, X_n) = c(X_1,\ldots, X_n)-n\vartheta +\sum_{i =1}^n X_i \log \vartheta.$$
\item {\color{red}Equation de vraisemblance}
$$-n+\sum_{i = 1}^n X_i \frac{1}{\vartheta} = 0,\;\;
\text{soit}\;\;
\boxed{\widehat \vartheta_n^{\,{\tt rv}} = \frac{1}{n}\sum_{i = 1}^n X_i=\overline{X}_n}$$
et on vérifie que $\widehat \vartheta_n^{\,{\tt rv}} =\estMV$.
\end{itemize}
\end{frame}

\begin{frame}
\frametitle{Exemple : mod\`ele de Laplace} L'expérience statistique
est engendrée par un $n$-échantillon de loi de Laplace de paramètre
$\vartheta \in \Theta = \R$. La densité par rapport à la mesure de
Lebesgue :
$$f(\vartheta,x) = \frac{1}{2\sigma}\exp\big(-\frac{|x-\vartheta|}{\sigma}\big),$$
où $\sigma >0$ est {\color{red}connu}.
\begin{itemize}
\item {\color{red}Vraisemblance}
$${\mathcal L}_n(\vartheta, X_1,\ldots, X_n) = (2\sigma)^{-n}
\exp\big(-\frac{1}{\sigma}\sum_{i = 1}^n \big|X_i-\vartheta\big|\big)$$
\item {\color{red}Log-vraisemblance}
$$\ell_n(\vartheta,X_1,\ldots, X_n) = - n \log(2\sigma)-
\frac{1}{\sigma}\sum_{i = 1}^n \big|X_i-\vartheta\big|.$$
\end{itemize}
\end{frame}

\begin{frame}
\frametitle{Exemple : mod\`ele de Laplace} Maximiser ${\mathcal
L}_n(\vartheta, X_1,\ldots, X_n)$ revient à minimiser la fonction
$\vartheta \leadsto \sum_{i = 1}^n \big|X_i-\vartheta\big|$,
dérivable presque partout de dérivée constante par morceaux.
{\color{red}Equation de vraisemblance:}
$$\sum_{i = 1}^n \text{sign}(X_i-\vartheta)=0.$$
Soit $X_{(1)}\leq \ldots \leq X_{(n)}$ la statistique d'ordre.
\begin{itemize}
\item
$n$ pair: $\estMV$ {\color{red}n'est pas unique}; tout point de
l'intervalle
$\big[X_{\big(\tfrac{n}{2}\big)},X_{\big(\tfrac{n}{2}+1\big)} \big]$
est un EMV.
\item $n$ impair: $\estMV=X_{\big(\tfrac{n+1}{2}\big)}$,
l'EMV est unique. Mais $\widehat \vartheta_n^{\,{\tt rv}}$ n'existe
pas.
\item {\color{red}pour tout} $n$, la
médiane empirique est un EMV.
\end{itemize}
\end{frame}

\begin{frame}
\frametitle{Exemple : mod\`ele de Cauchy}
\begin{itemize}
\item {\color{red}Vraisemblance}
$${\mathcal L}_n(\vartheta, X_1,\ldots, X_n) = \pi^{-n} \prod_{i =1}^n \frac{1}{1+(X_i-\vartheta)^2}.$$
\item {\color{red}Log-vraisemblance}
$$\ell_n(\vartheta,X_1,\ldots, X_n) = -n\log \pi -\sum_{i = 1}^n \log\big(1+(X_i-\vartheta)^2\big)$$
\item {\color{red}Equation de vraisemblance}
$$\boxed{\sum_{i = 1}^n \frac{X_i-\vartheta}{1+(X_i-\vartheta)^2}=0}$$
pas de solution explicite et admet en général plusieurs solutions.
\end{itemize}
\end{frame}


\begin{frame}
\frametitle{Maximum de vraisemblance = $M$-estimateur}
\begin{itemize}
\item \underline{Une inégalité de convexité} : $\mu$ mesure $\sigma$-finie sur $\R$ ; $f,g$ deux {\color{red}densités de probabilités} par rapport à $\mu$. Alors
$$\boxed{\int_{\R}f(x)\log f(x) \mu(dx) \geq \int_{\R} f(x) \log g(x) \mu(dx)}$$
(si les intégrales sont finies) avec égalité {\color{red}ssi} $f=g$ $\mu$-pp.
\item \underline{Preuve}: à montrer
$$\int_{\R} f(x) \log \frac{g(x)}{f(x)}\mu(dx) \leq 0.$$
(avec une convention de notation appropriée)
\end{itemize}
\end{frame}

\begin{frame}
\frametitle{Une inégalité de convexité}
\begin{itemize}
\item On a $\log(1+x)\leq x$ pour $x \geq -1$ avec égalité ssi $x=0$.
\item Donc
$$\log \frac{g(x)}{f(x)} = \log\Big(1+\big(\frac{g(x)}{f(x)}-1\big)\Big) \leq \frac{g(x)}{f(x)}-1$$
(avec égalité ssi $f(x)=g(x)$).
\item Finalement
\begin{align*}
\int_{\R}f(x)\log \frac{g(x)}{f(x)}\mu(dx)& \leq \int_{\R} f(x)\Big(\frac{g(x)}{f(x)}-1\Big)\mu(dx) \\
& = \int_{\R} g(x)\mu(dx)- \int_{\R}f(x) \mu(dx)\\
&=0.
\end{align*}
\end{itemize}
\end{frame}

\begin{frame}
\frametitle{Conséquence pour l'EMV}
\begin{itemize}
\item On pose
$$\boxed{\psi(a,x):=\log f(a,x),\;\;a \in \Theta,\;x\in\R}$$
(avec une convention pour le cas où on n'a pas $f(a,\cdot) >0$.)
\item La fonction
$$a \leadsto \E_\vartheta \big[\psi(a,X)\big]=\int_{\R}\log f(a,x) f(\vartheta,x) \mu(dx)$$
a un maximum en $a=\vartheta$ d'après {\color{red}l'inégalité de convexité}.
\end{itemize}
\end{frame}

\begin{frame}
%\frametitle{}
\begin{itemize}
\item Le $M$-estimateur associé à $\psi$ maximise la fonction
$$a \leadsto \sum_{i = 1}^n \log f(a, X_i) = \ell_n(a, X_1,\ldots, X_n)$$
c'est-à-dire la {\color{red} log-vraisemblance}. C'est {\color{red}l'estimateur du maximum de vraisemblance}.

\item C'est aussi un $Z$-estimateur si la fonction $\vartheta \leadsto \log f(\vartheta, \cdot)$ est régulière, associé à la fonction
$$\phi(\vartheta, x) = \partial_\vartheta \log f(\vartheta, x) = \frac{\partial_\vartheta f(\vartheta, x)}{f(\vartheta, x)},\;\vartheta \in \Theta, x\in \R$$
lorsque $\Theta \subset \R$, \`a condition que le maximum de
log-vraisemblance n'est pas atteint sur la fronti\`ere de $\Theta$.
(Se généralise en dimension $d$.)
\end{itemize}
\end{frame}

\begin{frame}
\frametitle{Choix de mod\`ele statistique}
\begin{itemize}
\item  Le statisticien a le choix de la famille
$\{\PP_{\vartheta}, \vartheta\in\Theta\}$. L'EMV d\'epend de ce
choix.
\item \underline{Exemple:} on a l'\'echantillon ($n=10$):
$$ \underbrace{0.92, -0.20, -1.80, 0.02,  0.49, 1.41, -1.59, -1.29,
0.34}_{\footnotesize{tirage \ de \ {\mathcal N}(0,1)}},
{\color{red}100}.$$
\item On prend $\PP_{\vartheta}(dx) = f(x-\vartheta)dx$ pour deux
$f$ diff\'erents:
\item $f$ densit\'e de la loi normale $\Rightarrow$
$\estMV={\overline X}_n= {\color{red}9.83}$.
\item $f$ densit\'e de loi de Laplace $\Rightarrow$
 tout point de l'intervalle $[0.02, 0.34]$ est un $\estMV$, en
particulier, la m\'ediane: $$\estMV = M_n=
(0.02+0.34)/2={\color{red}0.18}.$$
\item {\color{red}Autre choix de mod\`ele...}
\end{itemize}
\end{frame}

\begin{frame}
\frametitle{Un $M$-estimateur qui n'est pas un $Z$-estimateur}
\begin{itemize}
\item On observe $X_1,\ldots, X_n\sim_{\text{i.i.d}.}$ uniformes sur $[0,\vartheta]$, $\vartheta \in \Theta = \R_+\setminus \{0\}$.
\item On a $$\PP_{\color{red}\vartheta}(dx) = {\color{red}\vartheta}^{-1}1_{[0,{\color{red}\vartheta}]}(x)dx$$
et
\begin{align*}
{\mathcal L}_n(\vartheta, X_1,\ldots, X_n)& = \vartheta^{-n}\prod_{i = 1}^n 1_{[0,\vartheta]}(X_i) \\
& = \vartheta^{-n}1_{\{\max_{1 \leq i \leq n} X_i \leq \vartheta\}}
\end{align*}
\item La fonction de vraisemblance {\color{red}n'est pas régulière}.
\item {\color{red}L'estimateur du maximum de vraisemblance est}
$\estMV = \max_{1 \leq i \leq n}X_i$. %{\color{red} A suivre...}
\end{itemize}
\end{frame}

%\begin{frame}
%\frametitle{Un $M$-estimateur qui n'est pas un $Z$-estimateur}
%
%\begin{center}
%%\vspace{-8cm}
%\includegraphics[width=10cm]{mvunif.eps}
%\end{center}
%%
%%
%%$$
%%\begin{array}{c}
%%\epsffile{}
%%\\
%%\\
%$$\mbox{\footnotesize Modèle uniforme: $\estMV=X_{(n)}.$}$$
%%\end{array}
%%$$
%\begin{itemize}
%\item La fonction de vraisemblance {\color{red}n'est pas régulière}.
%\item {\color{red}L'estimateur du maximum de vraisemblance est}
%$\estMV = \max_{1 \leq i \leq n}X_i$. {\color{red} A suivre...}
%\end{itemize}
%\end{frame}

\subsection{Convergence des $Z$- et $M$- estimateurs $\rightarrow$
Cours 4}

%\begin{frame}
%\frametitle{Asymptotique des $Z$- et $M$- estimateurs}
%\end{frame}

%\subsection{Modèle de régression}






\end{document}
