\documentclass[a4paper,11pt,fleqn]{article}

\usepackage[francais]{babel}
%\usepackage[latin1]{inputenc}
%\usepackage[applemac]{inputenc}
\usepackage{a4wide,amsmath,amssymb,amsthm,bbm,fancyhdr,graphicx,bbm}

\RequirePackage[OT1]{fontenc}

\usepackage[utf8]{inputenc}
% THE variable
\newcommand{\thisyear}{Ann\'ee 2015-2016}

% Definitions (pas trop!)
\def\1{\mathbbm{1}}
\def\rme{\mathrm{e}}
\newcommand{\R}{\ensuremath{\mathbb{R}}}
\newcommand{\N}{\mathbb{N}}
\newcommand{\rset}{\ensuremath{\mathbb{R}}}
\renewcommand{\P}{\ensuremath{\operatorname{P}}}
\newcommand{\E}{\ensuremath{\mathbb{E}}}
\newcommand{\V}{\ensuremath{\mathbb{V}}}
\newcommand{\gaus}{\ensuremath{\mathcal{N}}}
\newcommand{\dlim}{\ensuremath{\stackrel{\mathcal{L}}{\longrightarrow}}}
\newcommand{\plim}{\ensuremath{\stackrel{\mathrm{P}}{\longrightarrow}}}
\newcommand{\PP}{\ensuremath{\mathbb{P}}}
\newcommand{\p}{\ensuremath{\mathbb{P}}}
\newcommand{\eps}{\varepsilon}
\newcommand{\pa}[1]{\left(#1\right)}

\theoremstyle{definition}
\newtheorem{exercice}{Exercice}
\newcommand{\coint}[1]{\left[#1\right)}
\newcommand{\ocint}[1]{\left(#1\right]}
\newcommand{\ooint}[1]{\left(#1\right)}
\newcommand{\ccint}[1]{\left[#1\right]}


% Style
\pagestyle{fancyplain}
\renewcommand{\sectionmark}[1]{\markright{#1}}
\renewcommand{\subsectionmark}[1]{}
\lhead[\fancyplain{}{\thepage}]{\fancyplain{}{\footnotesize {\sf
MAP433 Statistique, \thisyear, PC3}}}
\rhead[\fancyplain{}{\footnotesize {\sf MAP433 Statistique,
\thisyear / \rightmark}}]{\fancyplain{}{\thepage}}
\cfoot[\fancyplain{}{}]{\fancyplain{}{}}
\renewcommand{\thefootnote}{\fnsymbol{footnote}}

% Titre
\title{{\bf MAP433 Statistique}\\
{\bf PC3: maximum de vraisemblance}}

\date{11 septembre 2015}

\begin{document}

\maketitle

\begin{exercice}[Dur\'ees de connection]
On peut mod\'eliser la dur\'ee d'une connection sur le site {\tt www.Cpascher.com} par une loi gamma$(2,1/\theta)$ de densit\'e
$$\theta^{-2} x \rme^{-x/\theta } \1_{\coint{0,+\infty}}(x).$$
Pour fixer vos tarifs publicitaires, vous voulez estimer le
param\`etre $\theta$ \`a partir
 d'un \'echantillon $X_1,\ldots,X_n$  de $n$ dur\'ees de connexion. On vous
 donne $\E_{\theta}(X_{i})=2\theta$ et $\textrm{var}_{\theta}(X_{i})=2\theta^2$.
%  rappelle que $\int_0^{+\infty}x^pe^{-x/\theta}dx=p!\theta^{p+1}$ pour
%$p\in\mathbb N$.
\begin{enumerate}
%\item Proposez un estimateur par la m\'ethode des moments.
\item  Calculez l'estimateur du maximum de vraisemblance $\hat \theta_n$ de $\theta$.
\item  Calculez $\E(\hat \theta_n)$?  Quelle est la variance de $\hat \theta_n$?
\end{enumerate}
\end{exercice}

\begin{exercice}[Mod\`ele exponentiel]
Consid\'erons une famille de fonctions de r\'epartition $\{F_{\theta},\ \theta\in\Theta\}$ ayant une
densit\'e par rapport \`a la mesure de Lebesgue sur $\R$
de la forme
$$p_{\theta}(x)=c(\theta)\exp\pa{\theta f(x)+h(x)}.$$
On suppose que $\Theta$ est un intervalle ouvert de $\R$, et $c(\cdot)\in C^2$,
$c(\theta)>0$ pour tout $\theta\in \Theta$. On note
$\varphi(\theta):=\E_{\theta}\pa{f(X)}=-{d\over
d\theta}\log(c(\theta))$.
%\begin{enumerate}
%\item

Soit $X_{1},\ldots,X_{n}$ un \'echantillon i.i.d.\;de densit\'e $p_{\theta}$, avec $\theta$ inconnu.
 Calculez l'estimateur du maximum de vraisemblance $\hat \theta_n$ de $\theta$ (s'il existe).
%\end{enumerate}
\end{exercice}


\begin{exercice}
Soient $\{(Y_i,Z_i)\}_{i=1}^n$ $n$ vecteurs aléatoires i.i.d. ; On suppose que $Y_1$ et $Z_1$ sont indépendants et distribués suivant des lois exponentielles de paramètres $\lambda > 0$ and $\mu > 0$.
\begin{enumerate}
\item On  observe $\{(Y_i,Z_i) \}_{i=1}^n$. Donnez le modèle statistique et déterminez l'estimateur du MV de $\lambda$ et $\mu$.
\item Déterminer la distribution limite de cet estimateur.
\item On observe $\{(X_i,\Delta_i)\}_{i=1}^n$ où $X_i= \min(Y_i,Z_i)$ et $\Delta_i= 1$ si $Y_i = X_i$ $\Delta_i=0$ autrement. Donnez le modèle statistique et l'estimateur de vraisemblance de $\lambda$ et $\mu$ dans ce modèle.
\item Déterminer la distribution limite de cet estimateur.
\end{enumerate}
\end{exercice}

\begin{exercice}[Mod\`ele d'autor\'egression]

On consid\`ere les observations $X_1,\dots,X_n$, o\`u les $X_i$
sont issus du {\it mod\`ele d'autor\'egression}:
$$
X_i=\theta X_{i-1} + \xi_i, \quad i=1,\dots,n, \quad\quad X_0=0,
$$
avec $\xi_i$ i.i.d. de loi normale ${\mathcal N}(0,\sigma^2)$ et
$\theta\in \R$.
%\begin{enumerate}
%\item Explicitez la densit\'e de la loi jointe de $(X_{1},\ldots,X_{n})$ par rapport \`a la mesure de Lebesgue sur ${\bf R}^n$.
%\item
Calculez l'estimateur du maximum de vraisemblance  $(\hat \theta_n,\hat \sigma_{n}^2)$ de $(\theta,\sigma^2)$.
%\end{enumerate}


%\begin{figure}
%\includegraphics[width=13cm]{../images/AR.pdf}
%\caption{Figure 1. Trajectoires simul\'ees d'un AR(1) pour $\theta\in\{0,0.8,1\}$}
%\end{figure}
\end{exercice}

\begin{exercice}[R\'epartition de g\'enotypes dans une population]
Quand les fr\'equences de g\`enes sont en \'equilibre, les
g\'enotypes AA, Aa et aa se manifestent dans une population avec
probabilit\'es $(1-\theta)^2$, $2\theta(1-\theta)$ et $\theta^2$
respectivement, o\`u $\theta$ est un param\`etre inconnu. Plato {\it et
al.} (1964) ont publi\'e les donn\'ees suivantes sur le type de
haptoglobine dans un \'echantillon de 190 personnes:
$$
%\\
%\hline
\begin{array}{ccc}
&\text{Type \ de \ haptoglobine:}&\\\hspace{6mm}
&\text{------------------------------ }&\\
 Hp-AA& Hp-Aa & Hp-aa \\
 10&68&112
\end{array}
$$
\begin{enumerate}
\item Comment interpr\'eter le param\`etre $\theta$? Proposez un mod\`ele statistique pour ce probl\`eme.
\item  Calculez l'estimateur du maximum de vraisemblance $\hat \theta_n$ de $\theta$.
\item  Donnez la loi asymptotique de $\sqrt{n}(\hat \theta_n -
\theta)$.
%\item Proposez un intervalle de confiance de niveau asymptotique 95\% pour
%$\theta$.
\end{enumerate}
\end{exercice}

%\section{Analyse d'un canal de communication}
%Une variable al\'eatoire r\'eelle $X$ suit une loi de
%Pareto$(\alpha,\theta)$ avec $\alpha>1$ et $\theta>0$, si elle a
%pour densit\'e par rapport \`a la mesure de Lebesgue
%$$f_{X}(x)={\alpha-1\over \theta}\pa{\theta\over x}^{\alpha}{\bf
%1}_\coint{\theta,\infty}}(x).$$ Les paquets d'information arrivent
%al\'eatoirement dans un canal de communication et le temps entre
%deux paquets est mod\'elis\'e par une loi de Pareto. On dispose
%d'un \'echantillon $X_{1},\ldots,X_{n}$ de temps d'attente,
%suppos\'es ind\'ependants.
%\begin{enumerate}
%\item Comment interpr\'eter $\alpha$ et $\theta$? V\'erifier que pour $\alpha>3$
%$$\E(X)={\alpha-1\over \alpha-2}\theta \quad \textrm{et}\quad
%\textrm{Var}(X)={\alpha-1\over (\alpha-3)(\alpha-2)^2}\theta^2.$$
%\item On suppose $\alpha>3$ connu. Calculez l'estimateur $\hat \theta$ du maximum de
%vraisemblance de $\theta$.
%\item Calculez $P(\hat \theta>x).$ Quelle est la loi de $\hat \theta$? sa moyenne? sa variance?
%\item Que dire du comportement de $\hat \theta$ lorsque $n\to\infty$?
%\item Maintenant, on suppose $\theta$ connu, mais pas $\alpha$. Quel est
%l'estimateur $\hat \alpha$ du maximum de vraisemblance de
%$\alpha$?
%\item Quelle est la loi de $\log(X_{i}/ \theta)$? de $\sum_{i=1}^n\log(X_{i}/ \theta)$?
%\item Calculez le biais $\E(\hat \alpha)-\alpha$ de $\hat \alpha$? Proposez un
%estimateur non biais\'e de $\alpha$.
%\item Quelle est la variance de ce dernier estimateur? Quel est son comportement
%lorsque $n\to \infty$?
%\end{enumerate}

%\subsection*{5. Param\`etre vectoriel - vitesses de convergence diff\'erentes}
%Soient $X_1, \ldots,X_n$ des variables al\'eatoires i.i.d. de loi
%exponentielle translat\'ee dont le densit\'e est de la forme:
%$$f(x,\theta,\alpha)=\frac{1}{\theta}\text{exp}\left[-
%\frac{(x-\alpha)}{\theta}\right]I_{[\alpha,+\infty[}(x),$$ o\`u
%$\theta>0$ et $\alpha\in \R$ sont deux param\`etres inconnus.
%\begin{enumerate}
%\item Donner les estimateurs du maximum de vraisemblance $\hat{\alpha}_n$ et
%$\hat{\theta}_n$ de $\alpha$ et $\theta$.
%\item Quelle est la loi de $X_{i}-\alpha$? Calculer la loi (exacte) de $n(\hat{\alpha}_n-\alpha)$.
%\item D\'eterminer la loi limite de $\sqrt{n}(\hat{\theta}_n-\theta)$.
%%\item  Chercher la loi
%%du $n$-uplet
%%$$\big(nX_{(1)},(n-1)(X_{(2)}-X_{(1)}),\ldots,2(X_{(n-1)}-X_{(n-2)}),X_{(n)}-X_{(n-1)}\big).$$
%%En d\'eduire que $\hat{\alpha}_n$ et $\hat{\theta}_n$ sont
%%ind\'ependants pour tout $n$.
%\end{enumerate}
%
%\subsection*{6. La statistique d'ordre}
%
%Soient $X_1, \ldots,X_n$ des variables al\'eatoires i.i.d. de
%fonction de r\'epartition $F$. On suppose que $F$ admet une
%densit\'e $f$ par rapport \`a la mesure de Lebesgue. On note $X_{(1)}\leq X_{(2)}\leq\ldots\leq X_{(n)}$ les variables al\'eatoires $X_1, \ldots,X_n$ r\'eordonn\'ees par ordre croissant.
%\begin{enumerate}
%\item
%Donner l'expression de la loi de la statistique d'ordre
%$(X_{(1)},\ldots,X_{(n)})$ en fonction de $f$.
%\item D\'eterminer la fonction de r\'epartition $F_{k}(x)$ puis la densit\'e $f_k(x) $ de $X_{(k)}$.
%%Calculer la fonction de r\'epartition, not\'ee $G_k(x)$, de
%%$X_{(k)}$.
%\item Sans utiliser les r\'esultats des questions pr\'ec\'edentes,
%calculer les fonctions de r\'epartition de $X_{(1)}$, $X_{(n)}$,
%du couple $(X_{(1)}, X_{(n)})$ et la loi de la statistique $W=
%X_{(n)}- X_{(1)}$ (on appelle $W$ {\it \'etendue}). Les
%variables $X_{(1)}$ et $X_{(n)}$ sont--elles ind\'ependantes?
%\end{enumerate}
%

\end{document}
