%\documentclass[a4paper,11pt,fleqn]{article}

\documentclass[10pt]{article}
\setlength{\topmargin}{0cm} \setlength{\textheight}{23cm}
\setlength{\oddsidemargin}{-0.5cm} \setlength{\textwidth}{20cm}

\usepackage[francais]{babel}
%\usepackage[latin1]{inputenc}
\usepackage[applemac]{inputenc}
\usepackage{a4wide,amsmath,amssymb,bbm,fancyhdr, graphicx}

% THE variable
\newcommand{\thisyear}{Ann�e 2014-2015}

% Definitions (pas trop!)
\newcommand{\R}{\ensuremath{\mathbb{R}}}
\newcommand{\rset}{\ensuremath{\mathbb{R}}}
\renewcommand{\P}{\ensuremath{\operatorname{P}}}
\newcommand{\E}{\ensuremath{\mathbb{E}}}
\newcommand{\V}{\ensuremath{\mathbb{V}}}
\newcommand{\gaus}{\ensuremath{\mathcal{N}}}
\newcommand{\1}{\ensuremath{\mathbbm{1}}}
\newcommand{\dlim}{\ensuremath{\stackrel{\mathcal{L}}{\longrightarrow}}}
\newcommand{\plim}{\ensuremath{\stackrel{\mathrm{P}}{\longrightarrow}}}
\newcommand{\PP}{\ensuremath{\mathbb{P}}}
\newcommand{\eps}{\varepsilon}

% Style
\pagestyle{fancyplain}
\renewcommand{\sectionmark}[1]{\markright{#1}}
\renewcommand{\subsectionmark}[1]{}
\lhead[\fancyplain{}{\thepage}]{\fancyplain{}{\footnotesize {\sf MAP433 Statistique, \thisyear, PC1}}}
\rhead[\fancyplain{}{\footnotesize {\sf MAP433 Statistique, \thisyear\  / \rightmark}}]{\fancyplain{}{\thepage}}
\cfoot[\fancyplain{}{}]{\fancyplain{}{}}
\renewcommand{\thefootnote}{\fnsymbol{footnote}}
\parindent=0mm



% Titre
\title{\includegraphics[width=3.5cm]{../images/logo_x.jpg}\hfill {\bf MAP433 Statistique}\hfill \quad\quad\quad\quad \ }
\author{\bf PC1: Rappels de probabilit\'es}


\date{29 ao\^ut 2014}

\begin{document}


\maketitle

\vspace{-0.1cm}






\section{Th�or�me central limite} % 
Soit $\left( X_{n}\right)_{n}$ une suite de variables al�atoires
i.i.d. centr�es de variance $\sigma^2>0$. Soit
\begin{equation*}
  Z_{n}=\frac{1}{\sigma\sqrt{n}}\sum_{j=1}^{n} X_{j}\,.
\end{equation*}%
Par le th�or�me central limite, cette variable converge en loi vers la
loi normale centr�e r�duite, c'est-�-dire, pour tout $t \in \R$, on a   $
\lim_{n\rightarrow +\infty} \E[e^{itZ_{n}}]=e^{-\frac{t^{2}}{2}}$. L'objet de
cet exercice est de montrer que la suite $Z_{n}$ ne peut pas converger en probabilit�.
\begin{enumerate}
\item Calculer la fonction caract�ristique de $Z_{2n}-Z_{n}$ et montrer que
  cette diff�rence converge en loi.
\item En \'etudiant $\PP(|Z_{2n}-Z_{n}|\geq\eps)$, montrer que $Z_{n}$ ne converge pas en
  probabilit�.
\end{enumerate}

\section{Lemme de Slutsky} \label{slutski1}
\begin{enumerate}
\item Donner un exemple de suites $(X_{n})$ et $(Y_{n})$ telles que $X_{n}\stackrel{\textrm{loi}}{\to}X$ et $Y_{n}\stackrel{\textrm{loi}}{\to}Y$, mais $X_{n}+Y_{n}$ ne converge pas en loi vers $X+Y$.
\item Soient $(X_n)$, $(Y_n)$ deux suites
de variables al�atoires r\'eelles,
$X$ et $Y$ des variables al�atoires r\'eelles, telles que
\begin{enumerate}
\item[(i)]  $X_{n}\stackrel{\textrm{loi}}{\to}X$ et $Y_{n}\stackrel{\mathbb{P}}{\to}Y$,
\item[(ii)] $Y$ est ind�pendante de $(X_n)$ et $X$.
\end{enumerate}
Montrer que le couple $(X_n,Y_n)$ converge en loi vers $(X,Y)$.
\item En d\'eduire que si $(X_n)$ et $(Y_n)$ sont deux suites de variables al�atoires r�elles telles que $(X_n)$
converge en loi vers une limite $X$ et $(Y_n)$ converge en
probabilit� vers une constante $c$, alors $(X_n+Y_n)$ converge en
loi vers $X+c$ et $(X_n\, Y_n)$ converge en loi vers $c \, X$.
\end{enumerate}


\section{Estimateur de la variance}
Soient $X_{1},\ldots,X_{n}$ des variables al\'eatoires i.i.d.,
$X_i\sim f(\cdot-\theta)$, o\`u $f$ est une densit\'e de
probabilit\'e sur $\mathbb{R}$ sym\'etrique dont on note $\mu_k =
\int_{\R}x^kf(x)\, dx$ les moments d'ordre $k=2$ et $k=4$. On note
$\bar{X}_n = \frac{1}{n}\sum_{i=1}^n X_i$.
 Montrer que l'estimateur $\frac{1}{n}\sum_{i=1}^n (X_i -\bar{X}_n)^2$
 de la variance des $X_{i}$ v\'erifie un th\'eor\`eme central limite.
 
 \medskip


  {\it Indication}~: on montrera d'abord que l'on peut se ramener au cas
  o\`u $\theta=0$, puis on exprimera l'estimateur comme une transformation de $S_n
  = \frac{1}{n}\sum_{i=1}^n X_i^2$ et de $\bar{X}_n$.




\section{Taux de d\'efaillance}
Une cha\^{\i}ne de production  doit
garantir une qualit\'e minimale de ses produits. En particulier,
elle doit garantir que la proportion $\theta$ des produits
d\'efaillants reste inf\'erieure \`a un taux fix\'e par le client.
Un
\'echantillon de  $n$ produits  est pr\'elev\'e et
analys\'e. %On
%suppose que le nombre $n$ de composants pr\'elev\'es est faible
%devant le nombre de composants produits.
On note $\hat \theta_{n}$ la proportion de produits d\'efectueux
dans l'\'echantillon.
\begin{enumerate}
\item Proposer un mod\`ele statistique pour ce probl\`eme. Quelle est la loi de $n\hat\theta_{n}$?
\item Quelle information donne la loi des grands nombres et le th\'eor\`eme central limite sur le comportement asymptotique de $\hat\theta_{n}$?
\item On donne $\PP(N>1.64)=5\%$ pour $N\sim\mathcal{N}(0,1)$.
En d\'eduire $\eps_{n}$ (d\'ependant de $n$ et $\theta$) tel que
$\PP(\theta \geq \hat\theta_{n}+\eps_{n})\stackrel{n\to \infty}{\to} 5\%$.
\item La valeur $\eps_{n}$ pr\'ec\'edente d\'epend de $\theta$. A l'aide du lemme de Slutsky, donner $\eps'_{n}$ ne d\'ependant que de $n$ et $\hat\theta_{n}$ tel que $\PP(\theta \geq \hat\theta_{n}+\eps'_{n})\stackrel{n\to \infty}{\to} 5\%$.
\end{enumerate}

%\subsection*{7. Cas des d\'efaillances rares}
%La cha\^{\i}ne produit des composants \'electroniques utilis\'es dans le secteur
%a\'eronautique. Le taux de d\'efaillance doit donc \^etre tr\`es
%bas. En particulier, comme la taille de l'\'echantillon ne peut \^etre tr\`es grosse (question de co\^ut), il est attendu que $\theta$ soit du m\^eme ordre de grandeur que $1/n$. 
% On supposera donc par la suite que la proportion de composants
%d\'efectueux est $\theta_{n}=\lambda/n$ pour un certain $\lambda>0$
%et on cherche \`a estimer $\lambda$ par
%$\hat\lambda_{n}=n\hat\theta_{n}$. La valeur $\lambda$ est
%suppos\'ee ind\'ependante de $n$ (le cas int\'eressant est quand
%$\lambda$ est petit).
%\begin{enumerate}
%\item Quelle est la limite de
%$\PP(\hat\lambda_{n}=k)$ lorsque $n\rightarrow+\infty$? En
%d�duire que $\hat\lambda_n$ converge en loi vers une variable de
%Poisson de param\`etre $\lambda$.
%\item On suppose qu'il y a une proportion $\theta_{n}=3/n$ de
%composants d\'efectueux.  Sachant que $\PP(Z=0)\approx 5\%$ pour $Z$
%de loi de Poisson de param\`etre 3, montrer que $\PP(\theta_{n}>\hat
%\theta_{n}+2/n)\approx 5\%$ pour $n$ grand.
%\end{enumerate}

%\subsection*{3. Convergence dans $L^p$}
%Soit $(X_n)$ une suite de variables
%al�atoires r�elles bor\-n�es par une m�me constante. Montrer que si
%$(X_n)$ converge en probabilit�, alors $X_n$ converge dans $L^p$ pour tout $p \ge 1$.

%\subsection*{4. Loi conditionnelle}
%\label{loicond1}
%Soit $X$ une variable al�atoire qui suit une loi Gamma $\left( 2,\lambda
%\right) $ de densit�
%\begin{equation*}
%  f\left( x\right) =\lambda ^{2}xe^{-\lambda x}\1_{[0,+\infty)}(x)
%\end{equation*}%
%et soit $Y$ une variable al�atoire dont la loi conditionnelle � $X=x$ est
%uniforme sur $\left[ 0,x\right] .$
%
%\begin{enumerate}
%\item Donner la loi jointe de $\left( X,Y\right)$.
%\item Donner la loi marginale de $Y$ et montrer que $Y$ est ind�pendant de
%  $X-Y$.
%\end{enumerate}



\end{document}
